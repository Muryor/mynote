\examxtitle{zhejiang_lishui_2026_nov_preprocessed}

\section{单选题}

\begin{question}
已知复数\(z\),若\(\frac{z}{2 + \text{i}} = \text{i}\)(\(\text{i}\)为虚数单位),则\(|z| =\)(  )
\begin{choices}
  \item \(\sqrt{5}\)
  \item 5
  \item \(\sqrt{3}\)
  \item 3
\end{choices}
\topics{求复数的模;复数的除法运算}
\difficulty{0.85}
\answer{A}
\explain{由方程
\(\frac{z}{2 + \text{i}} = \text{i}\)得\(z = \text{i} \cdot (2 + \text{i}) = 2\text{i} + \text{i}^{2} = 2\text{i} + ( - 1) = - 1 + 2\text{i}\),
%
所以\(|z| = \sqrt{{( - 1)}^{2} + 2^{2}} = \sqrt{1 + 4} = \sqrt{5}\)}
\end{question}

\begin{question}
已知集合\(A = \{ x \mid - 2 \leq x < 1\},B = \left\{ m,3 \right\}\),且\(A \cap B\)的元素个数是一个,则实数\(m\)的取值范围是(  )
\begin{choices}
  \item \(( - 2,1)\)
  \item \(\lbrack - 2,1\rbrack\)
  \item \(\lbrack - 2,1)\)
  \item \(( - 2,1\rbrack\)
\end{choices}
\topics{根据元素与集合的关系求参数;根据交集结果求集合或参数}
\difficulty{0.85}
\answer{C}
\explain{由\(A \cap B\)的元素个数是一个,且\(3 \notin A\),得\(m \in A\),则\(- 2 \leq m < 1\),
所以实数\(m\)的取值范围是\(\lbrack - 2,1)\)}
\end{question}

\begin{question}
已知\(F_{1},F_{2}\)为双曲线\(C:\frac{x^{2}}{a^{2}} - \frac{y^{2}}{b^{2}} = 1(a > 0,b > 0)\)的左右焦点,点\(A\)的坐标为\((0,2b)\).若\(\bigtriangleup AF_{1}F_{2}\)为等边三角形,则双曲线\(C\)的离心率是(  )
\begin{choices}
  \item \(\sqrt{3}\)
  \item 2\(\sqrt{3}\)
  \item 2
  \item 3
\end{choices}
\topics{求双曲线的离心率或离心率的取值范围}
\difficulty{0.85}
\answer{C}
\explain{\(\because \bigtriangleup AF_{1}F_{2}\)为等边三角形,\(O\)为\(F_{1}F_{2}\)的中点,
%

% IMAGE_TODO_START id=zhejiang_lishui_2026_nov-Q3-img1 path=/Users/muryor/code/mynote/word\\_to\\_tex/output/figures/raw/media/image2.png width=60% inline=true question_index=3 sub_index=1
% CONTEXT_BEFORE: eup AF_{1}F_{2}\(为等边三角形,\)O\(为\)F_{1}F_{2}\(的中点,
% CONTEXT_AFTER: \)\therefore\
\begin{tikzpicture}[scale=0.8,baseline=-0.5ex]
  % TODO: AI_AGENT_REPLACE_ME (id=zhejiang_lishui_2026_nov-Q3-img1)
\end{tikzpicture}
% IMAGE_TODO_END id=zhejiang_lishui_2026_nov-Q3-img1\(\therefore\tan\angle AF_{1}O = \frac{2b}{c} = \sqrt{3}\),则\(\frac{b}{c} = \frac{\sqrt{3}}{2}\),
%
\(\because e^{2} = \frac{c^{2}}{a^{2}} = \frac{c^{2}}{c^{2} - b^{2}} = \frac{1}{1 - \left( \frac{b}{c} \right)^{2}} = \frac{1}{1 - \frac{3}{4}} = 4\),
%
\(\therefore e = 2\).}
\end{question}

\begin{question}
已知\(x,y \in \text{R}\),则下列条件中使\(x > y\)成立的充要条件是(  )
\begin{choices}
  \item \(|x| > y\)
  \item \(x^{2} > y^{2}\)
  \item \(a^{x} > a^{y}(a > 0,且a \neq 1)\)
  \item \(\text{ln}(x - y + 1) > 0\)
\end{choices}
\topics{探求命题为真的充要条件;对数函数单调性的应用;由指数函数的单调性解不等式}
\difficulty{0.65}
\answer{D}
\explain{对于A,当\(x = - 2,y = 1\)时,满足\(|x| > y\),但不满足\(x > y\),
所以\(|x| > y\)不是\(x > y\)的充要条件,故A错误;
对于B,当\(x = - 2,y = 1\)时,满足\(x^{2} > y^{2}\),但不满足\(x > y\),
所以\(x^{2} > y^{2}\)不是\(x > y\)的充要条件,故B错误;
对于C,当\(a > 1\)时,指数函数\(y = a^{x}\)为增函数,若\(a^{x} > a^{y}\),则\(x > y\),
当\(0 < a < 1\)时,指数函数\(y = a^{x}\)为减函数,若\(a^{x} > a^{y}\),则\(x < y\),
所以\(a^{x} > a^{y}\)(\(a > 0\)且\(a \neq 1\))不是\(x > y\)的充要条件,故C错误;
对于D,若\(\ln(x - y + 1) > 0 = \ln 1\),则\(x - y + 1 > 1 \Rightarrow x - y > 0\),即\(x > y\),
反之,若\(x > y\),则\(x - y > 0\),则\(x - y + 1 > 1\),所以\(\ln(x - y + 1) > \ln 1 = 0\),
所以\(\ln(x - y + 1) > 0\)是\(x > y\)的充要条件,故D正确}
\end{question}

\begin{question}
定义在\(R\)上的两个函数\(f(x),g(x)\),恒有\(f^{3}(x) = g\left( x^{2} \right)\),则(  )
\begin{choices}
  \item \(f(x)\)为奇函数
  \item \(f(x)\)为偶函数
  \item \(g(x)\)为奇函数
  \item \(g(x)\)为偶函数
\end{choices}
\topics{函数奇偶性的定义与判断}
\difficulty{0.85}
\answer{B}
\explain{由\(f^{3}(x) = g\left( x^{2} \right)\),则\(f^{3}( - x) = g\left\lbrack ( - x)^{2} \right\rbrack = g\left( x^{2} \right) = f^{3}(x)\),
则\(f( - x) = f(x)\),又\(f(x)\)定义域为\(R\),故\(f(x)\)为偶函数,故B正确;
由已知得不到\(g(x)\)与\(g( - x)\)关系,也得不到\(f(x) + f( - x)\)是否为\(0\),故A、C、D错误}
\end{question}

\begin{question}
若函数\(y = \text{sin}\left| \omega x + \frac{\text{π}}{6} \right|\)的图象向右平移\(\frac{\text{π}}{4}\)个单位后得到的图象关于\(y\)轴对称,则实数\(\omega\)可以是(  )
\begin{choices}
  \item \(\frac{2}{3}\)
  \item \(- \frac{2}{3}\)
  \item 2
  \item \(- 2\)
\end{choices}
\topics{利用正弦函数的对称性求参数}
\difficulty{0.65}
\answer{A}
\explain{因为函数\(y = \text{sin}\left| \omega x + \frac{\text{π}}{6} \right|\)的图象向右平移\(\frac{\text{π}}{4}\)个单位后得到的图象关于\(y\)轴对称,
%
可知函数\(y = \text{sin}\left| \omega x + \frac{\text{π}}{6} \right|\)关于直线\(x = - \frac{\text{π}}{4}\)对称,
%
若\(\omega = 0\),则函数\(y = \sin\frac{\text{π}}{6} = \frac{1}{2}\)关于直线\(x = - \frac{\text{π}}{4}\)对称,符合题意;
%
若\(\omega \neq 0\),设\(t = \omega x + \frac{\text{π}}{6}\),
%
则函数\(y = \sin\left| \omega x + \frac{\pi}{6} \right|\)的对称轴\(x = - \frac{\pi}{4}\)所对应的\(t\)值(\(t = - \frac{\pi}{4}\omega + \frac{\pi}{6}\))必为函数\(y = \sin|t|\)的对称轴,
%
又因为函数\(y = \sin|t|\)的对称轴为\(y\)轴,
%
则\(- \frac{\text{π}}{4}\omega + \frac{\text{π}}{6} = 0\),解得\(\omega = \frac{2}{3}\);
%
综上所述:\(\omega = 0\)或\(\omega = \frac{2}{3}\).
%
结合选项可知:A正确,BCD错误}
\end{question}

\begin{question}
已知三棱锥\(S - ABC\),满足\(SA = SB = SC\),且\(SA\),\(SB\),\(SC\)两两垂直.在底面\(\bigtriangleup ABC\)内有一动点\(P\)到三个侧面的距离依次成等差数列,则点\(P\)的轨迹是(  )
\begin{choices}
  \item 一个点
  \item 一条线段
  \item 一段圆弧
  \item 一段抛物线
\end{choices}
\topics{等差中项的应用;立体几何中的轨迹问题}
\difficulty{0.4}
\answer{B}
\explain{在三棱锥\(S - ABC\)中,\(SA = SB = SC\),且\(SA\),\(SB\),\(SC\)两两垂直,
%
\(\therefore AB = \sqrt{SA^{2} + SB^{2}},AC = \sqrt{SA^{2} + SC^{2}},BC = \sqrt{SC^{2} + SB^{2}}\),
%
\(\therefore AB = AC = BC\),即\(\bigtriangleup ABC\)为等边三角形,
%
设点\(P\)到平面\(SAB\)、平面\(SBC\)、平面\(SCA\)的距离依次为\(h_{1}\)、\(h_{2}\)、\(h_{3}\),如下图所示:
%

% IMAGE_TODO_START id=zhejiang_lishui_2026_nov-Q7-img1 path=/Users/muryor/code/mynote/word\\_to\\_tex/output/figures/raw/media/image3.png width=60% inline=true question_index=7 sub_index=1
% CONTEXT_BEFORE: $SCA\(的距离依次为\)h_{1}\(、\)h_{2}\(、\)h_{3}\(,如下图所示:
\begin{tikzpicture}[scale=0.8,baseline=-0.5ex]
  % TODO: AI_AGENT_REPLACE_ME (id=zhejiang_lishui_2026_nov-Q7-img1)
\end{tikzpicture}
% IMAGE_TODO_END id=zhejiang_lishui_2026_nov-Q7-img1{
%
由题意可知,\(h_{1} + h_{3} = 2h_{2}\),则\(\frac{1}{3}S_{\bigtriangleup SAB}h_{1} + \frac{1}{3}S_{\bigtriangleup SAC}h_{3} = 2 \times \frac{1}{3}S_{\bigtriangleup SBC}h_{2}\),
%
即\(V_{P - SAB} + V_{P - SAC} = 2V_{P - SBC}\),即\(V_{S - PAB} + V_{S - PAC} = 2V_{S - PBC}\),
%
所以,\(S_{\bigtriangleup PAB} + S_{\bigtriangleup PAC} = 2S_{\bigtriangleup PBC}\),
%
不妨设点\(P\)到边\(AB\)、\(BC\)、\(AC\)的距离分别为\(d_{1}\)、\(d_{2}\)、\(d_{3}\),
%
设等边\(\bigtriangleup ABC\)的边长为\(a\),则\(S_{\bigtriangleup ABC} = \frac{\sqrt{3}}{4}a^{2}\),
%
又因为\(S_{\bigtriangleup PAB} + S_{\bigtriangleup PBC} + S_{\bigtriangleup PAC} = S_{\bigtriangleup ABC}\),即\(\frac{1}{2}a\left( d_{1} + d_{2} + d_{3} \right) = \frac{\sqrt{3}}{4}a^{2}\),
%
所以,\(d_{1} + d_{2} + d_{3} = \frac{\sqrt{3}}{2}a\),①
%
由\(S_{\bigtriangleup PAB} + S_{\bigtriangleup PAC} = 2S_{\bigtriangleup PBC}\),可得\(\frac{1}{2}a\left( d_{1} + d_{3} \right) = 2 \times \frac{1}{2}ad_{2}\),可得\(d_{1} + d_{3} = 2d_{2}\),②
%
联立①②可得\(d_{2} = \frac{\sqrt{3}}{6}a\),
%
所以,点\(P\)的轨迹是一条与\(BC\)平行且与\(BC\)之间的距离为\(\frac{\sqrt{3}}{6}a\)的线段.}
\end{question}
%
\begin{question}
若关于\(x\)的方程\(\left( \text{e}^{x} - tx \right)\left( x - t\text{ln}x \right) = 0(t \in R)\)恰有四个不同的实根\(a,b,c,d(a < b < c < d)\),则(  )
\begin{choices}
  \item \(a + d < b + c\)
  \item \(a + d = b + c\)
  \item \(ad < bc\)
  \item \(ad = bc\)
\end{choices}
\topics{根据函数零点的个数求参数范围;用导数判断或证明已知函数的单调性;函数单调性;极值与最值的综合应用;利用导数研究方程的根}
\difficulty{0.4}
\answer{D}
\explain{由\(\left( \text{e}^{x} - tx \right)\left( x - t\text{ln}x \right) = 0\),则\(\text{e}^{x} - tx = 0\)或\(x - t\text{ln}x = 0\),
%
则\(t = \frac{\text{e}^{x}}{x}\)或\(t = \frac{x}{\ln x}\),令\(f(x) = \frac{\text{e}^{x}}{x}\),则\(f'(x) = \frac{\text{e}^{x}(x - 1)}{x^{2}}\),
%
当\(x \in ( - \infty,0) \cup (0,1)\)时,\(f'(x) < 0\),当\(x \in (1, + \infty)\)时,\(f'(x) > 0\),
%
故\(f(x)\)在\(( - \infty,0)\)、\((0,1)\)上单调递减,在\((1, + \infty)\)上单调递增,
%
又当\(x < 0\)时,\(f(x) = \frac{\text{e}^{x}}{x} < 0\),\(f(1) = \frac{\text{e}^{1}}{1} = \text{e}\),
%
故当\(t < 0\)或\(t = \text{e}\)时,\(t = \frac{\text{e}^{x}}{x}\)仅有一根,当\(t > \text{e}\)时,\(t = \frac{\text{e}^{x}}{x}\)有两根,
%
又\(f\left( \ln x \right) = \frac{\text{e}^{\ln x}}{\ln x} = \frac{x}{\ln x}\),则\(t = \frac{x}{\ln x}\)最多有两根,
%
由题意可得\(t = \frac{\text{e}^{x}}{x}\)与\(t = \frac{x}{\ln x}\)共有四个不同根,
%
故\(t > \text{e}\),设\(t = \frac{\text{e}^{x}}{x}\)两根分别为\(x_{1}\)、\(x_{2}\),且\(0 < x_{1} < 1 < x_{2}\),
%
则\(t = \frac{x}{\ln x}\)两根分别为\(\text{e}^{x_{1}}\)、\(\text{e}^{x_{2}}\),则\(\text{1} < \text{e}^{x_{1}} < \text{e} < \text{e}^{x_{2}}\),
%
则有\(0 < x_{1} < \text{e}^{x_{1}} < x_{2} < \text{e}^{x_{2}}\)或\(x_{1} < x_{2} < \text{e}^{x_{1}} < \text{e}^{x_{2}}\),
%
若\(0 < x_{1} < \text{e}^{x_{1}} < x_{2} < \text{e}^{x_{2}}\),则\(a = x_{1}\)、\(b = \text{e}^{x_{1}}\)、\(c = x_{2}\)、\(d = \text{e}^{x_{2}}\),
%
若\(x_{1} < x_{2} < \text{e}^{x_{1}} < \text{e}^{x_{2}}\),则\(a = x_{1}\)、\(b = x_{2}\)、\(c = \text{e}^{x_{1}}\)、\(d = \text{e}^{x_{2}}\),
%
故\(ad = x_{1}\text{e}^{x_{2}}\),\(bc = x_{2}\text{e}^{x_{1}}\),
%
由\(t = \frac{\text{e}^{x_{1}}}{x_{1}} = \frac{\text{e}^{x_{2}}}{x_{2}}\),则\(x_{1}\text{e}^{x_{2}} = x_{2}\text{e}^{x_{1}}\),即有\(ad = bc\),故D正确,C错误;
%
\(a + d = x_{1} + \text{e}^{x_{2}}\),\(b + c = x_{2} + \text{e}^{x_{1}}\),
%
则\((a + d) - (b + c) = x_{1} + \text{e}^{x_{2}} - x_{2} - \text{e}^{x_{1}} = \left( \text{e}^{x_{2}} - x_{2} \right) - \left( \text{e}^{x_{1}} - x_{1} \right)\),
%
令\(g(x) = \text{e}^{x} - x\),则\(g'(x) = \text{e}^{x} - 1\),
%
则当\(x > 0\)时,\(g'(x) > 0\),则\(g(x)\)在\((0, + \infty)\)上单调递增,
%
由\(x_{2} > x_{1} > 0\),则\(g\left( x_{2} \right) > g\left( x_{1} \right)\),即\(\left( \text{e}^{x_{2}} - x_{2} \right) - \left( \text{e}^{x_{1}} - x_{1} \right) > 0\),
%
即\((a + d) - (b + c) > 0\),即有\(a + d > b + c\),故A、B错误}
\end{question}
%
\section{多选题}
%
\begin{question}
已知随机变量\(\xi \sim N\left( 0,\sigma^{2} \right),\varphi(x) = P(\xi \leq x)\),则下列等式正确的是(  )
\begin{choices}
  \(\varphi( - x) = 1 - \varphi(x)\)

\begin{enumerate}[label=(\arabic*)]
  \item \(P\left( |\xi| \leq 1 \right) = 1 - 2\varphi(1)\)
  \item \(P\left( |\xi| \leq 1 \right) = 2\varphi(1) - 1\)
  \item \(P\left( |\xi| > 1 \right) = 2 - 2\varphi(1)\)
\end{choices}
\topics{正态曲线的性质}
\difficulty{0.85}
\answer{ACD}
\explain{因为随机变量\(\xi \sim N\left( 0,\sigma^{2} \right)\),
则正态分布曲线关于\(\xi = 0\)对称,
因为\(\varphi(x) = P(\xi \leq x)\),
则由正态分布曲线的对称性可得:\(\varphi( - x) + \varphi(x) = 1\),即\(\varphi( - x) = 1 - \varphi(x)\),故A正确;
又\(P\left( |\xi| \leq 1 \right) = P( - 1 \leq \xi \leq 1) = P(\xi \leq 1) - P(\xi \leq - 1) = \varphi(1) - \varphi( - 1)\),
由于\(\varphi( - 1) = 1 - \varphi(1)\),所以\(P\left( |\xi| \leq 1 \right) = 2\varphi(1) - 1\),故B不正确,C正确;
\(P\left( |\xi| > 1 \right) = 1 - P\left( |\xi| \leq 1 \right) = 1 - 2\varphi(1) + 1 = 2 - 2\varphi(1)\),故D正确}
\end{enumerate}
\end{question}
%
\begin{question}
设抛物线\(C:y^{2} = 2px(p > 0)\)的焦点为\(F\),准线为\(l\),过点\(F\)的直线交\(C\)于\(A\),\(B\)两点,以\(F\)为圆心,\(|FA|\)为半径的圆交\(l\)于\(M,N\)两点.若\(AM\bot l,|FA| = 6\),则(  )
\begin{choices}
  \item \(p = 2\)
  \item 直线\(AF\)的斜率是\(\pm \sqrt{3}\)
  \item \(|AB| = 8\)
  \item \(\bigtriangleup AMN\)的面积是\(18\sqrt{3}\)
\end{choices}
\topics{求直线与抛物线相交所得弦的弦长;抛物线中的三角形或四边形面积问题}
\difficulty{0.65}
\answer{BCD}
\explain{对于A,以\(F\)为圆心,\(|FA|\)为半径的圆交准线\(l\)于\(M,N\)两点,且\(AM\bot l,\)
%
故\(|FA| = |FM| = |AM|\),
%
所以\(\bigtriangleup AMF\)是等边三角形,所以\(\angle AMF = 60^{\circ}\),
%
设准线\(l\)与\(x\)轴交于点\(H\),则\(\angle HMF = 30^{\circ}\),
%
故\(p = |HF| = |MF|\sin 30^{\circ} = 3,\)故A错误;
%
对于B,因为\(\angle MAF = 60^{\circ}\),\(AM\)平行于\(x\)轴,
%
故\(\angle AFH = 120^{\circ}\),故当\(A\)点位于第一象限时,直线\(AF\)的倾斜角为\(60^{\circ}\);
%
当\(A\)点位于第三象限时,直线\(AF\)的倾斜角为\(120^{\circ}\);
%
所以直线\(AF\)的斜率是\(\pm \sqrt{3}\),故B正确;
%
对于C,因为直线\(AF\)的斜率是\(\pm \sqrt{3}\),且抛物线\(C:y^{2} = 6x\),
%
故直线\(AB\)的方程为:\(y = \pm \sqrt{3}\left( x - \frac{3}{2} \right)\),
%
联立方程得:\(3\left( x - \frac{3}{2} \right)^{2} = 6x\),即\(4x^{2} - 20x + 9 = 0,\)
%
设\(A\left( x_{1}\text{,}y_{1} \right)\text{,}B\left( x_{2}\text{,}y_{2} \right)\text{,}\)则\(x_{1}\text{+}x_{2} = 5\),
%
故\(|AB| = x_{1}\text{+}x_{2} + p = 8\),故C正确;
%
对于D,由A知, \(\angle HMF = 30^{\circ}\),
%
故\(|MH| = |MF|\cos 30^{\circ} = 3\sqrt{3},\)
%
故\(|MN| = 2|MH| = 6\sqrt{3},\)
%
\(S_{\bigtriangleup AMN} = \frac{1}{2} \times |MN| \times |AM| = 18\sqrt{3}\),故D正确.
%
% IMAGE_TODO_START id=zhejiang_lishui_2026_nov-Q10-img1 path=/Users/muryor/code/mynote/word\\_to\\_tex/output/figures/raw/media/image4.png width=60% inline=true question_index=10 sub_index=1
% CONTEXT_BEFORE: s |MN| \times |AM| = 18\),故D正确. 故选: BCD
% CONTEXT_AFTER: 
\begin{tikzpicture}[scale=0.8,baseline=-0.5ex]
  % TODO: AI_AGENT_REPLACE_ME (id=zhejiang_lishui_2026_nov-Q10-img1)
\end{tikzpicture}
% IMAGE_TODO_END id=zhejiang_lishui_2026_nov-Q10-img1{
}
\end{question}
%
\begin{question}
在\(\bigtriangleup ABC\)中,若\(C > B\),且\(\text{sin}^{2}B + \text{cos}^{2}C - \text{sin}B\text{cos}C = \frac{3}{4}\),则(  )
\begin{choices}
  \item \(C = \frac{\text{π}}{2}\)
  \item \(\text{sin}A = \frac{1}{2}\)
  \item \(\text{sin}A = \text{cos}C\)
  \item \(\text{2sin}B - \text{cos}C\)的最大值是\(\sqrt{3}\)
\end{choices}
\topics{二倍角的余弦公式;三角恒等变换的化简问题}
\difficulty{0.4}
\answer{BD}
\explain{由\(\text{sin}^{2}B + \text{cos}^{2}C - \text{sin}B\text{cos}C = \frac{3}{4}\),可得\(\frac{\text{1} - \cos 2B}{2} + \frac{\text{1} + \text{cos}2C}{2} - \text{sin}B\text{cos}C = \frac{3}{4}\),
%
所以\(\text{cos}2C - \cos 2B - 2\text{sin}B\text{cos}C = - \frac{1}{2}\),
%
所以\(\text{cos}\left\lbrack (C + B) + (C - B) \right\rbrack - \cos\left\lbrack (C + B) - (C - B) \right\rbrack - 2\text{sin}B\text{cos}C = - \frac{1}{2}\),
%
所以\(- 2\text{sin}(C + B)\sin(C - B) - 2\text{sin}B\text{cos}C = - \frac{1}{2}\),即\(- 2\sin A\sin(C - B) - 2\text{sin}B\text{cos}C = - \frac{1}{2}\),
%
所以\(- 2\sin A\sin(C - B) - \left\lbrack \sin(B + C) + \sin(B - C) \right\rbrack = - \frac{1}{2}\),
%
所以\(- 2\sin A\sin(C - B) - \sin A - \sin(B - C) = - \frac{1}{2}\),
%
所以\(2\sin A\sin(B - C) - \sin A - \sin(B - C) + \frac{1}{2} = 0\),
%
即\(4\sin A\sin(B - C) - 2\sin A - 2\sin(B - C) + 1 = 0\),
%
所以\(2\sin A\left\lbrack 2\sin(B - C) - 1 \right\rbrack - \left\lbrack 2\sin(B - C) - 1 \right\rbrack = 0\),
%
所以\(\left( 2\sin A - 1 \right)\left\lbrack 2\sin(B - C) - 1 \right\rbrack = 0\),
%
因为\(C > B\),所以\(- \text{π} < B - C < 0\),所以\(\sin(B - C) < 0\),
%
所以\(2\sin A - 1 = 0\),所以\(\text{sin}A = \frac{1}{2}\),故B正确;
%
又\(0 < A < \text{π}\),则\(A = \frac{\text{π}}{6}\)或\(A = \frac{\text{5π}}{6}\),
%
当\(A = \frac{\text{π}}{6}\)时,则\(B + C = \frac{\text{5π}}{6}\),不能得出\(C = \frac{\text{π}}{2}\),故A错误,
%
若\(C = \frac{\text{π}}{2}\),则\(B = \frac{\text{π}}{3}\)时,符合题意,但\(\text{cos}C = 0\),所以\(\text{sin}A \neq \text{cos}C\),故C错误;
%
由\(\text{sin}^{2}B + \text{cos}^{2}C - \text{sin}B\text{cos}C = \frac{3}{4}\),得\(\left( \text{sin}B - \frac{1}{2}\cos C \right)^{2} + \frac{3}{4}\cos^{2}C = \frac{3}{4}\),
%
所以\(\left( \text{sin}B - \frac{1}{2}\cos C \right)^{2} \leq \frac{3}{4}\),解得\(- \frac{\sqrt{3}}{2} \leq \text{sin}B - \frac{1}{2}\cos C \leq \frac{\sqrt{3}}{2}\),
%
所以\(\text{2sin}B - \text{cos}C \leq \sqrt{3}\),当且仅当\(\frac{3}{4}\cos^{2}C = 0\),即\(C = \frac{\text{π}}{2}\)时取等号,故D正确}
\end{question}
%
\section{填空题}
%
\begin{question}
\(\left( x^{3} - 1 \right)\left( x^{2} + \frac{1}{x} \right)^{6}\)展开式中的常数项是
.
\topics{两个二项式乘积展开式的系数问题}
\difficulty{0.85}
\answer{\(- 9\)}
\explain{由二项式\(\left( x^{2} + \frac{1}{x} \right)^{6}\)的展开式的通项公式为\(T_{r + 1} = C_{6}^{r}{(x^{2})}^{6 - r} \cdot {(\frac{1}{x})}^{r} = C_{6}^{r}x^{12 - 3r},r = 0,1,\cdots,6\),
%
所以\(\left( x^{3} - 1 \right)\left( x^{2} + \frac{1}{x} \right)^{6} = x^{3}\left( x^{2} + \frac{1}{x} \right)^{6} - \left( x^{2} + \frac{1}{x} \right)^{6} = x^{3}C_{6}^{r}x^{12 - 3r} - C_{6}^{r}x^{12 - 3r} = C_{6}^{r}x^{15 - 3r} - C_{6}^{r}x^{12 - 3r},r = 0,1,\cdots,6\),
%
所以当\(r = 5\)时有常数项\(C_{6}^{5}x^{15 - 3 \times 5} = C_{6}^{5}\),当\(r = 4\)时有常数项\(- C_{6}^{4}x^{12 - 3 \times 4} = - C_{6}^{4}\),
%
所以所求展开式的常数项为\(C_{6}^{5} + ( - 1) \cdot C_{6}^{4} = 6 - 15 = - 9\).\(- 9\).}
\end{question}
%
\begin{question}
已知平面向量\(\overrightarrow{a},\overrightarrow{b}\)满足\(\left| \overrightarrow{a} \right| = \sqrt{3},\left| \overrightarrow{a} - \overrightarrow{b} \right| = 1\),则\(\left| \overrightarrow{b} \right|\)的最大值是
.
\topics{已知数量积求模}
\difficulty{0.4}
\answer{\(\sqrt{3} + 1\)}
\explain{根据向量模长的三角不等式,有
\(\left| \left| \overrightarrow{a} \right| - \left| \overrightarrow{b} \right| \right| \leq \left| \overrightarrow{a} - \overrightarrow{b} \right| \leq \left| \overrightarrow{a} \right| + \left| \overrightarrow{b} \right|\),
又因为\(\overrightarrow{b} = \overrightarrow{a} - (\overrightarrow{a} - \overrightarrow{b})\),由三角不等式:
\(\left| \overrightarrow{b} \right| = \left| \overrightarrow{a} - (\overrightarrow{a} - \overrightarrow{b}) \right| \leq \left| \overrightarrow{a} \right| + \left| \overrightarrow{a} - \overrightarrow{b} \right|\),
则\(\left| \overrightarrow{b} \right| \leq \left| \overrightarrow{a} \right| + \left| \overrightarrow{a} - \overrightarrow{b} \right| = \sqrt{3} + 1\),得:
\(\left| \overrightarrow{b} \right| \leq \sqrt{3} + 1\)\(\sqrt{3} + 1\)}
\end{question}
%
\begin{question}
在Rt\(\bigtriangleup ABC\)中,\(C = \frac{\text{π}}{2},AC = 1,BC = \sqrt{3},D\)是\(AB\)的中点,把\(\bigtriangleup BCD\)沿\(CD\)翻折到\(\bigtriangleup B_{1}CD\),设二面角\(B_{1} - CD - A\)的平面角为\(\theta\),若\(\theta \in \left\lbrack \frac{\text{π}}{3},\frac{\text{π}}{2} \right\rbrack\),则三棱锥\(B_{1} - ACD\)外接球表面积的范围是
.
\topics{由导数求函数的最值(不含参);球的表面积的有关计算;多面体与球体内切外接问题;求二面角}
\difficulty{0.4}
\answer{\(\left\lbrack 4\text{π},\frac{13\text{π}}{3} \right\rbrack\)}
\explain{由题可得\(AB = \sqrt{\left( \sqrt{3} \right)^{2} + 1^{2}} = 2\),所以\(AC = CD = AD = BD = B_{1}D = 1,CB_{1} = \sqrt{3}\),
%
所以\(\angle ABD = \angle BCD = 30{^\circ}\),故\(\bigtriangleup ACD\)和\(\bigtriangleup CDB_{1}\)分别为等边三角形和等腰三角形,且\(\angle CDB_{1} = 120{^\circ}\),
%
如图,\(O_{1},O_{2}\)分别为\(\bigtriangleup ACD, \bigtriangleup CDB_{1}\)外接圆圆心,取\(CD\)中点\(G\),连接\(O_{2}G,AG\),
%
则\(O_{1}A = \frac{1}{2\sin 60{^\circ}} = \frac{\sqrt{3}}{3},O_{1}G = \frac{1}{3}AG = \frac{1}{3} \times \sqrt{1^{2} - \left( \frac{1}{2} \right)^{2}} = \frac{\sqrt{3}}{6}\),\(O_{2}C = O_{2}B_{1} = \frac{\sqrt{3}}{2\sin 120{^\circ}} = 1\),\(O_{2}G = \sqrt{O_{2}C^{2} - GC^{2}} = \frac{\sqrt{3}}{2}\),
%
且\(O_{2}G\bot CD,AG\bot CD\),故\(\angle AGO_{2}\)为二面角\(B_{1} - CD - A\)的平面角,所以\(\angle AGO_{2} = \theta\),
%
分别过\(O_{1},O_{2}\)作平面\(ACD\)和平面\(CDB_{1}\)的垂线,则球心均在两垂线上,两垂线的交点即为球心\emph{O},
%
如图,当\(\theta = \frac{\text{π}}{2}\)时,四边形\(GO_{2}OO_{1}\)为矩形,则\(O_{2}O = GO_{1} = \frac{\sqrt{3}}{6}\),
%
所以由\(B_{1}O^{2} = B_{1}O_{2}^{2} + OO_{2}^{2}\)得\(R^{2} = 1^{2} + \left( \frac{\sqrt{3}}{6} \right)^{2} = \frac{13}{12}\);
%
若\(\frac{\text{π}}{3} \leq \theta < \frac{\text{π}}{2}\),如图,连接\(O_{1}O_{2}\),则\(O_{1}O_{2}\)与\(O_{2}G\)相交于平面\(CDB_{1}\)一点\emph{H},
%

% IMAGE_TODO_START id=zhejiang_lishui_2026_nov-Q14-img1 path=/Users/muryor/code/mynote/word\\_to\\_tex/output/figures/raw/media/image5.png width=60% inline=true question_index=14 sub_index=1
% CONTEXT_BEFORE: 则\(O_{1}O_{2}\)与\(O_{2}G\)相交于平面\(CDB_{1}\)一点*H*,
\begin{tikzpicture}[scale=0.8,baseline=-0.5ex]
  % TODO: AI_AGENT_REPLACE_ME (id=zhejiang_lishui_2026_nov-Q14-img1)
\end{tikzpicture}
% IMAGE_TODO_END id=zhejiang_lishui_2026_nov-Q14-img1{
%
则\(\angle GHO_{1} = \angle O_{2}HO,\angle GO_{1}H = \angle OO_{2}H = 90{^\circ},\)所以\(\angle O_{2}OH = \angle AGO_{2} = \theta\),
%
设三棱锥\(B_{1} - ACD\)外接球半径为\emph{R},
%
则\(OO_{1} = \sqrt{OA^{2} - O_{1}A^{2}} = \sqrt{R^{2} - \frac{1}{3}}\),\(OO_{2} = \sqrt{OB_{1}^{2} - O_{2}B_{1}^{2}} = \sqrt{R^{2} - 1}\),\(HG = \frac{O_{1}G}{\cos\theta} = \frac{\sqrt{3}}{6\cos\theta},O_{2}H = OO_{2}\tan\theta = \sqrt{R^{2} - 1}\tan\theta\),
%
所以\(\frac{\sqrt{3}}{2} = \sqrt{R^{2} - 1}\tan\theta + \frac{\sqrt{3}}{6\cos\theta} \Rightarrow \sqrt{R^{2} - 1} = \frac{\frac{\sqrt{3}}{2} - \frac{\sqrt{3}}{6\cos\theta}}{\tan\theta} = \left( \frac{\sqrt{3}}{2} - \frac{\sqrt{3}}{6\cos\theta} \right)\frac{\cos\theta}{\sin\theta} = \frac{3\sqrt{3}\cos\theta - \sqrt{3}}{6\sin\theta}\),
%
所以\(R^{2} = \left( \frac{3\sqrt{3}\cos\theta - \sqrt{3}}{6\sin\theta} \right)^{2} + 1 = \frac{\left( 3\cos\theta - 1 \right)^{2}}{12\sin^{2}\theta} + 1 = \frac{13 - 6\cos\theta - 3\cos^{2}\theta}{12 - 12\cos^{2}\theta}\),
%
若\(\frac{\text{π}}{3} \leq \theta < \frac{\text{π}}{2}\),则\(0 < t = \cos\theta \leq \frac{1}{2}\),令\(f(t) = \frac{13 - 6t - 3t^{2}}{12 - 12t^{2}},0 < t \leq \frac{1}{2}\),
%
则\(f'(t) = \frac{( - 6 - 6t) \times \left( 12 - 12t^{2} \right) - \left( 13 - 6t - 3t^{2} \right) \times ( - 24t)}{\left( 12 - 12t^{2} \right)^{2}} = \frac{- 3t^{2} + 10t - 3}{12\left( t^{2} - 1 \right)^{2}} = - \frac{(3t - 1)(t - 3)}{12\left( t^{2} - 1 \right)^{2}}\),
%
所以\(t \in \left( 0,\frac{1}{3} \right)\)时\(f'(t) < 0\),\(t \in \left\lbrack \frac{1}{3},\frac{1}{2} \right\rbrack\)时\(f'(t) > 0\),所以\(f(t)\)在\(\left( 0,\frac{1}{3} \right)\)上单调递减,在\(\left\lbrack \frac{1}{3},\frac{1}{2} \right\rbrack\)上单调递增,
%
又\(f(0) = \frac{13}{12},f\left( \frac{1}{2} \right) = \frac{37}{36} < \frac{13}{12},f\left( \frac{1}{3} \right) = \frac{13 - 6t - 3t^{2}}{12 - 12t^{2}} = \frac{37}{36} = 1\),
%
所以\(f(t)_{\text{max}} \rightarrow f(0) = \frac{13}{12},f(t)_{\text{min}} = f\left( \frac{1}{3} \right) = 1\),
%
综上所述,\(R^{2}\)最小值为1,最大值为\(\frac{13}{12}\).
%

% IMAGE_TODO_START id=zhejiang_lishui_2026_nov-Q14-img2 path=/Users/muryor/code/mynote/word\\_to\\_tex/output/figures/raw/media/image6.png width=60% inline=true question_index=14 sub_index=1
% CONTEXT_BEFORE: 1\(, 综上所述,\)R^{2}\(最小值为1,最大值为\){12}\(.
\begin{tikzpicture}[scale=0.8,baseline=-0.5ex]
  % TODO: AI_AGENT_REPLACE_ME (id=zhejiang_lishui_2026_nov-Q14-img2)
\end{tikzpicture}
% IMAGE_TODO_END id=zhejiang_lishui_2026_nov-Q14-img2{
%
所以三棱锥\(B_{1} - ACD\)外接球表面积最小值为\(4\text{π}R^{2} = 4\text{π}\),最大值为\(4\text{π}R^{2} = 4\text{π} \times \frac{13}{12} = \frac{13\text{π}}{3}\).\(\left\lbrack 4\text{π},\frac{\text{13π}}{3} \right\rbrack\)}
\end{question}
%
\section{解答题}
%
\begin{question}
证明:数列\(\left\{ \frac{1}{a_{n}} \right\}\)为等差数列;

\begin{enumerate}[label=(\arabic*)]
\item 求数列\(\left\{ a_{n}a_{n + 1} \right\}\)的前\(n\)项和\(S_{n}\).
\topics{由递推关系证明数列是等差数列;裂项相消法求和}
\difficulty{0.65}
\answer{(1)证明见解析
(2)\(S_{n} = \frac{2n}{n + 2}\)}
\explain{(1)由\(\frac{a_{n + 1} - a_{n}}{a_{n}} = \frac{a_{n + 2} - a_{n + 1}}{a_{n + 2}}\),得\(\frac{a_{n + 1}}{a_{n}} - 1 = 1 - \frac{a_{n + 1}}{a_{n + 2}}\),
%
即\(\frac{a_{n + 1}}{a_{n}} + \frac{a_{n + 1}}{a_{n + 2}} = 2\),得\(\frac{1}{a_{n}} + \frac{1}{a_{n + 2}} = \frac{2}{a_{n + 1}}\),
%
所以数列\(\left\{ \frac{1}{a_{n}} \right\}\)为等差数列.
%
(2)设数列\(\left\{ \frac{1}{a_{n}} \right\}\)的公差为\(d\),则\(d = \frac{1}{a_{2}} - \frac{1}{a_{1}} = \frac{1}{2}\),
%
得\(\frac{1}{a_{n}} = \frac{1}{a_{1}} + (n - 1)d = 1 + (n - 1) \times \frac{1}{2} = \frac{n + 1}{2}\),故\(a_{n} = \frac{2}{n + 1}\),
%
故\(a_{n}a_{n + 1} = \frac{4}{(n + 1)(n + 2)} = 4\left( \frac{1}{n + 1} - \frac{1}{n + 2} \right)\),
%
则\(S_{n} = a_{1}a_{2} + a_{2}a_{3} + \cdots + a_{n}a_{n + 1}\)
%
\(= 4 \times \left( \frac{1}{2} - \frac{1}{3} + \frac{1}{3} - \frac{1}{4} + \cdots + \frac{1}{n + 1} - \frac{1}{n + 2} \right)\)
%
\(= 4 \times \left( \frac{1}{2} - \frac{1}{n + 2} \right) = \frac{2n}{n + 2}\),
%
故\(S_{n} = \frac{2n}{n + 2}\).}
\end{enumerate}
\end{question}
%
\begin{question}
求证:\(A_{1}B\bot AC_{1}\);

\begin{enumerate}[label=(\arabic*)]
\item 求平面\(ABC_{1}\)与平面\(BCC_{1}\)夹角的余弦值.
\topics{线面垂直证明线线垂直;面面角的向量求法}
\difficulty{0.4}
\answer{(1)证明见解析
(2)\(\frac{\sqrt{105}}{35}\)}
\explain{(1)取\(AC\)的中点\(O\),连接\(BO,A_{1}O,C_{1}O\),
%
因为\(BA = BC\),\(O\)为\(AC\)中点,所以\(BO\bot AC\),
%
又因为平面\(AA_{1}C_{1}C\bot\)平面\(ABC\),平面\(AA_{1}C_{1}C \cap\)平面\(ABC = AC\),\(BO \subset\)平面\(ABC\),
%
所以\(BO\bot\)平面\(AA_{1}C_{1}C\),而\(AC_{1} \subset\)平面\(AA_{1}C_{1}C\),则\(AC_{1}\bot BO\).
%
因为\(A_{1}C_{1}//AO,A_{1}C_{1} = AO = AA_{1}\),所以四边形\(A_{1}AOC_{1}\)是菱形,\(AC_{1}\bot A_{1}O\),
%
而\(A_{1}O \cap BO = O,A_{1}O,BO \subset\)平面\(A_{1}OB\),因此\(AC_{1}\bot\)平面\(A_{1}OB\),
%
因为\(A_{1}B \subset\)平面\(A_{1}OB\),所以\(A_{1}B\bot AC_{1}\).
%
(2)取\(A_{1}C_{1}\)中点\(M\),则\(OM\bot AC\),
%
由平面\(AA_{1}C_{1}C\bot\)平面\(ABC\),平面\(AA_{1}C_{1}C \cap\)平面\(ABC = AC\),\(OM \subset\)平面\(AA_{1}C_{1}C\),
%
则\(OM\bot\)平面\(ABC\),又\(OB \subset\)平面\(ABC\),所以\(OM\bot OB\),
%
则\(OB,OC,OM\)两两垂直,依题可建立如图所示空间直角坐标系\(O - xyz\).
%
在平面\(AA_{1}C_{1}C\)内作\(A_{1}H\bot AC\)于\(H\),连接\(BH\).
%

% IMAGE_TODO_START id=zhejiang_lishui_2026_nov-Q16-img2 path=/Users/muryor/code/mynote/word\\_to\\_tex/output/figures/raw/media/image8.png width=60% inline=true question_index=16 sub_index=1
% CONTEXT_BEFORE: A_{1}C_{1}C\)内作\(A_{1}H\bot AC\)于\(H\),连接\(BH\).
\begin{tikzpicture}[scale=0.8,baseline=-0.5ex]
  % TODO: AI_AGENT_REPLACE_ME (id=zhejiang_lishui_2026_nov-Q16-img2)
\end{tikzpicture}
% IMAGE_TODO_END id=zhejiang_lishui_2026_nov-Q16-img2{
%
因为平面\(AA_{1}C_{1}C\bot\)平面\(ABC\),所以\(A_{1}H\bot\)平面\(ABC\).
%
在梯形\(A_{1}C_{1}CA\)中,由题意\(AH = \frac{1}{4}AC = 1\),\(A_{1}H = \sqrt{3}\).
%
在\(Rt\Delta A_{1}HB\)中,\(BH = \sqrt{A_{1}B^{2} - A_{1}H^{2}} = \sqrt{5}\).
%
在\(Rt\Delta OHB\)中,\(OB = \sqrt{BH^{2} - OH^{2}} = 2\),
%
\(A(0, - 2,0)\),\(C(0,2,0)\),\(B(2,0,0)\),\(C_{1}(0,1,\sqrt{3})\),
%
\(\overrightarrow{BC_{1}} = ( - 2,1,\sqrt{3})\),\(\overrightarrow{AB} = (2,2,0)\),\(\overrightarrow{BC} = ( - 2,2,0)\),
%
设平面\(ABC_{1}\)的法向量\(\overrightarrow{n_{1}} = (x_{1},y_{1},z_{1})\),
%
则\(\left\{ \begin{array}{r}
\overrightarrow{AB} \cdot \overrightarrow{n_{1}} = 2x_{1} + 2y_{1} = 0 \\
\overrightarrow{BC_{1}} \cdot \overrightarrow{n_{1}} = - 2x_{1} + y_{1} + \sqrt{3}z_{1} = 0
\end{array} \right.,
%
取\)x_{1} = 1\(,得\)\overrightarrow{n_{1}} = (1, - 1,\sqrt{3})\(.
%
设平面\)BCC_{1}\(的法向量\)\overrightarrow{n_{2}} = (x_{2},y_{2},z_{2})\(,
%
则\)\left\{ \begin{array}{r}
\overrightarrow{BC} \cdot \overrightarrow{n_{2}} = - 2x_{2} + 2y_{2} = 0 \\
\overrightarrow{BC_{1}} \cdot \overrightarrow{n_{2}} = - 2x_{2} + y_{2} + \sqrt{3}z_{2} = 0
\end{array} \right.,
%
取\(x_{2} = 1\),得\(\overrightarrow{n_{2}} = (\sqrt{3},\sqrt{3},1)\),
%
所以\(\cos\left\langle \overrightarrow{n_{1}},\overrightarrow{n_{2}} \right\rangle = \frac{\overrightarrow{n_{1}} \cdot \overrightarrow{n_{2}}}{\left| \overrightarrow{n_{1}} \right| \cdot \left| \overrightarrow{n_{2}} \right|} = \frac{\sqrt{3}}{\sqrt{5} \cdot \sqrt{7}} = \frac{\sqrt{105}}{35}\),
%
因此平面\(ABC_{1}\)和\(BCC_{1}\)夹角的余弦值是\(\frac{\sqrt{105}}{35}\).}
\end{enumerate}
\end{question}
%
\begin{question}
按照上述规则摸球3次.当第1次选中的是甲袋,求摸到红球的个数\(X\)的分布列及期望\(E(X)\);

\begin{enumerate}[label=(\arabic*)]
\item 按照上述规则进行连续摸球,若摸到2次红球则停止摸球.求3次之内(含3次)停止摸球的概率.
\topics{利用二项分布求分布列;二项分布的均值;利用全概率公式求概率}
\difficulty{0.65}
\answer{(1)分布列见解析,\(\frac{3}{4}\)
(2)\(\frac{1}{4}\)}
\explain{(1)法一:由题意得\(X\)的可能取值为\(0,1,2,3\).
%
\(P(X = 0) = \text{C}_{3}^{0}{(\frac{3}{4})}^{3} = \frac{27}{64}\),\(P(X = 1) = \text{C}_{3}^{1}(\frac{1}{4}) \times {(\frac{3}{4})}^{2} = \frac{27}{64}\),
%
\(P(X = 2) = \text{C}_{3}^{2}{(\frac{1}{4})}^{2} \times (\frac{3}{4}) = \frac{9}{64}\),\(P(X = 3) = \text{C}_{3}^{3}{(\frac{1}{4})}^{3} = \frac{1}{64}\).
%
  ------------- ------------------- ------------------- ------------------ ------------------
      \(X\)              0                   1                  2                  3
%
      \(P\)      \(\frac{27}{64}\)   \(\frac{27}{64}\)   \(\frac{9}{64}\)   \(\frac{1}{64}\)
  ------------- ------------------- ------------------- ------------------ ------------------
%
因此\(E(X) = 0 \times \frac{27}{64} + 1 \times \frac{27}{64} + 2 \times \frac{9}{64} + 3 \times \frac{1}{64} = \frac{3}{4}\).
%
法二:由题意得\(X\)的可能取值为\(0,1,2,3\).
%
又\(X\sim B(3,\frac{1}{4})\),故\(P(X) = \text{C}_{3}^{k}{(\frac{1}{4})}^{k} \cdot {(\frac{3}{4})}^{3 - k}\)(\(k = 0,1,2,3\)).
%
因此\(E(X) = 3 \times \frac{1}{4} = \frac{3}{4}\).
%
(2)设事件\(M =\)"\(3\)次之内(含\(3\)次)停止摸球",
%
事件\(A =\)"第\(1\)次摸到红球,第\(2\)次摸到红球";
%
事件\(B =\)"第\(1\)次摸到红球,第\(2\)次摸到白球,第\(3\)次摸到红球";
%
事件\(C =\)"第\(1\)次摸到白球,第\(2\)次摸到红球,第\(3\)次摸到红球";
%
事件\(D_{i} =\)"首次选择甲袋是第\(i\)次摸球"(\(i = 1,2,3\)),
%
事件\(D_{0} =\)"一直没有选择甲袋".
%
则\(P(A) = P(D_{1})P(A\left| D_{1} \right.\ ) + P(D_{2})P(A\left| D_{2} \right.\ ) + P(D_{0})P(A\left| D_{0} \right.\ )\)
%
\(= \frac{1}{2} \times \frac{1}{4} \times \frac{1}{4} + \frac{1}{2} \times \frac{1}{2} \times \frac{1}{2} \times \frac{1}{4} + \frac{1}{2} \times \frac{1}{2} \times \frac{1}{2} \times \frac{1}{2} = \frac{1}{8}\).
%
\(P(B) = P(D_{1})P(B\left| D_{1} \right.\ ) + P(D_{2})P(B\left| D_{2} \right.\ ) + P(D_{3})P(B\left| D_{3} \right.\ ) + P(D_{0})P(B\left| D_{0} \right.\ )\)
%
\(= \frac{1}{2} \times \frac{1}{4} \times \frac{3}{4} \times \frac{1}{4} + \frac{1}{4} \times \frac{1}{2} \times \frac{3}{4} \times \frac{1}{4} + \frac{1}{8} \times \frac{1}{2} \times \frac{1}{2} \times \frac{1}{4} + \frac{1}{8} \times \frac{1}{2} \times \frac{1}{2} \times \frac{1}{2} = \frac{9}{128}\).
%
\(P(C) = P(D_{1})P(C\left| D_{1} \right.\ ) + P(D_{2})P(C\left| D_{2} \right.\ ) + P(D_{3})P(C\left| D_{3} \right.\ ) + P(D_{0})P(C\left| D_{0} \right.\ )\)
%
\(= \frac{1}{2} \times \frac{3}{4} \times \frac{1}{4} \times \frac{1}{4} + \frac{1}{2} \times \frac{1}{2} \times \frac{1}{2} \times \frac{1}{4} \times \frac{1}{4} + \frac{1}{2} \times \frac{1}{2} \times \frac{1}{2} \times \frac{1}{2} \times \frac{1}{2} \times \frac{1}{4} + \frac{1}{8} \times \frac{1}{2} \times \frac{1}{2} \times \frac{1}{2} = \frac{7}{128}\).
%
因此\(P(M) = P(A) + P(B) + P(C) = \frac{1}{8} + \frac{9}{128} + \frac{7}{128} = \frac{1}{4}\).}
\end{enumerate}
\end{question}
%
\begin{question}
\item 求椭圆\(C\)的标准方程;
\item 已知\(P,Q\)是椭圆\(C\)上的两动点,且\(P,Q\)的横坐标之和为\(\frac{50}{9}\),设直线\(l\)为线段\(PQ\)的中垂线,过点\(A\)作直线\(AM\bot l\),垂足为\(M\).求垂足\(M\)横坐标\(x_{M}\)的取值范围,并求\(M\)的轨迹方程.
\topics{轨迹问题------圆;求平面轨迹方程;根据椭圆过的点求标准方程}
\difficulty{0.4}
\answer{(1)\(\frac{x^{2}}{\text{25}}\text{+}\frac{y^{2}}{16} = 1\);
(2)\(x_{M} \in \lbrack - 5, - \frac{1}{3})\),\({(x + 2)}^{2} + y^{2} = 9\)且\(x \in \lbrack - 5, - \frac{1}{3})\).}
\explain{(1)由题意\(\left\{ \begin{array}{r}
a = \text{5} \\
\frac{16}{a^{2}}\text{+}\frac{\frac{1\text{44}}{\text{25}}}{b^{2}} = 1
\end{array} \right.解得\)\left\{ \begin{array}{r}
a = \text{5} \\
b = 4
\end{array} \right.,所以椭圆\(C\)的标准方程为\(\frac{x^{2}}{\text{25}}\text{+}\frac{y^{2}}{16} = 1\);
%
(2)设\(P(x_{1},y_{1}),Q(x_{2},y_{2})\),线段\(PQ\)的中点\(G(x_{0},y_{0})\),则\(x_{0} = \frac{\text{25}}{\text{9}}\),\(\frac{x_{1}^{2}}{\text{25}}\text{+}\frac{y_{1}^{2}}{16} = 1,\frac{x_{2}^{2}}{\text{25}}\text{+}\frac{y_{2}^{2}}{16} = 1\),
%

% IMAGE_TODO_START id=zhejiang_lishui_2026_nov-Q18-img1 path=/Users/muryor/code/mynote/word\\_to\\_tex/output/figures/raw/media/image9.png width=60% inline=true question_index=18 sub_index=1
% CONTEXT_AFTER:  ①当x_{1}
\begin{tikzpicture}[scale=0.8,baseline=-0.5ex]
  % TODO: AI_AGENT_REPLACE_ME (id=zhejiang_lishui_2026_nov-Q18-img1)
\end{tikzpicture}
% IMAGE_TODO_END id=zhejiang_lishui_2026_nov-Q18-img1{
%
①当\(x_{1} = x_{2},y_{1} = - y_{2}\)时,\(PQ\)的中垂线为\(x\)轴,过点\(A\)向中垂线作垂线,垂足\(M\)为点\(A( - 5,0)\)
%
②当\(x_{1} \neq x_{2}\)时,直线\(PQ\)的斜率\(k = \frac{y_{1} - y_{2}}{x_{1} - x_{2}} = - \frac{16}{25} \cdot \frac{x_{1} + x_{2}}{y_{1} + y_{2}} = - \frac{16}{25} \cdot \frac{x_{0}}{y_{0}}\),则\(k \cdot \frac{y_{0}}{x_{0}} = - \frac{16}{\text{25}}\),
%
所以\(k \cdot y_{0} = - \frac{16}{9}\),将\(x = \frac{25}{9}\)代入椭圆方程得\(y = \pm \frac{8\sqrt{14}}{9}\),
%
所以\(- \frac{8\sqrt{14}}{9} < y_{0} < \frac{8\sqrt{14}}{9}\),从而\(k < - \frac{\sqrt{14}}{7}\)或\(k > \frac{\sqrt{14}}{7}\),
%
线段\(PQ\)的中垂线方程为\(y + \frac{16}{9k} = - \frac{1}{k}(x - \frac{25}{9})\),即\(y = - \frac{x}{k} + \frac{1}{k} = - \frac{1}{k}(x - 1)\).
%
故线段\(PQ\)的中垂线过定点\((1,0)\)
%
故垂足\(M\)轨迹是在以\(( - 2,0)\)为圆心,半径为\(3\)的圆弧,其方程为\({(x + 2)}^{2} + y^{2} = 9\)
%
过点\(A\)与\(y = - \frac{1}{k}(x - 1)\)垂直的直线为\(y = k(x + 5)\),
%
联立方程组\(\left\{ \begin{array}{r}
y = - \frac{1}{k}(x - 1) \\
y = k(x + 5)
\end{array} \right.消去\)y\(得\)x_{M} = \frac{1 - 5k^{2}}{k^{2} + 1}\(,因为\)k^{2} > \frac{2}{7}\(,
%
所以\)x_{M} = - 5 + \frac{6}{k^{2} + 1} \in ( - 5, - \frac{1}{3})\(,综上①,②所得\)x_{M} \in \lbrack - 5, - \frac{1}{3})\(所以垂足\)M\(轨迹方程是\){(x + 2)}^{2} + y^{2} = 9\(,且\)x \in \lbrack - 5, - \frac{1}{3})\(.\)}
\end{question}
%
\begin{question}
\item 当\(a = 1\)时,
%
(i)求\(f(x)\)的单调递增区间;
%
(ii)记\(x_{n}\)为函数\(f(x)\)在\((0, + \infty)\)上从小到大排列的第\(n\)个极值点\(\left( n \in N^{\text{*}} \right)\),求数列\(\left\{ \frac{\sqrt{2}\text{e}^{\frac{3}{4}\text{π}}}{\text{π}}x_{n} \cdot \left| f\left( x_{n} \right) \right| \right\}\)的前20项和.
\item 当\(a \in (0,2)\)时,求证:对任意的\(x \in \left\lbrack 0,\frac{\text{π}}{4} \right\rbrack\),\(a\text{e}^{x} \cdot g'(x) \geq f(x) - f'(x)\)恒成立.
\topics{利用导数求函数的单调区间(不含参);利用导数证明不等式;利用定义求等差数列通项公式;错位相减法求和}
\difficulty{0.15}
\answer{(1)(i)\(\left\lbrack 2k\text{π} - \frac{3\text{π}}{4},2k\text{π} + \frac{\text{π}}{4} \right\rbrack,k \in Z\);(ii)\(\frac{77\text{e}^{22\text{π}} - 81\text{e}^{21\text{π}} + 3\text{e}^{2\text{π}} + \text{e}^{\text{π}}}{4\left( 1 - \text{e}^{\text{π}} \right)^{2}}\)
(2)证明见解析}
\explain{(1)由题可得:\(f'(x) = \text{e}^{x}\left( \cos x - \sin x \right) = \sqrt{2}\text{e}^{x}\cos\left( x + \frac{\text{π}}{4} \right)\),
%
(i)令\(f'(x) \geq 0\),则\(\cos\left( x + \frac{\text{π}}{4} \right) \geq 0\),
%
可得\(2k\text{π} - \frac{\text{π}}{2} \leq x + \frac{\text{π}}{4} \leq 2k\text{π} + \frac{\text{π}}{2},k \in Z\),解得\(2k\text{π} - \frac{3\text{π}}{4} \leq x \leq 2k\text{π} + \frac{\text{π}}{4},k \in Z\),
%
所以\(f(x)\)的单调递增区间为\(\left\lbrack 2k\text{π} - \frac{3\text{π}}{4},2k\text{π} + \frac{\text{π}}{4} \right\rbrack,k \in Z\);
%
(ii)令\(f'(x) \leq 0\),则\(\cos\left( x + \frac{\text{π}}{4} \right) \leq 0\),
%
可得\(2k\text{π} + \frac{\text{π}}{2} \leq x + \frac{\text{π}}{4} \leq 2k\text{π} + \frac{\text{3π}}{2},k \in Z\),解得\(2k\text{π} + \frac{\text{π}}{4} \leq x \leq 2k\text{π} + \frac{\text{5π}}{4},k \in Z\),
%
则函数\(f(x)\)的单调递增区间为\(\left\lbrack 2k\text{π} - \frac{3\text{π}}{4},2k\text{π} + \frac{\text{π}}{4} \right\rbrack\),单调递减区间为\(\left\lbrack 2k\text{π} + \frac{\text{π}}{4},2k\text{π} + \frac{\text{5π}}{4} \right\rbrack,k \in Z\),
%
可知函数\(f(x)\)的极值点为\(x = k\text{π} + \frac{\text{π}}{4},k \in Z\),
%
且\(x > 0\),则函数\(f(x)\)的极值点依次为\(\frac{\text{π}}{4}\),\(\frac{\text{5π}}{4}\),\(\frac{\text{9π}}{4}\),\(\cdot \cdot \cdot\),
%
可知数列\(\left\{ x_{n} \right\}\)是以首项为\(\frac{\text{π}}{4}\),公差为\(\text{π}\)的等差数列,
%
则\(x_{n} = \frac{\text{π}}{4} + (n - 1)\text{π} = \left( n - \frac{3}{4} \right)\text{π}\),\(\left| f\left( x_{n} \right) \right| = \left| \text{e}^{\left( n - \frac{3}{4} \right)\text{π}}\cos\left( n - \frac{3}{4} \right)\text{π} \right| = \frac{\sqrt{2}}{2}\text{e}^{\left( n - \frac{3}{4} \right)\text{π}}\),
%
令\(a_{n} = \frac{\sqrt{2}\text{e}^{\frac{3}{4}\text{π}}}{\text{π}}x_{n} \cdot \left| f\left( x_{n} \right) \right| = \frac{\sqrt{2}\text{e}^{\frac{3}{4}\text{π}}}{\text{π}}\left( n - \frac{3}{4} \right)\text{π}\frac{\sqrt{2}}{2}\text{e}^{\left( n - \frac{3}{4} \right)\text{π}} = \left( n - \frac{3}{4} \right)\text{e}^{n\text{π}}\),
%
记数列\(\left\{ a_{n} \right\}\)的前20项和为\(S_{20}\),
%
则\(S_{20} = \frac{1}{4}\text{e}^{\text{π}} + \frac{5}{4}\text{e}^{2\text{π}} + \frac{9}{4}\text{e}^{3\text{π}} + \cdot \cdot \cdot + \frac{77}{4}\text{e}^{20\text{π}}\),
%
可得\(\text{e}^{\text{π}}S_{20} = \frac{1}{4}\text{e}^{2\text{π}} + \frac{5}{4}\text{e}^{3\text{π}} + \frac{9}{4}\text{e}^{4\text{π}} + \cdot \cdot \cdot + \frac{77}{4}\text{e}^{21\text{π}}\),
%
两式相减得\(\left( 1 - \text{e}^{\text{π}} \right)S_{20} = \frac{1}{4}\text{e}^{\text{π}} + \text{e}^{2\text{π}} + \text{e}^{3\text{π}} + \text{e}^{4\text{π}} + \cdot \cdot \cdot + \text{e}^{20\text{π}} - \frac{77}{4}\text{e}^{21\text{π}}\)
%
\(= \frac{1}{4}\text{e}^{\text{π}} + \frac{\text{e}^{2\text{π}} - \text{e}^{21\text{π}}}{1 - \text{e}^{\text{π}}} - \frac{77}{4}\text{e}^{21\text{π}}\)
%
整理可得\(S_{20} = \frac{77\text{e}^{22\text{π}} - 81\text{e}^{21\text{π}} + 3\text{e}^{2\text{π}} + \text{e}^{\text{π}}}{4\left( 1 - \text{e}^{\text{π}} \right)^{2}}\).
%
(2)由题可得:\(a\text{\,\,}\text{e}^{x} \cdot g'(x) - f(x) + f'(x) = a\text{e}^{x}\left( \sin x + x\cos x \right) - \text{e}^{x}\cos(ax) + \left\lbrack \text{e}^{x}\cos(ax) - a\text{e}^{x}\sin(ax) \right\rbrack\)
%
\(= a\text{e}^{x}\left\lbrack \sin x + x\cos x - \sin(ax) \right\rbrack\),
%
因为\(a \in (0,2)\),\(x \in \left\lbrack 0,\frac{\text{π}}{4} \right\rbrack\),则\(0 \leq ax \leq \frac{\text{π}}{4}a < \frac{\text{π}}{2}\),\(0 \leq 2x \leq \frac{\text{π}}{2}\),\(ax \leq 2x\),
%
可得\(\sin(ax) \leq \sin(2x)\),
%
构造\(h(x) = x - \sin x,x \in \left\lbrack 0,\frac{\text{π}}{4} \right\rbrack\),则\(h'(x) = 1 - \cos x \geq 0\),
%
可知\(h(x)\)在\(\left\lbrack 0,\frac{\text{π}}{4} \right\rbrack\)内单调递增,则\(h(x) \geq h(0) = 0\),
%
即\(x - \sin x \geq 0,x \in \left\lbrack 0,\frac{\text{π}}{4} \right\rbrack\),
%
且\(\cos x > 0\),\(\sin x \geq 0\),\(1 - \cos x \geq 0\),
%
可得\(\sin x\left( 1 - \cos x \right) \geq 0\),\(\left( x - \sin x \right)\cos x \geq 0\),即\(\sin x\left( 1 - \cos x \right) + \left( x - \sin x \right)\cos x \geq 0\),
%
则\(a\text{e}^{x}\left\lbrack \sin x + x\cos x - \sin(ax) \right\rbrack \geq a\text{e}^{x}\left\lbrack \sin x + x\cos x - \sin(2x) \right\rbrack\)
%
\(= a\text{e}^{x}\left\lbrack \sin x\left( 1 - \cos x \right) + \left( x - \sin x \right)\cos x \right\rbrack \geq 0\),
%
即\(a\text{\,\,}\text{e}^{x} \cdot g'(x) - f(x) + f'(x) \geq 0\),
%
所以对任意的\(x \in \left\lbrack 0,\frac{\text{π}}{4} \right\rbrack\),\(a\text{e}^{x} \cdot g'(x) \geq f(x) - f'(x)\)恒成立.}
\end{question}
