\examxtitle{2025年高考全国一卷数学真题}

\section{单选题}

\begin{question}
\((1 + 5\text{i})\text{i}\)的虚部为(   )
\begin{choices}
  \item \(- 1\)
  \item 0
  \item 1
  \item 6
\end{choices}
\topics{求复数的实部与虚部;复数代数形式的乘法运算}
\difficulty{0.94}
\answer{C}
\explain{因为\(\left( 1 + 5\text{i} \right)\text{i} = \text{i} + 5\text{i}^{2} = - 5 + \text{i}\),所以其虚部为1}
\end{question}

\begin{question}
已知集合\(U = \{ x\left| x是小于9的正整数 \right.\ \}\),\(A = \{ 1,3,5\}\),则\(\complement_{U}A\)中元素个数为(   )
\begin{choices}
  \item 0
  \item 3
  \item 5
  \item 8
\end{choices}
\topics{补集的概念及运算}
\difficulty{0.94}
\answer{C}
\explain{因为\(U = \left\{ 1,2,3,4,5,6,7,8 \right\}\),所以\(\complement_{U}A = \left\{ 2,4,6,7,8 \right\}\),
\(\complement_{U}A\)中的元素个数为\(5\)}
\end{question}

\begin{question}
已知双曲线\emph{C}的虚轴长是实轴长的\(\sqrt{7}\)倍,则\emph{C}的离心率为(   )
\begin{choices}
  \item \(\sqrt{2}\)
  \item 2
  \item \(\sqrt{7}\)
  \item \(2\sqrt{2}\)
\end{choices}
\topics{求双曲线的离心率或离心率的取值范围}
\difficulty{0.94}
\answer{D}
\explain{设双曲线的实轴,虚轴,焦距分别为\(2a,2b,2c\),
由题知,\(b = \sqrt{7}a\),
于是\(a^{2} + b^{2} = c^{2} = a^{2} + 7a^{2} = 8a^{2}\),则\(c = 2\sqrt{2}a\),
即\(e = \frac{c}{a} = 2\sqrt{2}\)}
\end{question}

\begin{question}
已知点\((a,0)(a > 0)\)是函数\(y = 2\tan\left( x - \frac{\text{π}}{3} \right)\)的图象的一个对称中心,则\emph{a}的最小值为(   )
\begin{choices}
  \item \(\frac{\text{π}}{6}\)
  \item \(\frac{\text{π}}{3}\)
  \item \(\frac{\text{π}}{\text{2}}\)
  \item \(\frac{\text{4π}}{\text{3}}\)
\end{choices}
\topics{求正切(型)函数的对称中心;正切函数对称性的应用}
\difficulty{0.85}
\answer{B}
\explain{根据正切函数的性质,\(y = 2\tan(x - \frac{\text{π}}{3})\)的对称中心横坐标满足\(x - \frac{\text{π}}{3} = \frac{k\text{π}}{2},k \in Z\),
即\(y = 2\tan(x - \frac{\text{π}}{3})\)的对称中心是\(\left( \frac{\text{π}}{3} + \frac{k\text{π}}{2},0 \right),k \in Z\),
即\(a = \frac{\text{π}}{3} + \frac{k\text{π}}{2},k \in Z\),
又\(a > 0\),则\(k = 0\)时\(a\)最小,最小值是\(\frac{\text{π}}{3}\),
即\(a = \frac{\text{π}}{3}\)}
\end{question}

\begin{question}
已知\(f(x)\)是定义在\(R\)上且周期为2的偶函数,当\(2 \leq x \leq 3\)时,\(f(x) = 5 - 2x\),则\(f\left( - \frac{3}{4} \right) =\)(   )
\begin{choices}
  \item \(- \frac{1}{2}\)
  \item \(- \frac{1}{4}\)
  \item \(\frac{1}{4}\)
  \item \(\frac{1}{2}\)
\end{choices}
\topics{函数奇偶性的应用;由函数的周期性求函数值}
\difficulty{0.85}
\answer{A}
\explain{由题知\(f(x) = f( - x),f(x + 2) = f(x)\)对一切\(x \in R\)成立,
于是\(f( - \frac{3}{4}) = f(\frac{3}{4}) = f(\frac{11}{4}) = 5 - 2 \times \frac{11}{4} = - \frac{1}{2}\)}
\end{question}

\begin{question}
帆船比赛中,运动员可借助风力计测定风速的大小与方向,测出的结果在航海学中称为视风风速.视风风速对应的向量是真风风速对应的向量与船行风风速对应的向量之和,其中船行风风速对应的向量与船速对应的向量大小相等、方向相反.图1给出了部分风力等级、名称与风速大小的对应关系.已知某帆船运动员在某时刻测得的视风风速对应的向量与船速对应的向量如图2所示(线段长度代表速度大小,单位:\(m\)/s),则该时刻的真风为(   )

  ---------------------- ---------------------- -------------------------
           级数                   名称            风速大小(单位:\(m\)/s)

            2                     轻风                  1.6\~3.3

            3                     微风                  3.4\~5.4

            4                     和风                  5.5\~7.9

            5                     劲风                  8.0\~10.7
  ---------------------- ---------------------- -------------------------

% IMAGE_TODO_START id=gaokao_2025_national_1-Q6-img1 path=/Users/muryor/code/mynote/word\_to\_tex/output/figures/gaokao\_2025\_national\_1/media/image2.png width=60% inline=true question_index=6 sub_index=1
% CONTEXT_BEFORE: ---------------------- -------------------------
% CONTEXT_AFTER: 
\begin{tikzpicture}[scale=0.8,baseline=-0.5ex]
  % TODO: AI_AGENT_REPLACE_ME (id=gaokao_2025_national_1-Q6-img1)
\end{tikzpicture}
% IMAGE_TODO_END id=gaokao_2025_national_1-Q6-img1

\begin{choices}
  \item 轻风
  \item 微风
  \item 和风
  \item 劲风
\end{choices}
\topics{平面向量线性运算的坐标表示;向量坐标的线性运算解决几何问题;坐标计算向量的模}
\difficulty{0.94}
\answer{A}
\explain{由题意及图得,
视风风速对应的向量为:\(\overrightarrow{n} = (0,2) - (3,3) = ( - 3, - 1)\),
视风风速对应的向量,是真风风速对应的向量与船行风速对应的向量之和,
船速方向和船行风速的向量方向相反,
设真风风速对应的向量为\(\overrightarrow{n_{1}}\),船行风速对应的向量为\(\overrightarrow{n_{2}}\),
\(\therefore \overrightarrow{n} = \overrightarrow{n_{1}} + \overrightarrow{n_{2}}\),船行风速:\(\overrightarrow{n_{2}} = - \left\lbrack (3,3) - (2,0) \right\rbrack = ( - 1, - 3)\),
\(\therefore \overrightarrow{n_{1}} = \overrightarrow{n} - \overrightarrow{n_{2}} = ( - 3, - 1) - ( - 1, - 3) = ( - 2,2)\),
\(\left| \overrightarrow{n_{1}} \right| = \sqrt{( - 2)^{2} + 2^{2}} = 2\sqrt{2} \approx 2.828\),
\(\therefore\)由表得,真风风速为轻风}
\end{question}

\begin{question}
已知圆\(x^{2} + {(y + 2)}^{2} = r^{2}(r > 0)\)上到直线\(y = \sqrt{3}x + 2\)的距离为1的点有且仅有2个,则\emph{r}的取值范围是(   )
\begin{choices}
  \item \((0,1)\)
  \item \((1,3)\)
  \item \((3, + \infty)\)
  \item \((0, + \infty)\)
\end{choices}
\topics{求点到直线的距离;圆上点到定直线(图形)上的最值(范围);坐标法的应用------直线与圆的位置关系}
\difficulty{0.65}
\answer{B}
\explain{由题意,
在圆\(x^{2} + (y + 2)^{2} = r^{2}(r > 0)\)中,圆心\(E(0, - 2)\),半径为\(r\),
到直线\(y = \sqrt{3}x + 2\)的距离为\(1\)的点有且仅有 \(2\)个,
\(\because\)圆心\(E(0, - 2)\)到直线\(y = \sqrt{3}x + 2\)的距离为:\(d = \frac{\left| 0 \times \sqrt{3} - ( - 2) \times 1 + 2 \right|}{\sqrt{\left( \sqrt{3} \right)^{2} + ( - 1)^{2}}} = 2\),
% IMAGE_TODO_START id=gaokao_2025_national_1-Q7-img1 path=/Users/muryor/code/mynote/word\_to\_tex/output/figures/gaokao\_2025\_national\_1/media/image3.png width=60% inline=true question_index=7 sub_index=1
% CONTEXT_BEFORE: left( \sqrt{3} \right)^{2} + ( - 1)^{2}}} = 2\(,
% CONTEXT_AFTER: 
\begin{tikzpicture}[scale=0.8,baseline=-0.5ex]
  % TODO: AI_AGENT_REPLACE_ME (id=gaokao_2025_national_1-Q7-img1)
\end{tikzpicture}
% IMAGE_TODO_END id=gaokao_2025_national_1-Q7-img1

故由图可知,

当\(r = 1\)时,

圆\(x^{2} + (y + 2)^{2} = r^{2}(r > 0)\)上有且仅有一个点(\(A\)点)到直线\(y = \sqrt{3}x + 2\)的距离等于\(1\);

当\(r = 3\)时,

圆\(x^{2} + (y + 2)^{2} = r^{2}(r > 0)\)上有且仅有三个点(\(B,C,D\)点)到直线\(y = \sqrt{3}x + 2\)的距离等于\(1\);

当则\(r\)的取值范围为\((1,3)\)时,

圆\(x^{2} + (y + 2)^{2} = r^{2}(r > 0)\)上有且仅有两个点到直线\(y = \sqrt{3}x + 2\)的距离等于\(1\)}
\end{question}

\begin{question}
已知\(2 + \log_{2}x = 3 + \log_{3}y = 5 + \log_{5}z\),则\emph{x},\emph{y},\emph{z}的大小关系不可能是(   )
\begin{choices}
  \item \(x > y > z\)
  \item \(x > z > y\)
  \item \(y > x > z\)
  \item \(y > z > x\)
\end{choices}
\topics{对数的运算性质的应用;对数函数单调性的应用}
\difficulty{0.4}
\answer{B}
\explain{法一:设\(2 + \log_{2}x = 3 + \log_{3}y = 5 + \log_{5}z = m\),所以
令\(m = 2\),则\(x = 1,y = 3^{- 1} = \frac{1}{3},z = 5^{- 3} = \frac{1}{125}\),此时\(x > y > z\),A有可能;
令\(m = 5\),则\(x = 8,y = 9,z = 1\),此时\(y > x > z\),C有可能;
令\(m = 8\),则\(x = 2^{6} = 64,y = 3^{5} = 243,z = 5^{3} = 125\),此时\(y > z > x\),D有可能;
法二:设\(2 + \log_{2}x = 3 + \log_{3}y = 5 + \log_{5}z = m\),所以,\(x = 2^{m - 2},y = 3^{m - 3},z = 5^{m - 5}\)
根据指数函数的单调性,易知各方程只有唯一的根,
作出函数\(y = 2^{x - 2},y = 3^{x - 3},y = 5^{x - 5}\)的图象,以上方程的根分别是函数\(y = 2^{x - 2},y = 3^{x - 3},y = 5^{x - 5}\)的图象与直线\(x = m\)的交点纵坐标,如图所示:
% IMAGE_TODO_START id=gaokao_2025_national_1-Q8-img1 path=/Users/muryor/code/mynote/word\_to\_tex/output/figures/gaokao\_2025\_national\_1/media/image4.png width=60% inline=true question_index=8 sub_index=1
% CONTEXT_BEFORE: - 3},y = 5^{x - 5}\)的图象与直线\(x = m\)的交点纵坐标,如图所示:
% CONTEXT_AFTER: 
\begin{tikzpicture}[scale=0.8,baseline=-0.5ex]
  % TODO: AI_AGENT_REPLACE_ME (id=gaokao_2025_national_1-Q8-img1)
\end{tikzpicture}
% IMAGE_TODO_END id=gaokao_2025_national_1-Q8-img1

易知,随着\(m\)的变化可能出现:\(x > y > z\),\(y > x > z\),\(y > z > x\),\(z > y > x\)}
\end{question}

\section{多选题}

\begin{question}
在正三棱柱\(ABC - A_{1}B_{1}C_{1}\)中,\emph{D}为*BC*的中点,则(    )
\begin{choices}
  \item \(AD\bot A_{1}C\)
  \item \(B_{1}C_{1}\bot\)平面\(AA_{1}D\)
  \item \(AD//A_{1}B_{1}\)
  \item \(CC_{1}//\)平面\(AA_{1}D\)
\end{choices}
\topics{证明线面平行;证明线面垂直;空间位置关系的向量证明;求平面的法向量}
\difficulty{0.65}
\answer{BD}
\explain{