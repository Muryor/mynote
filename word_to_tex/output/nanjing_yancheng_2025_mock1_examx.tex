\examxtitle{江苏省南京市、盐城市2024-2025学年高三下学期第一次模拟考试数学试题}

\section{单选题}

\begin{question}
设集合\(A = \left\{ x \mid x^{2} - 4 \leq 0 \right\},B = \left\{ x \mid x + a \leq 0 \right\}\).若\(A \subseteq B\),则实数\(a\)的取值范围是(    )
\begin{choices}
  \item \(( - \infty,2)\)
  \item \(( - \infty,2\rbrack\)
  \item \(( - \infty, - 2)\)
  \item \(( - \infty, - 2\rbrack\)
\end{choices}
\topics{根据集合的包含关系求参数;解不含参数的一元二次不等式}
\difficulty{0.85}
\answer{D}
\explain{由\(x^{2} - 4 \leq 0\)可得\(A = \lbrack - 2,2\rbrack\),由\(x + a \leq 0\)可得\(B = ( - \infty, - a\rbrack\),
又\(A \subseteq B\),所以\(2 \leq - a\),即\(a \leq - 2\),故D正确}
\end{question}

\begin{question}
已知复数\(z\)满足\(\frac{1}{z + \text{i}} = \text{i}\)(\(\text{i}\)为虚数单位),则\(|z| =\)(    )
\begin{choices}
  \item 4
  \item 2
  \item 1
  \item \(\frac{1}{2}\)
\end{choices}
\topics{求复数的模;复数的除法运算}
\difficulty{0.85}
\answer{B}
\explain{因为\(\frac{1}{z + \text{i}} = \text{i}\),所以\(z + \text{i} = \frac{1}{\text{i}} = \frac{\text{i}}{\text{i}^{2}} = - \text{i}\),所以\(z = - 2\text{i}\),
\par
所以\(|z| = 2\)}
\end{question}

\begin{question}
已知\(\overrightarrow{a},\overrightarrow{b},\overrightarrow{c}\)均为单位向量.若\(\overrightarrow{a} = \overrightarrow{b} + \overrightarrow{c}\),则\(\overrightarrow{b}\)与\(\overrightarrow{c}\)夹角的大小是(    )
\begin{choices}
  \item \(\frac{\text{π}}{6}\)
  \item \(\frac{\text{π}}{3}\)
  \item \(\frac{2\text{π}}{3}\)
  \item \(\frac{5\text{π}}{6}\)
\end{choices}
\topics{用定义求向量的数量积;数量积的运算律;向量夹角的计算}
\difficulty{0.65}
\answer{C}
\explain{已知\(\overrightarrow{a} = \overrightarrow{b} + \overrightarrow{c}\),两边平方可得\({\overrightarrow{a}}^{2} = {(\overrightarrow{b} + \overrightarrow{c})}^{2}\).
\par
则\({(\overrightarrow{b} + \overrightarrow{c})}^{2} = {\overrightarrow{b}}^{2} + 2\overrightarrow{b} \cdot \overrightarrow{c} + {\overrightarrow{c}}^{2}\),所以\({\overrightarrow{a}}^{2} = {\overrightarrow{b}}^{2} + 2\overrightarrow{b} \cdot \overrightarrow{c} + {\overrightarrow{c}}^{2}\).
\par
因为\(\overrightarrow{a},\overrightarrow{b},\overrightarrow{c}\)均为单位向量,所以\(|\overrightarrow{a}| = |\overrightarrow{b}| = |\overrightarrow{c}| = 1\).
\par
根据\({\overrightarrow{a}}^{2} = |\overrightarrow{a}|^{2} = 1\),\({\overrightarrow{b}}^{2} = |\overrightarrow{b}|^{2} = 1\),\({\overrightarrow{c}}^{2} = |\overrightarrow{c}|^{2} = 1\).
\par
将其代入\({\overrightarrow{a}}^{2} = {\overrightarrow{b}}^{2} + 2\overrightarrow{b} \cdot \overrightarrow{c} + {\overrightarrow{c}}^{2}\)可得:\(1 = 1 + 2\overrightarrow{b} \cdot \overrightarrow{c} + 1\).
则\(\overrightarrow{b} \cdot \overrightarrow{c} = - \frac{1}{2}\).
\par
设\(\overrightarrow{b}\)与\(\overrightarrow{c}\)的夹角为\(\theta\),\(0 \leq \theta \leq \pi\),且\(|\overrightarrow{b}| = |\overrightarrow{c}| = 1\),\(\overrightarrow{b} \cdot \overrightarrow{c} = - \frac{1}{2}\),可得\(- \frac{1}{2} = 1 \times 1 \times \cos\theta\),即\(\cos\theta = - \frac{1}{2}\).
\par
因为\(0 \leq \theta \leq \pi\),所以\(\theta = \frac{2\pi}{3}\).
\par
则\(\overrightarrow{b}\)与\(\overrightarrow{c}\)夹角的大小是\(\frac{2\pi}{3}\)}
\end{question}

\begin{question}
某项比赛共有10个评委评分,若去掉一个最高分与一个最低分,则与原始数据相比,一定不变的是(    )
\begin{choices}
  \item 极差
  \item 45百分位数
  \item 平均数
  \item 众数
\end{choices}
\topics{计算几个数的众数;计算几个数的平均数;计算几个数据的极差;方差;标准差;总体百分位数的估计}
\difficulty{0.94}
\answer{B}
\explain{对A,若每个数据都不相同,则极差一定变化,故A错误;
对B,由\(10 \times 0.45\text{=}4.5 < 5\),所以将10个数据从小到大排列,45百分位数为第5个数据,
从10个原始评分中去掉1个最高分、1个最低分,得到8个有效评分,\(8 \times 0.45\text{=}3.6 < 4\),
所以45百分位数为8个数据从小到大排列后第4个数据,即为原来的第5个数据.
对C,去掉一个最高分一个最低分,平均数可能变化,故C错误;
对D,去掉一个最高分一个最低分,众数可能变化,故D错误}
\end{question}

\begin{question}
已知数列\(\left\{ a_{n} \right\}\)为等比数列,公比为2,且\(a_{1} + a_{2} = 3\).若\(a_{k} + a_{k + 1} + a_{k + 2} + \cdots + a_{k + 9} = 2^{14} - 2^{4}\),则正整数\(k\)的值是(    )
\begin{choices}
  \item 4
  \item 5
  \item 6
  \item 7
\end{choices}
\topics{等比数列通项公式的基本量计算;求等比数列前n项和}
\difficulty{0.85}
\answer{B}
\explain{因为数列\(\left\{ a_{n} \right\}\)为等比数列,公比为2,且\(a_{1} + a_{2} = 3\),所以\(a_{1} + 2a_{1} = 3\),解得\(a_{1} = 1\),
\par
故\(a_{n} = 2^{n - 1}\),因为\(a_{k} + a_{k + 1} + a_{k + 2} + \cdots + a_{k + 9} = a_{k}(1 + 2 + 2^{2} + ... + 2^{9})= 2^{k - 1} \cdot \frac{1 - 2^{10}}{1 - 2} = 2^{k + 9} - 2^{k - 1} = 2^{14} - 2^{4}\),解得\(k = 5\)}
\end{question}

\begin{question}
在锐角\(\bigtriangleup ABC\)中,角\(A,B,C\)所对的边分别为\(a,b,c\).若\(b - 2c = a\text{cos}C - 2a\text{cos}B\),则\(\frac{c}{b} =\)(    )
\begin{choices}
  \item \(\frac{1}{3}\)
  \item \(\frac{1}{2}\)
  \item 1
  \item 2
\end{choices}
\topics{正弦定理边角互化的应用}
\difficulty{0.85}
\answer{D}
\explain{