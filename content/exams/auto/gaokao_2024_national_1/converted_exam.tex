\examxtitle{2024年新课标全国Ⅰ卷数学真题}

\section{单选题}

\begin{question}
已知集合\(A = \left\{ x \mid - 5 < x^{3} < 5 \right\},B = \{ - 3, - 1,0,2,3\}\),
则\(A \cap B =\)(    )
\begin{choices}
  \item \(\{ - 1,0\}\)
  \item \(\{ 2,3\}\)
  \item \(\{ - 3, - 1,0\}\)
  \item \(\{ - 1,0,2\}\)
\end{choices}
\topics{交集的概念及运算;由幂函数的单调性解不等式}
\difficulty{0.94}
\answer{A}
\explain{因为\(A = \left\{ x| - \sqrt[3]{5} < x < \sqrt[3]{5} \right\},B = \left\{ - 3, - 1,0,2,3 \right\}\),
且注意到\(1 < \sqrt[3]{5} < 2\),\par
从而\(A \cap B =\) \(\left\{ - 1,0 \right\}\)}
\end{question}

\begin{question}
若\(\frac{z}{z - 1} = 1 + \text{i}\),则\(z =\)(    )
\begin{choices}
  \item \(- 1 - \text{i}\)
  \item \(- 1 + \text{i}\)
  \item \(1 - \text{i}\)
  \item \(1 + \text{i}\)
\end{choices}
\topics{复数的乘方;复数的除法运算}
\difficulty{0.94}
\answer{C}
\explain{因为\(\frac{z}{z - 1} = \frac{z - 1 + 1}{z - 1} = 1 + \frac{1}{z - 1} = 1 + \text{i}\),
所以\(z = 1 + \frac{1}{\text{i}} = 1 - \text{i}\)}
\end{question}

\begin{question}
已知向量\(\overrightarrow{a} = (0,1),\overrightarrow{b} = (2,x)\),
若\(\overrightarrow{b}\bot(\overrightarrow{b} - 4\overrightarrow{a})\),
则\(x =\)(    )
\begin{choices}
  \item \(- 2\)
  \item \(- 1\)
  \item 1
  \item 2
\end{choices}
\topics{平面向量线性运算的坐标表示;向量垂直的坐标表示}
\difficulty{0.85}
\answer{D}
\explain{因为\(\overrightarrow{b}\bot\left( \overrightarrow{b} - 4\overrightarrow{a} \right)\),
所以\(\overrightarrow{b} \cdot \left( \overrightarrow{b} - 4\overrightarrow{a} \right) = 0\),\par
所以\({\overrightarrow{b}}^{2} - 4\overrightarrow{a} \cdot \overrightarrow{b} = 0\)即\(4 + x^{2} - 4x = 0\),
故\(x = 2\)}
\end{question}

\begin{question}
已知\(\cos(\alpha + \beta) = m,\tan\alpha\tan\beta = 2\),
则\(\cos(\alpha - \beta) =\)(    )
\begin{choices}
  \item \(- 3m\)
  \item \(- \frac{m}{3}\)
  \item \(\frac{m}{3}\)
  \item \(3m\)
\end{choices}
\topics{三角函数的化简;求值------同角三角函数基本关系;用和;差角的余弦公式化简;求值}
\difficulty{0.85}
\answer{A}
\explain{因为\(\cos(\alpha + \beta) = m\),
所以\(\cos\alpha\cos\beta - \sin\alpha\sin\beta = m\),\par
而\(\tan\alpha\tan\beta = 2\),
所以\(\sin\alpha\sin\beta = 2\cos\alpha\cos\beta\),\par
故\(\cos\alpha\cos\beta - 2\cos\alpha\cos\beta = m\)即\(\cos\alpha\cos\beta = - m\),\par
从而\(\sin\alpha\sin\beta = - 2m\),
故\(\cos(\alpha - \beta) = - 3m\)}
\end{question}

\begin{question}
已知圆柱和圆锥的底面半径相等,侧面积相等,且它们的高均为\(\sqrt{3}\),则圆锥的体积为(    )
\begin{choices}
  \item \(2\sqrt{3}\pi\)
  \item \(3\sqrt{3}\pi\)
  \item \(6\sqrt{3}\pi\)
  \item \(9\sqrt{3}\pi\)
\end{choices}
\topics{圆柱表面积的有关计算;圆锥表面积的有关计算;锥体体积的有关计算}
\difficulty{0.85}
\answer{B}
\explain{设圆柱的底面半径为\(r\),
则圆锥的母线长为\(\sqrt{r^{2} + 3}\),\par
而它们的侧面积相等,
所以\(2\pi r \times \sqrt{3} = \pi r \times \sqrt{3 + r^{2}}\)即\(2\sqrt{3} = \sqrt{3 + r^{2}}\),\par
故\(r = 3\),
故圆锥的体积为\(\frac{1}{3}\pi \times 9 \times \sqrt{3} = 3\sqrt{3}\pi\)}
\end{question}

\begin{question}
已知函数\(f(x) = \left\{ \begin{array}{r}
 - x^{2} - 2ax - a,x < 0 \\
e^{x} + \ln(x + 1),x \geq 0
\end{array} \right.\)在\(\mathbf{R}\)上单调递增,则\(a\)的取值范围是(    )
\begin{choices}
  \item \(( - \infty,0\rbrack\)
  \item \(\lbrack - 1,0\rbrack\)
  \item \(\lbrack - 1,1\rbrack\)
  \item \(\lbrack 0, + \infty)\)
\end{choices}
\topics{判断指数函数的单调性;研究对数函数的单调性;根据分段函数的单调性求参数}
\difficulty{0.65}
\answer{B}
\explain{因为\(f(x)\)在\(R\)上单调递增,且\(x \geq 0\)时,
\(f(x) = e^{x} + \ln(x + 1)\)单调递增,\par
则需满足\(\left\{ \begin{array}{r}
 - \frac{- 2a}{2 \times ( - 1)} \geq 0 \\
 - a \leq e^{0} + \ln 1
\end{array} \right.\),解得\(- 1 \leq a \leq 0\),\par
即\(a\)的范围是\(\lbrack - 1,0\rbrack\)}
\end{question}

\begin{question}
当\(x \in \lbrack 0,2\pi\rbrack\)时,
曲线\(y = \sin x\)与\(y = 2\sin\left( 3x - \frac{\pi}{6} \right)\)的交点个数为(    )
\begin{choices}
  \item 3
  \item 4
  \item 6
  \item 8
\end{choices}

\begin{center}
% IMAGE_TODO_START id=gaokao_2024_national_1-Q7-img1 path=image2.png width=60%
\includegraphics[width=0.5\textwidth]{content/exams/auto/gaokao_2024_national_1/images/media/image2.png}
% IMAGE_TODO_END id=gaokao_2024_national_1-Q7-img1
\end{center}

\topics{正弦函数图象的应用;求函数零点或方程根的个数}
\difficulty{0.65}
\answer{C}
\explain{因为函数\(y = \sin x\)的最小正周期为\(T = 2\pi\),\par
函数\(y = 2\sin\left( 3x - \frac{\pi}{6} \right)\)的最小正周期为\(T = \frac{2\pi}{3}\),\par
所以在\(x \in \left\lbrack 0,2\pi \right\rbrack\)上函数\(y = 2\sin\left( 3x - \frac{\pi}{6} \right)\)有三个周期的图象,\par
在坐标系中结合五点法画出两函数图象,如图所示:\par
由图可知,两函数图象有6个交点}
\end{question}

\begin{question}
已知函数\(f(x)\)的定义域为\(\mathbf{R}\),
\(f(x) > f(x - 1) + f(x - 2)\),
且当\(x < 3\)时\(f(x) = x\),
则下列结论中一定正确的是(    )
\begin{choices}
  \item \(f(10) > 100\)
  \item \(f(20) > 1000\)
  \item \(f(10) < 1000\)
  \item \(f(20) < 10000\)
\end{choices}
\topics{求函数值;比较函数值的大小关系}
\difficulty{0.4}
\answer{B}
\explain{因为当\(x < 3\)时\(f(x) = x\),
所以\(f(1) = 1,f(2) = 2\),\par
又因为\(f(x) > f(x - 1) + f(x - 2)\),\par
则\(f(3) > f(2) + f(1) = 3,f(4) > f(3) + f(2) > 5\),\par
\(f(5) > f(4) + f(3) > 8,f(6) > f(5) + f(4) > 13,f(7) > f(6) + f(5) > 21\),\par
\(f(8) > f(7) + f(6) > 34,f(9) > f(8) + f(7) > 55,f(10) > f(9) + f(8) > 89\),\par
\(f(11) > f(10) + f(9) > 144,f(12) > f(11) + f(10) > 233,f(13) > f(12) + f(11) > 377f(14) > f(13) + f(12) > 610,f(15) > f(14) + f(13) > 987\),\par
\(f(16) > f(15) + f(14) > 1597 > 1000\),
则依次下去可知\(f(20) > 1000\),则B正确;\par
且无证据表明ACD一定正确}
\end{question}

\section{多选题}

\begin{question}
随着"一带一路"国际合作的深入,
某茶叶种植区多措并举推动茶叶出口.为了解推动出口后的亩收入(单位:万元)情况,
从该种植区抽取样本,
得到推动出口后亩收入的样本均值\(\overline{x} = 2.1\),
样本方差\(s^{2} = 0.01\),
已知该种植区以往的亩收入\(X\)服从正态分布\(N\left( 1.8,{0.1}^{2} \right)\),
假设推动出口后的亩收入\(Y\)服从正态分布\(N\left( \overline{x},s^{2} \right)\),
则(    )(若随机变量\(Z\)服从正态分布\(N\left( \mu,\sigma^{2} \right)\),
\(P(Z < \mu + \sigma) \approx 0.8413\))
\begin{choices}
  \item \(P(X > 2) > 0.2\)
  \item \(P(X > 2) < 0.5\)
  \item \(P(Y > 2) > 0.5\)
  \item \(P(Y > 2) < 0.8\)
\end{choices}
\topics{指定区间的概率;正态分布的实际应用}
\difficulty{0.85}
\answer{BC}
\explain{依题可知,\(\overline{x} = 2.1,s^{2} = 0.01\),
所以\(Y\sim N\left( 2.1,{0.1}^{2} \right)\),\par
故\(P(Y > 2) = P(Y > 2.1 - 0.1) = P(Y < 2.1 + 0.1) \approx 0.8413 > 0.5\),
C正确,D错误;\par
因为\(X\sim N\left( 1.8,{0.1}^{2} \right)\),
所以\(P(X > 2) = P(X > 1.8 + 2 \times 0.1)\),\par
因为\(P(X < 1.8 + 0.1) \approx 0.8413\),
所以\(P(X > 1.8 + 0.1) \approx 1 - 0.8413 = 0.1587 < 0.2\),\par
而\(P(X > 2) = P(X > 1.8 + 2 \times 0.1) < P(X > 1.8 + 0.1) < 0.2\),
B正确,A错误}
\end{question}

\begin{question}
设函数\(f(x) = {(x - 1)}^{2}(x - 4)\),则(    )
\begin{choices}
  \item \(x = 3\)是\(f(x)\)的极小值点
  \item 当\(0 < x < 1\)时,\(f(x) < f\left( x^{2} \right)\)
  \item 当\(1 < x < 2\)时,\(- 4 < f(2x - 1) < 0\)
  \item 当\(- 1 < x < 0\)时,\(f(2 - x) > f(x)\)
\end{choices}
\topics{利用导数求函数的单调区间(不含参);求已知函数的极值点}
\difficulty{0.65}
\answer{ACD}
\explain{对A,因为函数\(f(x)\)的定义域为\(\mathbf{R}\),
而\(f'(x) = 2(x - 1)(x - 4) + (x - 1)^{2} = 3(x - 1)(x - 3)\),\par
易知当\(x \in (1,3)\)时,\(f'(x) < 0\),
当\(x \in ( - \infty,1)\)或\(x \in (3, + \infty)\)时,
\(f'(x) > 0\)\par
函数\(f(x)\)在\(( - \infty,1)\)上单调递增,
在\((1,3)\)上单调递减,在\((3, + \infty)\)上单调递增,
故\(x = 3\)是函数\(f(x)\)的极小值点,正确;\par
对B,当\(0 < x < 1\)时,
\(x - x^{2} = x(1 - x) > 0\),
所以\(1 > x > x^{2} > 0\),\par
而由上可知,函数\(f(x)\)在\((0,1)\)上单调递增,
所以\(f(x) > f\left( x^{2} \right)\),错误;\par
对C,当\(1 < x < 2\)时,\(1 < 2x - 1 < 3\),
而由上可知,函数\(f(x)\)在\((1,3)\)上单调递减,\par
所以\(f(1) > f(2x - 1) > f(3)\),
即\(- 4 < f(2x - 1) < 0\),正确;\par
对D,当\(- 1 < x < 0\)时,
\(f(2 - x) - f(x) = (1 - x)^{2}( - 2 - x) - (x - 1)^{2}(x - 4) = (x - 1)^{2}(2 - 2x) > 0\),\par
所以\(f(2 - x) > f(x)\),正确}
\end{question}

\begin{question}
设计一条美丽的丝带,其造型
% IMAGE_TODO_START id=gaokao_2024_national_1-Q11-img1 path=image3.png (inline)
\includegraphics[height=1.2em]{content/exams/auto/gaokao_2024_national_1/images/media/image3.png}
% IMAGE_TODO_END id=gaokao_2024_national_1-Q11-img1
可以看作图中的曲线\(C\)的一部分.已知\(C\)过坐标原点\(O\).且\(C\)上的点满足:横坐标大于\(- 2\),
到点\(F(2,0)\)的距离与到定直线\(x = a(a < 0)\)的距离之积为4,
则(    )
\begin{choices}
  \item \(a = - 2\)
  \item 点\((2\sqrt{2},0)\)在\(C\)上
  \item \(C\)在第一象限的点的纵坐标的最大值为1
  \item 当点\(\left( x_{0},y_{0} \right)\)在\(C\)上时,\(y_{0} \leq \frac{4}{x_{0} + 2}\)
\end{choices}

\begin{center}
% IMAGE_TODO_START id=gaokao_2024_national_1-Q11-img2 path=image4.png width=60%
\includegraphics[width=0.4\textwidth]{content/exams/auto/gaokao_2024_national_1/images/media/image4.png}
% IMAGE_TODO_END id=gaokao_2024_national_1-Q11-img2
\end{center}

\topics{由方程研究曲线的性质;求平面轨迹方程}
\difficulty{0.65}
\answer{ABD}
\explain{对于A:设曲线上的动点\(P(x,y)\),
则\(x > - 2\)且\(\sqrt{(x - 2)^{2} + y^{2}} \times |x - a| = 4\),\par
因为曲线过坐标原点,
故\(\sqrt{(0 - 2)^{2} + 0^{2}} \times |0 - a| = 4\),解得\(a = - 2\),故A正确.\par
对于B:又曲线方程为\(\sqrt{(x - 2)^{2} + y^{2}} \times |x + 2| = 4\),
而\(x > - 2\),\par
故\(\sqrt{(x - 2)^{2} + y^{2}} \times (x + 2) = 4\).\par
当\(x = 2\sqrt{2},y = 0\)时,
\(\sqrt{\left( 2\sqrt{2} - 2 \right)^{2}} \times \left( 2\sqrt{2} + 2 \right) = 8 - 4 = 4\),\par
故\(\left( 2\sqrt{2},0 \right)\)在曲线上,
故B正确.\par
对于C:由曲线的方程可得\(y^{2} = \frac{16}{(x + 2)^{2}} - (x - 2)^{2}\),
取\(x = \frac{3}{2}\),\par
则\(y^{2} = \frac{64}{49} - \frac{1}{4}\),
而\(\frac{64}{49} - \frac{1}{4} - 1 = \frac{64}{49} - \frac{5}{4} = \frac{256 - 245}{49 \times 4} > 0\),
故此时\(y^{2} > 1\),\par
故\(C\)在第一象限内点的纵坐标的最大值大于1,故C错误.\par
对于D:当点\(\left( x_{0},y_{0} \right)\)在曲线上时,
由C的分析可得\(y_{0}^{2} = \frac{16}{\left( x_{0} + 2 \right)^{2}} - \left( x_{0} - 2 \right)^{2} \leq \frac{16}{\left( x_{0} + 2 \right)^{2}}\),\par
故\(- \frac{4}{x_{0} + 2} \leq y_{0} \leq \frac{4}{x_{0} + 2}\),
故D正确}
\end{question}

\section{填空题}

\begin{question}
设双曲线\(C:\frac{x^{2}}{a^{2}} - \frac{y^{2}}{b^{2}} = 1(a > 0,b > 0)\)的左右焦点分别为\(F_{1},F_{2}\),
过\(F_{2}\)作平行于\(y\)轴的直线交\(C\)于\(A\),
\(B\)两点,若\(|F_{1}A| = 13,|AB| = 10\),
则\(C\)的离心率为
.

\begin{center}
% IMAGE_TODO_START id=gaokao_2024_national_1-Q12-img1 path=image5.png width=60%
\includegraphics[width=0.4\textwidth]{content/exams/auto/gaokao_2024_national_1/images/media/image5.png}
% IMAGE_TODO_END id=gaokao_2024_national_1-Q12-img1
\end{center}

\topics{求双曲线的离心率或离心率的取值范围}
\difficulty{0.65}
\answer{\(\frac{3}{2}\)}
\explain{由题可知\(A,B,F_{2}\)三点横坐标相等,设\(A\)在第一象限,
将\(x = c\)代入\(\frac{x^{2}}{a^{2}} - \frac{y^{2}}{b^{2}} = 1\)\par
得\(y = \pm \frac{b^{2}}{a}\),
即\(A\left( c,\frac{b^{2}}{a} \right),B\left( c, - \frac{b^{2}}{a} \right)\),
故\(|AB| = \frac{2b^{2}}{a} = 10\),
\(\left| AF_{2} \right| = \frac{b^{2}}{a} = 5\),\par
又\(\left| AF_{1} \right| - \left| AF_{2} \right| = 2a\),
得\(\left| AF_{1} \right| = \left| AF_{2} \right| + 2a = 2a + 5 = 13\),
解得\(a = 4\),
代入\(\frac{b^{2}}{a} = 5\)得\(b^{2} = 20\),\par
故\(c^{2} = a^{2} + b^{2} = 36\),,
即\(c = 6\),
所以\(e = \frac{c}{a} = \frac{6}{4} = \frac{3}{2}\).\(\frac{3}{2}\)}
\end{question}

\begin{question}
若曲线\(y = e^{x} + x\)在点\((0,1)\)处的切线也是曲线\(y = \ln(x + 1) + a\)的切线,则\(a =\)
.
\topics{已知切线(斜率)求参数;两条切线平行;垂直;重合(公切线)问题}
\difficulty{0.65}
\answer{\(\ln 2\)}
\explain{由\(y = \mathrm{e}^{x} + x\)得\(y' = \mathrm{e}^{x} + 1\),
\(y'|_{x = 0} = \mathrm{e}^{0} + 1 = 2\),\par
故曲线\(y = \mathrm{e}^{x} + x\)在\((0,1)\)处的切线方程为\(y = 2x + 1\);\par
由\(y = \ln(x + 1) + a\)得\(y' = \frac{1}{x + 1}\),\par
设切线与曲线\(y = \ln(x + 1) + a\)相切的切点为\(\left( x_{0},\ln\left( x_{0} + 1 \right) + a \right)\),\par
由两曲线有公切线得\(y' = \frac{1}{x_{0} + 1} = 2\),
解得\(x_{0} = - \frac{1}{2}\),
则切点为\(\left( - \frac{1}{2},a + \ln\frac{1}{2} \right)\),\par
切线方程为\(y = 2\left( x + \frac{1}{2} \right) + a + \ln\frac{1}{2} = 2x + 1 + a - \ln 2\),\par
根据两切线重合,所以\(a - \ln 2 = 0\),
解得\(a = \ln 2\).\(\ln 2\)}
\end{question}

\begin{question}
甲、乙两人各有四张卡片,每张卡片上标有一个数字,甲的卡片上分别标有数字1,3,5,
7,乙的卡片上分别标有数字2,4,6,8,两人进行四轮比赛,在每轮比赛中,
两人各自从自己持有的卡片中随机选一张,并比较所选卡片上数字的大小,
数字大的人得1分,数字小的人得0分,
然后各自弃置此轮所选的卡片(弃置的卡片在此后的轮次中不能使用).则四轮比赛后,
甲的总得分不小于2的概率为
.
\topics{计算古典概型问题的概率;求离散型随机变量的均值;均值的性质}
\difficulty{0.4}
\answer{\(\frac{1}{2}\)/0.5}
\explain{设甲在四轮游戏中的得分分别为\(X_{1},X_{2},X_{3},X_{4}\),
四轮的总得分为\(X\).\par
对于任意一轮,甲乙两人在该轮出示每张牌的概率都均等,
其中使得甲得分的出牌组合有六种,
从而甲在该轮得分的概率\(P\left( X_{k} = 1 \right) = \frac{6}{4 \times 4} = \frac{3}{8}\),
所以\(E\left( X_{k} \right) = \frac{3}{8}(k = 1,2,3,4)\).\par
从而\(E(X) = E\left( X_{1} + X_{2} + X_{3} + X_{4} \right) = \sum_{k = 1}^{4}{E\left( X_{k} \right)} = \sum_{k = 1}^{4}\frac{3}{8} = \frac{3}{2}\).\par
记\(p_{k} = P(X = k)(k = 0,1,2,3)\).\par
如果甲得0分,则组合方式是唯一的:必定是甲出1,3,5,7分别对应乙出2,4,6,
8,
所以\(p_{0} = \frac{1}{\text{A}_{4}^{4}} = \frac{1}{24}\);\par
如果甲得3分,则组合方式也是唯一的:必定是甲出1,3,5,7分别对应乙出8,2,
4,6,
所以\(p_{3} = \frac{1}{\text{A}_{4}^{4}} = \frac{1}{24}\).\par
而\(X\)的所有可能取值是0,1,2,3,
故\(p_{0} + p_{1} + p_{2} + p_{3} = 1\),
\(p_{1} + 2p_{2} + 3p_{3} = E(X) = \frac{3}{2}\).\par
所以\(p_{1} + p_{2} + \frac{1}{12} = 1\),
\(p_{1} + 2p_{2} + \frac{1}{8} = \frac{3}{2}\),
两式相减即得\(p_{2} + \frac{1}{24} = \frac{1}{2}\),
故\(p_{2} + p_{3} = \frac{1}{2}\).\par
所以甲的总得分不小于2的概率为\(p_{2} + p_{3} = \frac{1}{2}\).\(\frac{1}{2}\).}
\end{question}

\section{解答题}

\begin{question}
记\(\triangle ABC\)的内角\(A\)、\(B\)、\(C\)的对边分别为\(a\),
\(b\),\(c\),已知\(\sin C = \sqrt{2}\cos B\),
\(a^{2} + b^{2} - c^{2} = \sqrt{2}ab\)
\begin{enumerate}[label=(\arabic*)]
  \item 求\(B\);
  \item 若\(\triangle ABC\)的面积为\(3 + \sqrt{3}\),求\(c\).
\end{enumerate}
\topics{已知两角的正;余弦,求和;差角的正弦;正弦定理解三角形;三角形面积公式及其应用;余弦定理解三角形}
\difficulty{0.65}
\answer{(1)\(B = \frac{\pi}{3}\)
(2)\(2\sqrt{2}\)}
\explain{(1)由余弦定理有\(a^{2} + b^{2} - c^{2} = 2ab\cos C\),
对比已知\(a^{2} + b^{2} - c^{2} = \sqrt{2}ab\),\par
可得\(\cos C = \frac{a^{2} + b^{2} - c^{2}}{2ab} = \frac{\sqrt{2}ab}{2ab} = \frac{\sqrt{2}}{2}\),\par
因为\(C \in \left( 0,\pi \right)\),
所以\(\sin C > 0\),\par
从而\(\sin C = \sqrt{1 - \cos^{2}C} = \sqrt{1 - \left( \frac{\sqrt{2}}{2} \right)^{2}} = \frac{\sqrt{2}}{2}\),\par
又因为\(\sin C = \sqrt{2}\cos B\),
即\(\cos B = \frac{1}{2}\),\par
注意到\(B \in \left( 0,\pi \right)\),\par
所以\(B = \frac{\pi}{3}\).\par
(2)由(1)可得\(B = \frac{\pi}{3}\),
\(\cos C = \frac{\sqrt{2}}{2}\),
\(C \in \left( 0,\pi \right)\),
从而\(C = \frac{\pi}{4}\),
\(A = \pi - \frac{\pi}{3} - \frac{\pi}{4} = \frac{5\pi}{12}\),\par
而\(\sin A = \sin\left( \frac{5\pi}{12} \right) = \sin\left( \frac{\pi}{4} + \frac{\pi}{6} \right) = \frac{\sqrt{2}}{2} \times \frac{\sqrt{3}}{2} + \frac{\sqrt{2}}{2} \times \frac{1}{2} = \frac{\sqrt{6} + \sqrt{2}}{4}\),\par
由正弦定理有\(\frac{a}{\sin\frac{5\pi}{12}} = \frac{b}{\sin\frac{\pi}{3}} = \frac{c}{\sin\frac{\pi}{4}}\),\par
从而\(a = \frac{\sqrt{6} + \sqrt{2}}{4} \cdot \sqrt{2}c = \frac{\sqrt{3} + 1}{2}c,b = \frac{\sqrt{3}}{2} \cdot \sqrt{2}c = \frac{\sqrt{6}}{2}c\),\par
由三角形面积公式可知,\(\triangle ABC\)的面积可表示为\par
\(S_{\triangle ABC} = \frac{1}{2}ab\sin C = \frac{1}{2} \cdot \frac{\sqrt{3} + 1}{2}c \cdot \frac{\sqrt{6}}{2}c \cdot \frac{\sqrt{2}}{2} = \frac{3 + \sqrt{3}}{8}c^{2}\),\par
由已知\(\triangle ABC\)的面积为\(3 + \sqrt{3}\),
可得\(\frac{3 + \sqrt{3}}{8}c^{2} = 3 + \sqrt{3}\),\par
所以\(c = 2\sqrt{2}\).}
\end{question}

\begin{question}
已知\(A(0,3)\)和\(P\left( 3,\frac{3}{2} \right)\)为椭圆\(C:\frac{x^{2}}{a^{2}} + \frac{y^{2}}{b^{2}} = 1(a > b > 0)\)上两点.
\begin{enumerate}[label=(\arabic*)]
  \item 求\(C\)的离心率;
  \item 若过\(P\)的直线\(l\)交\(C\)于另一点\(B\),且\(\triangle ABP\)的面积为9,求\(l\)的方程.
\end{enumerate}

\begin{center}
% IMAGE_TODO_START id=gaokao_2024_national_1-Q16-img1 path=image6.png width=60%
\includegraphics[width=0.4\textwidth]{content/exams/auto/gaokao_2024_national_1/images/media/image6.png}
% IMAGE_TODO_END id=gaokao_2024_national_1-Q16-img1
\end{center}

\topics{根据椭圆过的点求标准方程;求椭圆的离心率或离心率的取值范围;椭圆中三角形(四边形)的面积;根据韦达定理求参数}
\difficulty{0.65}
\answer{(1)\(\frac{1}{2}\)
(2)直线\(l\)的方程为\(3x - 2y - 6 = 0\)或\(x - 2y = 0\).}
\explain{(1)由题意得\(\left\{ \begin{array}{r}
b = 3 \\
\frac{9}{a^{2}} + \frac{\frac{9}{4}}{b^{2}} = 1
\end{array} \right.\),解得\(\left\{ \begin{array}{r}
b^{2} = 9 \\
a^{2} = 12
\end{array} \right.\),\par
所以\(e = \sqrt{1 - \frac{b^{2}}{a^{2}}} = \sqrt{1 - \frac{9}{12}} = \frac{1}{2}\).\par
(2)法一:\(k_{AP} = \frac{3 - \frac{3}{2}}{0 - 3} = - \frac{1}{2}\),则直线\(AP\)的方程为\(y = - \frac{1}{2}x + 3\),即\(x + 2y - 6 = 0\),\par
\(|AP| = \sqrt{(0 - 3)^{2} + \left( 3 - \frac{3}{2} \right)^{2}} = \frac{3\sqrt{5}}{2}\),由(1)知\(C:\frac{x^{2}}{12} + \frac{y^{2}}{9} = 1\),\par
设点\(B\)到直线\(AP\)的距离为\(d\),则\(d = \frac{2 \times 9}{\frac{3\sqrt{5}}{2}} = \frac{12\sqrt{5}}{5}\),\par
则将直线\(AP\)沿着与\(AP\)垂直的方向平移\(\frac{12\sqrt{5}}{5}\)单位即可,\par
此时该平行线与椭圆的交点即为点\(B\),\par
设该平行线的方程为:\(x + 2y + C = 0\),\par
则\(\frac{|C + 6|}{\sqrt{5}} = \frac{12\sqrt{5}}{5}\),解得\(C = 6\)或\(C = - 18\),\par
当\(C = 6\)时,联立\(\left\{ \begin{array}{r}
\frac{x^{2}}{12} + \frac{y^{2}}{9} = 1 \\
x + 2y + 6 = 0
\end{array} \right.\),解得\(\left\{ \begin{array}{r}
x = 0 \\
y = - 3
\end{array} \right.\)或\(\left\{ \begin{array}{r}
x = - 3 \\
y = - \frac{3}{2}
\end{array} \right.\),\par
即\(B(0, - 3)\)或\(\left( - 3, - \frac{3}{2} \right)\),\par
当\(B(0, - 3)\)时,此时\(k_{l} = \frac{3}{2}\),直线\(l\)的方程为\(y = \frac{3}{2}x - 3\),即\(3x - 2y - 6 = 0\),\par
当\(B\left( - 3, - \frac{3}{2} \right)\)时,此时\(k_{l} = \frac{1}{2}\),直线\(l\)的方程为\(y = \frac{1}{2}x\),即\(x - 2y = 0\),\par
当\(C = - 18\)时,联立\(\left\{ \begin{array}{r}
\frac{x^{2}}{12} + \frac{y^{2}}{9} = 1 \\
x + 2y - 18 = 0
\end{array} \right.\)得\(2y^{2} - 27y + 117 = 0\),\par
\(\Delta = 27^{2} - 4 \times 2 \times 117 = - 207 < 0\),此时该直线与椭圆无交点.\par
综上直线\(l\)的方程为\(3x - 2y - 6 = 0\)或\(x - 2y = 0\).\par
法二:同法一得到直线\(AP\)的方程为\(x + 2y - 6 = 0\),\par
点\(B\)到直线\(AP\)的距离\(d = \frac{12\sqrt{5}}{5}\),\par
设\(B\left( x_{0},y_{0} \right)\),则\(\left\{ \begin{array}{r}
\frac{\left| x_{0} + 2y_{0} - 6 \right|}{\sqrt{5}} = \frac{12\sqrt{5}}{5} \\
\frac{x_{0}^{2}}{12} + \frac{y_{0}^{2}}{9} = 1
\end{array} \right.\),解得\(\left\{ \begin{array}{r}
x_{0} = - 3 \\
y_{0} = - \frac{3}{2}
\end{array} \right.\)或\(\left\{ \begin{array}{r}
x_{0} = 0 \\
y_{0} = - 3
\end{array} \right.\),\par
即\(B(0, - 3)\)或\(\left( - 3, - \frac{3}{2} \right)\),以下同法一.\par
法三:同法一得到直线\(AP\)的方程为\(x + 2y - 6 = 0\),\par
点\(B\)到直线\(AP\)的距离\(d = \frac{12\sqrt{5}}{5}\),\par
设\(B\left( 2\sqrt{3}\cos\theta,3\sin\theta \right)\),其中\(\theta \in \lbrack 0,2\pi)\),则有\(\frac{\left| 2\sqrt{3}\cos\theta + 6\sin\theta - 6 \right|}{\sqrt{5}} = \frac{12\sqrt{5}}{5}\),\par
联立\(\cos^{2}\theta + \sin^{2}\theta = 1\),解得\(\left\{ \begin{array}{r}
\cos\theta = - \frac{\sqrt{3}}{2} \\
\sin\theta = - \frac{1}{2}
\end{array} \right.\)或\(\left\{ \begin{array}{r}
\cos\theta = 0 \\
\sin\theta = - 1
\end{array} \right.\),\par
即\(B(0, - 3)\)或\(\left( - 3, - \frac{3}{2} \right)\),以下同法一;\par
法四:当直线\(AB\)的斜率不存在时,此时\(B(0, - 3)\),\par
\(S_{\triangle PAB} = \frac{1}{2} \times 6 \times 3 = 9\),符合题意,此时\(k_{l} = \frac{3}{2}\),直线\(l\)的方程为\(y = \frac{3}{2}x - 3\),即\(3x - 2y - 6 = 0\),\par
当线\(AB\)的斜率存在时,设直线\(AB\)的方程为\(y = kx + 3\),\par
联立椭圆方程有\(\left\{ \begin{array}{r}
y = kx + 3 \\
\frac{x^{2}}{12} + \frac{y^{2}}{9} = 1
\end{array} \right.\),则\(\left( 4k^{2} + 3 \right)x^{2} + 24kx = 0\),其中\(k \neq k_{AP}\),即\(k \neq - \frac{1}{2}\),\par
解得\(x = 0\)或\(x = \frac{- 24k}{4k^{2} + 3}\),\(k \neq 0\),\(k \neq - \frac{1}{2}\),\par
令\(x = \frac{- 24k}{4k^{2} + 3}\),则\(y = \frac{- 12k^{2} + 9}{4k^{2} + 3}\),则\(B\left( \frac{- 24k}{4k^{2} + 3},\frac{- 12k^{2} + 9}{4k^{2} + 3} \right)\)\par
同法一得到直线\(AP\)的方程为\(x + 2y - 6 = 0\),\par
点\(B\)到直线\(AP\)的距离\(d = \frac{12\sqrt{5}}{5}\),\par
则\(\frac{\left| \frac{- 24k}{4k^{2} + 3} + 2 \times \frac{- 12k^{2} + 9}{4k^{2} + 3} - 6 \right|}{\sqrt{5}} = \frac{12\sqrt{5}}{5}\),解得\(k = \frac{3}{2}\),\par
此时\(B\left( - 3, - \frac{3}{2} \right)\),则得到此时\(k_{l} = \frac{1}{2}\),直线\(l\)的方程为\(y = \frac{1}{2}x\),即\(x - 2y = 0\),\par
综上直线\(l\)的方程为\(3x - 2y - 6 = 0\)或\(x - 2y = 0\).\par
法五:当\(l\)的斜率不存在时,\(l:x = 3\text{,}B\left( 3, - \frac{3}{2} \right)\text{,}|PB| = 3\text{,}A\)到\(PB\)距离\(d = 3\),\par
此时\(S_{\triangle ABP} = \frac{1}{2} \times 3 \times 3 = \frac{9}{2} \neq 9\)不满足条件.\par
当\(l\)的斜率存在时,设\(PB:y - \frac{3}{2} = k(x - 3)\),令\(P\left( x_{1},y_{1} \right)\text{,}B\left( x_{2},y_{2} \right)\),\par
\(\left\{ \begin{array}{r}
y = k(x - 3) + \frac{3}{2} \\
\frac{x^{2}}{12} + \frac{y^{2}}{9} = 1
\end{array} \right.\),消\(y\)可得\(\left( 4k^{2} + 3 \right)x^{2} - \left( 24k^{2} - 12k \right)x + 36k^{2} - 36k - 27 = 0\),\par
\(\Delta = \left( 24k^{2} - 12k \right)^{2} - 4\left( 4k^{2} + 3 \right)\left( 36k^{2} - 36k - 27 \right) > 0\),且\(k \neq k_{AP}\),即\(k \neq - \frac{1}{2}\),\par
\(\left\{ \begin{array}{r}
x_{1} + x_{2} = \frac{24k^{2} - 12k}{4k^{2} + 3} \\
x_{1}x_{2} = \frac{36k^{2} - 36k - 27}{4k^{2} + 3}
\end{array} \right.\),\(|PB| = \sqrt{k^{2} + 1}\sqrt{\left( x_{1} + x_{2} \right)^{2} - 4x_{1}x_{2}} = \frac{4\sqrt{3}\sqrt{k^{2} + 1}\sqrt{3k^{2} + 9k + \frac{27}{4}}}{4k^{2} + 3}\),\(A\)到直线\(PB\)距离\(d = \frac{\left| 3k + \frac{3}{2} \right|}{\sqrt{k^{2} + 1}}\),\(S_{\triangle PAB} = \frac{1}{2} \cdot \frac{4\sqrt{3}\sqrt{k^{2} + 1}\sqrt{3k^{2} + 9k + \frac{27}{4}}}{4k^{2} + 3} \cdot \frac{\left| 3k + \frac{3}{2} \right|}{\sqrt{k^{2} + 1}} = 9\),\(\therefore k = \frac{1}{2}\)或\(\frac{3}{2}\),均满足题意,\(\therefore l:y = \frac{1}{2}x\)或\(y = \frac{3}{2}x - 3\),即\(3x - 2y - 6 = 0\)或\(x - 2y = 0\)。\par
法六:当\(l\)的斜率不存在时,\(l:x = 3\),\(B\left( 3, - \frac{3}{2} \right)\),\(|PB| = 3\),\(A\)到\(PB\)距离\(d = 3\),\par
此时\(S_{\triangle ABP} = \frac{1}{2} \times 3 \times 3 = \frac{9}{2} \neq 9\)不满足条件。\par
当直线\(l\)斜率存在时,设\(l:y = k(x - 3) + \frac{3}{2}\),\par
设\(l\)与\(y\)轴的交点为\(Q\),令\(x = 0\),则\(Q\left( 0, - 3k + \frac{3}{2} \right)\),\par
联立\(\left\{ \begin{array}{r}
y = kx - 3k + \frac{3}{2} \\
3x^{2} + 4y^{2} = 36
\end{array} \right.\),则有\(\left( 3 + 4k^{2} \right)x^{2} - 8k\left( 3k - \frac{3}{2} \right)x + 36k^{2} - 36k - 27 = 0\),\par
\(\left( 3 + 4k^{2} \right)x^{2} - 8k\left( 3k - \frac{3}{2} \right)x + 36k^{2} - 36k - 27 = 0\),\par
其中\(\Delta = 8k^{2}\left( 3k - \frac{3}{2} \right)^{2} - 4\left( 3 + 4k^{2} \right)\left( 36k^{2} - 36k - 27 \right) > 0\),且\(k \neq - \frac{1}{2}\),\par
则\(3x_{B} = \frac{36k^{2} - 36k - 27}{3 + 4k^{2}},x_{B} = \frac{12k^{2} - 12k - 9}{3 + 4k^{2}}\),\par
则\(S = \frac{1}{2}|AQ|\left| x_{P} - x_{B} \right| = \frac{1}{2}\left| 3k + \frac{3}{2} \right|\left| \frac{12k + 18}{3 + 4k^{2}} \right| = 9\),解得\(k = \frac{1}{2}\)或\(k = \frac{3}{2}\),经代入判别式验证均满足题意.\par
则直线\(l\)为\(y = \frac{1}{2}x\)或\(y = \frac{3}{2}x - 3\),即\(3x - 2y - 6 = 0\)或\(x - 2y = 0\).}
\end{question}

\begin{question}
如图,四棱锥\(P - ABCD\)中,\(PA\bot\)底面\(ABCD\),
\(PA = AC = 2\),
\(BC = 1,AB = \sqrt{3}\).
\begin{enumerate}[label=(\arabic*)]
  \item 若\(AD\bot PB\),证明:\(AD\text{//}\)平面\(PBC\);
  \item 若\(AD\bot DC\),且二面角\(A - CP - D\)的正弦值为\(\frac{\sqrt{42}}{7}\),求\(AD\).
\end{enumerate}

\begin{center}
% IMAGE_TODO_START id=gaokao_2024_national_1-Q17-img1 path=image7.png width=60%
\includegraphics[width=0.35\textwidth]{content/exams/auto/gaokao_2024_national_1/images/media/image7.png}
% IMAGE_TODO_END id=gaokao_2024_national_1-Q17-img1
\end{center}


\begin{center}
% IMAGE_TODO_START id=gaokao_2024_national_1-Q17-img2 path=image8.png width=60%
\includegraphics[width=0.35\textwidth]{content/exams/auto/gaokao_2024_national_1/images/media/image8.png}
% IMAGE_TODO_END id=gaokao_2024_national_1-Q17-img2
\end{center}

\topics{证明线面平行;证明面面垂直;由二面角大小求线段长度或距离}
\difficulty{0.65}
\answer{(1)证明见解析
(2)\(\sqrt{3}\)}
\explain{(1)因为\(PA\bot\)平面\(ABCD\),
而\(AD \subset\)平面\(ABCD\),所以\(PA\bot AD\),\par
又\(AD\bot PB\),\(PB \cap PA = P\),
\(PB,PA \subset\)平面\(PAB\),
所以\(AD\bot\)平面\(PAB\),\par
而\(AB \subset\)平面\(PAB\),
所以\(AD\bot AB\)。\par
因为\(BC^{2} + AB^{2} = AC^{2}\),
所以\(BC\bot AB\),
根据平面知识可知\(AD//BC\),\par
又\(AD ⊄\)平面\(PBC\),
\(BC \subset\)平面\(PBC\),所以\(AD//\)平面\(PBC\)。\par
(2)如图所示,过点\(D\)作\(DE\bot AC\)于\(E\),
再过点\(E\)作\(EF\bot CP\)于\(F\),连接\(DF\),\par
因为\(PA\bot\)平面\(ABCD\),
所以平面\(PAC\bot\)平面\(ABCD\),
而平面\(PAC \cap\)平面\(ABCD = AC\),\par
所以\(DE\bot\)平面\(PAC\),又\(EF\bot CP\),
所以\(CP\bot\)平面\(DEF\),\par
根据二面角的定义可知,
\(\angle DFE\)即为二面角\(A - CP - D\)的平面角,\par
即\(\sin\angle DFE = \frac{\sqrt{42}}{7}\),
即\(\tan\angle DFE = \sqrt{6}\)。\par
因为\(AD\bot DC\),设\(AD = x\),
则\(CD = \sqrt{4 - x^{2}}\),由等面积法可得,
\(DE = \frac{x\sqrt{4 - x^{2}}}{2}\),\par
又\(CE = \sqrt{\left( 4 - x^{2} \right) - \frac{x^{2}\left( 4 - x^{2} \right)}{4}} = \frac{4 - x^{2}}{2}\),
而\(\triangle EFC\)为等腰直角三角形,
所以\(EF = \frac{4 - x^{2}}{2\sqrt{2}}\),\par
故\(\tan\angle DFE = \frac{\frac{x\sqrt{4 - x^{2}}}{2}}{\frac{4 - x^{2}}{2\sqrt{2}}} = \sqrt{6}\),
解得\(x = \sqrt{3}\),即\(AD = \sqrt{3}\)。}
\end{question}

\begin{question}
已知函数\(f(x) = \ln\frac{x}{2 - x} + ax + b{(x - 1)}^{3}\)
\begin{enumerate}[label=(\arabic*)]
  \item 若\(b = 0\),且\(f'(x) \geq 0\),求\(a\)的最小值;
  \item 证明:曲线\(y = f(x)\)是中心对称图形;
  \item 若\(f(x) > - 2\)当且仅当\(1 < x < 2\),求\(b\)的取值范围。
\end{enumerate}
\topics{判断或证明函数的对称性;简单复合函数的导数;利用导数证明不等式;利用导数研究不等式恒成立问题}
\difficulty{0.4}
\answer{(1)\(-2\)
(2)证明见解析
(3)\(b \geq - \frac{2}{3}\)}
\explain{(1)\(b = 0\)时,
\(f(x) = \ln\frac{x}{2 - x} + ax\),
其中\(x \in (0,2)\),\par
则\(f'(x) = \frac{1}{x} + \frac{1}{2 - x} + a = \frac{2}{x(2 - x)} + a,x \in (0,2)\),\par
因为\(x(2 - x) \leq \left( \frac{2 - x + x}{2} \right)^{2} = 1\),
当且仅当\(x = 1\)时等号成立,\par
故\(f'(x)_{\min} = 2 + a\),
而\(f'(x) \geq 0\)成立,
故\(a + 2 \geq 0\)即\(a \geq - 2\),\par
所以\(a\)的最小值为\(- 2\)。\par
(2)\(f(x) = \ln\frac{x}{2 - x} + ax + b(x - 1)^{3}\)的定义域为\((0,2)\),\par
设\(P(m,n)\)为\(y = f(x)\)图象上任意一点,\par
\(P(m,n)\)关于\((1,a)\)的对称点为\(Q(2 - m,2a - n)\),\par
因为\(P(m,n)\)在\(y = f(x)\)图象上,
故\(n = \ln\frac{m}{2 - m} + am + b(m - 1)^{3}\),\par
而\(f(2 - m) = \ln\frac{2 - m}{m} + a(2 - m) + b(2 - m - 1)^{3} = - \left\lbrack \ln\frac{m}{2 - m} + am + b(m - 1)^{3} \right\rbrack + 2a\),\par
\(= - n + 2a\),\par
所以\(Q(2 - m,2a - n)\)也在\(y = f(x)\)图象上,\par
由\(P\)的任意性可得\(y = f(x)\)图象为中心对称图形,
且对称中心为\((1,a)\)。\par
(3)因为\(f(x) > - 2\)当且仅当\(1 < x < 2\),
故\(x = 1\)为\(f(x) = - 2\)的一个解,\par
所以\(f(1) = - 2\)即\(a = - 2\),\par
先考虑\(1 < x < 2\)时,\(f(x) > - 2\)恒成立。\par
此时\(f(x) > - 2\)即为\(\ln\frac{x}{2 - x} + 2(1 - x) + b(x - 1)^{3} > 0\)在\((1,2)\)上恒成立,\par
设\(t = x - 1 \in (0,1)\),
则\(\ln\frac{t + 1}{1 - t} - 2t + bt^{3} > 0\)在\((0,1)\)上恒成立,\par
设\(g(t) = \ln\frac{t + 1}{1 - t} - 2t + bt^{3},t \in (0,1)\),\par
则\(g'(t) = \frac{2}{1 - t^{2}} - 2 + 3bt^{2} = \frac{t^{2}\left( - 3bt^{2} + 2 + 3b \right)}{1 - t^{2}}\),\par
当\(b \geq 0\),
\(- 3bt^{2} + 2 + 3b \geq - 3b + 2 + 3b = 2 > 0\),\par
故\(g'(t) > 0\)恒成立,
故\(g(t)\)在\((0,1)\)上为增函数,\par
故\(g(t) > g(0) = 0\)即\(f(x) > - 2\)在\((1,2)\)上恒成立。\par
当\(- \frac{2}{3} \leq b < 0\)时,
\(- 3bt^{2} + 2 + 3b \geq 2 + 3b \geq 0\),\par
故\(g'(t) \geq 0\)恒成立,
故\(g(t)\)在\((0,1)\)上为增函数,\par
故\(g(t) > g(0) = 0\)即\(f(x) > - 2\)在\((1,2)\)上恒成立。\par
当\(b < - \frac{2}{3}\),
则当\(0 < t < \sqrt{1 + \frac{2}{3b}} < 1\)时,\(g'(t) < 0\)\par
故在\(\left( 0,\sqrt{1 + \frac{2}{3b}} \right)\)上\(g(t)\)为减函数,
故\(g(t) < g(0) = 0\),不合题意,舍。\par
综上,
\(f(x) > - 2\)在\((1,2)\)上恒成立时\(b \geq - \frac{2}{3}\)。\par
而当\(b \geq - \frac{2}{3}\)时,
由上述过程可得\(g(t)\)在\((0,1)\)递增,
故\(g(t) > 0\)的解为\((0,1)\),\par
即\(f(x) > - 2\)的解为\((1,2)\)。\par
综上,\(b \geq - \frac{2}{3}\)。}
\end{question}

\begin{question}
设\(m\)为正整数,数列\(a_{1},a_{2},...,a_{4m + 2}\)是公差不为0的等差数列,若从中删去两项\(a_{i}\)和\(a_{j}(i < j)\)后剩余的\(4m\)项可被平均分为\(m\)组,且每组的4个数都能构成等差数列,则称数列\(a_{1},a_{2},...,a_{4m + 2}\)是\((i,j) -\)可分数列。
\begin{enumerate}[label=(\arabic*)]
  \item 写出所有的\((i,j)\),\(1 \leq i < j \leq 6\),使数列\(a_{1},a_{2},...,a_{6}\)是\((i,j) -\)可分数列;
  \item 当\(m \geq 3\)时,证明:数列\(a_{1},a_{2},...,a_{4m + 2}\)是\((2,13) -\)可分数列;
  \item 从\(1,2,...,4m + 2\)中任取两个数\(i\)和\(j(i < j)\),记数列\(a_{1},a_{2},...,a_{4m + 2}\)是\((i,j) -\)可分数列的概率为\(P_{m}\),证明:\(P_{m} > \frac{1}{8}\)。
\end{enumerate}
\topics{等差数列通项公式的基本量计算;数列新定义}
\difficulty{0.15}
\answer{(1)\((1,2),(1,6),(5,6)\)
(2)证明见解析
(3)证明见解析}
\explain{(1)首先,
我们设数列\(a_{1},a_{2},...,a_{4m + 2}\)的公差为\(d\),
则\(d \neq 0\)。\par
由于一个数列同时加上一个数或者乘以一个非零数后是等差数列,
当且仅当该数列是等差数列,\par
故我们可以对该数列进行适当的变形\({a'}_{k} = \frac{a_{k} - a_{1}}{d} + 1(k = 1,2,...,4m + 2)\),\par
得到新数列\({a'}_{k} = k(k = 1,2,...,4m + 2)\),
然后对\({a'}_{1},{a'}_{2},...,{a'}_{4m + 2}\)进行相应的讨论即可。\par
换言之,
我们可以不妨设\(a_{k} = k(k = 1,2,...,4m + 2)\),
此后的讨论均建立在该假设下进行。\par
回到原题,
第1小问相当于从\(1,2,3,4,5,6\)中取出两个数\(i\)和\(j(i < j)\),
使得剩下四个数是等差数列。\par
那么剩下四个数只可能是\(1,2,3,4\),或\(2,3,4,5\),
或\(3,4,5,6\)。\par
所以所有可能的\((i,j)\)就是\((1,2),(1,6),(5,6)\)。\par
(2)由于从数列\(1,2,...,4m + 2\)中取出\(2\)和\(13\)后,
剩余的\(4m\)个数可以分为以下两个部分,共\(m\)组,使得每组成等差数列:\par
①\(\left\{ 1,4,7,10 \right\},\left\{ 3,6,9,12 \right\},\left\{ 5,8,11,14 \right\}\),
共\(3\)组;\par
②\(\left\{ 15,16,17,18 \right\},\left\{ 19,20,21,22 \right\},...,\left\{ 4m - 1,4m,4m + 1,4m + 2 \right\}\),
共\(m - 3\)组.\par
(如果\(m - 3 = 0\),则忽略②)\par
故数列\(1,2,...,4m + 2\)是\((2,13) -\)可分数列.\par
(3)定义集合\(A = \left\{ \left. \ 4k + 1 \right|k = 0,1,2,...,m \right\} = \left\{ 1,5,9,13,...,4m + 1 \right\}\),
\(B = \left\{ \left. \ 4k + 2 \right|k = 0,1,2,...,m \right\} = \left\{ 2,6,10,14,...,4m + 2 \right\}\).\par
下面证明,对\(1 \leq i < j \leq 4m + 2\),
如果下面两个命题同时成立,\par
则数列\(1,2,...,4m + 2\)一定是\((i,j) -\)可分数列:\par
命题1:\(i \in A,j \in B\)或\(i \in B,j \in A\);\par
命题2:\(j - i \neq 3\).\par
我们分两种情况证明这个结论.\par
第一种情况:如果\(i \in A,j \in B\),
且\(j - i \neq 3\).\par
此时设\(i = 4k_{1} + 1\),\(j = 4k_{2} + 2\),
\(k_{1},k_{2} \in \left\{ 0,1,2,...,m \right\}\).\par
则由\(i < j\)可知\(4k_{1} + 1 < 4k_{2} + 2\),
即\(k_{2} - k_{1} > - \frac{1}{4}\),
故\(k_{2} \geq k_{1}\).\par
此时,
由于从数列\(1,2,...,4m + 2\)中取出\(i = 4k_{1} + 1\)和\(j = 4k_{2} + 2\)后,\par
剩余的\(4m\)个数可以分为以下三个部分,共\(m\)组,使得每组成等差数列:\par
①\(\left\{ 1,2,3,4 \right\},\left\{ 5,6,7,8 \right\},...,\left\{ 4k_{1} - 3,4k_{1} - 2,4k_{1} - 1,4k_{1} \right\}\),共\(k_{1}\)组;\par
②\(\left\{ 4k_{1} + 2,4k_{1} + 3,4k_{1} + 4,4k_{1} + 5 \right\},\)\\
\(\left\{ 4k_{1} + 6,4k_{1} + 7,4k_{1} + 8,4k_{1} + 9 \right\},...,\left\{ 4k_{2} - 2,4k_{2} - 1,4k_{2},4k_{2} + 1 \right\}\),共\(k_{2} - k_{1}\)组;\par
③\(\left\{ 4k_{2} + 3,4k_{2} + 4,4k_{2} + 5,4k_{2} + 6 \right\},\)\\
\(\left\{ 4k_{2} + 7,4k_{2} + 8,4k_{2} + 9,4k_{2} + 10 \right\},...,\left\{ 4m - 1,4m,4m + 1,4m + 2 \right\}\),共\(m - k_{2}\)组.\par
(如果某一部分的组数为\(0\),则忽略之)\par
故此时数列\(1,2,...,4m + 2\)是\((i,j) -\)可分数列.\par
第二种情况:如果\(i \in B,j \in A\),
且\(j - i \neq 3\).\par
此时设\(i = 4k_{1} + 2\),\(j = 4k_{2} + 1\),
\(k_{1},k_{2} \in \left\{ 0,1,2,...,m \right\}\).\par
则由\(i < j\)可知\(4k_{1} + 2 < 4k_{2} + 1\),
即\(k_{2} - k_{1} > \frac{1}{4}\),
故\(k_{2} > k_{1}\).\par
由于\(j - i \neq 3\),
故\(\left( 4k_{2} + 1 \right) - \left( 4k_{1} + 2 \right) \neq 3\),
从而\(k_{2} - k_{1} \neq 1\),
这就意味着\(k_{2} - k_{1} \geq 2\).\par
此时,
由于从数列\(1,2,...,4m + 2\)中取出\(i = 4k_{1} + 2\)和\(j = 4k_{2} + 1\)后,
剩余的\(4m\)个数可以分为以下四个部分,共\(m\)组,使得每组成等差数列:\par
①\(\left\{ 1,2,3,4 \right\},\left\{ 5,6,7,8 \right\},...,\left\{ 4k_{1} - 3,4k_{1} - 2,4k_{1} - 1,4k_{1} \right\}\),共\(k_{1}\)组;\par
②\(\left\{ 4k_{1} + 1,3k_{1} + k_{2} + 1,2k_{1} + 2k_{2} + 1,k_{1} + 3k_{2} + 1 \right\}\),\\
\(\left\{ 3k_{1} + k_{2} + 2,2k_{1} + 2k_{2} + 2,k_{1} + 3k_{2} + 2,4k_{2} + 2 \right\}\),共\(2\)组;\par
③全体\(\left\{ 4k_{1} + p,3k_{1} + k_{2} + p,2k_{1} + 2k_{2} + p,k_{1} + 3k_{2} + p \right\}\),其中\(p = 3,4,...,k_{2} - k_{1}\),共\(k_{2} - k_{1} - 2\)组;\par
④\(\left\{ 4k_{2} + 3,4k_{2} + 4,4k_{2} + 5,4k_{2} + 6 \right\},\)\\
\(\left\{ 4k_{2} + 7,4k_{2} + 8,4k_{2} + 9,4k_{2} + 10 \right\},...,\left\{ 4m - 1,4m,4m + 1,4m + 2 \right\}\),共\(m - k_{2}\)组.\par
(如果某一部分的组数为\(0\),则忽略之)\par
这里对②和③进行一下解释:将③中的每一组作为一个横排,
排成一个包含\(k_{2} - k_{1} - 2\)个行,
\(4\)个列的数表以后,\(4\)个列分别是下面这些数:\par
\(\left\{ 4k_{1} + 3,4k_{1} + 4,...,3k_{1} + k_{2} \right\}\),
\(\left\{ 3k_{1} + k_{2} + 3,3k_{1} + k_{2} + 4,...,2k_{1} + 2k_{2} \right\}\),
\(\left\{ 2k_{1} + 2k_{2} + 3,2k_{1} + 2k_{2} + 3,...,k_{1} + 3k_{2} \right\}\),
\(\left\{ k_{1} + 3k_{2} + 3,k_{1} + 3k_{2} + 4,...,4k_{2} \right\}\).\par
可以看出每列都是连续的若干个整数,它们再取并以后,
将取遍\(\left\{ 4k_{1} + 1,4k_{1} + 2,...,4k_{2} + 2 \right\}\)中除开五个集合\(\left\{ 4k_{1} + 1,4k_{1} + 2 \right\}\),
\(\left\{ 3k_{1} + k_{2} + 1,3k_{1} + k_{2} + 2 \right\}\),
\(\left\{ 2k_{1} + 2k_{2} + 1,2k_{1} + 2k_{2} + 2 \right\}\),
\(\left\{ k_{1} + 3k_{2} + 1,k_{1} + 3k_{2} + 2 \right\}\),
\(\left\{ 4k_{2} + 1,4k_{2} + 2 \right\}\)中的十个元素以外的所有数.\par
而这十个数中,
除开已经去掉的\(4k_{1} + 2\)和\(4k_{2} + 1\)以外,
剩余的八个数恰好就是②中出现的八个数.\par
这就说明我们给出的分组方式满足要求,
故此时数列\(1,2,...,4m + 2\)是\((i,j) -\)可分数列.\par
至此,我们证明了:对\(1 \leq i < j \leq 4m + 2\),
如果前述命题1和命题2同时成立,
则数列\(1,2,...,4m + 2\)一定是\((i,j) -\)可分数列.\par
然后我们来考虑这样的\((i,j)\)的个数.\par
首先,由于\(A \cap B = \varnothing\),
\(A\)和\(B\)各有\(m + 1\)个元素,
故满足命题1的\((i,j)\)总共有\((m + 1)^{2}\)个;\par
而如果\(j - i = 3\),假设\(i \in A,j \in B\),
则可设\(i = 4k_{1} + 1\),\(j = 4k_{2} + 2\),
代入得\(\left( 4k_{2} + 2 \right) - \left( 4k_{1} + 1 \right) = 3\).\par
但这导致\(k_{2} - k_{1} = \frac{1}{2}\),矛盾,
所以\(i \in B,j \in A\).\par
设\(i = 4k_{1} + 2\),\(j = 4k_{2} + 1\),
\(k_{1},k_{2} \in \left\{ 0,1,2,...,m \right\}\),
则\(\left( 4k_{2} + 1 \right) - \left( 4k_{1} + 2 \right) = 3\),
即\(k_{2} - k_{1} = 1\).\par
所以可能的\(\left( k_{1},k_{2} \right)\)恰好就是\((0,1),(1,2),...,(m - 1,m)\),
对应的\((i,j)\)分别是\((2,5),(6,9),...,(4m - 2,4m + 1)\),
总共\(m\)个.\par
所以这\((m + 1)^{2}\)个满足命题1的\((i,j)\)中,
不满足命题2的恰好有\(m\)个.\par
这就得到同时满足命题1和命题2的\((i,j)\)的个数为\((m + 1)^{2} - m\).\par
当我们从\(1,2,...,4m + 2\)中一次任取两个数\(i\)和\(j(i < j)\)时,
总的选取方式的个数等于\(\frac{(4m + 2)(4m + 1)}{2} = (2m + 1)(4m + 1)\).\par
而根据之前的结论,
使得数列\(a_{1},a_{2},...,a_{4m + 2}\)是\((i,j) -\)可分数列的\((i,j)\)至少有\((m + 1)^{2} - m\)个.\par
所以数列\(a_{1},a_{2},...,a_{4m + 2}\)是\((i,j) -\)可分数列的概率\(P_{m}\)一定满足\par
\(P_{m} \geq \frac{(m + 1)^{2} - m}{(2m + 1)(4m + 1)} = \frac{m^{2} + m + 1}{(2m + 1)(4m + 1)} > \frac{m^{2} + m + \frac{1}{4}}{(2m + 1)(4m + 2)} = \frac{\left( m + \frac{1}{2} \right)^{2}}{2(2m + 1)(2m + 1)} = \frac{1}{8}\).\par
这就证明了结论.}
\end{question}
