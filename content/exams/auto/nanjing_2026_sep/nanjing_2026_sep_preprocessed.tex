\examxtitle{南京2026年9月试卷}

\section{单选题}

\begin{question}
已知\(\text{i}\)是虚数单位,则复数\(z = \frac{1 + \text{i}}{\text{i}}\)的实部为(  )
\begin{choices}
  \item \(- 1\)
  \item 1
  \item \(- \text{i}\)
  \item i
\end{choices}
\topics{求复数的实部与虚部;复数的除法运算}
\difficulty{0.94}
\answer{B}
\explain{复数\(z = \frac{(1 + \text{i}) \cdot ( - \text{i})}{\text{i} \cdot ( - \text{i})} = 1 - \text{i}\),所以\(z\)的实部为1}
\end{question}

\begin{question}
有一组样本数据1,2,2,2,3,5,去掉1和5后,相较于原数据不变的是(  )
\begin{choices}
  \item 平均数
  \item 极差
  \item 方差
  \item 中位数
\end{choices}
\topics{计算几个数的中位数;计算几个数的平均数;计算几个数据的极差;方差;标准差}
\difficulty{0.94}
\answer{D}
\explain{样本数据1,2,2,2,3,5的平均数为\(\frac{5}{2}\),极差为4,中位数为2,
去掉1和5后的数据的平均数为\(\frac{9}{4}\),极差为1,中位数为2,故平均数和极差都发生变化,中位数不改变,
由于去掉1和5后,数据的波动性更小,故相比较于原数据,方差变小,故ABC错误,D正确}
\end{question}

\begin{question}
已知\(a \in \text{R}\),若集合\(M = \left\{ 0,a \right\},N = \left\{ - 2,0,1 \right\}\),则"\(a = 1\)"是"\(M \subseteq N\)"的(  )
\begin{choices}
  \item 充分且不必要条件
  \item 必要且不充分条件
  \item 充要条件
  \item 既不充分又不必要条件
\end{choices}
\topics{根据集合的包含关系求参数;判断命题的充分不必要条件}
\difficulty{0.94}
\answer{A}
\explain{由\(M = \{ 0,a\}\),得\(a \neq 0\),而\(N = \{ - 2,0,1\},M \subseteq N\),因此\(a = - 2\)或\(a = 1\),
所以"\(a = 1\)"是"\(M \subseteq N\)"的充分且不必要条件}
\end{question}

\begin{question}
\((x - 1)^{6}\)的展开式中\(x^{2}\)的系数为(  )
\begin{choices}
  \item -20
  \item -15
  \item 15
  \item 20
\end{choices}
\topics{求指定项的系数}
\difficulty{0.94}
\answer{C}
\explain{\((x - 1)^{6}\)展开式的通项为\(T_{r + 1} = \text{C}_{6}^{r}x^{6 - r}( - 1)^{r}\),
取\(r = 4\),得\(T_{5} = \text{C}_{6}^{4}x^{2}( - 1)^{4} = 15x^{2}\).
即\(x^{2}\)的系数为\(15\)}
\end{question}

\begin{question}
要得到函数\(y = 3 \cdot 2^{x}\)的图象,只需将函数\(y = 2^{x}\)的图象(   )
\begin{choices}
  \item 向左平移\(\text{log}_{2}3\)个单位长度
  \item 向右平移\(\text{log}_{2}3\)个单位长度
  \item 向上平移\(\text{log}_{2}3\)个单位长度
  \item 向下平移\(\text{log}_{2}3\)个单位长度
\end{choices}
\topics{函数图象的变换}
\difficulty{0.85}
\answer{A}
\explain{由题可得:\(y = 3 \cdot 2^{x} = 2^{x + \log_{2}3}\),所以只需将函数\(y = 2^{x}\)的图象向左平移\(\text{log}_{2}3\)个单位长度得到函数\(y = 3 \cdot 2^{x}\)的图象}
\end{question}

\begin{question}
设等比数列\(\left\{ a_{n} \right\}\)的前\(n\)项和为\(S_{n}\),若\(S_{2} = 2,S_{4} = 6\),则\(a_{5} + a_{6} =\)(  )
\begin{choices}
  \item 8
  \item 10
  \item 14
  \item 18
\end{choices}
\topics{等比数列片段和性质及应用}
\difficulty{0.85}
\answer{A}
\explain{等比数列\(\left\{ a_{n} \right\}\)中,\(S_{2},S_{4} - S_{2},S_{6} - S_{4}\)成等比数列,
\(\therefore 2,4,S_{6} - S_{4}\)成等比数列,
\(\therefore S_{6} - S_{4} = 8 = a_{5} + a_{6}\)}
\end{question}

\begin{question}
已知点\(A( - 1,1),B(3,3)\),线段\(AB\)为\(\odot M\)的一条直径.设过点\(C(2, - 1)\)且与\(\odot M\)相切的两条直线的斜率分别为\(k_{1},k_{2}\),则\(k_{1} + k_{2} =\)(  )
\begin{choices}
  \item \(- \frac{3}{2}\)
  \item \(- \frac{2}{3}\)
  \item \(\frac{2}{3}\)
  \item \(\frac{3}{2}\)
\end{choices}
\topics{过圆外一点的圆的切线方程}
\difficulty{0.65}
\answer{D}
\explain{由于点\(A( - 1,1),B(3,3)\),线段\(AB\)为\(\odot M\)的一条直径,故圆心\(M\left( \frac{- 1 + 3}{2},\frac{1 + 3}{2} \right)\),即\(M(1,2)\),圆的半径为\(r = \frac{1}{2}|AB| = \frac{1}{2}\sqrt{( - 1 - 3)^{2} + (1 - 3)^{2}} = \sqrt{5}\),
由题意可知两条切线的斜率均存在,故设切线方程为\(y = k(x - 2) - 1\),
由相切可得\(\frac{| - k - 1 - 2|}{\sqrt{1 + k^{2}}} = r = \sqrt{5}\),化简可得\(2k^{2} - 3k - 2 = 0\),
故\(k_{1},k_{2}\)是方程\(2k^{2} - 3k - 2 = 0\)的两个根,故\(k_{1} + k_{2} =\)
\(\frac{3}{2}\)}
\end{question}

\begin{question}
数学中有许多形状优美、寓意美好的曲线,如星形线等.某星形线如图所示,已知该曲线上一点\(P\left( x_{0},y_{0} \right)\)的坐标可以表示为\(\left( a\text{cos}^{3}\theta ,a\text{sin}^{3}\theta \right)(a > 0)\),若\(x_{0}y_{0} = \frac{8a^{2}}{125}\),且\(x_{0} + y_{0} = \frac{9}{5}\),则\(a =\)(  )


\begin{center}
\begin{tikzpicture}[scale=1.05,>=Stealth,line cap=round,line join=round]
  \node[draw, minimum width=6cm, minimum height=4cm] {图略(图 ID: 74994822-7a76-4465-b523-6b19e9aef9ba)};
\end{tikzpicture}
\end{center}

\begin{choices}
  \item \(\sqrt{2}\)
  \item \(\sqrt{3}\)
  \item 2
  \item \(\sqrt{5}\)
\end{choices}
\topics{sinα±cosα和sinα·cosα的关系;由方程研究曲线的性质}
\difficulty{0.65}
\answer{D}
\explain{\(\because x_{0}y_{0} = \frac{8a^{2}}{125}\),\(x_{0} + y_{0} = \frac{9}{5}\),
\(\therefore x_{0} > 0,y_{0} > 0\),
令\(x_{0} = a\cos^{3}\theta,y_{0} = a\sin^{3}\theta\),则\(\cos\theta > 0,\sin\theta > 0\),
\(\therefore a\cos^{3}\theta \cdot a\sin^{3}\theta = \frac{8a^{2}}{125}\),即\(\sin\theta\cos\theta = \frac{2}{5}\),
\(\therefore\left( \sin\theta + \cos\theta \right)^{2} = 1 + 2\sin\theta\cos\theta = 1 + \frac{4}{5} = \frac{9}{5}\),
\(\therefore\sin\theta + \cos\theta = \frac{3\sqrt{5}}{5}\),
\(\therefore x_{0} + y_{0} = a\cos^{3}\theta + a\sin^{3}\theta = a\left( \sin\theta + \cos\theta \right)\left( 1 - \sin\theta\cos\theta \right) = a \cdot \frac{3\sqrt{5}}{5} \times \frac{3}{5} = \frac{9}{5}\),
解得\(a = \sqrt{5}\)}
\end{question}

\section{多选题}

\begin{question}
若\(a < b < 0\),则(  )
\begin{choices}
  \item \(a^{2} < b^{2}\)
  \item \(\frac{1}{a} > \frac{1}{b}\)
  \item \(\text{ln}(b - a) > 0\)
  \item \(a^{3} < b^{3}\)
\end{choices}
\topics{由不等式的性质比较数(式)大小;作差法比较代数式的大小;比较对数式的大小}
\difficulty{0.85}
\answer{BD}
\explain{对于A,由\(a < b < 0\),则\(a - b < 0,a + b < 0\),由\(a^{2} - b^{2} = (a - b)(a + b) > 0\),
可得\(a^{2} > b^{2}\),故A错误;
对于B,由\(a < b < 0\),则\(b - a > 0,ab > 0\),故\(\frac{1}{a} - \frac{1}{b} = \frac{b - a}{ab} > 0\),即\(\frac{1}{a} > \frac{1}{b}\),故B正确;
对于C,因\(a < b < 0\),当\(0 < b - a < 1\)时,\(\text{ln}(b - a) < 0\),故C错误;
对于D,由\(a < b < 0\),可得\(- a > - b > 0\),利用不等式的性质可得\({( - a)}^{3} > {( - b)}^{3}\),
即\(- a^{3} > - b^{3}\),故\(a^{3} < b^{3}\),故D正确.}
\end{question}

\begin{question}
已知向量\(\overrightarrow{a} = (2,4),\overset{\rightarrow}{b} = \left( m,\frac{1}{m} \right),\overset{\rightarrow}{c} = (3,3)\),则下列说法正确的是(  )
\begin{choices}
  \item 若\(m = 1\),则\(\left( \overset{\rightarrow}{a} - \overset{\rightarrow}{c} \right)\bot\overset{\rightarrow}{b}\)
  \item 若\(\overset{\rightarrow}{a}\text{//}\overset{\rightarrow}{b}\),则\(m = \frac{\sqrt{2}}{2}\)
  \item \(\overset{\rightarrow}{a}\)在\(\overset{\rightarrow}{c}\)上的投影向量为\(\overset{\rightarrow}{c}\)
  \item \(\left| \overset{\rightarrow}{b} - \overset{\rightarrow}{c} \right|\)的最小值为\(\sqrt{7}\)
\end{choices}
\topics{由向量共线(平行)求参数;数量积的坐标表示;坐标计算向量的模;求投影向量}
\difficulty{0.65}
\answer{ACD}
\explain{若\(m = 1\),\(\overset{\rightarrow}{a} - \overset{\rightarrow}{c} = ( - 1,1)\),\(\overset{\rightarrow}{b} = (1,1)\),所以\(\left( \overset{\rightarrow}{a} - \overset{\rightarrow}{c} \right) \cdot \overset{\rightarrow}{b} = - 1 \times 1 + 1 \times 1 = 0\),
所以\(\left( \overset{\rightarrow}{a} - \overset{\rightarrow}{c} \right)\bot\overset{\rightarrow}{b}\),故A正确;
若\(\overset{\rightarrow}{a}\text{//}\overset{\rightarrow}{b}\),则\(\frac{2}{m} - 4m = 0\),解得\(m = \pm \frac{\sqrt{2}}{2}\),故B错误;
\(\overset{\rightarrow}{a}\)在\(\overset{\rightarrow}{c}\)上的投影向量为\(\frac{\overset{\rightarrow}{a} \cdot \overset{\rightarrow}{c}}{\left| \overset{\rightarrow}{c} \right|^{2}} \cdot \overset{\rightarrow}{c} = \frac{2 \times 3 + 4 \times 3}{3^{2} + 3^{2}} \cdot \overset{\rightarrow}{c} = \frac{18}{18}\overset{\rightarrow}{c} = \overset{\rightarrow}{c}\),故C正确;
因为\(\overset{\rightarrow}{b} - \overset{\rightarrow}{c} = \left( m - 3,\frac{1}{m} - 3 \right)\),所以\(\left| \overset{\rightarrow}{b} - \overset{\rightarrow}{c} \right| = \sqrt{(m - 3)^{2} + \left( \frac{1}{m} - 3 \right)^{2}} = \sqrt{\left( m + \frac{1}{m} \right)^{2} - 6\left( m + \frac{1}{m} \right) + 16}\),
令\(t = m + \frac{1}{m}\),当\(m > 0\)时,则由均值不等式,\(t = m + \frac{1}{m} \geq 2\sqrt{m \cdot \frac{1}{m}} = 2\),当且仅当\(m = 1\)时取等号,
当\(m < 0\)时,\(t = m + \frac{1}{m} = - \left( - m + \frac{1}{- m} \right) \leq - 2\sqrt{- m \cdot \frac{1}{- m}} = - 2\),当且仅当\(m = - 1\)时取等号,
则\(\left| \overset{\rightarrow}{b} - \overset{\rightarrow}{c} \right| = \sqrt{t^{2} - 6t + 16} = \sqrt{(t - 3)^{2} + 7}\)(\(t \geq 2\)或\(t \leq - 2\)),
所以当\(t = 3\)时,\(\left| \overset{\rightarrow}{b} - \overset{\rightarrow}{c} \right|\)有最小值\(\sqrt{7}\),故D正确}
\end{question}

\begin{question}
已知函数\(f(x) = \sqrt{|x + 1|} - \sqrt{|x - 1|}\),则(  ).
\begin{choices}
  \item \(f( - x) + f(x) = 0\)
  \item \(f(x)\)在\(( - 1,1)\)上单调递减
  \item \(f(x + 1) - f(x) \leq \sqrt{2}\)
  \item \(g(x) = f(x) - \text{sin}x\)在\(( - 1,1)\)上有且仅有1个零点
\end{choices}
\topics{函数奇偶性的定义与判断;利用导数研究不等式恒成立问题;利用导数研究函数的零点}
\difficulty{0.4}
\answer{ACD}
\explain{对于A,\(\because f( - x) = \sqrt{| - x + 1|} - \sqrt{| - x - 1|} = \sqrt{|x - 1|} - \sqrt{|x + 1|}\),
\(\therefore f( - x) + f(x) = 0\),故A正确;
对于B,\(x \in ( - 1,1)\)时,\(f(x) = \sqrt{x + 1} - \sqrt{1 - x}\),
又\(y = \sqrt{x + 1}\)在\(( - 1,1)\)上单调递增,\(y = - \sqrt{1 - x}\)在\(( - 1,1)\)上单调递增,
所以\(f(x) = \sqrt{x + 1} - \sqrt{1 - x}\)在\(( - 1,1)\)上单调递增,故B错误;
对于C,令\(h(x) = f(x + 1) - f(x) = \sqrt{|x + 2|} - \sqrt{|x|} - \sqrt{|x + 1|} + \sqrt{|x - 1|}\),
\(h( - 1 - x) = \sqrt{|x - 1|} - \sqrt{|x + 1|} - \sqrt{|x|} + \sqrt{|x + 2|} = h(x)\),
所以\(h(x)\)关于\(x = - \frac{1}{2}\)对称,
当\(x \geq 1\)时,\(h(x) = \sqrt{x + 2} + \sqrt{x - 1} - \left( \sqrt{x} + \sqrt{x + 1} \right)\),
又\(\left( \sqrt{x + 2} + \sqrt{x - 1} \right)^{2} - \left( \sqrt{x} + \sqrt{x + 1} \right)^{2} = 2\sqrt{x^{2} + x - 2} - 2\sqrt{x^{2} + x} < 0\),
所以\(x \geq 1\)时,\(h(x) < 0\),
\(0 \leq x < 1\)时,\(h(x) = \sqrt{x + 2} - \sqrt{x} - \sqrt{x + 1} + \sqrt{1 - x}\)
\(h'(x) = \frac{1}{2\sqrt{x + 2}} - \frac{1}{2\sqrt{x}} - \frac{1}{2\sqrt{x + 1}} - \frac{1}{2\sqrt{1 - x}}\)
\(= \frac{\sqrt{x} - \sqrt{x + 2}}{2\sqrt{x(x + 2)}} - \frac{1}{2\sqrt{x + 1}} - \frac{1}{2\sqrt{1 - x}} < 0\),
所以\(h(x)\)在\(\lbrack 0,1)\)上单调递减,则\(h(x) \leq h(0) = \sqrt{2} - 0 - 1 + 1 = \sqrt{2}\),
当\(- \frac{1}{2} \leq x < 0\)时,\(h(x) = \sqrt{x + 2} - \sqrt{- x} - \sqrt{x + 1} + \sqrt{1 - x}\),
\(h'(x) = \frac{1}{2\sqrt{x + 2}} + \frac{1}{2\sqrt{- x}} - \frac{1}{2\sqrt{x + 1}} - \frac{1}{2\sqrt{1 - x}}\)
\(= \frac{\sqrt{x + 1}\sqrt{1 - x}\left( \sqrt{x + 2} + \sqrt{- x} \right) - \sqrt{x + 2}\sqrt{- x}\left( \sqrt{1 - x} + \sqrt{x + 1} \right)}{2\sqrt{x + 2}\sqrt{- x}\sqrt{x + 1}\sqrt{1 - x}}\)
\(\left\lbrack \sqrt{x + 1}\sqrt{1 - x}\left( \sqrt{x + 2} + \sqrt{- x} \right) \right\rbrack^{2} - \left\lbrack \sqrt{x + 2}\sqrt{- x}\left( \sqrt{1 - x} + \sqrt{x + 1} \right) \right\rbrack^{2}\)
\(= \left( 1 - x^{2} \right)\left( 2 + 2\sqrt{- x^{2} - 2x} \right) - \left( - x^{2} - 2x \right)\left( 2 + 2\sqrt{1 - x^{2}} \right)\)
\(= 2(1 + 2x) + 2\sqrt{1 - x^{2}}\sqrt{- x^{2} - 2x}\left( \sqrt{1 - x^{2}} - \sqrt{- x^{2} - 2x} \right)\),
\(\because - \frac{1}{2} \leq x < 0\),\(\therefore 1 + 2x \geq 0\),
\(\left( 1 - x^{2} \right) - \left( - x^{2} - 2x \right) = 1 + 2x \geq 0\),即\(\sqrt{1 - x^{2}} - \sqrt{- x^{2} - 2x} \geq 0\),
所以\(2(1 + 2x) + 2\sqrt{1 - x^{2}}\sqrt{- x^{2} - 2x}\left( \sqrt{1 - x^{2}} - \sqrt{- x^{2} - 2x} \right) \geq 0\),
则\(\sqrt{x + 1}\sqrt{1 - x}\left( \sqrt{x + 2} + \sqrt{- x} \right) - \sqrt{x + 2}\sqrt{- x}\left( \sqrt{1 - x} + \sqrt{x + 1} \right) \geq 0\),
即\(h'(x) \geq 0\),所以\(h(x)\)在\(\left\lbrack - \frac{1}{2},0 \right)\)上单调递增,此时\(h(x) < h(0) = \sqrt{2}\),
又\(h(x)\)关于\(x = - \frac{1}{2}\)对称,所以\(h(x) = f(x + 1) - f(x) \leq \sqrt{2}\)成立,故C正确;
对于D,\(x \in ( - 1,1)\)时,\(g(x) = f(x) - \text{sin}x = \sqrt{x + 1} - \sqrt{1 - x} - \sin x\),
\(g'(x) = \frac{1}{2\sqrt{x + 1}} + \frac{1}{2\sqrt{1 - x}} - \cos x\),
又\(\frac{1}{2\sqrt{x + 1}} + \frac{1}{2\sqrt{1 - x}} \geq 2\sqrt{\frac{1}{4\sqrt{1 - x^{2}}}} \geq 1\),当\(x = 0\)时取等,\(\cos x \leq 1\),
所以\(g'(x) = \frac{1}{2\sqrt{x + 1}} + \frac{1}{2\sqrt{1 - x}} - \cos x \geq 0\),
即\(g(x)\)在\(( - 1,1)\)上单调递增,且\(g(0) = 0\),
所以\(g(x) = f(x) - \text{sin}x\)在\(( - 1,1)\)上有且仅有1个零点,故D正确}
\end{question}

\section{填空题}

\begin{question}
已知椭圆\(\frac{x^{2}}{m + 2} + \frac{y^{2}}{m - 1} = 1\)的离心率为\(\frac{1}{2}\),则实数\(m =\)
.
\topics{根据离心率求椭圆的标准方程}
\difficulty{0.85}
\answer{10}
\explain{\(e^{2} = \frac{c^{2}}{a^{2}} = \frac{a^{2} - b^{2}}{a^{2}} = 1 - \frac{b^{2}}{a^{2}} = \frac{1}{4} \Rightarrow 3a^{2} = 4b^{2}\).
\(\because\(\)\(m\) + 2 > \(m\) - 1\),
\(\therefore\)椭圆的焦点在\emph{x}轴上,
\(\therefore\(\)\left\{ \begin{array}{r}
\(m\) + 2 > \(m\) - 1 > 0 \\
\(a\)^{2} = \(m\) + 2 \\
\(b\)^{2} = \(m\) - 1 \\
3(\(m\) + 2) = 4(\(m\) - 1)
\end{array} \right.\\),
解得\(m = 10\).10.}
\end{question}

\begin{question}
记\(\bigtriangleup ABC\)的内角\(A,B,C\)的对边分别为\(a,b,c\),若\(B = \frac{\text{π}}{4},c = 4,b\text{sin}A = 1\),则\(b =\)
.
\topics{正弦定理解三角形;余弦定理解三角形}
\difficulty{0.85}
\answer{\(\sqrt{10}\)}
\explain{由正弦定理\(\frac{a}{\sin A} = \frac{b}{\sin B}\)可得\(a = \frac{b\sin A}{\sin B} = \frac{1}{\frac{\sqrt{2}}{2}} = \sqrt{2}\),
则\(b^{2} = a^{2} + c^{2} - 2ac\cos B = 2 + 16 - 8\sqrt{2} \times \frac{\sqrt{2}}{2} = 10\),即\(b = \sqrt{10}\).\(\sqrt{10}\).}
\end{question}

\begin{question}
已知球\(O\)的半径为\(3,P,Q\)是球面上两点,过\(P,Q\)的平面与球面的交线为圆\(O_{1}\),且\(P,Q,O,O_{1}\)四点不共面.若平面\(PQO_{1}\)与平面\(PQO\)的夹角为\(60^{\circ}\),则四面体\(PQOO_{1}\)体积的最大值为
.
\topics{由导数求函数的最值(不含参);面积;体积最大问题}
\difficulty{0.4}
\answer{\(\frac{3}{2}\)}
\explain{取\(PQ\)中点\(M\),因为\(O_{1}P = O_{1}Q\),则\(Q_{1}M\bot PQ\),同理可得\(OM\bot PQ\),
因为\(Q_{1}M \subset\)平面\(PQO_{1}\),\(OM \subset\)平面\(PQO\),且平面\(PQO_{1} \cap\)平面\(PQO = PQ\),
则平面\(PQO_{1}\)与平面\(PQO\)夹角为\(\angle OMO_{1} = 60^{{^\circ}}\),
令\(O_{1}M = t\),因为\(OO_{1}\bot\)圆面\(O_{1}\),而\(O_{1}M \subset\)圆面\(O_{1}\),所以\(OO_{1}\bot O_{1}M\),
则\(OO_{1} = \sqrt{3}t,OM = 2t,PQ = 2\sqrt{9 - 4t^{2}}\),则\(9 - 4t^{2} > 0\),且\(t > 0\),故\(0 < t < \frac{3}{2}\),
\(S_{\bigtriangleup O_{1}PQ} = \frac{1}{2} \cdot t \cdot 2\sqrt{9 - 4t^{2}} = t\sqrt{9 - 4t^{2}}\),
\(V = \frac{1}{3} \cdot \sqrt{3}t \cdot t\sqrt{9 - 4t^{2}} = \frac{\sqrt{3}}{3}\sqrt{9t^{4} - 4t^{6}}\),
令\(f(x) = 9x^{4} - 4x^{6}\),\(0 < x < \frac{3}{2}\),\(f'(x) = 36x^{3} - 24x^{5} = 12x^{3}\left( 3 - 2x^{2} \right) = 0\),
令\(f'(x) = 0\),解得\(x = 0\)(舍)或\(x = - \frac{\sqrt{6}}{2}\)(舍)或\(x = \frac{\sqrt{6}}{2}\),
当\(x \in \left( 0,\frac{\sqrt{6}}{2} \right)\)时,\(f'(x) > 0\),当\(x \in \left( \frac{\sqrt{6}}{2},\frac{3}{2} \right)\)时,\(f'(x) < 0\),
所以\(f(x)\)在\(\left( 0,\frac{\sqrt{6}}{2} \right)\)单调递增,\(\left( \frac{\sqrt{6}}{2},\frac{3}{2} \right)\)单调递减,所以\(f{(x)}_{\max} = f\left( \frac{\sqrt{6}}{2} \right) = \frac{27}{4}\),
则\(V_{\max} = \frac{\sqrt{3}}{3} \times \frac{3\sqrt{3}}{2} = \frac{3}{2}\).
[公式:9f7a4efb-81f8-442b-b4a7-00f1ef94652d]\(\frac{3}{2}\).}
\end{question}

\section{解答题}

\begin{question}
\item 若每次抽取后都放回,设取到黄球的个数为\(X\),求\(P(X \geq 1)\);
\item 若每次抽取后都不放回,设取到黄球的个数为\(Y\),求\(Y\)的分布列和数学期望.
\topics{独立重复试验的概率问题;超几何分布的分布列}
\difficulty{0.65}
\answer{(1)\(\frac{39}{64}\)
(2)分布列间解析;\(E(Y) = \frac{3}{4}\).}
\explain{(1)每次抽取后都放回,则取到黄球的个数\(X \sim B\left( 2,\frac{3}{8} \right)\),
所以\(P(X = 1) = \text{C}_{2}^{1} \times \frac{3}{8} \times \frac{5}{8} = \frac{15}{32}\),\(P(X = 2) = \text{C}_{2}^{2} \times \left( \frac{3}{8} \right)^{2} = \frac{9}{64}\),
所以\(P(X \geq 1) = P(X = 1) + P(X = 2)\)
\(= \frac{15}{32} + \frac{9}{64} = \frac{39}{64}\).
(2)每次抽取后都不放回则取到黄球的个数\(Y\)的值可能为:0,1,2.
且\(P(Y = 0) = \frac{\text{C}_{5}^{2}}{\text{C}_{8}^{2}} = \frac{5}{14}\),\(P(Y = 1) = \frac{\text{C}_{3}^{1} \cdot \text{C}_{5}^{1}}{\text{C}_{8}^{2}} = \frac{15}{28}\),\(P(Y = 2) = \frac{\text{C}_{3}^{2}}{\text{C}_{8}^{2}} = \frac{3}{28}\).
所以\(Y\)的分布列为:
  ------------ ------------------ ------------------- ------------------
     \(Y\)             0                   1                  2
     \(P\)      \(\frac{5}{14}\)   \(\frac{15}{28}\)   \(\frac{3}{28}\)
  ------------ ------------------ ------------------- ------------------
所以\(E(Y) = 0 \times \frac{5}{14} + 1 \times \frac{15}{28} + 2 \times \frac{3}{28} = \frac{21}{28} = \frac{3}{4}\).}
\end{question}

\begin{question}
\item 已知\(a_{n} = n^{2} + n + 1\),证明:\(\left\{ a_{n} \right\}\)的差分数列为等差数列;
\item 已知\(\left\{ a_{n} \right\}\)的差分数列为\(\left\{ \frac{2n^{2} + 2n + 1}{n^{2} + n} \right\} ,a_{1} = 1\),求\(\left\{ a_{n} \right\}\)的通项公式.
\topics{累加法求数列通项;判断等差数列;裂项相消法求和;数列新定义}
\difficulty{0.65}
\answer{(1)证明见解析
(2)\(a_{n} = 2n - \frac{1}{n}\)}
\explain{(1)\(\bigtriangleup a_{n} = (n + 1)^{2} + (n + 1) + 1 - \left( n^{2} + n + 1 \right) = 2n + 2\),其中\(n \geq 1\),
故\(\bigtriangleup a_{n} - \bigtriangleup a_{n - 1} = 2\),故\(\left\{ a_{n} \right\}\)的差分数列为等差数列.
(2)由题设有\(a_{n + 1} - a_{n} = \frac{2n^{2} + 2n + 1}{n^{2} + n} = 2 + \frac{1}{n} - \frac{1}{n + 1}\),
故\(a_{n} - a_{n - 1} = 2 + \frac{1}{n - 1} - \frac{1}{n}(n \geq 2)\),由累加法可得\(a_{n} - a_{1} = 2(n - 1) + 1 - \frac{1}{n}\),
而\(a_{1} = 1\),所以\(a_{n} = 1 + 2(n - 1) + 1 - \frac{1}{n} = 2n - \frac{1}{n}\),
而\(a_{1} = 1\)也满足该式,故\(a_{n} = 2n - \frac{1}{n}\).}
\end{question}

\begin{question}
\item 证明:\(MN\text{//}\)平面\(ACC_{1}A_{1}\);
\item 若\(AC = 2\sqrt{2},AB = 2\sqrt{3},\angle ACB = 90{^\circ},MN\bot A_{1}C\),求\(A_{1}C\)与平面\(CMN\)所成角的正弦值.
\topics{证明线面平行;线面角的向量求法}
\difficulty{0.65}
\answer{(1)证明见解析
(2)\(\frac{\sqrt{21}}{7}\)}
\explain{(1)取\(A_{1}C_{1}\)的中点为\(S\),连接\(SN,AS\),
因为\(A_{1}S = SC_{1}\),\(B_{1}N = NC_{1}\),故\(SN//A_{1}B_{1},SN = \frac{1}{2}A_{1}B_{1}\),
由直三棱柱的性质可得\(AM//A_{1}B_{1},AM = \frac{1}{2}A_{1}B_{1}\),故\(SN//AM,SN = AM\),
故四边形\(SNMA\)为平行四边形,故\(AS//NM\),
而\(AS \subset\)平面\(A_{1}C_{1}CA\),\(NM ⊄\)平面\(A_{1}C_{1}CA\),故\(NM//\)平面\(A_{1}C_{1}CA\).

\begin{center}
\begin{tikzpicture}[scale=1.05,>=Stealth,line cap=round,line join=round]
  \node[draw, minimum width=6cm, minimum height=4cm] {图略(图 ID: 210c8f48-b899-4bc5-a203-46aaf17cb775)};
\end{tikzpicture}
\end{center}

(2)因为\(\angle ACB = 90{^\circ}\),故\(AC\bot CB\),故\(BC = \sqrt{12 - 8} = 2\),设\(AA_{1} = a\).
由直三棱柱可得\(CC_{1}\bot\)平面\(ABC\),故可建立如图所示的空间直角坐标系,
则\(C(0,0,0),A\left( 2\sqrt{2},0,0 \right),B(0,2,0),A_{1}\left( 2\sqrt{2},0,a \right),B_{1}(0,2,a),C_{1}(0,0,a)\),
故\(M\left( \sqrt{2},1,0 \right),N(0,1,a)\),且\(\overrightarrow{CA_{1}} = \left( 2\sqrt{2},0,a \right),\overrightarrow{MN} = \left( - \sqrt{2},0,a \right)\).
因为\(MN\bot A_{1}C\),故\(\overrightarrow{CA_{1}} \cdot \overrightarrow{MN} = 0\)即\(- 4 + a^{2} = 0\),故\(a = 2\)(\(a = - 2\)舍去),
故\(\overrightarrow{CA_{1}} = \left( 2\sqrt{2},0,2 \right)\),\(\overrightarrow{MN} = \left( - \sqrt{2},0,2 \right)\),又\(\overrightarrow{CM} = \left( \sqrt{2},1,0 \right)\).
设平面\(CMN\)的法向量为\(\overrightarrow{n} = (x,y,z)\),则\(\left\{ \begin{array}{r}
\overrightarrow{n}\bot\overrightarrow{MN} \\
\overrightarrow{n}\bot\overrightarrow{CM}
\end{array} \right.\\),
所以\(\left\{ \begin{array}{r}
 - \sqrt{2}x + 2z = 0 \\
\sqrt{2}x + y = 0
\end{array} \right.\\),取\(\overrightarrow{n} = \left( \sqrt{2}, - 2,1 \right)\),
故\(A_{1}C\)与平面\(CMN\)所成角的正弦值为\(\left| \frac{\overrightarrow{n} \cdot \overrightarrow{CA_{1}}}{\left| \overrightarrow{n} \right| \cdot \left| \overrightarrow{CA_{1}} \right|} \right| = \frac{6}{\sqrt{7} \times 2\sqrt{3}} = \frac{\sqrt{21}}{7}\).
[公式:2913820e-4dc6-483a-b9f2-9baac169de9a]}
\end{question}

\begin{question}
\item 求\(C\)的方程;
\item 若\(A,B\)均在\(C\)的右支上,且\(\bigtriangleup ABF_{1}\)的周长为\(16\sqrt{2}\),求\(l\)的方程;
\item 是否存在\(x\)轴上的定点\(M\),使得\(\overrightarrow{MA} \cdot \overrightarrow{MB}\)为定值?若存在,求出点\(M\)的坐标;若不存在,请说明理由.
\topics{根据a;b;c求双曲线的标准方程;求双曲线中的弦长;双曲线中存在定点满足某条件问题}
\difficulty{0.65}
\answer{(1)\(x^{2} - y^{2} = 2\)
(2)\(\sqrt{2}x \pm y - 2\sqrt{2} = 0\)
(3)存在点\(M(1,0)\),使得\(\overrightarrow{MA} \cdot \overrightarrow{MB}\)为定值.}
\explain{(1)因为\(\left| F_{1}F_{2} \right| = 4\) \(\Rightarrow\)
\(2c = 4\),所以\(c = 2\),
又\(c^{2} = 2a^{2}\),所以\(a^{2} = 2\).
所以双曲线\(C\)的方程为:\(x^{2} - y^{2} = 2\).
(2)因为\(A,B\)均在\(C\)的右支上,且\(\bigtriangleup ABF_{1}\)的周长为\(16\sqrt{2}\),
所以\(|AB| + \left| AF_{1} \right| + \left| BF_{1} \right| = 16\sqrt{2}\)
\(\Rightarrow\)
\(|AB| + \left| AF_{2} \right| + 2\sqrt{2} + \left| BF_{2} \right| + 2\sqrt{2} = 16\sqrt{2}\)
\(\Rightarrow\) \(|AB| = 6\sqrt{2}\).
如图:

\begin{center}
\begin{tikzpicture}[scale=1.05,>=Stealth,line cap=round,line join=round]
  \node[draw, minimum width=6cm, minimum height=4cm] {图略(图 ID: 0df61d26-0be6-4fb5-bb2b-444d7221a1c2)};
\end{tikzpicture}
\end{center}

因为\(F_{2}(2,0)\),设直线\(l\):\(x = ty + 2\),代入\(x^{2} - y^{2} = 2\)得:
\((ty + 2)^{2} - y^{2} = 2\),
整理得:\(\left( t^{2} - 1 \right)y^{2} + 4ty + 2 = 0\).
设\(A\left( x_{1},y_{1} \right)\),\(B\left( x_{2},y_{2} \right)\),
因为\(A,B\)均在\(C\)的右支上,所以\(t^{2} - 1 \neq 0\),且\(y_{1}y_{2} = \frac{2}{t^{2} - 1} < 0\),所以\(t^{2} < 1\),
\(y_{1} + y_{2} = - \frac{4t}{t^{2} - 1}\).
所以\(\left( y_{1} - y_{2} \right)^{2} = \left( y_{1} + y_{2} \right)^{2} - 4y_{1}y_{2}\)
\(= \frac{16t^{2}}{\left( t^{2} - 1 \right)^{2}} - \frac{4 \times 2}{t^{2} - 1}\)
\(= \frac{16t^{2} - 8\left( t^{2} - 1 \right)}{\left( t^{2} - 1 \right)^{2}}\)
\(= \frac{8\left( t^{2} + 1 \right)}{\left( t^{2} - 1 \right)^{2}}\).
所以\(|AB| = \sqrt{1 + t^{2}} \cdot \left| y_{1} - y_{2} \right|\)
\(= \frac{2\sqrt{2}\left( t^{2} + 1 \right)}{1 - t^{2}}\)
\(= 6\sqrt{2}\) \(\Rightarrow\) \(t^{2} = \frac{1}{2}\).
所以\(t = \pm \frac{\sqrt{2}}{2}\).
所以直线\(l\)的方程为:\(x = \pm \frac{\sqrt{2}}{2}y + 2\),即\(\sqrt{2}x \pm y - 2\sqrt{2} = 0\).
(3)假设存在\(x\)轴上的定点\(M\)
\((m,0)\),使得\(\overrightarrow{MA} \cdot \overrightarrow{MB}\)为定值.
因为\(\overrightarrow{MA} = \left( x_{1} - m,y_{1} \right)\),\(\overrightarrow{MB} = \left( x_{2} - m,y_{2} \right)\),
所以\(\overrightarrow{MA} \cdot \overrightarrow{MB}\)
\(= \left( x_{1} - m,y_{1} \right) \cdot \left( x_{2} - m,y_{2} \right)\)
\(= \left( x_{1} - m \right) \cdot \left( x_{2} - m \right) + y_{1}y_{2}\)
\(= \left( ty_{1} + 2 - m \right) \cdot \left( ty_{2} + 2 - m \right) + y_{1}y_{2}\)
\(= \left( t^{2} + 1 \right)y_{1}y_{2} + (2 - m)t\left( y_{1} + y_{2} \right) + (2 - m)^{2}\)
\(= \frac{2\left( t^{2} + 1 \right)}{t^{2} - 1} - \frac{4t^{2}(2 - m)}{t^{2} - 1} + (2 - m)^{2}\)
\(= \frac{(4m - 6)t^{2} + 2}{t^{2} - 1} + (2 - m)^{2}\).
因为\(\overrightarrow{MA} \cdot \overrightarrow{MB}\)为常数,所以\((4m - 6) + 2 = 0\)
\(\Rightarrow\) \(m = 1\).
此时\(\overrightarrow{MA} \cdot \overrightarrow{MB} = - 2 + 1 = - 1\).
所以存在点\(M(1,0)\),使得\(\overrightarrow{MA} \cdot \overrightarrow{MB}\)为定值.}
\end{question}

\begin{question}
\item 当\(a = 1\)时,若直线\(y = - x + b\)是曲线\(y = f(x)\)的一条切线,求\(b\)的值;
\item 讨论\(f(x)\)的单调性;
\item 若集合\(\{ x \mid f(x) < 1,x \in \text{Z}\}\)中有且仅有一个元素,求\(a\)的取值范围.
\topics{已知切线(斜率)求参数;函数单调性;极值与最值的综合应用;利用导数求函数(含参)的单调区间}
\difficulty{0.4}
\answer{(1)\(b = 2 - \ln 2\)
(2)当\(a > 0\)时,\(f(x)\)在\((0,\frac{- 1 + \sqrt{1 + 4a}}{2a})\)上单调递减,在\((\frac{- 1 + \sqrt{1 + 4a}}{2a}, + \infty)\)上单调递增;当\(a < 0\)时,\(f(x)\)在\(( - \infty,0)\)上单调递减
(3)\(\left( - 1, - \frac{1}{2} \right\rbrack \cup \left\lbrack \frac{1}{2},1 \right)\)}
\explain{(1)当\(a = 1\)时,\(f(x) = \ln x + \frac{1}{x} + x - 1(x > 0)\),则\(f'(x) = \frac{1}{x} - \frac{1}{x^{2}} + 1\),
当\(f'(x) = \frac{1}{x} - \frac{1}{x^{2}} + 1 = - 1\)时,解得\(x = \frac{1}{2}\)或\(x = - 1\)(舍),
则\(f\left( \frac{1}{2} \right) = \ln\frac{1}{2} + 2 + \frac{1}{2} - 1 = \frac{3}{2} - \ln 2\),可得切点\(\left( \frac{1}{2},\frac{3}{2} - \ln 2 \right)\),
代入切线方程得\(\frac{3}{2} - \ln 2 = - \frac{1}{2} + b\),解得\(b = 2 - \ln 2\).
(2)已知\(f(x) = \text{ln}(ax) + \frac{1}{x} + a(x - 1)\),得\(f'(x) = \frac{1}{x} - \frac{1}{x^{2}} + a = \frac{ax^{2} + x - 1}{x^{2}}\);
当\(a > 0\)时,定义域为\((0, + \infty)\),
\(f'(x) = \frac{1}{x} - \frac{1}{x^{2}} + a = \frac{ax^{2} + x - 1}{x^{2}}\),二次函数\(y = ax^{2} + x - 1\)图象开口向上,且\(\Delta = 1 + 4a > 0\),
令\(g(x) = ax^{2} + x - 1 = 0\),在\((0, + \infty)\)必有解\(x = \frac{- 1 + \sqrt{1 + 4a}}{2a}\),
当\(0 < x < \frac{- 1 + \sqrt{1 + 4a}}{2a}\)时,\(g(x) < 0\),\(f'(x) < 0\),\(f(x)\)在\((0,\frac{- 1 + \sqrt{1 + 4a}}{2a})\)上单调递减,
当\(x > \frac{- 1 + \sqrt{1 + 4a}}{2a}\)时,\(g(x) > 0\),\(f'(x) > 0\),\(f(x)\)在\((\frac{- 1 + \sqrt{1 + 4a}}{2a}, + \infty)\)上单调递增;
当\(a < 0\)时,定义域为\(( - \infty,0)\),则\(f'(x) = \frac{ax^{2} + x - 1}{x^{2}} < 0\)恒成立,\(f(x)\)在\(( - \infty,0)\)上单调递减,
综上所述:
当\(a > 0\)时,\(f(x)\)在\((0,\frac{- 1 + \sqrt{1 + 4a}}{2a})\)上单调递减,在\((\frac{- 1 + \sqrt{1 + 4a}}{2a}, + \infty)\)上单调递增;
当\(a < 0\)时,\(f(x)\)在\(( - \infty,0)\)上单调递减.
(3)由\(f(x) = \text{ln}(ax) + \frac{1}{x} + a(x - 1)\)可知,\(f\left( \frac{1}{a} \right) = \text{ln}1 + a + a\left( \frac{1}{a} - 1 \right) = 1\),
当\(a < 0\)时,\(f(x)\)在\(( - \infty,0)\)上单调递减,若集合\(\{ x \mid f(x) < 1,x \in \text{Z}\}\)中有且仅有一个元素,
则\(\left\{ \begin{array}{r}
f( - 1) < 1 \\
f( - 2) \geq 1
\end{array} \right.\\),即\(\left\{ \begin{array}{r}
f( - 1) < f\left( \frac{1}{a} \right) \\
f( - 2) \geq f\left( \frac{1}{a} \right)
\end{array} \right.\\),即\(\left\{ \begin{array}{r}
 - 1 > \frac{1}{a} \\
 - 2 \leq \frac{1}{a}
\end{array} \right.\\),
解得\(- 1 < a \leq - \frac{1}{2}\);
当\(a > 0\)时,\(f(x)\)在\((0,\frac{- 1 + \sqrt{1 + 4a}}{2a})\)上单调递减,在\((\frac{- 1 + \sqrt{1 + 4a}}{2a}, + \infty)\)上单调递增;
可知\(0 < \frac{- 1 + \sqrt{1 + 4a}}{2a} < 1\),所以集合\(\{ x \mid f(x) < 1,x \in \text{Z}\}\)中有且仅有一个元素,
则\(\left\{ \begin{array}{r}
f(1) < 1 \\
f(2) \geq 1
\end{array} \right.\\),即\(\left\{ \begin{array}{r}
f(1) < f\left( \frac{1}{a} \right) \\
f(2) \geq f\left( \frac{1}{a} \right)
\end{array} \right.\\),即\(\left\{ \begin{array}{r}
1 < \frac{1}{a} \\
2 \geq \frac{1}{a}
\end{array} \right.\\),解得\(\frac{1}{2} \leq a < 1\);
综上所述,\(a\)的取值范围为\(\left( - 1, - \frac{1}{2} \right\rbrack \cup \left\lbrack \frac{1}{2},1 \right)\).}
\end{question}
