\setexamdir{content/exams/auto/gaokao_2024_national_2}

\examxtitle{2024年新课标全国Ⅱ卷数学真题}

\section{单选题}

\begin{question}
已知\(z = - 1 - i\),则\(|z| =\)(    )
\begin{choices}
  \item 0
  \item 1
  \item \(\sqrt{2}\)
  \item 2
\end{choices}
\topics{求复数的模}
\difficulty{0.94}
\answer{C}
\explain{若\(z = - 1 - \text{i}\),则\(|z| = \sqrt{( - 1)^{2} + ( - 1)^{2}} = \sqrt{2}\)}
\end{question}

\begin{question}
已知命题\(p\):\(\forall x \in R\),
\(|x + 1| > 1\);
命题\(q\):\(\exists x > 0\),\(x^{3} = x\),
则(    )
\begin{choices}
  \item \(p\)和\(q\)都是真命题
  \item \(\neg p\)和\(q\)都是真命题
  \item \(p\)和\(\neg q\)都是真命题
  \item \(\neg p\)和\(\neg q\)都是真命题
\end{choices}
\topics{判断命题的真假;全称命题的否定及其真假判断;特称命题的否定及其真假判断}
\difficulty{0.94}
\answer{B}
\explain{对于\(p\)而言,取\(x = - 1\),
则有\(|x + 1| = 0 < 1\),故\(p\)是假命题,
\(\neg p\)是真命题,\par
对于\(q\)而言,取\(x = 1\),
则有\(x^{3} = 1^{3} = 1 = x\),故\(q\)是真命题,
\(\neg q\)是假命题,\par
综上,\(\neg p\)和\(q\)都是真命题}
\end{question}

\begin{question}
已知向量\(\overrightarrow{a},\overrightarrow{b}\)满足\(\left| \overrightarrow{a} \right| = 1,\left| \overrightarrow{a} + 2\overrightarrow{b} \right| = 2\),
且\(\left( \overrightarrow{b} - 2\overrightarrow{a} \right)\bot\overrightarrow{b}\),
则\(\left| \overrightarrow{b} \right| =\)(    )
\begin{choices}
  \item \(\frac{1}{2}\)
  \item \(\frac{\sqrt{2}}{2}\)
  \item \(\frac{\sqrt{3}}{2}\)
  \item 1
\end{choices}
\topics{数量积的运算律;已知数量积求模;垂直关系的向量表示}
\difficulty{0.85}
\answer{B}
\explain{因为\(\left( \overrightarrow{b} - 2\overrightarrow{a} \right)\bot\overrightarrow{b}\),
所以\(\left( \overrightarrow{b} - 2\overrightarrow{a} \right) \cdot \overrightarrow{b} = 0\),
即\({\overrightarrow{b}}^{2} = 2\overrightarrow{a} \cdot \overrightarrow{b}\),\par
又因为\(\left| \overrightarrow{a} \right| = 1,\left| \overrightarrow{a} + 2\overrightarrow{b} \right| = 2\),\par
所以\(1 + 4\overrightarrow{a} \cdot \overrightarrow{b} + 4{\overrightarrow{b}}^{2} = 1 + 6{\overrightarrow{b}}^{2} = 4\),\par
从而\(\left| \overrightarrow{b} \right| = \frac{\sqrt{2}}{2}\)}
\end{question}

\begin{question}
某农业研究部门在面积相等的100块稻田上种植一种新型水稻,
得到各块稻田的亩产量(单位:kg)并整理如下表

\begin{center}
\small
\begin{tabular}{|c|c|c|c|c|c|c|}
\hline
亩产量 & {[}900,950) & {[}950,1000) & {[}1000,1050) & {[}1050,1100) & {[}1100,1150) & {[}1150,1200) \\
\hline
频数 & 6 & 12 & 18 & 30 & 24 & 10 \\
\hline
\end{tabular}
\end{center}

根据表中数据,下列结论中正确的是(    )
\begin{choices}
  \item 100块稻田亩产量的中位数小于1050kg
  \item 100块稻田中亩产量低于1100kg的稻田所占比例超过80%
  \item 100块稻田亩产量的极差介于200kg至300kg之间
  \item 100块稻田亩产量的平均值介于900kg至1000kg之间
\end{choices}
\topics{计算几个数的中位数;计算几个数的平均数;计算几个数据的极差;方差;标准差}
\difficulty{0.85}
\answer{C}
\explain{对于 A, 根据频数分布表可知, \(6 + 12 + 18 = 36 < 50\),\par
所以亩产量的中位数不小于 \(1050\text{kg}\), 故 A 错误;\par
对于B,
亩产量不低于\(1100\text{kg}\)的频数为\(24 + 10 = 34\),\par
所以低于\(1100\text{kg}\)的稻田占比为\(\frac{100 - 34}{100} = 66\%\),
故B错误;\par
对于C,稻田亩产量的极差最大为\(1200 - 900 = 300\),
最小为\(1150 - 950 = 200\),故C正确;\par
对于D,由频数分布表可得,
平均值为\(\frac{1}{100} \times (6 \times 925 + 12 \times 975 + 18 \times 1025 + 30 \times 1075 + 24 \times 1125 + 10 \times 1175) = 1067\),
故D错误.\par
故选;C.}
\end{question}

\begin{question}
已知曲线\(C\):\(x^{2} + y^{2} = 16\ (y > 0)\),
从\(C\)上任意一点\(P\)向\(x\)轴作垂线段\(PP'\),
\(P'\)为垂足,则线段\(PP'\)的中点\(M\)的轨迹方程为(    )
\begin{choices}
  \item \(\frac{x^{2}}{16} + \frac{y^{2}}{4} = 1\ (y > 0)\)
  \item \(\frac{x^{2}}{16} + \frac{y^{2}}{8} = 1\ (y > 0)\)
  \item \(\frac{y^{2}}{16} + \frac{x^{2}}{4} = 1\ (y > 0)\)
  \item \(\frac{y^{2}}{16} + \frac{x^{2}}{8} = 1\ (y > 0)\)
\end{choices}
\topics{求平面轨迹方程;轨迹问题------椭圆}
\difficulty{0.85}
\answer{A}
\explain{设点\(M(x,y)\),则\(P(x,y_{0}),P'(x,0)\),\par
因为\(M\)为\(PP'\)的中点,所以\(y_{0} = 2y\),
即\(P(x,2y)\),\par
又\(P\)在圆\(x^{2} + y^{2} = 16(y > 0)\)上,\par
所以\(x^{2} + 4y^{2} = 16(y > 0)\),
即\(\frac{x^{2}}{16} + \frac{y^{2}}{4} = 1(y > 0)\),\par
即点\(M\)的轨迹方程为\(\frac{x^{2}}{16} + \frac{y^{2}}{4} = 1(y > 0)\)}
\end{question}

\begin{question}
设函数\(f(x) = a{(x + 1)}^{2} - 1\),
\(g(x) = \cos x + 2ax\),
当\(x \in ( - 1,1)\)时,
曲线\(y = f(x)\)与\(y = g(x)\)恰有一个交点,
则\(a =\)(    )
\begin{choices}
  \item \(- 1\)
  \item \(\frac{1}{2}\)
  \item 1
  \item 2
\end{choices}
\topics{函数奇偶性的定义与判断;函数奇偶性的应用;根据函数零点的个数求参数范围;求余弦(型)函数的奇偶性}
\difficulty{0.65}
\answer{D}
\explain{解法一:令\(f(x) = g(x)\),
即\(a{(x + 1)}^{2} - 1 = \cos x + 2ax\),
可得\(ax^{2} + a - 1 = \cos x\),\par
令\(F(x) = ax^{2} + a - 1,G(x) = \cos x\),\par
原题意等价于当\(x \in ( - 1,1)\)时,
曲线\(y = F(x)\)与\(y = G(x)\)恰有一个交点,\par
注意到\(F(x),G(x)\)均为偶函数,可知该交点只能在\(y\)轴上,\par
可得\(F(0) = G(0)\),即\(a - 1 = 1\),
解得\(a = 2\),\par
若\(a = 2\),令\(F(x) = G(x)\),
可得\(2x^{2} + 1 - \cos x = 0\)\par
因为\(x \in ( - 1,1)\),
则\(2x^{2} \geq 0,1 - \cos x \geq 0\),
当且仅当\(x = 0\)时,等号成立,\par
可得\(2x^{2} + 1 - \cos x \geq 0\),
当且仅当\(x = 0\)时,等号成立,\par
则方程\(2x^{2} + 1 - \cos x = 0\)有且仅有一个实根0,
即曲线\(y = F(x)\)与\(y = G(x)\)恰有一个交点,\par
所以\(a = 2\)符合题意;\par
综上所述:\(a = 2\).\par
解法二:令\(h(x) = f(x) - g(x) = ax^{2} + a - 1 - \cos x,x \in ( - 1,1)\),\par
原题意等价于\(h(x)\)有且仅有一个零点,\par
因为\(h( - x) = a( - x)^{2} + a - 1 - \cos( - x) = ax^{2} + a - 1 - \cos x = h(x)\),\par
则\(h(x)\)为偶函数,\par
根据偶函数的对称性可知\(h(x)\)的零点只能为0,\par
即\(h(0) = a - 2 = 0\),解得\(a = 2\),\par
若\(a = 2\),
则\(h(x) = 2x^{2} + 1 - \cos x,x \in ( - 1,1)\),\par
又因为\(2x^{2} \geq 0,1 - \cos x \geq 0\)当且仅当\(x = 0\)时,
等号成立,\par
可得\(h(x) \geq 0\),当且仅当\(x = 0\)时,等号成立,\par
即\(h(x)\)有且仅有一个零点0,所以\(a = 2\)符合题意}
\end{question}

\begin{question}
已知正三棱台\(ABC - A_{1}B_{1}C_{1}\)的体积为\(\frac{52}{3}\),
\(AB = 6\),\(A_{1}B_{1} = 2\),
则\(A_{1}A\)与平面\(ABC\)所成角的正切值为(    )
\begin{choices}
  \item \(\frac{1}{2}\)
  \item 1
  \item 2
  \item 3
\end{choices}

\examimage{images/media/image2.png}{0.30}


\examimage{images/media/image3.png}{0.30}

\topics{锥体体积的有关计算;台体体积的有关计算;求线面角}
\difficulty{0.65}
\answer{B}
\explain{解法一:分别取\(BC,B_{1}C_{1}\)的中点\(D,D_{1}\),
则\(AD = 3\sqrt{3},A_{1}D_{1} = \sqrt{3}\),\par
可知\(S_{\triangle ABC} = \frac{1}{2} \times 6 \times 6 \times \frac{\sqrt{3}}{2} = 9\sqrt{3},S_{\triangle A_{1}B_{1}C_{1}} = \frac{1}{2} \times 2 \times \sqrt{3} = \sqrt{3}\),\par
设正三棱台\(ABC - A_{1}B_{1}C_{1}\)的为\(h\),\par
则\(V_{ABC - A_{1}B_{1}C_{1}} = \frac{1}{3}\left( 9\sqrt{3} + \sqrt{3} + \sqrt{9\sqrt{3} \times \sqrt{3}} \right)h = \frac{52}{3}\),
解得\(h = \frac{4\sqrt{3}}{3}\),\par
如图,分别过\(A_{1},D_{1}\)作底面垂线,垂足为\(M,N\),
设\(AM = x\),\par
则\(AA_{1} = \sqrt{AM^{2} + A_{1}M^{2}} = \sqrt{x^{2} + \frac{16}{3}}\),
\(DN = AD - AM - MN = 2\sqrt{3} - x\),\par
可得\(DD_{1} = \sqrt{DN^{2} + D_{1}N^{2}} = \sqrt{\left( 2\sqrt{3} - x \right)^{2} + \frac{16}{3}}\),\par
结合等腰梯形\(BCC_{1}B_{1}\)可得\(BB_{1}^{2} = \left( \frac{6 - 2}{2} \right)^{2} + DD_{1}^{2}\),\par
即\(x^{2} + \frac{16}{3} = \left( 2\sqrt{3} - x \right)^{2} + \frac{16}{3} + 4\),
解得\(x = \frac{4\sqrt{3}}{3}\),\par
所以\(A_{1}A\)与平面\(ABC\)所成角的正切值为\(\tan\angle A_{1}AD = \frac{A_{1}M}{AM} = 1\);\par
解法二:将正三棱台\(ABC - A_{1}B_{1}C_{1}\)补成正三棱锥\(P - ABC\),\par
则\(A_{1}A\)与平面\(ABC\)所成角即为\(PA\)与平面\(ABC\)所成角,\par
因为\(\frac{PA_{1}}{PA} = \frac{A_{1}B_{1}}{AB} = \frac{1}{3}\),
则\(\frac{V_{P - A_{1}B_{1}C_{1}}}{V_{P - ABC}} = \frac{1}{27}\),\par
可知\(V_{ABC - A_{1}B_{1}C_{1}} = \frac{26}{27}V_{P - ABC} = \frac{52}{3}\),
则\(V_{P - ABC} = 18\),\par
设正三棱锥\(P - ABC\)的高为\(d\),
则\(V_{P - ABC} = \frac{1}{3}d \times \frac{1}{2} \times 6 \times 6 \times \frac{\sqrt{3}}{2} = 18\),
解得\(d = 2\sqrt{3}\),\par
取底面\(ABC\)的中心为\(O\),则\(PO\bot\)底面\(ABC\),
且\(AO = 2\sqrt{3}\),\par
所以\(PA\)与平面\(ABC\)所成角的正切值\(\tan\angle PAO = \frac{PO}{AO} = 1\)}
\end{question}

\begin{question}
设函数\(f(x) = (x + a)\ln(x + b)\),若\(f(x) \geq 0\),则\(a^{2} + b^{2}\)的最小值为(    )
\begin{choices}
  \item \(\frac{1}{8}\)
  \item \(\frac{1}{4}\)
  \item \(\frac{1}{2}\)
  \item 1
\end{choices}
\topics{由对数函数的单调性解不等式;函数不等式恒成立问题}
\difficulty{0.4}
\answer{C}
\explain{解法一:由题意可知:\(f(x)\)的定义域为\(( - b, + \infty)\),\par
令\(x + a = 0\)解得\(x = - a\);
令\(\ln(x + b) = 0\)解得\(x = 1 - b\);\par
若\(- a \leq - b\),
当\(x \in ( - b,1 - b)\)时,
可知\(x + a > 0,\ln(x + b) < 0\),\par
此时\(f(x) < 0\),不合题意;\par
若\(- b < - a < 1 - b\),
当\(x \in ( - a,1 - b)\)时,
可知\(x + a > 0,\ln(x + b) < 0\),\par
此时\(f(x) < 0\),不合题意;\par
若\(- a = 1 - b\),
当\(x \in ( - b,1 - b)\)时,
可知\(x + a < 0,\ln(x + b) < 0\),
此时\(f(x) > 0\);\par
当\(x \in \lbrack 1 - b, + \infty)\)时,
可知\(x + a \geq 0,\ln(x + b) \geq 0\),
此时\(f(x) \geq 0\);\par
可知若\(- a = 1 - b\),符合题意;\par
若\(- a > 1 - b\),
当\(x \in (1 - b, - a)\)时,
可知\(x + a < 0,\ln(x + b) > 0\),\par
此时\(f(x) < 0\),不合题意;\par
综上所述:\(- a = 1 - b\),即\(b = a + 1\),\par
则\(a^{2} + b^{2} = a^{2} + (a + 1)^{2} = 2\left( a + \frac{1}{2} \right)^{2} + \frac{1}{2} \geq \frac{1}{2}\),
当且仅当\(a = - \frac{1}{2},b = \frac{1}{2}\)时,
等号成立,\par
所以\(a^{2} + b^{2}\)的最小值为\(\frac{1}{2}\);\par
解法二:由题意可知:\(f(x)\)的定义域为\(( - b, + \infty)\),\par
令\(x + a = 0\)解得\(x = - a\);
令\(\ln(x + b) = 0\)解得\(x = 1 - b\);\par
则当\(x \in ( - b,1 - b)\)时,
\(\ln(x + b) < 0\),故\(x + a \leq 0\),
所以\(1 - b + a \leq 0\);\par
\(x \in (1 - b, + \infty)\)时,
\(\ln(x + b) > 0\),故\(x + a \geq 0\),
所以\(1 - b + a \geq 0\);\par
故\(1 - b + a = 0\),
则\(a^{2} + b^{2} = a^{2} + (a + 1)^{2} = 2\left( a + \frac{1}{2} \right)^{2} + \frac{1}{2} \geq \frac{1}{2}\),\par
当且仅当\(a = - \frac{1}{2},b = \frac{1}{2}\)时,
等号成立,\par
所以\(a^{2} + b^{2}\)的最小值为\(\frac{1}{2}\)}
\end{question}

\section{多选题}

\begin{question}
对于函数\(f(x) = \sin 2x\)和\(g(x) = \sin(2x - \frac{\pi}{4})\),下列说法中正确的有(    )
\begin{choices}
  \item \(f(x)\)与\(g(x)\)有相同的零点
  \item \(f(x)\)与\(g(x)\)有相同的最大值
  \item \(f(x)\)与\(g(x)\)有相同的最小正周期
  \item \(f(x)\)与\(g(x)\)的图象有相同的对称轴
\end{choices}
\topics{求含sinx(型)函数的值域和最值;求正弦(型)函数的最小正周期;求正弦(型)函数的对称轴及对称中心;求函数零点或方程根的个数}
\difficulty{0.65}
\answer{BC}
\explain{A选项,令\(f(x) = \sin 2x = 0\),
解得\(x = \frac{k\pi}{2},k \in Z\),即为\(f(x)\)零点,\par
令\(g(x) = \sin(2x - \frac{\pi}{4}) = 0\),
解得\(x = \frac{k\pi}{2} + \frac{\pi}{8},k \in Z\),
即为\(g(x)\)零点,\par
显然\(f(x),g(x)\)零点不同,A选项错误;\par
B选项,
显然\(f{(x)}_{\max} = g{(x)}_{\max} = 1\),B选项正确;\par
C选项,根据周期公式,
\(f(x),g(x)\)的周期均为\(\frac{2\pi}{2} = \pi\),C选项正确;\par
D选项,
根据正弦函数的性质\(f(x)\)的对称轴满足\(2x = k\pi + \frac{\pi}{2} \Leftrightarrow x = \frac{k\pi}{2} + \frac{\pi}{4},k \in Z\),\par
\(g(x)\)的对称轴满足\(2x - \frac{\pi}{4} = k\pi + \frac{\pi}{2} \Leftrightarrow x = \frac{k\pi}{2} + \frac{3\pi}{8},k \in Z\),\par
显然\(f(x),g(x)\)图像的对称轴不同,D选项错误}
\end{question}

\begin{question}
抛物线\(C\):\(y^{2} = 4x\)的准线为\(l\),
\(P\)为\(C\)上的动点,
过\(P\)作\(\odot A:x^{2} + {(y - 4)}^{2} = 1\)的一条切线,
\(Q\)为切点,过\(P\)作\(l\)的垂线,垂足为\(B\),
则(    )
\begin{choices}
  \item \(l\)与\(\odot A\)相切
  \item 当\(P\),\(A\),\(B\)三点共线时,\(|PQ| = \sqrt{15}\)
  \item 当\(|PB| = 2\)时,\(PA\bot AB\)
  \item 满足\(|PA| = |PB|\)的点\(P\)有且仅有2个
\end{choices}

\examimage{images/media/image4.png}{0.30}

\topics{切线长;根据抛物线方程求焦点或准线;直线与抛物线交点相关问题}
\difficulty{0.65}
\answer{ABD}
\explain{A选项,抛物线\(y^{2} = 4x\)的准线为\(x = - 1\),\par
\(\odot A\)的圆心\((0,4)\)到直线\(x = - 1\)的距离显然是\(1\),
等于圆的半径,\par
故准线\(l\)和\(\odot A\)相切,A选项正确;\par
B选项,\(P,A,B\)三点共线时,即\(PA\bot l\),
则\(P\)的纵坐标\(y_{P} = 4\),\par
由\(y_{P}^{2} = 4x_{P}\),得到\(x_{P} = 4\),
故\(P(4,4)\),\par
此时切线长\(|PQ| = \sqrt{|PA|^{2} - r^{2}} = \sqrt{4^{2} - 1^{2}} = \sqrt{15}\),
B选项正确;\par
C选项,当\(|PB| = 2\)时,\(x_{P} = 1\),
此时\(y_{P}^{2} = 4x_{P} = 4\),
故\(P(1,2)\)或\(P(1, - 2)\),\par
当\(P(1,2)\)时,\(A(0,4),B( - 1,2)\),
\(k_{PA} = \frac{4 - 2}{0 - 1} = - 2\),
\(k_{AB} = \frac{4 - 2}{0 - ( - 1)} = 2\),\par
不满足\(k_{PA}k_{AB} = - 1\);\par
当\(P(1, - 2)\)时,\(A(0,4),B( - 1, - 2)\),
\(k_{PA} = \frac{4 - ( - 2)}{0 - 1} = - 6\),
\(k_{AB} = \frac{4 - ( - 2)}{0 - ( - 1)} = 6\),\par
不满足\(k_{PA}k_{AB} = - 1\);\par
于是\(PA\bot AB\)不成立,C选项错误;\par
D选项,方法一:利用抛物线定义转化\par
根据抛物线的定义,\(|PB| = |PF|\),这里\(F(1,0)\),\par
于是\(|PA| = |PB|\)时\(P\)点的存在性问题转化成\(|PA| = |PF|\)时\(P\)点的存在性问题,\par
\(A(0,4),F(1,0)\),
\(AF\)中点\(\left( \frac{1}{2},2 \right)\),
\(AF\)中垂线的斜率为\(- \frac{1}{k_{AF}} = \frac{1}{4}\),\par
于是\(AF\)的中垂线方程为:\(y = \frac{2x + 15}{8}\),
与抛物线\(y^{2} = 4x\)联立可得\(y^{2} - 16y + 30 = 0\),\par
\(\Delta = 16^{2} - 4 \times 30 = 136 > 0\),
即\(AF\)的中垂线和抛物线有两个交点,\par
即存在两个\(P\)点,使得\(|PA| = |PF|\),D选项正确.\par
方法二:(设点直接求解)\par
设\(P\left( \frac{t^{2}}{4},t \right)\),
由\(PB\bot l\)可得\(B( - 1,t)\),又\(A(0,4)\),
又\(|PA| = |PB|\),\par
根据两点间的距离公式,
\(\sqrt{\frac{t^{4}}{16} + {(t - 4)}^{2}} = \frac{t^{2}}{4} + 1\),
整理得\(t^{2} - 16t + 30 = 0\),\par
\(\Delta = 16^{2} - 4 \times 30 = 136 > 0\),
则关于\(t\)的方程有两个解,\par
即存在两个这样的\(P\)点,D选项正确}
\end{question}

\begin{question}
设函数\(f(x) = 2x^{3} - 3ax^{2} + 1\),则(    )
\begin{choices}
  \item 当\(a > 1\)时,\(f(x)\)有三个零点
  \item 当\(a\text{<0}\)时,\(x = 0\)是\(f(x)\)的极大值点
  \item 存在\(a\),\(b\),使得\(x = b\)为曲线\(y = f(x)\)的对称轴
  \item 存在\(a\),使得点\(\left( 1,f(1) \right)\)为曲线\(y = f(x)\)的对称中心
\end{choices}
\topics{函数对称性的应用;函数单调性;极值与最值的综合应用;利用导数研究函数的零点;判断零点所在的区间}
\difficulty{0.4}
\answer{AD}
\explain{A选项,\(f'(x) = 6x^{2} - 6ax = 6x(x - a)\),
由于\(a > 1\),\par
故\(x \in ( - \infty,0) \cup (a, + \infty)\)时\(f'(x) > 0\),
故\(f(x)\)在\(( - \infty,0),(a, + \infty)\)上单调递增,\par
\(x \in (0,a)\)时,\(f'(x) < 0\),
\(f(x)\)单调递减,\par
则\(f(x)\)在\(x = 0\)处取到极大值,
在\(x = a\)处取到极小值,\par
由\(f(0) = 1 > 0\),
\(f(a) = 1 - a^{3} < 0\),
则\(f(0)f(a) < 0\),\par
根据零点存在定理\(f(x)\)在\((0,a)\)上有一个零点,\par
又\(f( - 1) = - 1 - 3a < 0\),
\(f(2a) = 4a^{3} + 1 > 0\),
则\(f( - 1)f(0) < 0,f(a)f(2a) < 0\),\par
则\(f(x)\)在\(( - 1,0),(a,2a)\)上各有一个零点,
于是\(a > 1\)时,\(f(x)\)有三个零点,A选项正确;\par
B选项,\(f'(x) = 6x(x - a)\),\(a < 0\)时,
\(x \in (a,0),f'(x) < 0\),\(f(x)\)单调递减,\par
\(x \in (0, + \infty)\)时\(f'(x) > 0\),
\(f(x)\)单调递增,\par
此时\(f(x)\)在\(x = 0\)处取到极小值,B选项错误;\par
C选项,假设存在这样的\(a,b\),
使得\(x = b\)为\(f(x)\)的对称轴,\par
即存在这样的\(a,b\)使得\(f(x) = f(2b - x)\),\par
即\(2x^{3} - 3ax^{2} + 1 = 2{(2b - x)}^{3} - 3a{(2b - x)}^{2} + 1\),\par
根据二项式定理,
等式右边\({(2b - x)}^{3}\)展开式含有\(x^{3}\)的项为\(\text{2C}_{3}^{3}{(2b)}^{0}{( - x)}^{3} = - 2x^{3}\),\par
于是等式左右两边\(x^{3}\)的系数都不相等,原等式不可能恒成立,\par
于是不存在这样的\(a,b\),使得\(x = b\)为\(f(x)\)的对称轴,
C选项错误;\par
D选项,\par
\textbf{方法一:利用对称中心的表达式化简}\par
\(f(1) = 3 - 3a\),若存在这样的\(a\),
使得\((1,3 - 3a)\)为\(f(x)\)的对称中心,\par
则\(f(x) + f(2 - x) = 6 - 6a\),事实上,\par
\(f(x) + f(2 - x) = 2x^{3} - 3ax^{2} + 1 + 2{(2 - x)}^{3} - 3a{(2 - x)}^{2} + 1 = (12 - 6a)x^{2} + (12a - 24)x + 18 - 12a\),\par
于是\(6 - 6a = (12 - 6a)x^{2} + (12a - 24)x + 18 - 12a\)\par
即\(\left\{ \begin{array}{r}
12 - 6a = 0 \\
12a - 24 = 0 \\
18 - 12a = 6 - 6a
\end{array} \right.\),解得\(a = 2\),即存在\(a = 2\)使得\((1,f(1))\)是\(f(x)\)的对称中心,D选项正确.\par
\textbf{方法二:直接利用拐点结论}\par
任何三次函数都有对称中心,对称中心的横坐标是二阶导数的零点,\par
\(f(x) = 2x^{3} - 3ax^{2} + 1\),\(f'(x) = 6x^{2} - 6ax\),\(f''(x) = 12x - 6a\),\par
由\(f''(x) = 0 \Leftrightarrow x = \frac{a}{2}\),于是该三次函数的对称中心为\(\left( \frac{a}{2},f\left( \frac{a}{2} \right) \right)\),\par
由题意\((1,f(1))\)也是对称中心,故\(\frac{a}{2} = 1 \Leftrightarrow a = 2\),\par
即存在\(a = 2\)使得\((1,f(1))\)是\(f(x)\)的对称中心,D选项正确}
\end{question}

\section{填空题}

\begin{question}
记\(S_{n}\)为等差数列\(\{ a_{n}\}\)的前\(n\)项和,
若\(a_{3} + a_{4} = 7\),
\(3a_{2} + a_{5} = 5\),则\(S_{10} =\)
.
\topics{等差数列通项公式的基本量计算;求等差数列前n项和}
\difficulty{0.85}
\answer{95}
\explain{因为数列\(a_{n}\)为等差数列,则由题意得\(\left\{ \begin{array}{r}
a_{1} + 2d + a_{1} + 3d = 7 \\
3\left( a_{1} + d \right) + a_{1} + 4d = 5
\end{array} \right.\),解得\(\left\{ \begin{array}{r}
a_{1} = - 4 \\
d = 3
\end{array} \right.\),\par
则\(S_{10} = 10a_{1} + \frac{10 \times 9}{2}d = 10 \times ( - 4) + 45 \times 3 = 95\).\(95\).}
\end{question}

\begin{question}
已知\(\alpha\)为第一象限角,\(\beta\)为第三象限角,
\(\tan\alpha + \tan\beta = 4\),
\(\tan\alpha\tan\beta = \sqrt{2} + 1\),
则\(\sin(\alpha + \beta) =\)
.
\topics{用和;差角的正切公式化简;求值}
\difficulty{0.85}
\answer{\(- \frac{2\sqrt{2}}{3}\)}
\explain{法一:由题意得\(\tan(\alpha + \beta) = \frac{\tan\alpha + \tan\beta}{1 - \tan\alpha\tan\beta} = \frac{4}{1 - \left( \sqrt{2} + 1 \right)} = - 2\sqrt{2}\),\par
因为\(\alpha \in \left( 2k\pi,2k\pi + \frac{\pi}{2} \right),\beta \in \left( 2m\pi + \pi,2m\pi + \frac{3\pi}{2} \right)\),
\(k,m \in Z\),\par
则\(\alpha + \beta \in \left( (2m + 2k)\pi + \pi,(2m + 2k)\pi + 2\pi \right)\),
\(k,m \in Z\),\par
又因为\(\tan(\alpha + \beta) = - 2\sqrt{2} < 0\),\par
则\(\alpha + \beta \in \left( (2m + 2k)\pi + \frac{3\pi}{2},(2m + 2k)\pi + 2\pi \right)\),
\(k,m \in Z\),
则\(\sin(\alpha + \beta) < 0\),\par
则\(\frac{\sin(\alpha + \beta)}{\cos(\alpha + \beta)} = - 2\sqrt{2}\),
联立
\(\sin^{2}(\alpha + \beta) + \cos^{2}(\alpha + \beta) = 1\),
解得\(\sin(\alpha + \beta) = - \frac{2\sqrt{2}}{3}\).\par
法二:
因为\(\alpha\)为第一象限角,\(\beta\)为第三象限角,
则\(\cos\alpha > 0,\cos\beta < 0\),\par
\(\cos\alpha = \frac{\cos\alpha}{\sqrt{\sin^{2}\alpha + \cos^{2}\alpha}} = \frac{1}{\sqrt{1 + \tan^{2}\alpha}}\),
\(\cos\beta = \frac{\cos\beta}{\sqrt{\sin^{2}\beta + \cos^{2}\beta}} = \frac{- 1}{\sqrt{1 + \tan^{2}\beta}}\),\par
则\(\sin(\alpha + \beta) = \sin\alpha\cos\beta + \cos\alpha\sin\beta = \cos\alpha\cos\beta(\tan\alpha + \tan\beta)= 4\cos\alpha\cos\beta = \frac{- 4}{\sqrt{1 + \tan^{2}\alpha}\sqrt{1 + \tan^{2}\beta}} = \frac{- 4}{\sqrt{{(\tan\alpha + \tan\beta)}^{2} + {(\tan\alpha\tan\beta - 1)}^{2}}} = \frac{- 4}{\sqrt{4^{2} + 2}} = - \frac{2\sqrt{2}}{3}\)\(- \frac{2\sqrt{2}}{3}\).}
\end{question}

\begin{question}
在如图的4×4的方格表中选4个方格,要求每行和每列均恰有一个方格被选中,则共有
种选法,在所有符合上述要求的选法中,选中方格中的4个数之和的最大值是 .

\examimage{images/media/image5.png}{0.30}

\topics{全排列问题;写出样本空间}
\difficulty{0.4}
\answer{24 112}
\explain{由题意知,选4个方格,每行和每列均恰有一个方格被选中,\par
则第一列有4个方格可选,第二列有3个方格可选,\par
第三列有2个方格可选,第四列有1个方格可选,\par
所以共有\(4 \times 3 \times 2 \times 1 = 24\)种选法;\par
每种选法可标记为\((a,b,c,d)\),
\(a,b,c,d\)分别表示第一、二、三、四列的数字,\par
则所有的可能结果为:\par
\((11,22,33,44),(11,22,34,43),(11,22,33,44),(11,22,34,42),(11,24,33,43),(11,24,33,42)\),\par
\((12,21,33,44),(12,21,34,43),(12,22,31,44),(12,22,34,40),(12,24,31,43),(12,24,33,40)\),\par
\((13,21,33,44),(13,21,34,42),(13,22,31,44),(13,22,34,40),(13,24,31,42),(13,24,33,40)\),\par
\((15,21,33,43),(15,21,33,42),(15,22,31,43),(15,22,33,40),(15,22,31,42),(15,22,33,40)\),\par
所以选中的方格中,\((15,21,33,43)\)的4个数之和最大,
为\(15 + 21 + 33 + 43 = 112\).24;112}
\end{question}

\section{解答题}

\begin{question}
记\(\triangle ABC\)的内角\(A\),\(B\),
\(C\)的对边分别为\(a\),\(b\),\(c\),
已知\(\sin A + \sqrt{3}\cos A = 2\).
\begin{enumerate}[label=(\arabic*)]
  \item 求\(A\).若\(a = 2\),
\(\sqrt{2}b\sin C = c\sin 2B\),
求\(\triangle ABC\)的周长.
\end{enumerate}
\topics{辅助角公式;正弦定理解三角形;正弦定理边角互化的应用}
\difficulty{0.65}
\answer{(1)\(A = \frac{\pi}{6}\)
(2)\(2 + \sqrt{6} + 3\sqrt{2}\)}
\explain{(1)\textbf{方法一:常规方法(辅助角公式)}\par
由\(\sin A + \sqrt{3}\cos A = 2\)可得\(\frac{1}{2}\sin A + \frac{\sqrt{3}}{2}\cos A = 1\),
即\(\sin(A + \frac{\pi}{3}) = 1\),\par
由于\(A \in (0,\pi) \Rightarrow A + \frac{\pi}{3} \in (\frac{\pi}{3},\frac{4\pi}{3})\),
故\(A + \frac{\pi}{3} = \frac{\pi}{2}\),
解得\(A = \frac{\pi}{6}\)\par
\textbf{方法二:常规方法(同角三角函数的基本关系)}\par
由\(\sin A + \sqrt{3}\cos A = 2\),
又\(\sin^{2}A + \cos^{2}A = 1\),
消去\(\sin A\)得到:\par
\(4\cos^{2}A - 4\sqrt{3}\cos A + 3 = 0 \Leftrightarrow {(2\cos A - \sqrt{3})}^{2} = 0\),
解得\(\cos A = \frac{\sqrt{3}}{2}\),\par
又\(A \in (0,\pi)\),
故\(A = \frac{\pi}{6}\)\par
\textbf{方法三:利用极值点求解}\par
设\(f(x) = \sin x + \sqrt{3}\cos x(0 < x < \pi)\),
则\(f(x) = 2\sin\left( x + \frac{\pi}{3} \right)(0 < x < \pi)\),\par
显然\(x = \frac{\pi}{6}\)时,
\(f{(x)}_{\max} = 2\),
注意到\(f(A) = \sin A + \sqrt{3}\cos A = 2 = 2\sin(A + \frac{\pi}{3})\),\par
\(f{(x)}_{\max} = f(A)\),
在开区间\((0,\pi)\)上取到最大值,于是\(x = A\)必定是极值点,\par
即\(f'(A) = 0 = \cos A - \sqrt{3}\sin A\),
即\(\tan A = \frac{\sqrt{3}}{3}\),\par
又\(A \in (0,\pi)\),
故\(A = \frac{\pi}{6}\)\par
\textbf{方法四:利用向量数量积公式(柯西不等式)}\par
设\(\overrightarrow{a} = (1,\sqrt{3}),\overrightarrow{b} = (\sin A,\cos A)\),
由题意,
\(\overrightarrow{a} \cdot \overrightarrow{b} = \sin A + \sqrt{3}\cos A = 2\),\par
根据向量的数量积公式,
\(\overrightarrow{a} \cdot \overrightarrow{b} = |\overrightarrow{a}||\overrightarrow{b}|\cos\left\langle \overrightarrow{a},\overrightarrow{b} \right\rangle = 2\cos\left\langle \overrightarrow{a},\overrightarrow{b} \right\rangle\),\par
则\(2\cos\overrightarrow{a},\overrightarrow{b} = 2 \Leftrightarrow \cos\overrightarrow{a},\overrightarrow{b} = 1\),
此时\(\overrightarrow{a},\overrightarrow{b} = 0\),
即\(\overrightarrow{a},\overrightarrow{b}\)同向共线,\par
根据向量共线条件,
\(1 \cdot \cos A = \sqrt{3} \cdot \sin A \Leftrightarrow \tan A = \frac{\sqrt{3}}{3}\),\par
又\(A \in (0,\pi)\),
故\(A = \frac{\pi}{6}\)\par
\textbf{方法五:利用万能公式求解}\par
设\(t = \tan\frac{A}{2}\),根据万能公式,
\(\sin A + \sqrt{3}\cos A = 2 = \frac{2t}{1 + t^{2}} + \frac{\sqrt{3}(1 - t^{2})}{1 + t^{2}}\),\par
整理可得,
\(t^{2} - 2(2 - \sqrt{3})t + {(2 - \sqrt{3})}^{2} = 0 = {(t - (2 - \sqrt{3}))}^{2}\),\par
解得\(\tan\frac{A}{2} = t = 2 - \sqrt{3}\),
根据二倍角公式,
\(\tan A = \frac{2t}{1 - t^{2}} = \frac{\sqrt{3}}{3}\),\par
又\(A \in (0,\pi)\),
故\(A = \frac{\pi}{6}\)\par
(2)由题设条件和正弦定理\par
\(\sqrt{2}b\sin C = c\sin 2B \Leftrightarrow \sqrt{2}\sin B\sin C = 2\sin C\sin B\cos B\),\par
又\(B,C \in (0,\pi)\),
则\(\sin B\sin C \neq 0\),
进而\(\cos B = \frac{\sqrt{2}}{2}\),得到\(B = \frac{\pi}{4}\),\par
于是\(C = \pi - A - B = \frac{7\pi}{12}\),\par
\(\sin C = \sin(\pi - A - B) = \sin(A + B) = \sin A\cos B + \sin B\cos A = \frac{\sqrt{2} + \sqrt{6}}{4}\),\par
由正弦定理可得,
\(\frac{a}{\sin A} = \frac{b}{\sin B} = \frac{c}{\sin C}\),
即\(\frac{2}{\sin\frac{\pi}{6}} = \frac{b}{\sin\frac{\pi}{4}} = \frac{c}{\sin\frac{7\pi}{12}}\),\par
解得\(b = 2\sqrt{2},c = \sqrt{6} + \sqrt{2}\),\par
故\(\triangle ABC\)的周长为\(2 + \sqrt{6} + 3\sqrt{2}\)}
\end{question}

\begin{question}
已知函数\(f(x) = \mathrm{e}^{x} - ax - a^{3}\).
\begin{enumerate}[label=(\arabic*)]
  \item 当\(a = 1\)时,求曲线\(y = f(x)\)在点\(\left( 1,f(1) \right)\)处的切线方程;
  \item 若\(f(x)\)有极小值,且极小值小于0,求\(a\)的取值范围.
\end{enumerate}
\topics{求在曲线上一点处的切线方程(斜率);根据极值求参数}
\difficulty{0.65}
\answer{(1)\(\left( \text{e} - 1 \right)x - y - 1 = 0\)
(2)\((1, + \infty)\)}
\explain{(1)当\(a = 1\)时,
则\(f(x) = \mathrm{e}^{x} - x - 1\),
\(f'(x) = \mathrm{e}^{x} - 1\),\par
可得\(f(1) = \text{e} - 2\),
\(f'(1) = \text{e} - 1\),\par
即切点坐标为\(\left( 1,\text{e} - 2 \right)\),
切线斜率\(k = \text{e} - 1\),\par
所以切线方程为\(y - \left( \text{e} - 2 \right) = \left( \text{e} - 1 \right)(x - 1)\),
即\(\left( \text{e} - 1 \right)x - y - 1 = 0\).\par
(2)解法一:因为\(f(x)\)的定义域为\(R\),
且\(f'(x) = \mathrm{e}^{x} - a\),\par
若\(a \leq 0\),
则\(f'(x) \geq 0\)对任意\(x \in R\)恒成立,\par
可知\(f(x)\)在\(R\)上单调递增,无极值,不合题意;\par
若\(a > 0\),令\(f'(x) > 0\),
解得\(x > \ln a\);令\(f'(x) < 0\),
解得\(x < \ln a\);\par
可知\(f(x)\)在\(\left( - \infty,\ln a \right)\)内单调递减,
在\(\left( \ln a, + \infty \right)\)内单调递增,\par
则\(f(x)\)有极小值\(f\left( \ln a \right) = a - a\ln a - a^{3}\),
无极大值,\par
由题意可得:\(f\left( \ln a \right) = a - a\ln a - a^{3} < 0\),
即\(a^{2} + \ln a - 1 > 0\),\par
构建\(g(a) = a^{2} + \ln a - 1,a > 0\),
则\(g'(a) = 2a + \frac{1}{a} > 0\),\par
可知\(g(a)\)在\((0, + \infty)\)内单调递增,
且\(g(1) = 0\),\par
不等式\(a^{2} + \ln a - 1 > 0\)等价于\(g(a) > g(1)\),
解得\(a > 1\),\par
所以\(a\)的取值范围为\((1, + \infty)\);\par
解法二:因为\(f(x)\)的定义域为\(R\),
且\(f'(x) = \mathrm{e}^{x} - a\),\par
若\(f(x)\)有极小值,
则\(f'(x) = \mathrm{e}^{x} - a\)有零点,\par
令\(f'(x) = \mathrm{e}^{x} - a = 0\),
可得\(\mathrm{e}^{x} = a\),\par
可知\(y = \mathrm{e}^{x}\)与\(y = a\)有交点,
则\(a > 0\),\par
若\(a > 0\),令\(f'(x) > 0\),
解得\(x > \ln a\);令\(f'(x) < 0\),
解得\(x < \ln a\);\par
可知\(f(x)\)在\(\left( - \infty,\ln a \right)\)内单调递减,
在\(\left( \ln a, + \infty \right)\)内单调递增,\par
则\(f(x)\)有极小值\(f\left( \ln a \right) = a - a\ln a - a^{3}\),
无极大值,符合题意,\par
由题意可得:\(f\left( \ln a \right) = a - a\ln a - a^{3} < 0\),
即\(a^{2} + \ln a - 1 > 0\),\par
构建\(g(a) = a^{2} + \ln a - 1,a > 0\),\par
因为则\(y = a^{2},y = \ln a - 1\)在\((0, + \infty)\)内单调递增,\par
可知\(g(a)\)在\((0, + \infty)\)内单调递增,
且\(g(1) = 0\),\par
不等式\(a^{2} + \ln a - 1 > 0\)等价于\(g(a) > g(1)\),
解得\(a > 1\),\par
所以\(a\)的取值范围为\((1, + \infty)\).}
\end{question}

\begin{question}
如图,平面四边形\(ABCD\)中,\(AB = 8\),\(CD = 3\),
\(AD = 5\sqrt{3}\),
\(\angle ADC = 90^{{^\circ}}\),
\(\angle BAD = 30^{{^\circ}}\),点\(E\),
\(F\)满足\(\overrightarrow{AE} = \frac{2}{5}\overrightarrow{AD}\),
\(\overrightarrow{AF} = \frac{1}{2}\overrightarrow{AB}\),
将\(\triangle AEF\)沿\(EF\)翻折至\(\triangle PEF\),
使得\(PC = 4\sqrt{3}\).
\begin{enumerate}[label=(\arabic*)]
  \item 证明:\(EF\bot PD\);
  \item 求平面\(PCD\)与平面\(PBF\)所成的二面角的正弦值.
\end{enumerate}

\examimage{images/media/image6.png}{0.30}


\examimage{images/media/image7.png}{0.30}

\topics{证明线面垂直;线面垂直证明线线垂直;求平面的法向量;面面角的向量求法}
\difficulty{0.65}
\answer{(1)证明见解析
(2)\(\frac{8\sqrt{65}}{65}\)}
\explain{(1)由\(AB = 8,AD = 5\sqrt{3},\overrightarrow{AE} = \frac{2}{5}\overrightarrow{AD},\overrightarrow{AF} = \frac{1}{2}\overrightarrow{AB}\),\par
得\(AE = 2\sqrt{3},AF = 4\),
又\(\angle BAD = 30^{{^\circ}}\),
在\(\triangle AEF\)中,\par
由余弦定理得\(EF = \sqrt{AE^{2} + AF^{2} - 2AE \cdot AF\cos\angle BAD} = \sqrt{16 + 12 - 2 \cdot 4 \cdot 2\sqrt{3} \cdot \frac{\sqrt{3}}{2}} = 2\),\par
所以\(AE^{2} + EF^{2} = AF^{2}\),
则\(AE\bot EF\),即\(EF\bot AD\),\par
所以\(EF\bot PE,EF\bot DE\),
又\(PE \cap DE = E,PE,DE \subset\)平面\(PDE\),\par
所以\(EF\bot\)平面\(PDE\),
又\(PD \subset\)平面\(PDE\),\par
故\(EF\bot\) \(PD\);\par
(2)连接\(CE\),
由\(\angle ADC = 90^{{^\circ}},ED = 3\sqrt{3},CD = 3\),
则\(CE^{2} = ED^{2} + CD^{2} = 36\),\par
在\(\triangle \text{PEC}\)中,
\(PC = 4\sqrt{3},PE = 2\sqrt{3},EC = 6\),
得\(EC^{2} + PE^{2} = PC^{2}\),\par
所以\(PE\bot EC\),由(1)知\(PE\bot EF\),
又\(EC \cap EF = E,EC,EF \subset\)平面\(ABCD\),\par
所以\(PE\bot\)平面\(ABCD\),
又\(ED \subset\)平面\(ABCD\),\par
所以\(PE\bot ED\),则\(PE,EF,ED\)两两垂直,
建立如图空间直角坐标系\(E - xyz\),\par
则\(E(0,0,0),P(0,0,2\sqrt{3}),D(0,3\sqrt{3},0),C(3,3\sqrt{3},0),F(2,0,0),A(0, - 2\sqrt{3},0)\),\par
由\(F\)是\(AB\)的中点,得\(B(4,2\sqrt{3},0)\),\par
所以\(\overrightarrow{PC} = (3,3\sqrt{3}, - 2\sqrt{3}),\overrightarrow{PD} = (0,3\sqrt{3}, - 2\sqrt{3}),\overrightarrow{PB} = (4,2\sqrt{3}, - 2\sqrt{3}),\overrightarrow{PF} = (2,0, - 2\sqrt{3})\),\par
设平面\(PCD\)和平面\(PBF\)的一个法向量分别为\(\overrightarrow{n} = (x_{1},y_{1},z_{1}),\overrightarrow{m} = (x_{2},y_{2},z_{2})\),\par
则\(\left\{ \begin{array}{r}
\overrightarrow{n} \cdot \overrightarrow{PC} = 3x_{1} + 3\sqrt{3}y_{1} - 2\sqrt{3}z_{1} = 0 \\
\overrightarrow{n} \cdot \overrightarrow{PD} = 3\sqrt{3}y_{1} - 2\sqrt{3}z_{1} = 0
\end{array} \right.\),\(\left\{ \begin{array}{r}
\overrightarrow{m} \cdot \overrightarrow{PB} = 4x_{2} + 2\sqrt{3}y_{2} - 2\sqrt{3}z_{2} = 0 \\
\overrightarrow{m} \cdot \overrightarrow{PF} = 2x_{2} - 2\sqrt{3}z_{2} = 0
\end{array} \right.\),\par
令\(y_{1} = 2,x_{2} = \sqrt{3}\),得\(x_{1} = 0,z_{1} = 3,y_{2} = - 1,z_{2} = 1\),\par
所以\(\overrightarrow{n} = (0,2,3),\overrightarrow{m} = (\sqrt{3}, - 1,1)\),\par
所以\(\left| \cos\left\langle \overrightarrow{m},\overrightarrow{n} \right\rangle \right| = \frac{\left| \overrightarrow{m} \cdot \overrightarrow{n} \right|}{\left| \overrightarrow{m} \right|\left| \overrightarrow{n} \right|} = \frac{1}{\sqrt{5} \cdot \sqrt{13}} = \frac{\sqrt{65}}{65}\),\par
设平面\(PCD\)和平面\(PBF\)所成角为\(\theta\),则\(\sin\theta = \sqrt{1 - \cos^{2}\theta} = \frac{8\sqrt{65}}{65}\),\par
即平面\(PCD\)和平面\(PBF\)所成角的正弦值为\(\frac{8\sqrt{65}}{65}\).}
\end{question}

\begin{question}
某投篮比赛分为两个阶段,每个参赛队由两名队员组成,
比赛具体规则如下:第一阶段由参赛队中一名队员投篮3次,若3次都未投中,
则该队被淘汰,比赛成绩为0分;若至少投中一次,
则该队进入第二阶段.第二阶段由该队的另一名队员投篮3次,每次投篮投中得5分,
未投中得0分.该队的比赛成绩为第二阶段的得分总和.某参赛队由甲、乙两名队员组成,
设甲每次投中的概率为\(p\),乙每次投中的概率为\(q\),
各次投中与否相互独立.
\begin{enumerate}[label=(\arabic*)]
  \item 若\(p = 0.4\),\(q = 0.5\),甲参加第一阶段比赛,
求甲、乙所在队的比赛成绩不少于5分的概率.
  \item 假设\(0 < p < q\),

(i)为使得甲、乙所在队的比赛成绩为15分的概率最大,应该由谁参加第一阶段比赛?

(ii)为使得甲、乙所在队的比赛成绩的数学期望最大,应该由谁参加第一阶段比赛?
\end{enumerate}
\topics{利用对立事件的概率公式求概率;独立事件的乘法公式;求离散型随机变量的均值}
\difficulty{0.4}
\answer{(1)\(0.686\)
(2)(i)由甲参加第一阶段比赛;(i)由甲参加第一阶段比赛;}
\explain{(1)甲、乙所在队的比赛成绩不少于5分,则甲第一阶段至少投中1次,
乙第二阶段也至少投中1次,\par
\(\therefore\)比赛成绩不少于5分的概率\(P = \left( 1 - {0.6}^{3} \right)\left( 1 - {0.5}^{3} \right) = 0.686\).\par
(2)(i)若甲先参加第一阶段比赛,
则甲、乙所在队的比赛成绩为15分的概率为\(P_{甲} = \left\lbrack 1 - {(1 - p)}^{3} \right\rbrack q^{3}\),\par
若乙先参加第一阶段比赛,
则甲、乙所在队的比赛成绩为15分的概率为\(P_{乙} = \left\lbrack 1 - {(1 - q)}^{3} \right\rbrack \cdot p^{3}\),\par
\(\because 0 < p < q\),\par
\(\therefore P_{甲} - P_{乙} = q^{3} - {(q - pq)}^{3} - p^{3} + {(p - pq)}^{3}= (q - p)\left( q^{2} + pq + p^{2} \right) + (p - q) \cdot \left\lbrack {(p - pq)}^{2} + {(q - pq)}^{2} + (p - pq)(q - pq) \right\rbrack= (p - q)\left( 3p^{2}q^{2} - 3p^{2}q - 3pq^{2} \right)= 3pq(p - q)(pq - p - q) = 3pq(p - q)\lbrack(1 - p)(1 - q) - 1\rbrack > 0\),\par
\(\therefore P_{甲} > P_{乙}\),
应该由甲参加第一阶段比赛.\par
(ii)若甲先参加第一阶段比赛,比赛成绩\(X\)的所有可能取值为0,5,10,
15,\par
\(P(X = 0) = {(1 - p)}^{3} + \left\lbrack 1 - {(1 - p)}^{3} \right\rbrack \cdot {(1 - q)}^{3}\),\par
\(P(X = 5) = \left\lbrack 1 - {(1 - p)}^{3} \right\rbrack C_{3}^{1}q \cdot {(1 - q)}^{2}\),\par
\(P(X = 10) = \left\lbrack 1 - {(1 - p)}^{3} \right\rbrack \cdot \mathbb{C}_{3}^{2}q^{2}(1 - q)\),\par
\(P(X = 15) = \left\lbrack 1 - {(1 - p)}^{3} \right\rbrack \cdot q^{3}\),\par
\(\therefore E(X) = 15\left\lbrack 1 - {(1 - p)}^{3} \right\rbrack q = 15\left( p^{3} - 3p^{2} + 3p \right) \cdot q\)\par
记乙先参加第一阶段比赛,比赛成绩\(Y\)的所有可能取值为0,5,10,15,\par
同理\(E(Y) = 15\left( q^{3} - 3q^{2} + 3q \right) \cdot p\therefore E(X) - E(Y) = 15\lbrack pq(p + q)(p - q) - 3pq(p - q)\rbrack= 15(p - q)pq(p + q - 3)\),\par
因为\(0 < p < q\),则\(p - q < 0\),
\(p + q - 3 < 1 + 1 - 3 < 0\),\par
则\((p - q)pq(p + q - 3) > 0\),\par
\(\therefore\)应该由甲参加第一阶段比赛.}
\end{question}

\begin{question}
已知双曲线\(C:x^{2} - y^{2} = m(m > 0)\),
点\(P_{1}(5,4)\)在\(C\)上,\(k\)为常数,
\(0 < k < 1\).按照如下方式依次构造点\(P_{n}(n = 2,3,...)\):过\(P_{n - 1}\)作斜率为\(k\)的直线与\(C\)的左支交于点\(Q_{n - 1}\),
令\(P_{n}\)为\(Q_{n - 1}\)关于\(y\)轴的对称点,
记\(P_{n}\)的坐标为\(\left( x_{n},y_{n} \right)\).
\begin{enumerate}[label=(\arabic*)]
  \item 若\(k = \frac{1}{2}\),求\(x_{2},y_{2}\);
  \item 证明:数列\(\left\{ x_{n} - y_{n} \right\}\)是公比为\(\frac{1 + k}{1 - k}\)的等比数列;
  \item 设\(S_{n}\)为\(\triangle P_{n}P_{n + 1}P_{n + 2}\)的面积,
证明:对任意正整数\(n\),\(S_{n} = S_{n + 1}\).
\end{enumerate}

\examimage{images/media/image8.png}{0.30}

\topics{由递推关系证明等比数列;求直线与双曲线的交点坐标;向量夹角的坐标表示}
\difficulty{0.4}
\answer{(1)\(x_{2} = 3\),\(y_{2} = 0\)
(2)证明见解析
(3)证明见解析}
\explain{(1)\par
由已知有\(m = 5^{2} - 4^{2} = 9\),
故\(C\)的方程为\(x^{2} - y^{2} = 9\).\par
当\(k = \frac{1}{2}\)时,
过\(P_{1}(5,4)\)且斜率为\(\frac{1}{2}\)的直线为\(y = \frac{x + 3}{2}\),
与\(x^{2} - y^{2} = 9\)联立得到\(x^{2} - \left( \frac{x + 3}{2} \right)^{2} = 9\).\par
解得\(x = - 3\)或\(x = 5\),
所以该直线与\(C\)的不同于\(P_{1}\)的交点为\(Q_{1}( - 3,0)\),
该点显然在\(C\)的左支上.\par
故\(P_{2}(3,0)\),从而\(x_{2} = 3\),
\(y_{2} = 0\).\par
(2)方法一:由于过\(P_{n}\left( x_{n},y_{n} \right)\)且斜率为\(k\)的直线为\(y = k\left( x - x_{n} \right) + y_{n}\),
与\(x^{2} - y^{2} = 9\)联立,
得到方程\(x^{2} - \left( k\left( x - x_{n} \right) + y_{n} \right)^{2} = 9\).\par
展开即得\(\left( 1 - k^{2} \right)x^{2} - 2k\left( y_{n} - kx_{n} \right)x - \left( y_{n} - kx_{n} \right)^{2} - 9 = 0\),
由于\(P_{n}\left( x_{n},y_{n} \right)\)已经是直线\(y = k\left( x - x_{n} \right) + y_{n}\)和\(x^{2} - y^{2} = 9\)的公共点,
故方程必有一根\(x = x_{n}\).\par
从而根据韦达定理,
另一根\(x = \frac{2k\left( y_{n} - kx_{n} \right)}{1 - k^{2}} - x_{n} = \frac{2ky_{n} - x_{n} - k^{2}x_{n}}{1 - k^{2}}\),
相应的\(y = k\left( x - x_{n} \right) + y_{n} = \frac{y_{n} + k^{2}y_{n} - 2kx_{n}}{1 - k^{2}}\).\par
所以该直线与\(C\)的不同于\(P_{n}\)的交点为\(Q_{n}\left( \frac{2ky_{n} - x_{n} - k^{2}x_{n}}{1 - k^{2}},\frac{y_{n} + k^{2}y_{n} - 2kx_{n}}{1 - k^{2}} \right)\),
而注意到\(Q_{n}\)的横坐标亦可通过韦达定理表示为\(\frac{- \left( y_{n} - kx_{n} \right)^{2} - 9}{\left( 1 - k^{2} \right)x_{n}}\),
故\(Q_{n}\)一定在\(C\)的左支上.\par
所以\(P_{n + 1}\left( \frac{x_{n} + k^{2}x_{n} - 2ky_{n}}{1 - k^{2}},\frac{y_{n} + k^{2}y_{n} - 2kx_{n}}{1 - k^{2}} \right)\).\par
这就得到\(x_{n + 1} = \frac{x_{n} + k^{2}x_{n} - 2ky_{n}}{1 - k^{2}}\),
\(y_{n + 1} = \frac{y_{n} + k^{2}y_{n} - 2kx_{n}}{1 - k^{2}}\).\par
所以\(x_{n + 1} - y_{n + 1} = \frac{x_{n} + k^{2}x_{n} - 2ky_{n}}{1 - k^{2}} - \frac{y_{n} + k^{2}y_{n} - 2kx_{n}}{1 - k^{2}}= \frac{x_{n} + k^{2}x_{n} + 2kx_{n}}{1 - k^{2}} - \frac{y_{n} + k^{2}y_{n} + 2ky_{n}}{1 - k^{2}} = \frac{1 + k^{2} + 2k}{1 - k^{2}}\left( x_{n} - y_{n} \right) = \frac{1 + k}{1 - k}\left( x_{n} - y_{n} \right)\).\par
再由\(x_{1}^{2} - y_{1}^{2} = 9\),
就知道\(x_{1} - y_{1} \neq 0\),
所以数列\(\left\{ x_{n} - y_{n} \right\}\)是公比为\(\frac{1 + k}{1 - k}\)的等比数列.\par
方法二:因为\(P_{n}\left( x_{n},y_{n} \right)\),
\(Q_{n}\left( - x_{n + 1},y_{n + 1} \right)\),
\(k = - \frac{y_{n + 1} - y_{n}}{x_{n + 1} + x_{n}}\),
则\(\frac{1 + k}{1 - k} = \frac{x_{n + 1} + x_{n} - y_{n + 1} + y_{n}}{x_{n + 1} + x_{n} + y_{n + 1} - y_{n}}\),\par
由于\(\left\{ \begin{array}{r}
{x_{n + 1}}^{2} - {y_{n + 1}}^{2} = 9 \\
{x_{n}}^{2} - {y_{n}}^{2} = 9
\end{array} \right.\),作差得\(\left( x_{n + 1} - y_{n + 1} \right)\left( x_{n + 1} + y_{n + 1} \right) = \left( x_{n} - y_{n} \right)\left( x_{n} + y_{n} \right)\),\par
\(\frac{x_{n + 1} - y_{n + 1}}{x_{n} - y_{n}} = \frac{x_{n} + y_{n}}{x_{n + 1} + y_{n + 1}}\),利用合比性质知\(\frac{x_{n + 1} - y_{n + 1}}{x_{n} - y_{n}} = \frac{x_{n} + y_{n}}{x_{n + 1} + y_{n + 1}} = \frac{x_{n + 1} - y_{n + 1} + x_{n} + y_{n}}{x_{n} - y_{n} + x_{n + 1} + y_{n + 1}} = \frac{1 + k}{1 - k}\),\par
因此\(\left\{ x_{n} - y_{n} \right\}\)是公比为\(\frac{1 + k}{1 - k}\)的等比数列.\par
(3)方法一:先证明一个结论:对平面上三个点\(U,V,W\),若\(\overrightarrow{UV} = (a,b)\),\(\overrightarrow{UW} = (c,d)\),则\(S_{\triangle UVW} = \frac{1}{2}|ad - bc|\).(若\(U,V,W\)在同一条直线上,约定\(S_{\triangle UVW} = 0\))\par
证明:\(S_{\triangle UVW} = \frac{1}{2}\left| \overrightarrow{UV} \right| \cdot \left| \overrightarrow{UW} \right|\sin\overrightarrow{UV},\overrightarrow{UW} = \frac{1}{2}\left| \overrightarrow{UV} \right| \cdot \left| \overrightarrow{UW} \right|\sqrt{1 - \cos^{2}\overrightarrow{UV},\overrightarrow{UW}}= \frac{1}{2}\left| \overrightarrow{UV} \right| \cdot \left| \overrightarrow{UW} \right|\sqrt{1 - \left( \frac{\overrightarrow{UV} \cdot \overrightarrow{UW}}{\left| \overrightarrow{UV} \right| \cdot \left| \overrightarrow{UW} \right|} \right)^{2}} = \frac{1}{2}\sqrt{\left| \overrightarrow{UV} \right|^{2} \cdot \left| \overrightarrow{UW} \right|^{2} - \left( \overrightarrow{UV} \cdot \overrightarrow{UW} \right)^{2}}= \frac{1}{2}\sqrt{\left( a^{2} + b^{2} \right)\left( c^{2} + d^{2} \right) - (ac + bd)^{2}}= \frac{1}{2}\sqrt{a^{2}c^{2} + a^{2}d^{2} + b^{2}c^{2} + b^{2}d^{2} - a^{2}c^{2} - b^{2}d^{2} - 2abcd}= \frac{1}{2}\sqrt{a^{2}d^{2} + b^{2}c^{2} - 2abcd} = \frac{1}{2}\sqrt{(ad - bc)^{2}} = \frac{1}{2}|ad - bc|\).\par
证毕,回到原题.\par
由于上一小问已经得到\(x_{n + 1} = \frac{x_{n} + k^{2}x_{n} - 2ky_{n}}{1 - k^{2}}\),\(y_{n + 1} = \frac{y_{n} + k^{2}y_{n} - 2kx_{n}}{1 - k^{2}}\),\par
故\(x_{n + 1} + y_{n + 1} = \frac{x_{n} + k^{2}x_{n} - 2ky_{n}}{1 - k^{2}} + \frac{y_{n} + k^{2}y_{n} - 2kx_{n}}{1 - k^{2}} = \frac{1 + k^{2} - 2k}{1 - k^{2}}\left( x_{n} + y_{n} \right) = \frac{1 - k}{1 + k}\left( x_{n} + y_{n} \right)\).\par
再由\(x_{1}^{2} - y_{1}^{2} = 9\),就知道\(x_{1} + y_{1} \neq 0\),所以数列\(\left\{ x_{n} + y_{n} \right\}\)是公比为\(\frac{1 - k}{1 + k}\)的等比数列.\par
所以对任意的正整数\(m\),都有\par
\(x_{n}y_{n + m} - y_{n}x_{n + m}= \frac{1}{2}\left( \left( x_{n}x_{n + m} - y_{n}y_{n + m} \right) + \left( x_{n}y_{n + m} - y_{n}x_{n + m} \right) \right) - \frac{1}{2}\left( \left( x_{n}x_{n + m} - y_{n}y_{n + m} \right) - \left( x_{n}y_{n + m} - y_{n}x_{n + m} \right) \right)= \frac{1}{2}\left( x_{n} - y_{n} \right)\left( x_{n + m} + y_{n + m} \right) - \frac{1}{2}\left( x_{n} + y_{n} \right)\left( x_{n + m} - y_{n + m} \right)= \frac{1}{2}\left( \frac{1 - k}{1 + k} \right)^{m}\left( x_{n} - y_{n} \right)\left( x_{n} + y_{n} \right) - \frac{1}{2}\left( \frac{1 + k}{1 - k} \right)^{m}\left( x_{n} + y_{n} \right)\left( x_{n} - y_{n} \right)= \frac{1}{2}\left( \left( \frac{1 - k}{1 + k} \right)^{m} - \left( \frac{1 + k}{1 - k} \right)^{m} \right)\left( x_{n}^{2} - y_{n}^{2} \right)= \frac{9}{2}\left( \left( \frac{1 - k}{1 + k} \right)^{m} - \left( \frac{1 + k}{1 - k} \right)^{m} \right)\).\par
而又有\(\overrightarrow{P_{n + 1}P_{n}} = \left( - \left( x_{n + 1} - x_{n} \right), - \left( y_{n + 1} - y_{n} \right) \right)\),\(\overrightarrow{P_{n + 1}P_{n + 2}} = \left( x_{n + 2} - x_{n + 1},y_{n + 2} - y_{n + 1} \right)\),\par
故利用前面已经证明的结论即得\par
\(S_{n} = S_{\triangle P_{n}P_{n + 1}P_{n + 2}} = \frac{1}{2}\left| - \left( x_{n + 1} - x_{n} \right)\left( y_{n + 2} - y_{n + 1} \right) + \left( y_{n + 1} - y_{n} \right)\left( x_{n + 2} - x_{n + 1} \right) \right|= \frac{1}{2}\left| \left( x_{n + 1} - x_{n} \right)\left( y_{n + 2} - y_{n + 1} \right) - \left( y_{n + 1} - y_{n} \right)\left( x_{n + 2} - x_{n + 1} \right) \right|= \frac{1}{2}\left| \left( x_{n + 1}y_{n + 2} - y_{n + 1}x_{n + 2} \right) + \left( x_{n}y_{n + 1} - y_{n}x_{n + 1} \right) - \left( x_{n}y_{n + 2} - y_{n}x_{n + 2} \right) \right|= \frac{1}{2}\left| \frac{9}{2}\left( \frac{1 - k}{1 + k} - \frac{1 + k}{1 - k} \right) + \frac{9}{2}\left( \frac{1 - k}{1 + k} - \frac{1 + k}{1 - k} \right) - \frac{9}{2}\left( \left( \frac{1 - k}{1 + k} \right)^{2} - \left( \frac{1 + k}{1 - k} \right)^{2} \right) \right|\).\par
这就表明\(S_{n}\)的取值是与\(n\)无关的定值,所以\(S_{n} = S_{n + 1}\).\par
方法二:由于上一小问已经得到\(x_{n + 1} = \frac{x_{n} + k^{2}x_{n} - 2ky_{n}}{1 - k^{2}}\),\(y_{n + 1} = \frac{y_{n} + k^{2}y_{n} - 2kx_{n}}{1 - k^{2}}\),\par
故\(x_{n + 1} + y_{n + 1} = \frac{x_{n} + k^{2}x_{n} - 2ky_{n}}{1 - k^{2}} + \frac{y_{n} + k^{2}y_{n} - 2kx_{n}}{1 - k^{2}} = \frac{1 + k^{2} - 2k}{1 - k^{2}}\left( x_{n} + y_{n} \right) = \frac{1 - k}{1 + k}\left( x_{n} + y_{n} \right)\).\par
再由\(x_{1}^{2} - y_{1}^{2} = 9\),就知道\(x_{1} + y_{1} \neq 0\),所以数列\(\left\{ x_{n} + y_{n} \right\}\)是公比为\(\frac{1 - k}{1 + k}\)的等比数列.\par
所以对任意的正整数\(m\),都有\par
\(x_{n}y_{n + m} - y_{n}x_{n + m}= \frac{1}{2}\left( \left( x_{n}x_{n + m} - y_{n}y_{n + m} \right) + \left( x_{n}y_{n + m} - y_{n}x_{n + m} \right) \right) - \frac{1}{2}\left( \left( x_{n}x_{n + m} - y_{n}y_{n + m} \right) - \left( x_{n}y_{n + m} - y_{n}x_{n + m} \right) \right)= \frac{1}{2}\left( x_{n} - y_{n} \right)\left( x_{n + m} + y_{n + m} \right) - \frac{1}{2}\left( x_{n} + y_{n} \right)\left( x_{n + m} - y_{n + m} \right)= \frac{1}{2}\left( \frac{1 - k}{1 + k} \right)^{m}\left( x_{n} - y_{n} \right)\left( x_{n} + y_{n} \right) - \frac{1}{2}\left( \frac{1 + k}{1 - k} \right)^{m}\left( x_{n} + y_{n} \right)\left( x_{n} - y_{n} \right)= \frac{1}{2}\left( \left( \frac{1 - k}{1 + k} \right)^{m} - \left( \frac{1 + k}{1 - k} \right)^{m} \right)\left( x_{n}^{2} - y_{n}^{2} \right)= \frac{9}{2}\left( \left( \frac{1 - k}{1 + k} \right)^{m} - \left( \frac{1 + k}{1 - k} \right)^{m} \right)\).\par
这就得到\(x_{n + 2}y_{n + 3} - y_{n + 2}x_{n + 3} = \frac{9}{2}\left( \frac{1 - k}{1 + k} - \frac{1 + k}{1 - k} \right) = x_{n}y_{n + 1} - y_{n}x_{n + 1}\),\par
以及\(x_{n + 1}y_{n + 3} - y_{n + 1}x_{n + 3} = \frac{9}{2}\left( \left( \frac{1 - k}{1 + k} \right)^{2} - \left( \frac{1 + k}{1 - k} \right)^{2} \right) = x_{n}y_{n + 2} - y_{n}x_{n + 2}\).\par
两式相减,即得\(\left( x_{n + 2}y_{n + 3} - y_{n + 2}x_{n + 3} \right) - \left( x_{n + 1}y_{n + 3} - y_{n + 1}x_{n + 3} \right) = \left( x_{n}y_{n + 1} - y_{n}x_{n + 1} \right) - \left( x_{n}y_{n + 2} - y_{n}x_{n + 2} \right)\).\par
移项得到\(x_{n + 2}y_{n + 3} - y_{n}x_{n + 2} - x_{n + 1}y_{n + 3} + y_{n}x_{n + 1} = y_{n + 2}x_{n + 3} - x_{n}y_{n + 2} - y_{n + 1}x_{n + 3} + x_{n}y_{n + 1}\).\par
故\(\left( y_{n + 3} - y_{n} \right)\left( x_{n + 2} - x_{n + 1} \right) = \left( y_{n + 2} - y_{n + 1} \right)\left( x_{n + 3} - x_{n} \right)\).\par
而\(\overrightarrow{P_{n}P_{n + 3}} = \left( x_{n + 3} - x_{n},y_{n + 3} - y_{n} \right)\),\(\overrightarrow{P_{n + 1}P_{n + 2}} = \left( x_{n + 2} - x_{n + 1},y_{n + 2} - y_{n + 1} \right)\).\par
所以\(\overrightarrow{P_{n}P_{n + 3}}\)和\(\overrightarrow{P_{n + 1}P_{n + 2}}\)平行,这就得到\(S_{\triangle P_{n}P_{n + 1}P_{n + 2}} = S_{\triangle P_{n + 1}P_{n + 2}P_{n + 3}}\),即\(S_{n} = S_{n + 1}\).\par
方法三:由于\(\left\{ \begin{array}{r}
{x_{n + 1}}^{2} - {y_{n + 1}}^{2} = 9 \\
{x_{n}}^{2} - {y_{n}}^{2} = 9
\end{array} \right.\),作差得\({x_{n + 1}}^{2} - {x_{n}}^{2} = {y_{n + 1}}^{2} - {y_{n}}^{2}\),\par
变形得\(k_{P_{n}P_{n + 1}} = \frac{y_{n + 1} - y_{n}}{x_{n + 1} - x_{n}} = \frac{x_{n + 1} + x_{n}}{y_{n + 1} + y_{n}} = \frac{x_{n + 1} + y_{n + 1} + x_{n} - y_{n}}{x_{n + 1} + y_{n + 1} - \left( x_{n} - y_{n} \right)}\)①,\par
同理可得\(k_{P_{n - 1}P_{n + 2}} = \frac{y_{n + 2} - y_{n - 1}}{x_{n + 2} - x_{n - 1}} = \frac{x_{n + 2} + x_{n - 1}}{y_{n + 2} + y_{n - 1}} = \frac{x_{n + 2} + y_{n + 2} + x_{n - 1} - y_{n - 1}}{x_{n + 2} + y_{n + 2} - \left( x_{n - 1} - y_{n - 1} \right)}\),\par
由(2)知\(\left\{ x_{n} - y_{n} \right\}\)是公比为\(\frac{1 + k}{1 - k}\)的等比数列,令\(q = \frac{1 + k}{1 - k}\)则\(x_{n} - y_{n} = q\left( x_{n - 1} - y_{n - 1} \right)\)②,\par
同时\(\left\{ x_{n} + y_{n} \right\}\)是公比为\(\frac{1}{q}\)的等比数列,则\(x_{n + 1} + y_{n + 1} = q\left( x_{n + 2} + y_{n + 2} \right)\)③,\par
将②③代入①,\par
\(\frac{y_{n + 1} - y_{n}}{x_{n + 1} - x_{n}} = \frac{x_{n + 1} + y_{n + 1} + x_{n} - y_{n}}{x_{n + 1} + y_{n + 1} - \left( x_{n} - y_{n} \right)} = \frac{q\left( x_{n + 2} + y_{n + 2} \right) + q\left( x_{n - 1} - y_{n - 1} \right)}{q\left( x_{n + 2} + y_{n + 2} \right) - q\left( x_{n - 1} - y_{n - 1} \right)} = \frac{x_{n + 2} + y_{n + 2} + x_{n - 1} - y_{n - 1}}{x_{n + 2} + y_{n + 2} - \left( x_{n - 1} - y_{n - 1} \right)}\)\par
即\(k_{P_{n}P_{n + 1}} = k_{P_{n - 1}P_{n + 2}}\),从而\(S_{n - 1} = S_{n}\),即\(S_{n} = S_{n + 1}\).}
\end{question}
