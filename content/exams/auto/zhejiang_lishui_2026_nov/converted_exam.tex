\examxtitle{浙江省丽水、湖州、衢州三地市2026届高三上学期11月教学质量检测数学试题}

\section{单选题}

\begin{question}
已知复数\(z\),若\(\frac{z}{2 + \text{i}} = \text{i}\)(\(\text{i}\)为虚数单位),则\(|z| =\)(  )
\begin{choices}
  \item \(\sqrt{5}\)
  \item 5
  \item \(\sqrt{3}\)
  \item 3
\end{choices}
\topics{求复数的模;复数的除法运算}
\difficulty{0.85}
\answer{A}
\explain{由方程
\(\frac{z}{2 + \text{i}} = \text{i}\)得\(z = \text{i} \cdot (2 + \text{i}) = 2\text{i} + \text{i}^{2} = 2\text{i} + ( - 1) = - 1 + 2\text{i}\),
所以\(|z| = \sqrt{{( - 1)}^{2} + 2^{2}} = \sqrt{1 + 4} = \sqrt{5}\)}
\end{question}

\begin{question}
已知集合\(A = \{ x \mid - 2 \leq x < 1\},B = \left\{ m,3 \right\}\),且\(A \cap B\)的元素个数是一个,则实数\(m\)的取值范围是(  )
\begin{choices}
  \item \(( - 2,1)\)
  \item \(\lbrack - 2,1\rbrack\)
  \item \(\lbrack - 2,1)\)
  \item \(( - 2,1\rbrack\)
\end{choices}
\topics{根据元素与集合的关系求参数;根据交集结果求集合或参数}
\difficulty{0.85}
\answer{C}
\explain{由\(A \cap B\)的元素个数是一个,且\(3 \notin A\),得\(m \in A\),则\(- 2 \leq m < 1\),
所以实数\(m\)的取值范围是\(\lbrack - 2,1)\)}
\end{question}

\begin{question}
已知\(F_{1},F_{2}\)为双曲线\(C:\frac{x^{2}}{a^{2}} - \frac{y^{2}}{b^{2}} = 1(a > 0,b > 0)\)的左右焦点,点\(A\)的坐标为\((0,2b)\).若\(\bigtriangleup AF_{1}F_{2}\)为等边三角形,则双曲线\(C\)的离心率是(  )
\begin{choices}
  \item \(\sqrt{3}\)
  \item 2\(\sqrt{3}\)
  \item 2
  \item 3
\end{choices}
\topics{求双曲线的离心率或离心率的取值范围}
\difficulty{0.85}
\answer{C}
\explain{\(\because \bigtriangleup AF_{1}F_{2}\)为等边三角形,\(O\)为\(F_{1}F_{2}\)的中点,%
% IMAGE_TODO_START id=zhejiang_lishui_2026_nov-Q3-img1 path=/Users/muryor/code/mynote/word\_to\_tex/output/figures/raw/media/image2.png width=60% inline=true question_index=3 sub_index=1
% CONTEXT_BEFORE: eup AF_{1}F_{2}\(为等边三角形,\)O\(为\)F_{1}F_{2}\(的中点,
% CONTEXT_AFTER: \)\therefore\
\begin{tikzpicture}[scale=0.8,baseline=-0.5ex]
  % TODO: AI_AGENT_REPLACE_ME (id=zhejiang_lishui_2026_nov-Q3-img1)
\end{tikzpicture}%
% IMAGE_TODO_END id=zhejiang_lishui_2026_nov-Q3-img1
\(\therefore\tan\angle AF_{1}O = \frac{2b}{c} = \sqrt{3}\),则\(\frac{b}{c} = \frac{\sqrt{3}}{2}\),\(\because e^{2} = \frac{c^{2}}{a^{2}} = \frac{c^{2}}{c^{2} - b^{2}} = \frac{1}{1 - \left( \frac{b}{c} \right)^{2}} = \frac{1}{1 - \frac{3}{4}} = 4\),\(\therefore e = 2\).}
\end{question}

\begin{question}
已知\(x,y \in \text{R}\),则下列条件中使\(x > y\)成立的充要条件是(  )
\begin{choices}
  \item \(|x| > y\)
  \item \(x^{2} > y^{2}\)
  \item \(a^{x} > a^{y}(a > 0,\text{且}a \neq 1)\)
  \item \(\text{ln}(x - y + 1) > 0\)
\end{choices}
\topics{探求命题为真的充要条件;对数函数单调性的应用;由指数函数的单调性解不等式}
\difficulty{0.65}
\answer{D}
\explain{对于A,当\(x = - 2,y = 1\)时,满足\(|x| > y\),但不满足\(x > y\),
所以\(|x| > y\)不是\(x > y\)的充要条件,故A错误;
对于B,当\(x = - 2,y = 1\)时,满足\(x^{2} > y^{2}\),但不满足\(x > y\),
所以\(x^{2} > y^{2}\)不是\(x > y\)的充要条件,故B错误;
对于C,当\(a > 1\)时,指数函数\(y = a^{x}\)为增函数,若\(a^{x} > a^{y}\),则\(x > y\),
当\(0 < a < 1\)时,指数函数\(y = a^{x}\)为减函数,若\(a^{x} > a^{y}\),则\(x < y\),
所以\(a^{x} > a^{y}\)(\(a > 0\)且\(a \neq 1\))不是\(x > y\)的充要条件,故C错误;
对于D,若\(\ln(x - y + 1) > 0 = \ln 1\),则\(x - y + 1 > 1 \Rightarrow x - y > 0\),即\(x > y\),
反之,若\(x > y\),则\(x - y > 0\),则\(x - y + 1 > 1\),所以\(\ln(x - y + 1) > \ln 1 = 0\),
所以\(\ln(x - y + 1) > 0\)是\(x > y\)的充要条件,故D正确}
\end{question}

\begin{question}
定义在\(R\)上的两个函数\(f(x),g(x)\),恒有\(f^{3}(x) = g\left( x^{2} \right)\),则(  )
\begin{choices}
  \item \(f(x)\)为奇函数
  \item \(f(x)\)为偶函数
  \item \(g(x)\)为奇函数
  \item \(g(x)\)为偶函数
\end{choices}
\topics{函数奇偶性的定义与判断}
\difficulty{0.85}
\answer{B}
\explain{由\(f^{3}(x) = g\left( x^{2} \right)\),则\(f^{3}( - x) = g\left\lbrack ( - x)^{2} \right\rbrack = g\left( x^{2} \right) = f^{3}(x)\),
则\(f( - x) = f(x)\),又\(f(x)\)定义域为\(R\),故\(f(x)\)为偶函数,故B正确;
由已知得不到\(g(x)\)与\(g( - x)\)关系,也得不到\(f(x) + f( - x)\)是否为\(0\),故A、C、D错误}
\end{question}

\begin{question}
若函数\(y = \text{sin}\left| \omega x + \frac{\text{π}}{6} \right|\)的图象向右平移\(\frac{\text{π}}{4}\)个单位后得到的图象关于\(y\)轴对称,则实数\(\omega\)可以是(  )
\begin{choices}
  \item \(\frac{2}{3}\)
  \item \(- \frac{2}{3}\)
  \item 2
  \item \(- 2\)
\end{choices}
\topics{利用正弦函数的对称性求参数}
\difficulty{0.65}
\answer{A}
\explain{因为函数\(y = \text{sin}\left| \omega x + \frac{\text{π}}{6} \right|\)的图象向右平移\(\frac{\text{π}}{4}\)个单位后得到的图象关于\(y\)轴对称,
可知函数\(y = \text{sin}\left| \omega x + \frac{\text{π}}{6} \right|\)关于直线\(x = - \frac{\text{π}}{4}\)对称,
若\(\omega = 0\),则函数\(y = \sin\frac{\text{π}}{6} = \frac{1}{2}\)关于直线\(x = - \frac{\text{π}}{4}\)对称,符合题意;
若\(\omega \neq 0\),设\(t = \omega x + \frac{\text{π}}{6}\),
则函数\(y = \sin\left| \omega x + \frac{\pi}{6} \right|\)的对称轴\(x = - \frac{\pi}{4}\)所对应的\(t\)值(\(t = - \frac{\pi}{4}\omega + \frac{\pi}{6}\))必为函数\(y = \sin|t|\)的对称轴,
又因为函数\(y = \sin|t|\)的对称轴为\(y\)轴,
则\(- \frac{\text{π}}{4}\omega + \frac{\text{π}}{6} = 0\),解得\(\omega = \frac{2}{3}\);
综上所述:\(\omega = 0\)或\(\omega = \frac{2}{3}\).
结合选项可知:A正确,BCD错误}
\end{question}

\begin{question}
已知三棱锥\(S - ABC\),满足\(SA = SB = SC\),且\(SA\),\(SB\),\(SC\)两两垂直.在底面\(\bigtriangleup ABC\)内有一动点\(P\)到三个侧面的距离依次成等差数列,则点\(P\)的轨迹是(  )
\begin{choices}
  \item 一个点
  \item 一条线段
  \item 一段圆弧
  \item 一段抛物线
\end{choices}
\topics{等差中项的应用;立体几何中的轨迹问题}
\difficulty{0.4}
\answer{B}
\explain{待补充解析}
\end{question}