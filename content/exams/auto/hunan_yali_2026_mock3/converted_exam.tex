\examxtitle{湖南省长沙市雅礼中学2026届高三上学期月考(三)数学试题}

\section{单选题}

\begin{question}
若复数\(z\)满足\(\left( 1 + 2\text{i} \right)z = 4 + 3\text{i}\),
则\(\overline{z}\)的实部为(  )
\begin{choices}
  \item 1
  \item -1
  \item 2
  \item -2
\end{choices}
\topics{求复数的实部与虚部;复数的除法运算}
\difficulty{0.65}
\answer{C}
\explain{因为\(\left( 1 + 2\text{i} \right)z = 4 + 3\text{i}\),\par
所以\(z = \frac{4 + 3\text{i}}{1 + 2\text{i}} = \frac{\left( 4 + 3\text{i} \right)\left( 1 - 2\text{i} \right)}{\left( 1 + 2\text{i} \right)\left( 1 - 2\text{i} \right)} = \frac{10 - 5\text{i}}{5} = 2 - \text{i}\),\par
所以\(\overline{z} = 2 + \text{i}\),\par
所以\(\overline{z}\)的实部为2}
\end{question}

\begin{question}
已知集合\(U = \mathbb{R}\),
集合\(A = \left\{ \left. \ x \right|0 \leq x \leq 2 \right\},B = \left\{ \left. \ x \right| - 3 < x < 1 \right\}\),
则图中阴影部分表示的集合为(   )
\begin{choices}
  \item \(( - 3,0)\)
  \item \(( - 1,0)\)
  \item \((0,1)\)
  \item \((2,3)\)
\end{choices}

\begin{center}
% PNG: hunan_yali_2026_mock3-Q2-img1
\includegraphics[width=0.4\textwidth]{content/exams/auto/hunan_yali_2026_mock3/images/image2.png}
\end{center}

\topics{交并补混合运算}
\difficulty{0.94}
\answer{A}
\explain{\(\complement_{U}A = \left\{ x|x < 0 \right.\text{ 或 }\left. x > 2 \right\}\),
又\(B = \left\{ x - 3 < x < 1 \right\}\),\par
阴影部分表示\(\left( \complement_{U}A \right) \cap B = ( - 3,0)\)}
\end{question}

\begin{question}
已知\(a,b \in R\),则"\(a^{3} > b^{3}\)"是"\(\log_{2}a > \log_{2}b\)"的(   )
\begin{choices}
  \item 充分不必要条件
  \item 必要不充分条件
  \item 充要条件
  \item 既不充分也不必要条件
\end{choices}
\topics{判断命题的必要不充分条件;由不等式的性质比较数(式)大小;由对数函数的单调性解不等式}
\difficulty{0.85}
\answer{B}
\explain{由\(a^{3} > b^{3}\),当\(a > b > 0\)时,
可得\(\log_{2}a > \log_{2}b\);\par
当\(0 > a > b\),
不能得到\(\log_{2}a > \log_{2}b\),所以充分性不成立;\par
反之:由\(\log_{2}a > \log_{2}b\),
可得\(a > b > 0\),\par
根据不等式的基本性质,可得\(a^{3} > b^{3}\)成立,所以必要性成立,\par
所以"\(a^{3} > b^{3}\)"是"\(\log_{2}a > \log_{2}b\)"的必要不充分条件.}
\end{question}

\begin{question}
已知\(\cos\left( \alpha + \frac{\pi}{6} \right) = \frac{\sqrt{3}}{3}\),
\(0 < \alpha < \pi\),
则\(\sin\left( \frac{5\pi}{6} - \alpha \right) =\)(    )
\begin{choices}
  \item \(\frac{\sqrt{6}}{3}\)
  \item \(- \frac{\sqrt{6}}{3}\)
  \item \(\frac{\sqrt{3}}{3}\)
  \item \(- \frac{\sqrt{3}}{3}\)
\end{choices}
\topics{已知正(余)弦求余(正)弦;诱导公式二;三;四}
\difficulty{0.85}
\answer{A}
\explain{因为\(0 < \alpha < \pi\),
所以\(\frac{\pi}{6} < \alpha + \frac{\pi}{6} < \frac{7\pi}{6}\),\par
因为\(\cos\left( \alpha + \frac{\pi}{6} \right) = \frac{\sqrt{3}}{3}\),
所以\(\frac{\pi}{6} < \alpha + \frac{\pi}{6} < \frac{\pi}{2}\),\par
所以\(\sin\left( \alpha + \frac{\pi}{6} \right) = \sqrt{1 - \cos^{2}\left( \alpha + \frac{\pi}{6} \right)} = \frac{\sqrt{6}}{3}\),\par
所以\(\sin\left( \frac{5\pi}{6} - \alpha \right) = \sin\left\lbrack \pi - \left( \frac{\pi}{6} + \alpha \right) \right\rbrack = \sin\left( \frac{\pi}{6} + \alpha \right) = \frac{\sqrt{6}}{3}\)}
\end{question}

\begin{question}
下列说法正确的是(    )
\begin{choices}
  \item 当\(x \in \left( 0,1 \right)\)时,\(x + \frac{1}{x}\)的最小值为2
  \item 当\(x \in \left( 0,\pi \right)\)时,\(\sin x + \frac{4}{\sin x}\)的最小值为4
  \item \(x^{2} + 1\)的最小值为\(2x\)
  \item 当\(x \in (0,1)\)时,\(4x(1 - x)\)的最大值为1
\end{choices}
\topics{基本不等式求积的最大值;对勾函数求最值}
\difficulty{0.65}
\answer{D}
\explain{对于A,当\(0 < x < 1\)时,
\(y = x + \frac{1}{x}\)单调递减,所以\(y > 2\),所以A错误;\par
对于B,当\(0 < x < \pi\)时,
可得\(\sin x \in (0,1\rbrack\),令\(t = \sin x\),\par
则\(y = t + \frac{4}{t}\)在\((0,1\rbrack\)上单调递减,
\(t = 1\)时,函数取得最小值5,所以B错误;\par
对于C,由\(2x\)不是定值,
所以\(2x\)不是\(x^{2} + 1\)的最小值,所以C错误;\par
对于D,当\(0 < x < 1\)时,
\(4x(1 - x) \leq 4 \times \left( \frac{x + 1 - x}{2} \right)^{2} = 1\),
当且仅当\(x = 1 - x\),
即\(x = \frac{1}{2}\)时取等号,所以D正确.}
\end{question}

\begin{question}
已知奇函数\(f(x)\)的定义域为\(R\),
且函数\(y = f(x)\)图象关于\(x = 2\)对称.当\(x \in \lbrack 0,2\rbrack\)时,
\(f(x) = x\),则\(f(13) =\)(  )
\begin{choices}
  \item 1
  \item -1
  \item 2
  \item -2
\end{choices}
\topics{函数奇偶性的应用;函数对称性的应用;由函数的周期性求函数值}
\difficulty{0.65}
\answer{B}
\explain{因为函数\(y = f(x)\)图象关于\(x = 2\)对称,
所以\(f(2 + x) = f(2 - x)\),\par
因为\(f(x)\)为奇函数,所以\(f( - x) = - f(x)\),\par
所以\(f(2 - x) = - f(x - 2)\),
即\(f(x + 2) = - f(x - 2)\),\par
所以\(f(x + 4) = - f(x)\),
所以\(f(x + 8) = - f(x + 4) = f(x)\),
所以\(f(x)\)的周期为8,\par
所以\(f(13) = f(5 + 8) = f(5) = f( - 3 + 8) = f( - 3) = - f(3)\),\par
而\(f(3) = f(2 + 1) = f(2 - 1) = f(1)\),\par
又因为当\(x \in \lbrack 0,2\rbrack\)时,
\(f(x) = x\),所以\(f(1) = 1\),
即\(f(3) = f(1) = 1\),\par
所以\(f(13) = - f(3) = - f(1) = - 1\)}
\end{question}

\begin{question}
志愿者甲参加第\(21\)届文博会的服务工作,
甲从住所到文博会选择乘地铁、乘公交车、骑共享单车的概率分别为\(\frac{1}{4}\),
\(\frac{1}{4}\),\(\frac{1}{2}\),
且乘地铁、乘公交车、骑共享单车按时到达文博会的概率分别为\(\frac{4}{5}\),
\(\frac{3}{4}\),
\(\frac{2}{3}\).若某一天甲按时到达文博会,则他骑共享单车的概率为(    )
\begin{choices}
  \item \(\frac{173}{240}\)
  \item \(\frac{93}{173}\)
  \item \(\frac{80}{173}\)
  \item \(\frac{1}{3}\)
\end{choices}
\topics{计算条件概率;利用全概率公式求概率}
\difficulty{0.85}
\answer{C}
\explain{设事件\(A\)表示"甲乘地铁",事件\(B\)表示"甲乘公交车",
事件\(C\)表示"甲骑共享单车",事件\(D\)表示"甲按时到达文博会",\par
则\(P(A) = \frac{1}{4}\),
\(P(B) = \frac{1}{4}\),\(P(C) = \frac{1}{2}\),
\(P\left( \left. \ D \right|A \right) = \frac{4}{5}\),
\(P\left( \left. \ D \right|B \right) = \frac{3}{4}\),
\(P\left( \left. \ D \right|C \right) = \frac{2}{3}\),\par
则\(P(D) = P(A)P\left( \left. \ D \right|A \right) + P(B)P\left( \left. \ D \right|B \right) + P(C) \cdot P\left( \left. \ D \right|C \right)= \frac{1}{4} \times \frac{4}{5} + \frac{1}{4} \times \frac{3}{4} + \frac{1}{2} \times \frac{2}{3} = \frac{173}{240}\),\par
\(P(CD) = P(C)P\left( D\left| C \right.\  \right) = \frac{1}{3}\),\par
所以若某一天甲按时到达文博会,\par
则他骑共享单车的概率为\(P\left( \left. \ C \right|D \right) = \frac{P(CD)}{P(D)} = \frac{80}{173}\).}
\end{question}

\begin{question}
已知直线\(l:2x\cos\alpha + 2y\sin\alpha - 1 = 0\)与圆\(C:(x - 1)^{2} + \left( y - \sqrt{3} \right)^{2} = \frac{9}{4}\)相切,
则满足条件的直线有(    )
\begin{choices}
  \item 1条
  \item 2条
  \item 3条
  \item 4条
\end{choices}
\topics{由直线与圆的位置关系求参数;圆的公切线条数}
\difficulty{0.65}
\answer{C}
\explain{法一:原点\(O\)到直线\(l\)的距离为\(d = \frac{1}{\sqrt{4\cos^{2}\alpha + 4\sin^{2}\alpha}} = \frac{1}{2}\),\par
从而满足条件的直线\(l\)为圆\(O:x^{2} + y^{2} = \frac{1}{4}\)和圆\(C:(x - 1)^{2} + \left( y - \sqrt{3} \right)^{2} = \frac{9}{4}\)的公切线,\par
因为\(|OC| = 2 = \frac{1}{2} + \frac{3}{2}\),
故两圆外切,共有三条公切线.\par
法二:由题设知,圆心\(C\)到直线\(l\)的距离为:\par
\(d = \frac{\left| 2\cos\alpha + 2\sqrt{3}\sin\alpha - 1 \right|}{\sqrt{4\cos^{2}\alpha + 4\sin^{2}\alpha}} = \frac{\left| 4\sin\left( \alpha + \frac{\pi}{6} \right) - 1 \right|}{2} = \frac{3}{2}\),\par
化简得\(4\sin\left( \alpha + \frac{\pi}{6} \right) - 1 = \pm 3\),
即\(\sin\left( \alpha + \frac{\pi}{6} \right) = 1\)或\(- \frac{1}{2}\),\par
当\(\sin\left( \alpha + \frac{\pi}{6} \right) = 1\)时,
\(\alpha = \frac{\pi}{3} + 2k\pi,k \in Z\);\par
当\(\sin\left( \alpha + \frac{\pi}{6} \right) = - \frac{1}{2}\)时,
\(\alpha = \frac{5\pi}{3} + 2k\pi\)或\(\pi + 2k\pi,k \in Z\).\par
故满足条件的直线有三条.}
\end{question}

\section{多选题}

\begin{question}
已知向量\(\overrightarrow{a} = (1,1),\overrightarrow{b} = ( - 2,0)\),
则下列结论正确的是(    )
\begin{choices}
  \item \(|\overrightarrow{a}| = |\overrightarrow{b}|\)
  \item \(\overrightarrow{a}\)与\(\overrightarrow{b}\)的夹角为\(\frac{3}{4}\pi\)
  \item \(\left( \overrightarrow{a} + \overrightarrow{b} \right)\bot\overrightarrow{a}\)
  \item \(\overrightarrow{b}\)在\(\overrightarrow{a}\)上的投影向量是\(( - 1, - 1)\)
\end{choices}
\topics{向量垂直的坐标表示;向量夹角的坐标表示;利用坐标求向量的模;求投影向量}
\difficulty{0.85}
\answer{BCD}
\explain{对于A:\(\left| \overrightarrow{a} \right| = \sqrt{1^{2} + 1^{2}} = \sqrt{2},\left| \overrightarrow{b} \right| = \sqrt{( - 2)^{2} + 0^{2}} = 2\),
故A错误.\par
对于B:\(\cos\left\langle \overrightarrow{a},\overrightarrow{b} \right\rangle = \frac{\overrightarrow{a} \cdot \overrightarrow{b}}{\left| \overrightarrow{a} \right|\left| \overrightarrow{b} \right|} = \frac{- 2}{2\sqrt{2}} = - \frac{\sqrt{2}}{2}\),
因为\(\left\langle \overrightarrow{a},\overrightarrow{b} \right\rangle \in \left\lbrack 0,\pi \right\rbrack\),
所以\(\left\langle \overrightarrow{a},\overrightarrow{b} \right\rangle = \frac{3\pi}{4}\),
故B正确;\par
对于C:\(\left( \overrightarrow{a} + \overrightarrow{b} \right) \cdot \overrightarrow{a} = {\overrightarrow{a}}^{2} + \overrightarrow{a} \cdot \overrightarrow{b} = 2 - 2 = 0\),
则\(\left( \overrightarrow{a} + \overrightarrow{b} \right)\bot\overrightarrow{a}\),
故C正确;\par
对于D:\(\overrightarrow{b}\)在\(\overrightarrow{a}\)上的投影向量是\(\left| \overrightarrow{b} \right|\cos\left\langle \overrightarrow{a},\overrightarrow{b} \right\rangle \cdot \frac{\overrightarrow{a}}{\left| \overrightarrow{a} \right|} = 2 \times \left( - \frac{\sqrt{2}}{2} \right) \cdot \frac{\overrightarrow{a}}{\sqrt{2}} = - \overrightarrow{a} = ( - 1, - 1)\),
故D正确}
\end{question}

\begin{question}
下列结论错误的是(    )
\begin{choices}
  \item 在回归模型中,决定系数\(R^{2}\)越大,则回归拟合的效果越好
  \item 已知随机变量\(X \sim N\left( 120,9^{2} \right)\),随机变量\(Y \sim N\left( 140,10^{2} \right)\),则\(P(X < 140) < P(Y < 160)\)
  \item 在对两个分类变量进行\(\chi^{2}\)独立性检验时,如果列联表中所有数据都缩小为原来的十分之一,在相同的检验标准下,再去判断两变量的关联性时,结论不会发生改变(\(\chi^{2} = \frac{n{(ad - bc)}^{2}}{(a + b)(c + d)(a + c)(b + d)}\))
  \item 由两个分类变量\(X,Y\)的成对样本数据计算得到\(\chi^{2} = 8.612\),依据\(\alpha = 0.005\)的独立性检验(\(x_{0.005} = 7.879\)),可判断\(X,Y\)独立
\end{choices}
\topics{相关系数的意义及辨析;独立性检验的概念及辨析;指定区间的概率}
\difficulty{0.4}
\answer{BCD}
\explain{对于A选项,在回归模型中,决定系数\(R^{2}\)越接近1,
说明回归模型对数据的拟合效果越好,\(R^{2}\)越大(越接近1),
则回归拟合的效果越好,该选项正确;\par
对于B选项,
\(P(X < 140) > P(X < \mu + 2\sigma) = P(X < 138)\),
\(P(Y < 160) = P(Y < \mu + 2\sigma)\),\par
所以\(P(X < 140) > P(Y < 160)\),故B错误;\par
对于C选项,若列联表中所有数据都缩小为原来的\(\frac{1}{10}\),
则\(\chi^{2}\)的值变为原来的\(\frac{1}{10}\),
所以结论可能会发生改变,故C错误;\par
对于D选项,
由\(\chi^{2} = 8.612 > x_{0.005}\)可得出"零假设\(H_{0}:X\)与\(Y\)独立"不成立,
所以有\(99.5\text{\%}\)的把握说\(X,Y\)有关,故D错误.}
\end{question}

\begin{question}
如图,在正方体\(ABCD - A_{1}B_{1}C_{1}D_{1}\)中,
点\(E,F,G\)分别在棱\(AA_{1},AB,AD\)上(不与棱的端点重合),
\(M\)为棱\(CC_{1}\)的中点,则下列说法正确的是(    )

\begin{center}
% PNG: hunan_yali_2026_mock3-Q11-img1
\includegraphics[width=0.4\textwidth]{content/exams/auto/hunan_yali_2026_mock3/images/image3.png}
\end{center}

\begin{choices}
  \choice 若平面\(EFG\)与正方体的每条棱的夹角都相同,则直线\(BD\) \(\parallel\)平面\(EFG\)
  \choice 三角形\(EFG\)不可能为直角三角形
  \choice 若\(G,F\)分别为棱\(AD,AB\)的中点,则存在点\(E\),使得\(AM\bot\)平面\(EGF\)
  \choice 若二面角\(A - GF - E\)的余弦值为\(\frac{1}{2}\),二面角\(A - EF - G\)的余弦值为\(\frac{\sqrt{3}}{3}\),则二面角\(A - EG - F\)的余弦值为\(\frac{\sqrt{15}}{6}\)
\end{choices}

\begin{center}
% PNG: hunan_yali_2026_mock3-Q11-img2
\includegraphics[width=0.4\textwidth]{content/exams/auto/hunan_yali_2026_mock3/images/image4.png}
\end{center}

\begin{center}
% PNG: hunan_yali_2026_mock3-Q11-img3
\includegraphics[width=0.4\textwidth]{content/exams/auto/hunan_yali_2026_mock3/images/image5.png}
\end{center}

\topics{面面平行证明线面平行;空间位置关系的向量证明;已知线面角求其他量;面面角的向量求法}
\difficulty{0.4}
\answer{ABD}
\explain{对于A,在正方体\(ABCD - A_{1}B_{1}C_{1}D_{1}\)中,
令点\(A\)到平面\(EFG\)的距离为\(h\),\par
由平面\(EFG\)与正方体每条棱的夹角都相同,
得棱\(AB,AD,AA_{1}\)与平面\(EFG\)所成角都相等,\par
则\(\frac{h}{AF} = \frac{h}{AG} = \frac{h}{AE}\),
于是\(AF = AG = AE\),而\(AB = AD\),
则\(\frac{AF}{AB} = \frac{AG}{AD}\),\(FG//BD\),\par
而\(BD \not\subset\)平面\(EFG\),
\(FG \subset\)平面\(EFG\),因此\(BD//\)平面\(EFG\),
A正确;\par
对于B,设\(AF = a,AE = b,AG = c\),
则\(EF^{2} = b^{2} + a^{2},GF^{2} = c^{2} + a^{2},EG^{2} = b^{2} + c^{2}\),\par
从而\(EF^{2} < GF^{2} + EG^{2}\),
则\(\angle EGF < 90^{\circ}\),同理,
\(\angle EFG < 90^{\circ},\angle GEF < 90^{\circ}\),\par
因此三角形\(EFG\)为锐角三角形,B正确;\par
对于C,以\(A\)为坐标原点建立空间直角坐标系,如图,不妨设正方体棱长为2,\par
则\(A(0,0,0),M(2,2,1),E(0,0,t),F(1,0,0)\),
\(\overrightarrow{AM} = (2,2,1),\overrightarrow{EF} = (1,0, - t)\),\par
假设存在点\(E\),使得\(AM\bot\)平面\(EGF\),
则\(AM\bot EF\),
\(\overrightarrow{AM} \cdot \overrightarrow{EF} = 2 - t = 0\),\par
解得\(t = 2\),即\(E\)与\(A_{1}\)重合,不符合题意,
因此不存在点\(E\),使得\(AM\bot\)平面\(EGF\),C错误;\par
对于D,由选项C得,
平面\(AEF\)的法向量为\(\overrightarrow{a} = (0,1,0)\),
平面\(AGF\)的法向量为\(\overrightarrow{b} = (0,0,1)\),\par
平面\(AEG\)的法向量为\(\overrightarrow{t} = (1,0,0)\),
设平面\(EFG\)的法向量为\(\overrightarrow{c} = (x,y,z),x,y,z > 0\),\par
则\(\frac{\sqrt{3}}{3} = \frac{\overrightarrow{a} \cdot \overrightarrow{c}}{|\overrightarrow{a}||\overrightarrow{c}|} = \frac{y}{\sqrt{x^{2} + y^{2} + z^{2}}}\),
解得\(2y^{2} = x^{2} + z^{2}\),\par
\(\frac{1}{2} = \frac{\overrightarrow{b} \cdot \overrightarrow{c}}{|\overrightarrow{b}||\overrightarrow{c}|} = \frac{z}{\sqrt{x^{2} + y^{2} + z^{2}}}\),
解得\(3z^{2} = x^{2} + y^{2}\),
因此\(y^{2} = \frac{4}{5}x^{2},z^{2} = \frac{3}{5}x^{2}\),\par
则\(\cos\langle t,\overrightarrow{c}\rangle = \frac{\overrightarrow{t} \cdot \overrightarrow{c}}{|\overrightarrow{t}||\overrightarrow{c}|} = \frac{x}{\sqrt{x^{2} + y^{2} + z^{2}}} = \frac{x}{\sqrt{\frac{12}{5}x^{2}}} = \frac{\sqrt{15}}{6}\),\par
所以二面角\(A - EG - F\)的余弦值为\(\frac{\sqrt{15}}{6}\),
D正确.}
\end{question}

\section{填空题}

\begin{question}
\({(x - y)}^{6}\)的展开式中,含\(x^{3}y^{3}\)项的系数为 .
\topics{求指定项的系数}
\difficulty{0.85}
\answer{\(- 20\)}
\explain{由二项式定理得\({(x - y)}^{6}\)的展开式的通项为\(T_{r + 1} = \mathbb{C}_{6}^{r} \cdot {( - 1)}^{r}x^{6 - r}y^{r}\),\par
令\(6 - r = 3\),解得\(r = 3\),
则\(x^{3}y^{3}\)项的系数为\(\mathbb{C}_{6}^{3} \cdot {( - 1)}^{3} = - 20\).\(- 20\)}
\end{question}

\begin{question}
已知双曲线\(C:\frac{x^{2}}{a^{2}} - \frac{y^{2}}{b^{2}} = 1(a > 0,b > 0)\)的左、右焦点分别为\(F_{1},F_{2}\),
一条渐近线为\(l\),
设过点\(F_{1}\)且与\(l\)垂直的直线为\(l_{1}\),
\(l_{1}\)交\(l\)于点\(N\),
\(l_{1}\)交双曲线\(C\)右支于点\(M\),
若\(\overrightarrow{ON} = \frac{1}{2}\left( \overrightarrow{OM} + \overrightarrow{OF_{1}} \right)\),
则双曲线\(C\)的两条渐近线方程为
.

\begin{center}
% PNG: hunan_yali_2026_mock3-Q13-img1
\includegraphics[width=0.4\textwidth]{content/exams/auto/hunan_yali_2026_mock3/images/image6.png}
\end{center}

\topics{根据a,b,c齐次式关系求渐近线方程}
\difficulty{0.65}
\answer{\(y = \pm 2x\)}
\explain{由题可得渐近线\(l\)的一条方程为\(y = - \frac{b}{a}x\),
即\(bx - ay = 0\),\par
则由题可得\(F( - c,0)\)到该渐近线的距离为\(\left| NF_{1} \right| = \frac{| - bc|}{\sqrt{b^{2} + ( - a)^{2}}} = \frac{bc}{\sqrt{c^{2}}} = b\),\par
又\(\left| OF_{1} \right| = c\),
所以\(|ON| = a\),\par
由题知\(N\)为\(MF_{1}\)的中点,则\(ON//MF_{2}\),
所以\(|ON| = \frac{1}{2}\left| MF_{2} \right|\),\par
故\(\left| MF_{2} \right| = 2a,\left| MF_{1} \right| = 2b\),\par
又因为\(\left| MF_{1} \right| - \left| MF_{2} \right| = 2a\),
所以\(\left| MF_{1} \right| = 4a\),
所以\(2b = 4a\)即\(\frac{b}{a} = 2\),\par
所以双曲线\(C\)的渐近线方程为\(y = \pm 2x\).\(y = \pm 2x\).}
\end{question}

\begin{question}
盒中有5张卡片,其中标有数字1的有2张,标有数字2的有1张,标有数字3的有2张,
每次从中取1张,不放回,
直到取出所有标有1的卡片为止.设此过程中取到标有数字3的卡片张数为\(X\),
则\(E(X) =\)
.
\topics{全排列问题;实际问题中的组合计数问题;计算古典概型问题的概率;求离散型随机变量的均值}
\difficulty{0.65}
\answer{\(\frac{4}{3}\)/\(1\frac{1}{3}\)}
\explain{由题意知\(X\)的可能取值为0,1,2,\par
\(X = 0\),取两次,两次均为标有数字1的卡片,
此时概率为\(\frac{2 \times 1}{\text{A}_{5}^{2}}\),\par
或取三次,前两次中一次取到1,一次取到2,第三次取到1,
此时概率为\(\frac{\mathbb{C}_{2}^{1}\text{A}_{2}^{2}}{\text{A}_{5}^{3}}\),\par
故\(P(X = 0) = \frac{2 \times 1}{\text{A}_{5}^{2}} + \frac{\mathbb{C}_{2}^{1}\text{A}_{2}^{2}}{\text{A}_{5}^{3}} = \frac{1}{6}\),\par
\(X = 1\),取三次,前两次中一次取到1,一次取到3,第三次取到1,
此时概率为\(\frac{\mathbb{C}_{2}^{1}\mathbb{C}_{2}^{1}\text{A}_{2}^{2}}{\text{A}_{5}^{3}}\),\par
或取四次,前三次中一次取到1,一次取到3,一次取到2,第四次取到1,\par
此时概率为\(\frac{\mathbb{C}_{2}^{1}\mathbb{C}_{2}^{1}\text{A}_{3}^{3}}{\text{A}_{5}^{4}}\),\par
故\(P(X = 1) = \frac{\mathbb{C}_{2}^{1}\mathbb{C}_{2}^{1}\text{A}_{2}^{2}}{\text{A}_{5}^{3}} + \frac{\mathbb{C}_{2}^{1}\mathbb{C}_{2}^{1}\text{A}_{3}^{3}}{\text{A}_{5}^{4}} = \frac{1}{3}\),\par
\(X = 2\),取四次,前三次中一次取到1,两次取到3,第四次取到1,
此时概率为\(\frac{\mathbb{C}_{2}^{1}\text{A}_{3}^{3}}{\text{A}_{5}^{4}}\),\par
或取五次,前四次中一次取到1,两次取到3,一次取到2,第四次取到1,\par
此时概率为\(\frac{\mathbb{C}_{2}^{1}\text{A}_{4}^{4}}{\text{A}_{5}^{5}}\),\par
故\(P(X = 2) = \frac{\mathbb{C}_{2}^{1}\text{A}_{3}^{3}}{\text{A}_{5}^{4}} + \frac{\mathbb{C}_{2}^{1}\text{A}_{4}^{4}}{\text{A}_{5}^{5}} = \frac{1}{2}\)(或\(P(X = 2) = 1 - P(X = 0) - P(X = 1) = \frac{1}{2}\)),\par
所以\(E(X) = 0 \times \frac{1}{6} + 1 \times \frac{1}{3} + 2 \times \frac{1}{2} = \frac{4}{3}\).\(\frac{4}{3}\)}
\end{question}

\section{解答题}

\begin{question}
在\(\triangle ABC\)中,内角\(A\),\(B\),
\(C\)所对的边分别为\emph{a},\emph{b},\emph{c},
\((a + b - c)(b - a + c) - 3ac = 0\).
\begin{enumerate}[label=(\arabic*)]
  \item 求\(B\);
  \item 若\(\angle ABC\)的角平分线交\(AC\)于点\(D\),且\(AD = 2DC = \frac{2\sqrt{7}}{3}\),求\(BD\).
\end{enumerate}
\topics{三角形面积公式及其应用;余弦定理解三角形}
\difficulty{0.65}
\answer{(1)\(\frac{2\pi}{3}\);
(2)\(\frac{2}{3}\).}
\explain{(1)因为\((a + b - c)(b - a + c) - 3ac = 0\),
所以\(b^{2} - (a - c)^{2} - 3ac = 0\),\par
所以\(- ac = a^{2} + c^{2} - b^{2}\),
所以\(\cos B = \frac{a^{2} + c^{2}\text{-}b^{2}}{2ac} = - \frac{1}{2}\),
因为\(B \in \left( 0,\pi \right)\),
所以\(B = \frac{2\pi}{3}\);\par
(2)因为\(AD = 2DC = \frac{2\sqrt{7}}{3}\),
所以\(b = \sqrt{7}\),\par
因为\(BD\)平分\(\angle ABC\),
所以\(\frac{c}{a} = \frac{AB}{BC} = \frac{AD}{DC} = 2\),
即\(c = 2a\),\par
由(1)知,\(- ac = a^{2} + c^{2} - b^{2}\),
解得\(a = 1\),\(c = 2\),\par
因为\(S_{\triangle ABC} = S_{\triangle ABD} + S_{\triangle DBC}\),
所以\(\frac{1}{2}ac\sin B = \frac{1}{2}a \cdot BD\sin\frac{B}{2} + \frac{1}{2}c \cdot BD\sin\frac{B}{2}\),\par
整理得\(BD = \frac{ac}{a + c} = \frac{2}{3}\).}
\end{question}

\begin{question}
已知数列\(\left\{ a_{n} \right\}\)是等差数列,
记其前\emph{n}项和为\(S_{n}\),且\(S_{3} = a_{5}\),
\(a_{2n} = 2a_{n} + \frac{1}{4}\).
\begin{enumerate}[label=(\arabic*)]
  \item 求数列\(\left\{ a_{n} \right\}\)的通项公式;
  \item 将数列\(\left\{ a_{n} \right\}\)与\(\left\{ \sqrt{S_{n}} \right\}\)的所有项从小到大排列得到数列\(\left\{ b_{n} \right\}\).证明:\(\frac{1}{b_{1}^{2}} + \frac{1}{b_{2}^{2}} + \cdots + \frac{1}{b_{n}^{2}} < 32\).
\end{enumerate}
\topics{等差数列通项公式的基本量计算;求等差数列前n项和;裂项相消法求和}
\difficulty{0.65}
\answer{(1)\(a_{n} = \frac{1}{2}n - \frac{1}{4}\)
(2)证明见解析}
\explain{(1)因为数列\(\left\{ a_{n} \right\}\)是等差数列,
前\(n\)项和为\(S_{n}\),且\(S_{3} = a_{5}\),\par
所以\(3a_{1} + 3d = a_{1} + 4d\),
可得\(2a_{1} = d\),\par
又因为\(a_{2n} = 2a_{n} + \frac{1}{4}\),
所以\(a_{2} = 2a_{1} + \frac{1}{4} = a_{1} + d\),
所以\(a_{1} = d - \frac{1}{4}\),\par
联立方程组\(\left\{ \begin{array}{r}
2a_{1} = d \\
a_{1} = d - \frac{1}{4}
\end{array} \right.\),解得\(a_{1} = \frac{1}{4},d = \frac{1}{2}\),\par
所以数列\(\left\{ a_{n} \right\}\)的通项公式\(a_{n} = \frac{1}{2}n - \frac{1}{4}\).\par
(2)证明:由(1)知,数列\(\left\{ a_{n} \right\}\)的通项公式\(a_{n} = \frac{1}{2}n - \frac{1}{4}\),\par
可得\(S_{n} = \frac{1}{4}n + \frac{n(n - 1)}{2} \times \frac{1}{2} = \frac{1}{4}n^{2}\),所以\(\sqrt{S_{n}} = \frac{1}{2}n = \frac{2n}{4}\),\par
因为\(a_{n} = \frac{1}{2}n - \frac{1}{4} = \frac{2n - 1}{4}\),\par
所以对任意正整数\(n\)有\(a_{n} = \frac{2n - 1}{4} < \sqrt{S_{n}} = \frac{2n}{4} < a_{n + 1} = \frac{2n + 1}{4}\),\par
所以将数列\(\{ a_{n}\}\)与\(\{\sqrt{S_{n}}\}\)的所有项从小到大排列得到的数列\(\{ b_{n}\}\)的通项公式为\(b_{n} = \frac{n}{4}\),\par
则\(b_{n}^{2} = \frac{1}{16}n^{2}\),所以当\(n = 1\)时,\(\frac{1}{b_{1}^{2}} = 16 < 32\);\par
所以当\(n \geq 2\)时,\(\frac{1}{b_{n}^{2}} = \frac{16}{n^{2}} < \frac{16}{n(n - 1)} = 16\left( \frac{1}{n - 1} - \frac{1}{n} \right)\),\par
所以\(\frac{1}{b_{1}^{2}} + \frac{1}{b_{2}^{2}} + + \frac{1}{b_{n}^{2}} < 16 + 16\left\lbrack \left( 1 - \frac{1}{2} \right) + \left( \frac{1}{2} - \frac{1}{3} \right) + + \left( \frac{1}{n - 1} - \frac{1}{n} \right) \right\rbrack = 32 - \frac{16}{n} < 32\),\par
综上可得\(\frac{1}{b_{1}^{2}} + \frac{1}{b_{2}^{2}} + + \frac{1}{b_{n}^{2}} < 32\).}
\end{question}

\begin{question}
已知平行六面体\(ABCD - A_{1}B_{1}C_{1}D_{1}\)中,
\(\angle AA_{1}D_{1} = 90{^\circ},\angle D_{1}A_{1}B_{1} = 45{^\circ},AA_{1} = A_{1}D_{1} = 1,A_{1}B_{1} = \sqrt{2}\).
\begin{enumerate}[label=(\arabic*)]
  \item 求证:\(B_{1}D\bot A_{1}B\);
  \item 若\(B_{1}D = \sqrt{2}\).
  \begin{enumerate}[label=(\roman*)]
    \item 求\(\angle AA_{1}B_{1}\)的大小;
    \item 求直线\(B_{1}D\)与平面\(A_{1}BC_{1}\)所成角.
  \end{enumerate}
\end{enumerate}

\begin{center}
% PNG: hunan_yali_2026_mock3-Q17-img1
\includegraphics[width=0.4\textwidth]{content/exams/auto/hunan_yali_2026_mock3/images/image7.png}
\end{center}


\begin{center}
% PNG: hunan_yali_2026_mock3-Q17-img2
\includegraphics[width=0.4\textwidth]{content/exams/auto/hunan_yali_2026_mock3/images/image8.png}
\end{center}

\topics{线面垂直证明线线垂直;空间位置关系的向量证明;线面角的向量求法}
\difficulty{0.65}
\answer{(1)证明见解析
(2)(i)\(\angle AA_{1}B_{1} = 90^{\circ}\) ;(ii)\(60^{\circ}\)}
\explain{(1)证明:法1:设\(\overrightarrow{A_{1}A} = \overrightarrow{a},\overrightarrow{A_{1}B_{1}} = \overrightarrow{b},\overrightarrow{A_{1}D_{1}} = \overrightarrow{c}\),
则\(\overrightarrow{B_{1}D} = \overrightarrow{a} + \overrightarrow{c} - \overrightarrow{b},\overrightarrow{A_{1}B} = \overrightarrow{a} + \overrightarrow{b}\),\par
从而\(\overrightarrow{B_{1}D} \cdot \overrightarrow{A_{1}B} = \left( \overrightarrow{a} + \overrightarrow{c} - \overrightarrow{b} \right) \cdot \left( \overrightarrow{a} + \overrightarrow{b} \right) = 1 + \overrightarrow{a} \cdot \overrightarrow{b} + \overrightarrow{c} \cdot \overrightarrow{b} - \overrightarrow{a} \cdot \overrightarrow{b} - 2 = \overrightarrow{c} \cdot \overrightarrow{b} - 1 = 1 \times \sqrt{2} \times \frac{\sqrt{2}}{2} - 1 = 0\),
所以\(B_{1}D\bot A_{1}B\).\par
法2:在\(\triangle A_{1}B_{1}D_{1}\)中,
\(\angle D_{1}A_{1}B_{1} = 45{^\circ}\),
\(A_{1}D_{1} = 1\)且\(A_{1}B_{1} = \sqrt{2}\),\par
由余弦定理可得\(B_{1}D_{1} = 1\),
所以\(B_{1}D_{1}^{2} + A_{1}D_{1}^{2} = A_{1}B_{1}^{2}\),
所以\(A_{1}D_{1}\bot B_{1}D_{1}\),\par
又由\(\angle AA_{1}D_{1} = 90{^\circ}\),
可得\(A_{1}D_{1}\bot DD_{1}\),\par
因为\(DD_{1} \cap B_{1}D_{1} = D_{1}\),
且\(DD_{1},B_{1}D_{1} \subset\)平面\(DD_{1}B_{1}\),
所以\(A_{1}D_{1}\bot\)平面\(DD_{1}B_{1}\),\par
又因为\(DB_{1} \subset\)平面\(DD_{1}B_{1}\),
所以\(A_{1}D_{1}\bot DB_{1}\),\par
因为\(DD_{1} = B_{1}D_{1} = 1\),
可得\(DD_{1}//B_{1}B,DD_{1} = B_{1}B\),
所以四边形\(DD_{1}B_{1}B\)为菱形,\par
可得\(B_{1}D\bot D_{1}B\),
且\(A_{1}D_{1} \cap BD_{1} = D_{1}\),
\(A_{1}D_{1},BD_{1} \subset\)平面\(A_{1}D_{1}B\),\par
所以\(B_{1}D\bot\)平面\(A_{1}D_{1}B\),\par
又因为\(A_{1}B \subset\)平面\(A_{1}D_{1}B\),
所以\(B_{1}D\bot A_{1}B\).\par
法3:由(1)知\(A_{1}D_{1}\bot DD_{1},A_{1}D_{1}\bot B_{1}D_{1}\),\par
因为\(DD_{1} \cap B_{1}D_{1} = D_{1}\),
\(DD_{1},B_{1}D_{1} \subset\)平面\(DD_{1}B_{1}\),
所以\(A_{1}D_{1}\bot\)平面\(DD_{1}B_{1}\),\par
以\(D_{1}\)为坐标原点,
以\(D_{1}A_{1},D_{1}B_{1},D_{1}D\)所在直线分别为\(x,y,z\)轴,
建立空间直角坐标系,\par
如图所示,
则\(D_{1}(0,0,0),B_{1}(0,1,0),A_{1}(1,0,0)\),
设\(D(0,a,b)\),可得\(B(0,a + 1,b)\),\par
因为\(D_{1}D = 1\),则\(a^{2} + b^{2} = 1\),
可得\(\overrightarrow{A_{1}B} = ( - 1,a + 1,b),\overrightarrow{B_{1}D} = (0,a - 1,b)\),\par
又因为\(\overrightarrow{A_{1}B} \cdot \overrightarrow{B_{1}D} = a^{2} + b^{2} - 1 = 0\),
所以\(B_{1}D\bot A_{1}B\).\par
(2)解:(i)法1:因为\(\angle AA_{1}D_{1} = 90{^\circ},\angle D_{1}A_{1}B_{1} = 45{^\circ},AA_{1} = A_{1}D_{1} = 1,A_{1}B_{1} = \sqrt{2}\),\par
设\(\overrightarrow{A_{1}A} = \overrightarrow{a},\overrightarrow{A_{1}B_{1}} = \overrightarrow{b},\overrightarrow{A_{1}D_{1}} = \overrightarrow{c}\),
可得\(\overrightarrow{B_{1}D} = \overrightarrow{a} + \overrightarrow{c} - \overrightarrow{b}\),\par
则\({\overrightarrow{B_{1}D}}^{2} = {(\overrightarrow{a} + \overrightarrow{c} - \overrightarrow{b})}^{2} = 2 - 2\overrightarrow{a} \cdot \overrightarrow{b}\),\par
因为\(B_{1}D = \sqrt{2}\),
所以\(2 - 2\overrightarrow{a} \cdot \overrightarrow{b} = 2\),
所以\(\overrightarrow{a} \cdot \overrightarrow{b} = 0\),
可得\(\overrightarrow{a}\bot\overrightarrow{b}\),
所以\(\angle AA_{1}B_{1} = 90^{\circ}\).\par
法2:因为\(B_{1}D = \sqrt{2},DD_{1} = D_{1}B_{1} = 1\),
故\(B_{1}D^{2} = DD_{1}^{2} + D_{1}B_{1}^{2}\),
所以\(DD_{1}\bot B_{1}D_{1}\),\par
又因为\(DD_{1}\bot A_{1}D_{1},A_{1}D_{1} \cap B_{1}D_{1} = D_{1},A_{1}D_{1},B_{1}D_{1} \subset\)平面\(A_{1}D_{1}B_{1}\),\par
所以\(DD_{1}\bot\)平面\(A_{1}B_{1}D_{1}\),
因为\(A_{1}B_{1} \subset\)平面\(A_{1}B_{1}D_{1}\),
所以\(DD_{1}\bot A_{1}B_{1}\),\par
又因为\(DD_{1}//AA_{1}\),
所以\(AA_{1}\bot A_{1}B_{1}\),
所以\(\angle AA_{1}B_{1} = 90^{\circ}\).\par
(ii)由(i)知\(AA_{1}\bot A_{1}B_{1}\),
则\(DD_{1}\bot A_{1}B_{1}\),
又\(DD_{1}\bot A_{1}D_{1},A_{1}D_{1} \cap A_{1}B_{1} = A_{1}\),\par
所以\(DD_{1}\bot\)平面\(A_{1}B_{1}D_{1}\),
又\(A_{1}D_{1}\bot D_{1}B_{1}\),
以\(D_{1}\)为坐标原点建系,\par
如图所示,
则\(A_{1}(1,0,0),C_{1}( - 1,1,0),B_{1}(0,1,0),D_{1}(0,0,0),D(0,0,1),B(0,1,1)\),\par
则\(\overrightarrow{B_{1}D} = (0, - 1,1),\overrightarrow{A_{1}B} = ( - 1,1,1),\overrightarrow{A_{1}C_{1}} = ( - 2,1,0)\),\par
设面\(A_{1}BC_{1}\)的法向量为\(\overrightarrow{m} = (x,y,z)\),
则\(\left\{ \begin{array}{r}
\overrightarrow{m} \cdot \overrightarrow{A_{1}B} = - x + y + z = 0 \\
\overrightarrow{m} \cdot \overrightarrow{A_{1}C_{1}} = - 2x + y = 0
\end{array} \right.\),\par
令\(x = 1\),可得\(y = 2,z = - 1\),所以\(\overrightarrow{m} = (1,2, - 1)\),\par
所以\(\cos\left\langle \overrightarrow{B_{1}D},\overrightarrow{m} \right\rangle = \frac{- 3}{\sqrt{2} \cdot \sqrt{6}} = - \frac{\sqrt{3}}{2}\),故直线\(B_{1}D\)与平面\(A_{1}BC_{1}\)所成角为\(60^{\circ}\).}
\end{question}

\begin{question}
已知椭圆\(C\)过点\(P\left( 1,\frac{\sqrt{2}}{2} \right)\),
且与双曲线\(4x^{2} - \frac{4}{3}y^{2} = 1\)有相同的焦点.
\begin{enumerate}[label=(\arabic*)]
  \item 求椭圆\(C\)的方程;
  \item 如图,\(A\)为椭圆\(C\)的上顶点,\(MN\)为椭圆\(C\)上的两点,点\(N\)关于\(y\)轴的对称点为\(N'\),设\(MN'\)的中点为点\(T\).
  \begin{enumerate}[label=\textcircled{\arabic*}]
    \item 设直线\(OT\)的斜率为\(k_{0}\),直线\(MN'\)的斜率为\(k\),求\(kk_{0}\)的值;
    \item 若射线\(AF_{1}\)为\(\angle MAN\)的平分线,求\(|AT|\)的取值范围.
  \end{enumerate}
\end{enumerate}

\begin{center}
% PNG: hunan_yali_2026_mock3-Q18-img1
\includegraphics[width=0.4\textwidth]{content/exams/auto/hunan_yali_2026_mock3/images/image9.png}
\end{center}

\topics{根据a;b;c求椭圆标准方程;求双曲线的焦点坐标;求椭圆中的参数及范围;由弦中点求弦方程或斜率}
\difficulty{0.4}
\answer{(1)\(\frac{x^{2}}{2} + y^{2} = 1\)
(2)①\(- \frac{1}{2}\) ;②\(|AT| \in \left( 1,\frac{4}{3} \right)\)}
\explain{(1)设椭圆\(C\)的标准方程为\(\frac{x^{2}}{a^{2}} + \frac{y^{2}}{b^{2}} = 1(a > b > 0)\),
\(c = \sqrt{a^{2} - b^{2}}\),
左右焦点为\(F_{1},F_{2}\),\par
双曲线\(4x^{2} - \frac{4}{3}y^{2} = 1\)的焦点为\(F_{1}( - 1,0)\)和\(F_{2}(1,0)\),
则\(c = 1\),\par
因为\(\left| PF_{1} \right| + \left| PF_{2} \right| = \sqrt{4 + \frac{1}{2}} + \sqrt{\frac{1}{2}} = 2\sqrt{2}\),\par
所以\(2a = 2\sqrt{2}\),可得\(a = \sqrt{2}\),
\(b^{2} = a^{2} - c^{2} = 1\),\par
故椭圆\(C\)的方程为\(\frac{x^{2}}{2} + y^{2} = 1\).\par
(2)①设\(M\left( x_{1},y_{1} \right)\)、\(N'\left( x_{2},y_{2} \right)\)、\(T\left( x_{0},y_{0} \right)\),\par
则有\(\left\{ \begin{array}{r}
\frac{x_{1}^{2}}{2} + y_{1}^{2} = 1 \\
\frac{x_{2}^{2}}{2} + y_{2}^{2} = 1
\end{array} \right.\),两式作差得\(\frac{x_{1}^{2} - x_{2}^{2}}{2} + y_{1}^{2} - y_{2}^{2} = 0\),\par
即\(\frac{\left( x_{1} - x_{2} \right)\left( x_{1} + x_{2} \right)}{2} = - \left( y_{1} - y_{2} \right)\left( y_{1} + y_{2} \right)\),即\(\frac{\left( x_{1} - x_{2} \right) \cdot 2x_{0}}{2} = - \left( y_{1} - y_{2} \right) \cdot 2y_{0}\),\par
由题设知\(x_{0} \neq 0\),\(x_{1} \neq x_{2}\),故\(\frac{y_{1} - y_{2}}{x_{1} - x_{2}} = - \frac{x_{0}}{2y_{0}}\),即\(k = - \frac{x_{0}}{2y_{0}}\),\par
又\(k_{0} = \frac{y_{0}}{x_{0}}\),则\(kk_{0} = \frac{y_{0}}{x_{0}} \cdot \frac{- x_{0}}{2y_{0}} = - \frac{1}{2}\).\par
②设直线\(AM\)、\(AN\)的倾斜角分别为\(\alpha\)和\(\beta\),则直线\(AN'\)的倾斜角为\(180^{\circ} - \beta\),\par
由题设知\(\alpha\)和\(\beta\)均不等于\(90^{\circ}\).\par
又直线\(AF_{1}\)的斜率为\(1\),故其倾斜角为\(45^{\circ}\),从而有\(\alpha + \beta = 90^{\circ}\),\par
从而\(\alpha = 90^{\circ} - \beta\),\(\tan\alpha = \tan\left( 90^{\circ} - \beta \right) = \frac{\sin\left( 90^{\circ} - \beta \right)}{\cos\left( 90^{\circ} - \beta \right)} = \frac{\cos\beta}{\sin\beta} = \frac{1}{\tan\beta}\),即\(\tan\alpha\tan\beta = 1\),\par
又\(\tan\beta = - \tan\left( 180^{\circ} - \beta \right)\),故\(\tan\left( 180^{\circ} - \beta \right)\tan\alpha = - 1\).\par
设直线\(AM\)、\(AN'\)的斜率分别为\(k_{1}\)、\(k_{2}\),则\(k_{1}k_{2} = - 1\).\par
设直线\(AM\)、\(AN'\)的方程分别为\(y = k_{1}x + 1\)、\(y = k_{2}x + 1\),\par
联立直线\(AM\)和曲线\(C\)得\(\left( 2k_{1}^{2} + 1 \right)x^{2} + 4k_{1}x = 0\),解得\(x_{1} = \frac{- 4k_{1}}{2k_{1}^{2} + 1}\),\par
代入直线得\(y_{1} = \frac{1 - 2k_{1}^{2}}{2k_{1}^{2} + 1}\),故点\(M\left( \frac{- 4k_{1}}{2k_{1}^{2} + 1},\frac{1 - 2k_{1}^{2}}{2k_{1}^{2} + 1} \right)\),同理可得\(N'\left( \frac{- 4k_{2}}{2k_{2}^{2} + 1},\frac{1 - 2k_{2}^{2}}{2k_{2}^{2} + 1} \right)\),\par
设直线\(MN'\)的方程为\(y = mx + n\),代入点\(M\)坐标得\(\frac{1 - 2k_{1}^{2}}{2k_{1}^{2} + 1} = m \cdot \frac{- 4k_{1}}{2k_{1}^{2} + 1} + n\),\par
化简得\(k_{1}^{2}(2n + 2) - 4mk_{1} + n - 1 = 0\),同理有\(k_{2}^{2}(2n + 2) - 4mk_{2} + n - 1 = 0\),\par
故\(k_{1}\)、\(k_{2}\)是方程\((2n + 2)x^{2} - 4mx + n - 1 = 0\)的两个根,故\(k_{1}k_{2} = \frac{n - 1}{2n + 2} = - 1\),\par
解得\(n = - \frac{1}{3}\),故直线\(MN'\)方程为\(y = m \cdot x - \frac{1}{3}\),过定点\(Q\left( 0, - \frac{1}{3} \right)\).\par
法1:由①知\(kk_{0} = - \frac{1}{2}\),即\(k_{CT}k_{QT} = - \frac{1}{2}\),即\(\frac{y_{0}}{x_{0}} \cdot \frac{y_{0} + \frac{1}{3}}{x_{0}} = - \frac{1}{2}\),即\(y_{0}\left( y_{0} + \frac{1}{3} \right) = - \frac{1}{2}x_{0}^{2}\),\par
所以\(- 2y_{0}\left( y_{0} + \frac{- 1}{3} \right) = x_{0}^{2}\),且\(- \frac{1}{3} < y_{0} < 0\),\par
从而\(|AT|^{2} = x_{0}^{2} + \left( y_{0} - 1 \right)^{2} = - 2y_{0}\left( y_{0} + \frac{1}{3} \right) + \left( y_{0} - 1 \right)^{2} = - y_{0}^{2} - \frac{8}{3}y_{0} + 1\),\par
因为\(- \frac{1}{3} < y_{0} < 0\),而\(y = - y_{0}^{2} - \frac{8}{3}y_{0} + 1\)在区间\(\left( - \frac{1}{3},0 \right)\)内单调递减,\par
所以\(|AT|^{2} \in \left( 1,\frac{16}{9} \right)\),故\(|AT| \in \left( 1,\frac{4}{3} \right)\).\par
法2:因为\(k_{1}k_{2} = - 1\),故\(AM\bot AN'\),故\(|AT| = \frac{1}{2}\left| MN' \right|\),只需求\(\left| MN' \right|\)的范围.\par
设直线\(MN'\)的方程为\(y = tx - \frac{1}{3}(t \neq 0)\),\par
与曲线\(C\)联立得\(\left( 18t^{2} + 9 \right)x^{2} - 12tx - 16 = 0\),\par
因为点\(Q\left( 0, - \frac{1}{3} \right)\)在曲线\(C\)内部,则必有\(\text{Δ} > 0\),则\(x_{1} + x_{2} = \frac{12t}{18t^{2} + 9}\),\(x_{1}x_{2} = \frac{- 16}{18t^{2} + 9}\),\par
从而\(\left| MN' \right| = \sqrt{1 + t^{2}}\left| x_{1} - x_{2} \right| = \sqrt{1 + t^{2}} \cdot \sqrt{\frac{144t^{2} + 64\left( 18t^{2} + 9 \right)}{\left( 18t^{2} + 9 \right)^{2}}} = \frac{4}{3} \cdot \frac{\sqrt{\left( 9t^{2} + 4 \right)\left( 1 + t^{2} \right)}}{2t^{2} + 1}\),\par
令\(2t^{2} + 1 = s \in (1, + \infty)\),则\(\left| MN' \right| = \frac{2}{3}\sqrt{- \frac{1}{s^{2}} + \frac{8}{s} + 9} = \frac{2}{3}\sqrt{- \left( \frac{1}{s} - 4 \right)^{2} + 25}\),\par
因为\(s > 1\),则\(\frac{1}{s} \in (0,1)\),则\(- 4 < \frac{1}{s} - 4 < - 3\),即\(9 < \left( \frac{1}{s} - 4 \right)^{2} < 16\),\par
故\(\left| MN' \right| = \frac{2}{3}\sqrt{- \left( \frac{1}{s} - 4 \right)^{2} + 25} \in \left( 2,\frac{8}{3} \right)\),所以\(|AT| \in \left( 1,\frac{4}{3} \right)\).}
\end{question}

\begin{question}
已知\(a > 2\),
函数\(f(x) = \ln|1 - x| + ax,x > 0\)且\(x \neq 1\).
\begin{enumerate}[label=(\arabic*)]
  \item 求函数\(f(x)\)的单调区间;
  \item 若\(a \in (2,3)\),\(f\left( x_{1} \right) = f\left( x_{2} \right) = f\left( x_{3} \right)\),且\(\frac{2}{5} < x_{1} < x_{2} < x_{3}\).
  \begin{enumerate}[label=\textcircled{\arabic*}]
    \item 求证:\(1 + \frac{1}{5a} < x_{3} < 1 + \frac{1}{2a}\);
    \item 求证:\(x_{1} + x_{3} < x_{1}x_{3} + 1.13\).
  \end{enumerate}
  (附参考数据:\(\ln3 \approx 1.1\))
\end{enumerate}
\topics{函数单调性;极值与最值的综合应用;利用导数证明不等式;利用导数求函数(含参)的单调区间}
\difficulty{0.15}
\answer{(1)\(f(x)\)在区间\(\left( 0,1 - \frac{1}{a} \right)\)和\((1, + \infty)\)内单调递增,在区间\(\left( 1 - \frac{1}{a},1 \right)\)内单调递减
(2)①证明见解析 ;②证明见解析}
\explain{(1)①
当\(x \in (0,1)\)时,
可得\(f(x) = \ln(1 - x) + ax\),
可得\(f'(x) = \frac{- 1}{1 - x} + a = \frac{ax - a + 1}{x - 1}\),\par
令\(f'(x) = 0\),
解得\(x = 1 - \frac{1}{a} \in \left( \frac{1}{2},1 \right)\),\par
当\(x \in \left( 0,1 - \frac{1}{a} \right)\)时,
\(f'(x) > 0\);
当\(x \in \left( 1 - \frac{1}{a},1 \right)\)时,
\(f'(x) < 0\),\par
所以\(f(x)\)在区间\(\left( 0,1 - \frac{1}{a} \right)\)上单调递增,
在区间\(\left( 1 - \frac{1}{a},1 \right)\)上单调递减;\par
②
当\(x \in (1, + \infty)\)时,
\(f(x) = \ln(x - 1) + ax\),
可得\(f'(x) = \frac{1}{x - 1} + a > 0\),\par
所以\(f(x)\)在区间\((1, + \infty)\)上单调递增,\par
综上所述,
\(f(x)\)在区间\(\left( 0,1 - \frac{1}{a} \right)\)和\((1, + \infty)\)内单调递增,
在区间\(\left( 1 - \frac{1}{a},1 \right)\)内单调递减.\par
(2)①由(1)知,当\(x arrow 1\)时,
\(f(x) arrow - \infty,f(0) = 0,f\left( 1 - \frac{1}{a} \right) = a - 1 - \ln a\),\par
因为\(f\left( x_{1} \right) = f\left( x_{2} \right) = f\left( x_{3} \right)\),\par
由函数\(f(x)\)的单调性,
可得\(\frac{2}{5} < x_{1} < 1 - \frac{1}{a},1 < x_{3},f\left( \frac{2}{5} \right) < f\left( x_{3} \right) < f\left( 1 - \frac{1}{a} \right)\),\par
又因为\(f\left( 1 + \frac{1}{2a} \right) = a + \frac{1}{2} - \ln2a,f\left( 1 + \frac{1}{5a} \right) = a + \frac{1}{5} - \ln5a\),\par
则\(f\left( 1 + \frac{1}{2a} \right) - f\left( 1 - \frac{1}{a} \right) = a + \frac{1}{2} - \ln2a - a + 1 + \ln a = \frac{3}{2} - \ln2 > 0\),\par
则\(f\left( 1 + \frac{1}{2a} \right) > f\left( x_{3} \right)\),
可得\(1 + \frac{1}{2a} > x_{3}\),
且\(f\left( \frac{2}{5} \right) - f\left( 1 + \frac{1}{5a} \right) = \ln3a - \frac{3}{5}a - \frac{1}{5}\),\par
令\(g(a) = \ln3a - \frac{3}{5}a - \frac{1}{5}\),
则\(g'(a) = \frac{1}{a} - \frac{3}{5} < 0\),
故\(g(a)\)在区间\((2,3)\)上单调递减,\par
所以\(g(a) > g(3) = 2\ln3 - 2 > 0\),
故\(f\left( \frac{2}{5} \right) > f\left( 1 + \frac{1}{5a} \right)\),\par
所以\(f\left( x_{3} \right) > f\left( 1 + \frac{1}{5a} \right)\),
可得\(1 + \frac{1}{5a} < x_{3}\),\par
综上可得,
\(1 + \frac{1}{5a} < x_{3} < 1 + \frac{1}{2a}\).\par
②因为\(f\left( x_{1} \right) = f\left( x_{2} \right) = f\left( x_{3} \right)\),
可得\(\ln\left( 1 - x_{1} \right) + ax_{1} = \ln\left( x_{3} - 1 \right) + ax_{3}\),\par
即\(\ln\left( 1 - x_{1} \right) - \ln\left( x_{3} - 1 \right) = ax_{3} - ax_{1} = a\left\lbrack \left( x_{3} - 1 \right) + \left( 1 - x_{1} \right) \right\rbrack\),\par
即\(a = \frac{\ln\left( 1 - x_{1} \right) - \ln\left( x_{3} - 1 \right)}{\left( x_{3} - 1 \right) + \left( 1 - x_{1} \right)} = \frac{\ln\frac{1 - x_{1}}{x_{3} - 1}}{\left( x_{3} - 1 \right) + \left( 1 - x_{1} \right)}\),\par
所以\(a\sqrt{\left( x_{3} - 1 \right)\left( 1 - x_{1} \right)} = \frac{\sqrt{\left( x_{3} - 1 \right)\left( 1 - x_{1} \right)} \cdot \ln\frac{1 - x_{1}}{x_{3} - 1}}{\left( x_{3} - 1 \right) + \left( 1 - x_{1} \right)} = \frac{\ln\frac{1 - x_{1}}{x_{3} - 1}}{\sqrt{\frac{x_{3} - 1}{1 - x_{1}}} + \sqrt{\frac{1 - x_{1}}{x_{3} - 1}}}\),\par
法1:令\(t = \sqrt{\frac{1 - x_{1}}{x_{3} - 1}}\),
可知\(\frac{1}{a} < 1 - x_{1} < \frac{3}{5},\frac{1}{5a} < x_{3} - 1 < \frac{1}{2a}\),
所以\(2 < \frac{1 - x_{1}}{x_{3} - 1} < 3a\),\par
则\(a\sqrt{\left( x_{3} - 1 \right)\left( 1 - x_{1} \right)} = \frac{2\ln t}{t + \frac{1}{t}} = \frac{2t\ln t}{t^{2} + 1},t \in \left( \sqrt{2},\sqrt{3a} \right)\),\par
令\(h(t) = \frac{2t\ln t}{t^{2} + 1}\),
则\(h'(t) = 2 \cdot \frac{t^{2} + 1 + \left( 1 - t^{2} \right)\ln t}{\left( t^{2} + 1 \right)^{2}} = 2\left( 1 - t^{2} \right)\frac{\frac{t^{2} + 1}{1 - t^{2}} + \ln t}{\left( t^{2} + 1 \right)^{2}}\),\par
令\(H(t) = \frac{t^{2} + 1}{1 - t^{2}} + \ln t = \frac{2}{1 - t^{2}} - 1 + \ln t\),
则\(H'(t) = \frac{1}{t} + \frac{4t}{\left( 1 - t^{2} \right)^{2}} > 0\),\par
故\(H(t)\)在区间\(\left( \sqrt{2}, + \infty \right)\)上单调递增,
又\(H(3) = \ln3 - 1.25 < 0\),\par
所以\(H(t) < 0,t \in \left( \sqrt{2},3 \right)\),\par
因为\(1 - t^{2} < 0\),
故\(h'(t) > 0,t \in \left( \sqrt{2},3 \right)\),
则\(h(t)\)在区间\(\left( \sqrt{2},3 \right)\)上单调递增,\par
则\(h(t) < h\left( \sqrt{3a} \right) < h(3) = \frac{6\ln3}{10} = \frac{3\ln3}{5}\),\par
所以\(a\sqrt{\left( x_{3} - 1 \right)\left( 1 - x_{1} \right)} < \frac{3\ln3}{5}\),
可得\(\left( x_{3} - 1 \right)\left( 1 - x_{1} \right) < \frac{9 \times {(\ln3)}^{2}}{25a^{2}}\),\par
可得\(x_{1} + x_{3} < x_{1}x_{3} + 1 + \frac{9 \times {(\ln3)}^{2}}{25a^{2}}\),\par
又因为\(\frac{9 \times {(\ln3)}^{2}}{25a^{2}} < \frac{9}{25} \times \left( \frac{\ln3}{2} \right)^{2} \approx 0.1089\),
故\(x_{1} + x_{3} < x_{1}x_{3} + 1 + 0.1089 < x_{1}x_{3} + 1.13\).\par
法2:根据对数均值不等式,
可得\(\sqrt{\left( x_{3} - 1 \right)\left( 1 - x_{1} \right)} < \frac{\left( 1 - x_{1} \right) - \left( x_{3} - 1 \right)}{\ln\left( 1 - x_{1} \right) - \ln\left( x_{3} - 1 \right)} = \frac{\left( 1 - x_{1} \right) - \left( x_{3} - 1 \right)}{\ln\frac{1 - x_{1}}{x_{3} - 1}}\),\par
因为\(a = \frac{\ln\frac{1 - x_{1}}{x_{3} - 1}}{\left( x_{3} - 1 \right) + \left( 1 - x_{1} \right)}\),
则\(a\sqrt{\left( x_{3} - 1 \right)\left( 1 - x_{1} \right)} < \frac{\left( 1 - x_{1} \right) - \left( x_{3} - 1 \right)}{\left( x_{3} - 1 \right) + \left( 1 - x_{1} \right)} = \frac{\frac{1 - x_{1}}{x_{3} - 1} - 1}{\frac{1 - x_{1}}{x_{3} - 1} + 1}\),\par
令\(t = \frac{1 - x_{1}}{x_{3} - 1} \in (2,3a)\),
则\(a\sqrt{\left( x_{3} - 1 \right)\left( 1 - x_{1} \right)} < \frac{t - 1}{t + 1} = 1 - \frac{2}{t + 1} < 1 - \frac{2}{3a + 1}\),\par
故\(\left( x_{3} - 1 \right)\left( 1 - x_{1} \right) < \left\lbrack \frac{1}{a} - \frac{2}{a(3a + 1)} \right\rbrack^{2}\),\par
令\(h(a) = \frac{1}{a} - \frac{2}{a(3a + 1)}\),
则\(h'(a) = \frac{- 9a^{2} + 6a + 1}{a^{2}{(3a + 1)}^{2}} < 0\),\par
所以\(h(a)\)在区间\((2,3)\)上单调递减,
故\(h(a) < h(2) = \frac{5}{14}\),\par
可得\(\left( x_{3} - 1 \right)\left( 1 - x_{1} \right) < \frac{25}{196} \approx 0.128\),
故\(x_{1} + x_{3} < x_{1}x_{3} + 1 + \frac{25}{196}\),\par
因为\(1 + \frac{25}{196} \approx 1.128 < 1.13\),
所以\(x_{1} + x_{3} < x_{1}x_{3} + 1.13\),得证.}
\end{question}
