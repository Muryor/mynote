% content/exams/exam02.tex

% Set per-exam title (appears in header)
\examxtitle{高一上学期期中数学试题}

\section*{Ⅰ. 单选题}

% Example 1: Using the \mcq convenience macro (recommended for simple MCQs)
\mcq[B]{已知集合 $A=\{x\mid \log_2 x < 1\},\, B=\{x\mid x<1\}$,则 $A\cap B$ 等于}
{$(-\infty,1)$}{$(0,1)$}{$(-\infty,2)$}{$(0,2)$}[
\topics{交集;不等式与函数单调性}
\difficulty{0.40}
\explain{由 $\log_2 x<1\Rightarrow 0<x<2$,与 $x<1$ 取交得 $(0,1)$。}
\source{自编}
]

% Example 2: Using traditional exam-zh syntax (for more control)
\begin{question}
已知函数 $f(x)=2^x$,则 $f(0)+f(1)$ 等于 \paren[C]
\begin{choices}
  \choice $1$
  \choice $2$
  \choice $3$
  \choice $4$
\end{choices}
\topics{指数函数;基本运算}
\difficulty{0.20}
\explain{$f(0)=2^0=1$,$f(1)=2^1=2$,故 $f(0)+f(1)=3$。}
\end{question}


\begin{question}
已知向量$\vec{a}=(0,1),\vec{b}=(2,x)$ ,若$\vec{b}\perp(\vec{b}-4\vec{a})$ ,则x=()\paren[C]
\begin{choices}
 \choice $1$
 \choice $-2$
 \choice $-1$
 \choice $2$
\end{choices}
\topics{平面向量线性运算的坐标表示;向量垂直的坐标表示}
\difficulty{0.85}
\explain{
根据向量垂直的坐标运算可求$x$ 的值
因为$\vec{b}\perp\left(\vec{b}-4\vec{a}\right)$ ,所以$\vec{b}\cdot\left(\vec{b}-4\vec{a}\right)=0$ 所以$\vec{b}^{2}-4\vec{a}\cdot\vec{b}=0$ 即$4+x^{2}-4x=0$ ,故$x=2$ 故选:D.
本题答案:D。
}\end{question}



\section*{Ⅱ. 填空题}

\begin{question}
  若指数函数 $f(x)=2^x$,则 $f(0)+f(1)=\fillin[3]{}$。
  \topics{指数函数;基本运算}
  \difficulty{0.20}
\end{question}

\begin{question}
  函数 $f(x)=x^2-4x+3$ 的最小值为 \fillin[-1]{}。
  \topics{二次函数;顶点式}
  \difficulty{0.40}
  \explain{$f(x)=(x-2)^2-1$,最小值为 $-1$。}
\end{question}
