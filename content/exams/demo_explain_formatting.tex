% 详解格式化功能演示 - 展示 \exstep 和段落分隔的使用
\examxtitle{详解格式化功能演示}

\section{单选题}

\begin{question}
已知函数 $f(x) = 5\cos x - \cos 5x$ 在区间 $\left[0, \frac{\pi}{4}\right]$ 的最大值为(   )
\begin{choices}
  \item $3\sqrt{2}$
  \item $3\sqrt{3}$
  \item $4$
  \item $5$
\end{choices}
\topics{导数求最值;三角函数}
\difficulty{0.65}
\answer{B}
\explain{
对 $f(x)$ 求导:$f'(x) = -5\sin x + 5\sin 5x = 10\cos 3x\sin 2x$。

因为 $x \in \left[0, \frac{\pi}{4}\right]$,故 $\sin 2x \geq 0$。当 $0 < x < \frac{\pi}{6}$ 时,$\cos 3x > 0$ 即 $f'(x) > 0$;当 $\frac{\pi}{6} < x < \frac{\pi}{4}$ 时,$\cos 3x < 0$ 即 $f'(x) < 0$。

故 $f(x)$ 在 $\left[0, \frac{\pi}{6}\right)$ 上递增,在 $\left(\frac{\pi}{6}, \frac{\pi}{4}\right]$ 上递减。

因此最大值为 $f\left(\frac{\pi}{6}\right) = 5\cos\frac{\pi}{6} - \cos\frac{5\pi}{6} = \frac{5\sqrt{3}}{2} + \frac{\sqrt{3}}{2} = 3\sqrt{3}$。
}
\end{question}

\begin{question}
已知双曲线 $C$ 的虚轴长是实轴长的 $\sqrt{7}$ 倍,则 $C$ 的离心率为(   )
\begin{choices}
  \item $\sqrt{2}$
  \item $2$
  \item $\sqrt{7}$
  \item $2\sqrt{2}$
\end{choices}
\topics{求双曲线的离心率}
\difficulty{0.85}
\answer{D}
\explain{
设双曲线的实轴、虚轴、焦距分别为 $2a, 2b, 2c$。
\exstep
由题知 $b = \sqrt{7}a$。
\exstep
由双曲线性质 $c^2 = a^2 + b^2 = a^2 + 7a^2 = 8a^2$,则 $c = 2\sqrt{2}a$。
\exstep
因此离心率 $e = \frac{c}{a} = 2\sqrt{2}$。
}
\end{question}

\section{解答题}

\begin{question}
在四棱锥 $P-ABCD$ 中,$PA \perp$ 平面 $ABCD$,$AB \bot AD$,$BC \parallel AD$。

(1) 证明:平面 $PAB \bot$ 平面 $PAD$;

(2) 设 $PA = AB = \sqrt{2}$,$BC = 2$,$AD = 1 + \sqrt{3}$,且点 $P, B, C, D$ 均在球 $O$ 的球面上。

(i) 证明:点 $O$ 在平面 $ABCD$ 内;

(ii) 求直线 $AC$ 与 $PO$ 所成角的余弦值。
\topics{面面垂直;外接球;异面直线夹角}
\difficulty{0.65}
\answer{(1)证明见解析;(2)(i)证明见解析;(ii)$\frac{\sqrt{2}}{3}$}
\explain{
\exstep[证明(1)]
在四棱锥 $P-ABCD$ 中,$PA \perp$ 平面 $ABCD$,$AB \bot AD$。

因为 $AB \subset$ 平面 $ABCD$,$AD \subset$ 平面 $ABCD$,所以 $AP \bot AB$,$AP \bot AD$。

因为 $AP \subset$ 平面 $PAD$,$AD \subset$ 平面 $PAD$,$AP \cap AD = A$,所以 $AB \bot$ 平面 $PAD$。

因为 $AB \subset$ 平面 $PAB$,所以平面 $PAB \bot$ 平面 $PAD$。

\exstep[证明(2)(i)]
建立以 $A$ 为原点的空间直角坐标系,则 $A(0,0,0)$,$B(\sqrt{2},0,0)$,$C(\sqrt{2},2,0)$,$D(0,1+\sqrt{3},0)$,$P(0,0,\sqrt{2})$。

设球心为 $O(0,y_0,0)$(在平面 $xAy$ 上),由 $|OP| = |OB| = |OC| = |OD|$ 可得:

从 $|OP|^2 = |OB|^2$ 得:$y_0^2 + 2 = 2 + y_0^2$,恒成立。

从 $|OB|^2 = |OC|^2$ 得:$2 + y_0^2 = 2 + (y_0-2)^2$,解得 $y_0 = 1$。

验证:$|OP| = |OB| = |OC| = |OD| = \sqrt{3}$,成立。

因此点 $O(0,1,0) \in$ 平面 $ABCD$。

\exstep[计算(2)(ii)]
$\vec{AC} = (\sqrt{2}, 2, 0)$,$\vec{PO} = (0, 1, -\sqrt{2})$。

设直线 $AC$ 与 $PO$ 所成角为 $\theta$,则:

$$\cos\theta = \frac{|\vec{AC} \cdot \vec{PO}|}{|\vec{AC}| \cdot |\vec{PO}|} = \frac{|0 + 2 \times 1 + 0|}{\sqrt{2 + 4 + 0} \times \sqrt{0 + 1 + 2}} = \frac{2}{\sqrt{6} \times \sqrt{3}} = \frac{2}{3\sqrt{2}} = \frac{\sqrt{2}}{3}$$

因此所求角的余弦值为 $\frac{\sqrt{2}}{3}$。
}
\end{question}

\begin{question}
已知椭圆 $C$ 的中心在坐标原点,焦点在 $x$ 轴上,点 $A(0,-1)$,$B(a,0)$ 在椭圆 $C$ 上,$|AB| = \sqrt{10}$,离心率 $e = \frac{2\sqrt{2}}{3}$。

(1) 求 $C$ 的方程;

(2) 已知动点 $P$ 不在 $y$ 轴上,点 $R$ 在射线 $AP$ 上,且满足 $|AP| \cdot |AR| = 3$。

(i) 设 $P(m,n)$,求 $R$ 的坐标(用 $m, n$ 表示);

(ii) 设 $O$ 为坐标原点,$Q$ 是 $C$ 上的动点,直线 $OR$ 的斜率为直线 $OP$ 的斜率的 3 倍,求 $|PQ|$ 的最大值。
\topics{椭圆标准方程;椭圆中的最值问题}
\difficulty{0.40}
\answer{(1)$\frac{x^2}{9} + y^2 = 1$;(2)(i)$\left(\frac{3m}{m^2+(n+1)^2}, \frac{n+2-m^2-n^2}{m^2+(n+1)^2}\right)$;(ii)$3(\sqrt{3}+\sqrt{2})$}
\explain{
\exstep[求解(1)]
由题可知 $A(0,-b)$,$B(a,0)$,所以:
$$\begin{cases}
\sqrt{a^2 + b^2} = \sqrt{10} \\
e = \frac{c}{a} = \frac{2\sqrt{2}}{3} \\
c^2 = a^2 - b^2
\end{cases}$$

解得 $a^2 = 9$,$b^2 = 1$,$c^2 = 8$。

故椭圆 $C$ 的标准方程为 $\frac{x^2}{9} + y^2 = 1$。

\exstep[求解(2)(i)]
设 $R(x_0, y_0)$,易知 $m \neq 0$。

因为 $R$ 在射线 $AP$ 上,设 $\vec{AR} = \lambda\vec{AP}$($\lambda > 0$)。

由 $|AR| \cdot |AP| = 3$ 得:$\lambda \cdot [m^2 + (n+1)^2] = 3$,所以 $\lambda = \frac{3}{m^2+(n+1)^2}$。

因此:
$$\vec{AR} = \lambda\vec{AP} = \lambda(m, n+1) = \left(\frac{3m}{m^2+(n+1)^2}, \frac{3(n+1)}{m^2+(n+1)^2}\right)$$

由 $R = A + \vec{AR}$,得:
$$R\left(\frac{3m}{m^2+(n+1)^2}, \frac{n+2-m^2-n^2}{m^2+(n+1)^2}\right)$$

\exstep[求解(2)(ii)]
由 $k_{OR} = 3k_{OP}$,即 $\frac{n+2-m^2-n^2}{3m} = 3 \cdot \frac{n}{m}$。

化简得 $m^2 + n^2 + 8n - 2 = 0$,即 $m^2 + (n+4)^2 = 18$($m \neq 0$)。

所以点 $P$ 在以 $N(0,-4)$ 为圆心、$3\sqrt{2}$ 为半径的圆上(除去两点)。

\exstep[求最大值]
$|PQ|$ 的最大值为 $Q$ 到圆心 $N$ 的距离加上半径。

设 $Q(x_Q, y_Q)$,则 $\frac{x_Q^2}{9} + y_Q^2 = 1$。

$$|QN|^2 = x_Q^2 + (y_Q+4)^2 = 9-9y_Q^2 + y_Q^2 + 8y_Q + 16 = -8y_Q^2 + 8y_Q + 25$$
$$= -8\left(y_Q - \frac{1}{2}\right)^2 + 27 \leq 27$$

当且仅当 $y_Q = \frac{1}{2}$ 时取等号。

因此 $|PQ|_{\max} = \sqrt{27} + 3\sqrt{2} = 3\sqrt{3} + 3\sqrt{2} = 3(\sqrt{3} + \sqrt{2})$。
}
\end{question}

\begin{question}
设 $b \in \mathbb{R}$,若存在 $\varphi \in \mathbb{R}$ 使得 $5\cos x - \cos(5x + \varphi) \leq b$ 对 $x \in \mathbb{R}$ 恒成立,求 $b$ 的最小值。
\topics{导数求最值;三角恒等变换}
\difficulty{0.15}
\answer{$3\sqrt{3}$}
\explain{
\exstep[分析]
设 $g_\varphi(x) = 5\cos x - \cos(5x + \varphi)$。若存在 $\varphi$ 使得 $g_\varphi(x) \leq b$ 恒成立,则对这样的 $\varphi$,同样有 $g_\varphi(x) = -g_\varphi(x+\pi) \geq -b$。

所以 $|g_\varphi(x)| \leq b$ 对任意 $x$ 恒成立,这直接得到 $b \geq 0$。

\exstep[构造不等式]
设 $\frac{\varphi}{6} - \frac{\pi}{6} = m$,根据 $|g_\varphi(x)| \leq b$ 恒成立,有:

$$b \geq \left|g_\varphi\left(-\frac{\varphi}{6} + \frac{\pi}{6}\right)\right| = \left|6\cos m\right| = 6|\cos m|$$
$$b \geq \left|g_\varphi\left(-\frac{\varphi}{6} - \frac{\pi}{6}\right)\right| = 6\left|\cos\left(m + \frac{\pi}{3}\right)\right|$$
$$b \geq \left|g_\varphi\left(-\frac{\varphi}{6} + \frac{\pi}{2}\right)\right| = 6\left|\cos\left(m - \frac{\pi}{3}\right)\right|$$

所以 $|\cos m|, \left|\cos(m+\frac{\pi}{3})\right|, \left|\cos(m-\frac{\pi}{3})\right|$ 均不超过 $\frac{b}{6}$。

\exstep[利用二倍角公式]
结合 $\cos 2x = 2|\cos x|^2 - 1$,得到 $\cos 2m, \cos(2m+\frac{2\pi}{3}), \cos(2m-\frac{2\pi}{3})$ 均不超过 $\frac{b^2}{18} - 1$。

\exstep[反证法]
假设 $b < 3\sqrt{3}$,则 $\frac{b^2}{18} - 1 < \frac{27}{18} - 1 = \frac{1}{2}$。

故 $\cos 2m, \cos(2m+\frac{2\pi}{3}), \cos(2m-\frac{2\pi}{3}) \in [-1, \frac{1}{2})$。

但这三个角将单位圆三等分,三个点不可能都在直线 $x = \frac{1}{2}$ 左侧,矛盾。

\exstep[验证最小值]
所以 $b \geq 3\sqrt{3}$。另一方面,取 $\varphi = 0$,由第 (1) 问知 $f(x) = 5\cos x - \cos 5x \leq 3\sqrt{3}$。

因此 $b$ 的最小值是 $3\sqrt{3}$。
}
\end{question}
