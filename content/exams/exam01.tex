\examxtitle{无锡高三期中测试}
\section{单选题}

\begin{question}
设复数 $z=(1+\mathrm{i}a)(2-\mathrm{i})$,若 $z$ 的实部与虚部相等,则实数 $a$ 的值为(\quad)
\begin{choices}
  \item $3$
  \item $1$
  \item $-1$
  \item $-3$
\end{choices}
\topics{求复数的实部与虚部;复数代数式的乘法运算}
\difficulty{0.85}
\explain{分析:根据复数的乘法运算求出 $z=2+a+(2a-1)\mathrm{i}$,由题意列出方程,即可求得答案。\\
详解:因为 $z=(1+\mathrm{i}a)(2-\mathrm{i})=2+a+(2a-1)\mathrm{i}$,且 $z$ 的实部与虚部相等,故 $2+a=2a-1$,解得 $a=3$,故选:A。}
\end{question}


\begin{question}
已知集合 $M=\{\,x\mid x^2-2x-3\le 0\,\}$,$N=\{\,x\mid y=\ln(x-4)\,\}$,则 $(\complement_{\mathbf{R}}M)\cap N$ 等于 (\quad)
\begin{choices}
  \item $(-1,4]$
  \item $(-1,3]$
  \item $(3,+\infty)$
  \item $(4,+\infty)$
\end{choices}
\topics{交集的概念及运算;补集的概念及运算;求对数型复合函数的定义域;解不含参数的一元二次不等式}
\difficulty{0.85}
\explain{分析:先解一元二次不等式求出集合 $M$,再求对数型复合函数定义域求出集合 $N$,再根据补集和交集运算求解即可。\\
详解:$M=\{\,x\mid x^2-2x-3\le 0\,\}=\{\,x\mid -1\le x\le 3\,\}$,所以 $\complement_{\mathbf{R}} M=\{\,x\mid x<-1\text{ 或 }x>3\,\}$;又 $N=\{\,x\mid y=\ln(x-4)\,\}=\{\,x\mid x>4\,\}$,所以 $(\complement_{\mathbf{R}}M)\cap N=\{\,x\mid x>4\,\}=(4,+\infty)$。故选:D。}
\end{question}

\begin{question}
已知 $a\ne0$,命题 $P:x=1$ 是一元二次方程 $ax^2+bx+c=0$ 的一个根,命题 $q:a+b+c=0$,则 $P$ 是 $q$ 的 (\quad)
\begin{choices}
  \item 充分不必要条件
  \item 必要不充分条件
  \item 充分必要条件
  \item 既不充分也不必要条件
\end{choices}

\topics{充分条件、必要条件、充要条件的证明}
\difficulty{0.85}
\explain{分析:根据充分、必要性的定义判断命题间的推出关系,即可得答案。\\
详解:对于命题 $P$,$x=1$ 为方程的根,则 $a+b+c=0$,充分性成立;对命题 $q$,$a+b+c=0$ 且 $a\ne0$,则 $x=1$ 必是题设方程的一个根,必要性成立;所以 $P$ 是 $q$ 的充分必要条件。故选:C。}
\end{question}

\begin{question}
在 $\triangle ABC$ 中,$D$ 是 $BC$ 的中点,$E$ 是 $AD$ 的中点。若 $\overrightarrow{BE}=\lambda\overrightarrow{AB}+\mu\overrightarrow{AC}$,则 $\dfrac{\lambda}{\mu}=$ (\quad)
\begin{choices}
  \item $3$
  \item $-3$
  \item $2$
  \item $-2$
\end{choices}

\topics{平面向量的混合运算;用基底表示向量}
\difficulty{0.85}
\explain{
  \begin{center}
  \begin{tikzpicture}[scale=1.05,>=Stealth,line cap=round,line join=round]
    \coordinate (B) at (0,0);
    \coordinate (C) at (5.2,0.35);
    \coordinate (A) at (2.9,3.6);
    \path (B) -- (C) coordinate[pos=0.5] (D);
    \path (A) -- (D) coordinate[pos=0.5] (E);
    \draw[thick] (B)--(A)--(C)--cycle;
    \draw[thick] (A)--(D);
    \draw[thick] (B)--(E);
    \fill (A) circle (1.25pt) node[above] {$A$};
    \fill (B) circle (1.25pt) node[below left=-1pt] {$B$};
    \fill (C) circle (1.25pt) node[below right=-1pt] {$C$};
    \fill (D) circle (1.25pt) node[below=2pt] {$D$};
    \fill (E) circle (1.25pt) node[right=2pt] {$E$};
  \end{tikzpicture}
  \end{center}
  分析:利用向量的线性运算求解即可。\\
  详解:$\overrightarrow{BE}=\dfrac12(\overrightarrow{BA}+\overrightarrow{BD})=-\dfrac12\overrightarrow{AB}+\dfrac14\overrightarrow{BC}=-\dfrac12\overrightarrow{AB}+\dfrac14(\overrightarrow{AC}-\overrightarrow{AB})=-\dfrac34\overrightarrow{AB}+\dfrac14\overrightarrow{AC}$,所以 $\lambda=-\dfrac34,\ \mu=\dfrac14$,所以 $\dfrac{\lambda}{\mu}=-3$。故选:B。}
\end{question}

\begin{question}
已知函数 $f(x)$ 是定义在 $\mathbb{R}$ 上的奇函数,当 $x<0$ 时,$f(x)=10^x$,则 $f(\lg3)=$ (\quad)
\begin{choices}
  \item $-3$
  \item $3$
  \item $-\dfrac13$
  \item $\dfrac13$
\end{choices}

\topics{函数奇偶性的应用;对数的运算}
\difficulty{0.85}
\explain{分析:利用函数为奇函数,将 $f(\lg3)$ 化为 $-f\!\left(\lg\dfrac13\right)$,结合当 $x<0$ 时 $f(x)=10^x$,代入求值,即可求得答案。\\
详解:由题意知函数 $f(x)$ 是定义在 $\mathbb{R}$ 上的奇函数,故 $f(\lg3)=-f(-\lg3)=-f\!\left(\lg\dfrac13\right)$,而 $\lg\dfrac13<0$,故 $f\!\left(\lg\dfrac13\right)=10^{\lg(1/3)}=\dfrac13$,则 $f(\lg3)=-\dfrac13$,故选:C。}
\end{question}

\begin{question}
在数列 $\{a_n\}$ 中,$a_1=1$,若数列 $\{a_1,a_3,\dots,a_{2n-1}\}$ 是公比为 $2$ 的等比数列,则 $a_1+a_3+a_5+\cdots+a_9=$ (\quad)
\begin{choices}
  \item $2048$
  \item $2047$
  \item $1024$
  \item $1023$
\end{choices}
\end{question}
\topics{由定义判定等比数列;求等比数列前 $n$ 项和}
\difficulty{0.65}
\explain{分析:根据等比数列定义得数列 $\{a_n\}$ 的奇数项是以 $1$ 为首项、$2$ 为公比的等比数列,再应用等比数列前 $n$ 项和公式求解即可。\\
详解:数列 $\{a_1,a_{n-1}\}$ 是公比为 $2$ 的等比数列,则有 $\dfrac{a_{n+1}\cdot a_{n+2}}{a_n\cdot a_{n+1}}=2$,所以 $\dfrac{a_{n+2}}{a_n}=2$,因此数列 $\{a_n\}$ 的奇数项是以 $1$ 为首项、$2$ 为公比的等比数列,所以 $a_1+a_3+a_5+\cdots+a_9=\dfrac{1-2^{10}}{1-2}=1023$,故选:D。}

\begin{question}
在四边形 $ABCD$ 中,$\overrightarrow{AB}=\lambda\overrightarrow{DC}$,$\overrightarrow{AB}=(1,-\sqrt2)$,$\overrightarrow{AD}=(4,2\sqrt2)$。若四边形 $ABCD$ 的面积为 $12\sqrt2$,则实数 $\lambda$ 的值为(\quad)
\begin{choices}
  \item $3$
  \item $\sqrt3$
  \item $\dfrac13$
  \item $\dfrac{\sqrt3}{3}$
\end{choices}

\topics{向量的线性运算的几何应用;坐标计算向量的模;向量垂直的坐标表示}
\difficulty{0.65}
\explain{分析:根据向量的模长及数量积为 $0$,可得四边形 $ABCD$ 为直角梯形,然后结合模长关系利用面积建立方程求解即可。\\
详解:因为 $AB=\lambda DC$,$|AB|=\sqrt{1+2}=\sqrt3$,$|AD|=\sqrt{16+8}=2\sqrt6$,所以四边形 $ABCD$ 为梯形;又 $AB\cdot AD=1\times4+(-\sqrt2)\times2\sqrt2=0$,所以 $AB\perp AD$,所以四边形 $ABCD$ 为直角梯形,则 $S_{\triangle ABD}=\dfrac12\sqrt3\times2\sqrt6=3\sqrt2$;又 $S_{ABCD}=12\sqrt2$,得 $S_{\triangle BCD}=9\sqrt2$,于是 $\dfrac12|CD|\cdot|AD|=9\sqrt2$,即 $\dfrac12\cdot\dfrac{\sqrt3}{\lambda}\cdot2\sqrt6=9\sqrt2$,故 $\dfrac1\lambda=\dfrac13$,选 C。}
\end{question}

\begin{question}
已知函数 $f(x)=\dfrac1x-\dfrac1{1-x}+x+k$ 的三个零点为 $a,b,c$,且 $a<b<c$,则下列结论不正确的是 \paren[D]
\begin{choices}
  \item $f(x)$ 在 $(0,1)$ 上单调递减
  \item 曲线 $y=f(x)$ 是中心对称图形
  \item $\forall k\in\mathbb{R}$,都有 $c>b+1$
  \item $\forall k\in\mathbb{R}$,都有 $c>a+3$
\end{choices}

\topics{判断或证明函数的对称性;函数单调性的应用;用导数判断或证明已知函数的单调性;利用导数研究函数的零点}
\difficulty{0.4}
\explain{分析:求导后结合定义域即可得 A;计算 $f(x)+f(-x+1)$ 为定值可得 B;结合函数单调性可得 $a,b,c$ 取值范围,再计算出 $f(b+1)$,结合 $f(x)$ 在 $(1,+\infty)$ 上单调性即可得 C;举出反例即可得 D。\\
详解:对 A:由 $f'(x)=-\dfrac1{x^2}-\dfrac1{(1-x)^2}-1$,当 $x\in(-\infty,0)\cup(0,1)\cup(1,+\infty)$ 时 $f'(x)<0$,故 $f(x)$ 在上述区间单调递减,A 正确;\\
对 B:$f(x)+f(-x+1)=\dfrac1x+\dfrac1{1-x}-x+k+\dfrac1{-x+1}+x-1+k=2k-1$,定义域为 $(-\infty,0)\cup(0,1)\cup(1,+\infty)$,故曲线 $y=f(x)$ 关于点 $\left(\dfrac12,k-\dfrac12\right)$ 对称,B 正确;\\
对 C:$a\in(-\infty,0)$,$b\in(0,1)$,$c\in(1,+\infty)$,且 $f$ 在 $(1,+\infty)$ 上单调递减,$b+1\in(1,2)$,由 $f(b)=0$ 得 $k=b+\dfrac1{1-b}-\dfrac1b$,进而 $f(b+1)=\dfrac2{1-b^2}-1>1$,故 $c>b+1$ 恒成立,C 正确;\\
对 D:取 $k=\dfrac12$,得 $f(-1)=0$、$f(2)=0$,此时 $a=-1, c=2$,$c=a+3$,故"$\forall k$ 都有 $c>a+3$"为假,D 错误。}
\end{question}

\section{多选题}

\begin{question}
若 $\dfrac1a<\dfrac1b<0$,则下列不等式正确的是(\quad)。
\begin{choices}
  \item $|a|>|b|$
  \item $\sqrt{-a}<\sqrt{-b}$
  \item $a+b>ab$
  \item $a^2-a<b^2-b$
\end{choices}
\topics{由已知条件判断所给不等式是否正确;由不等式的性质比较数(式)大小;作差法比较代数式的大小}
\difficulty{0.65}
\explain{答案:BD。\\
详解:由 $\dfrac1a<\dfrac1b<0$ 可知,$b<a<0$,所以 $-b>-a>0$,即 $|b|>|a|$,故 A 错误;$\sqrt{-b}>\sqrt{-a}$,故 B 正确;$a+b<0,ab>0$,所以 $a+b<ab$,故 C 错误;$a^2-a-b^2+b=(a+b)(a-b)-(a-b)=(a-b)(a+b-1)$,由以上可知,$a-b>0$,$a+b-1<0$,所以 $a^2-a-b^2+b<0$,即 $a^2-a<b^2-b$,故 D 正确。}
\end{question}

\begin{question}
在直角坐标系 $xOy$ 中,已知 $O$ 是以 $O$ 为圆心的单位圆,点 $A$ 的坐标为 $(1,0)$,角 $\theta$ 的始边为射线 $OA$,终边 $OB$ 交圆于点 $B$,过点 $B$ 作直线 $OA$ 的垂线,垂足为 $C$。若将点 $C$ 到直线 $OB$ 的距离表示为 $\theta$ 的函数 $h(\theta)$,则(\quad)。
\begin{choices}
  \item $h\!\left(\dfrac\pi{12}\right)=\dfrac14$
  \item $h(\theta)$ 的最小正周期为 $\dfrac\pi4$
  \item $\left[\dfrac{3\pi}4,\pi\right]$ 是 $h(\theta)$ 的一个单调减区间
  \item $h(\theta)+h\!\left(\dfrac\pi4+\theta\right)$ 的最大值为 $\dfrac{\sqrt2}{2}$
\end{choices}
\topics{求含 $\sin x(\cos x)$ 函数的值域和取值;求正弦(型)函数的最小正周期、二倍角的正弦公式;求点到直线的距离}
\difficulty{0.4}
\explain{答案:ACD。\\
详解:利用点到直线的距离公式及二倍角正弦公式得 $h(\theta)=\dfrac12\lvert\sin2\theta\rvert$,代入求得 $h\!\left(\dfrac\pi{12}\right)=\dfrac14$,正确;对 B,因为 $h\!\left(\theta+\dfrac\pi4\right)=\dfrac12\lvert\sin(2\theta+\dfrac\pi2)\rvert\ne h(\theta)$,所以 $\dfrac\pi4$ 不是 $h(\theta)$ 的周期,错误;对 C,$\theta\in\left[\dfrac{3\pi}4,\pi\right]$,所以 $2\theta\in\left[\dfrac{3\pi}2,2\pi\right]$,所以 $h(\theta)=-\dfrac12\sin2\theta$,因 $y=\sin x$ 在 $\left[\dfrac{3\pi}2,2\pi\right]$ 单调递增,所以 $h(\theta)=-\dfrac12\sin2\theta$ 在 $\theta\in\left[\dfrac{3\pi}4,\pi\right]$ 单调递减,正确;对 D,$h(\theta)+h\!\left(\dfrac\pi4+\theta\right)=\dfrac12\left(|\sin2\theta|+|\cos2\theta|\right)\le\dfrac{\sqrt2}{2}$,当且仅当 $|\sin2\theta|=|\cos2\theta|$ 时等号成立,正确。}
\end{question}

\section{填空题}

\begin{question}
已知向量 $\vec a=(m+2n,2)$,$\vec b=(1,2)$,其中 $mn>0$。若 $\vec a\parallel\vec b$,则 $mn$ 的最大值为\fillin[$\dfrac18$]。
\topics{由向量共线(平行)求参数;基本不等式求积的最大值}
\difficulty{0.65}
\explain{根据向量平行的坐标公式得到 $m,n$ 的关系,再利用基本不等式求解即可。\\
详解:$\vec a=(m+2n,2)$,$\vec b=(1,2)$,因为 $\vec a\parallel\vec b$,所以 $1\times2=2(m+2n)$,即 $m+2n=1$,$mn>0$。所以 $2mn\le\left(\dfrac{m+2n}{2}\right)^2=\dfrac18$,当且仅当 $m=\dfrac12,\ n=\dfrac14$ 时等号成立。故 $mn$ 的最大值为 $\dfrac18$。}
\end{question}

\begin{question}
在 2025 年江苏省“苏超”足球联赛的一场激烈比赛中,某城市队的 10 号球员从 $A$ 出发,以 $2.5$ 米/秒的速度做匀速直线运动,到达 $B$ 点时,发现足球在点 $C$ 处正以 $3$ 倍于自己的速度向点 $A$ 侧匀速直线运动。已知 $AB=4\sqrt3$ 米,$AC=20$ 米,$\angle BAC=30^\circ$。若忽略球员转身所需的时间,则该球员按原来的速度最快截住足球所用的时间为\fillin[$\dfrac85$]秒。
\topics{余弦定理;距离最短(时间最短)模型}
\difficulty{0.65}
\explain{
  \begin{center}
\begin{tikzpicture}[scale=1.05,>=Stealth,line cap=round,line join=round]
  \coordinate (A) at (0,0);
  \coordinate (B) at (5.1,0);
  \coordinate (C) at (2.6,4.4);
  \path (A) -- (C) coordinate[pos=0.35] (D);
  \draw (A)--(B)--(C)--cycle;
  \draw (B)--(D);
  \draw ($(A)+(0.7,0)$) arc[start angle=0,end angle=60,radius=0.7];
  \node at (0.95,0.35) {$30^\circ$};
  \node[below] at ($(A)!0.5!(B)$) {$4\sqrt{3}$};
  \fill (A) circle (1.1pt) node[below left] {$A$};
  \fill (B) circle (1.1pt) node[below right] {$B$};
  \fill (C) circle (1.1pt) node[above] {$C$};
  \fill (D) circle (1.1pt) node[left] {$D$};
\end{tikzpicture}
\end{center}
  设在 $D$ 处截住足球,时间设为 $t$ 秒,则 $BD=2.5t,\ CD=7.5t$,则 $AD=20-7.5t$;又 $AB=4\sqrt3,\ \angle BAC=30^\circ$,在 $\triangle ABD$ 中,利用余弦定理可知,$BD^2=AD^2+AB^2-2AD\cdot AB\cdot\sin\angle BAC$,则 $(2.5t)^2=(20-7.5t)^2+48-2\times4\sqrt3\times(20-7.5t)\times\dfrac{\sqrt3}{2}$,化简得 $25t^2-110t+104=0$,解得 $t=\dfrac85$ 或 $t=\dfrac{13}5$,所以该球员按原来的速度最快截住足球所用的时间为 $\dfrac85$ 秒。}
\end{question}

\begin{question}
设函数 $f(x)=\sin x-kx-b,\ x\in\left[0,\dfrac\pi2\right)$,其中 $k,\ b\in\mathbb{R}$。若 $k=\dfrac2\pi,\ b=0$,则 $f(x)$ 的最小值为\fillin[0];若 $f'(x)\ge0$ 恒成立,则 $-k^2+\pi k+b$ 的最大值为\fillin[$\dfrac{\pi^2}{16}+1$]。

\topics{由导数求函数的最值(不含参);一元二次不等式在实数集上恒成立问题;函数不等式恒成立问题}
\difficulty{0.4}
\explain{
  \begin{center}
\begin{tikzpicture}[scale=1.05,>=Stealth,line cap=round,line join=round]
  \draw[->] (-0.5,0) -- (5.2,0) node[right] {$x$};
  \draw[->] (0,-2.2) -- (0,3.2) node[above] {$y$};
  \draw[thick,domain=0:4.5,samples=200] plot (\x,{1.6*sin(deg(\x/1.6))});
  \node at (2.2,1.8) {$y=\sin x$};
  \draw[thick] (-0.2,-1.4) -- (5,3);
  \node at (3.9,2.5) {$y=kx+b$};
  \draw[dashed] ({pi/2},-2.2) -- ({pi/2},3.2);
  \node[below] at ({pi/2},0) {$\dfrac{\pi}{2}$};
  \fill (0,0) circle (1.1pt) node[below left] {$O$};
\end{tikzpicture}
\end{center}
若 $k=\dfrac2\pi,\ b=0$,求导,利用导数判断 $f(x)$ 的单调性和最值;若 $f'(x)\ge0$ 恒成立,即 $\sin x\ge kx+b$,结合图象可得 $b\le0$,$\pi k+2b-2\le0$;令 $-k^2+\pi k+b=t$,消去 $b$ 结合 $a$ 判别式运算求解即可。\\
详解:当 $k=\dfrac2\pi,\ b=0$,则 $f(x)=\sin x-\dfrac2\pi x,\ f'(x)=\cos x-\dfrac2\pi$,因为 $f'(x)$ 在 $\left(0,\dfrac\pi2\right)$ 内存在唯一零点 $x_0$,当 $0<x<x_0$ 时,$f'(x)>0$;当 $x_0<x<\dfrac\pi2$ 时,$f'(x)<0$;可知 $f(x)$ 在 $(0,x_0)$ 内单调递增,在 $\left(x_0,\dfrac\pi2\right)$ 内单调递减,且 $f(0)=f\!\left(\dfrac\pi2\right)=0$,所以 $f(x)$ 的最小值为 $0$;若 $f(x)=\sin x-kx-b\ge0$,即 $\sin x\ge kx+b$,可知当 $x\in\left[0,\dfrac\pi2\right]$ 时,$y=\sin x$ 在线段 $y=kx+b$ 的上方,结合图象可得 $\sin0\ge0\ge b$,即 $b\le0$;$\sin\dfrac\pi2=1\ge\dfrac\pi2k+b$,即 $\pi k+2b-2\le0$,可知 $k\in\mathbb{R}$,设 $-k^2+\pi k+b=t$,则 $b=k^2-\pi k+t$,可得 $\begin{cases}k^2-\pi k+t\le0\\2k^2-\pi k+2t-2\le0\end{cases}$,因为该不等式组在 $k\in\mathbb{R}$ 内有解,则 $\begin{aligned}\Delta_1&=\pi^2-4t\ge0\\ \Delta_2&=\pi^2-8(2t-2)\ge0\end{aligned}$,解得 $t\le\dfrac{\pi^2}{16}+1$,所以当 $k=\dfrac\pi4,\ b=1-\dfrac{\pi^2}{8}$ 时,$-k^2+\pi k+b$ 取得最大值 $\dfrac{\pi^2}{16}+1$。}
\end{question}

\section{解答题}

\begin{question}
在 $\triangle ABC$ 中,角 $A,B,C$ 所对的边为 $a,b,c$,且 $\dfrac{b}{\cos B}=\dfrac{a+b+c}{\cos A+\cos B+\cos C}$。
\begin{enumerate}[label=(\arabic*)]
  \item 求角 $B$ 的大小;
  \item 若 $\cos A=\dfrac{2\sqrt7}{7}$,求 $\tan(2A-B)$ 的值。
\end{enumerate}
\topics{用和、差的正弦公式化简、求值;用和、差的正切公式化简、求值;二倍角的正切公式;正弦定理边角互化的应用}
\difficulty{0.65}
\explain{(1) 根据正弦定理及两角差的正弦定理化简得 $\sin(B-A)=\sin(C-B)$,解得 $A+C=2B$,即可求得 $B=\dfrac\pi3$;\\
(2) 利用同角三角函数关系求得 $\tan A=\dfrac{\sqrt3}{2}$,利用二倍角正切公式得 $\tan2A=4\sqrt3$,然后利用两角和的正切公式求解即可。\\
答案:(1) $B=\dfrac\pi3$;(2) $\dfrac{3\sqrt3}{13}$。}
\end{question}

\begin{question}
已知向量 $\vec m=(\sin x,\sqrt3)$,$\vec n=\!\left(\cos x,\dfrac12\cos2x\right)$,设函数 $f(x)=\vec m\cdot\vec n$。
\begin{enumerate}[label=(\arabic*)]
  \item 若 $\left(\dfrac x2-\dfrac\pi{12}\right)=\dfrac35,\ \dfrac\pi3<x<\dfrac{5\pi}6$,求 $\cos x$ 的值;
  \item 将函数 $f(x)$ 的图像上所有的点向右平移 $\varphi\ (0<\varphi<\dfrac\pi2)$ 个单位长度,再把所得各点的纵坐标缩短到原来的 $\dfrac12$ 倍(纵坐标不变),得到函数 $y=g(x)$ 的图像。若函数 $y=g(x)$ 的图像关于直线 $x=\dfrac\pi{12}$ 对称,求 $g(x)$ 在 $\left[0,\dfrac\pi4\right]$ 上的最大值和最小值。
\end{enumerate}
\topics{求含 $\sin x(\cos x)$ 函数的值域和最值;做图象变换前(后)的解析式;给值求值型题;数量积的坐标表示}
\difficulty{0.65}
\explain{(1) 根据向量数量积运算及辅助角公式化简得到 $f(x)$,再结合 $\cos x=\cos\!\left[\left(x+\dfrac\pi6\right)-\dfrac\pi6\right]$ 求得 $\cos x$;\\
(2) 根据三角函数图象变换得到 $g(x)$ 解析式,结合 $x$ 范围求解 $g(x)$ 取值范围。\\
答案:(1) $\dfrac{3-4\sqrt3}{10}$;(2) 最大值为 $1$,最小值为 $\dfrac12$。}
\end{question}

\begin{question}
已知函数 $g(x)=\dfrac1x$。
\begin{enumerate}[label=(\arabic*)]
  \item 求过点 $A(-4,2)$ 且与曲线 $y=g(x)$ 相切的直线方程;
  \item 令 $f(x)=\dfrac{x^2-ax+1}{g(x)}$,若 $f(x)$ 有两个极值点 $x_1,\ x_2$,记过两点 $P(x_1,f(x_1)),Q(x_2,f(x_2))$ 的直线斜率为 $k$。是否存在实数 $a$,使得 $k=\dfrac53-a$?若存在,求 $a$ 的值;若不存在,试说明理由。
\end{enumerate}
\topics{求过一点的切线方程;根据极值点求参数}
\difficulty{0.65}
\explain{(1) 设切点 $\left(x_0,\dfrac1{x_0}\right)$,然后利用导数的几何意义及点斜式求出切线方程 $y=\dfrac1{x_0}-\dfrac1{x_0^2}(x-x_0)$,将点 $A(-4,2)$ 的坐标代入求得 $x_0=2$ 或 $x_0=-1$,即切线方程:$y=-\dfrac14x+1$ 或 $y=-x-2$;\\
(2) 求出导函数,由题意 $3x^2-2ax+1=0$ 有两个不等的实根 $x_1,\ x_2$,进而有 $x_1+x_2=\dfrac{2a}{3},\ x_1x_2=\dfrac13$,结合两点斜率公式及 $k=\dfrac53-a$ 列式得 $-\,\dfrac{2a^2}{9}+\dfrac23=\dfrac53-a$,解得 $a=3$。}
\end{question}

\begin{question}
设数列 $\{a_n\}$ 的前 $n$ 项和为 $S_n$,已知 $(3n-5)a_{n+1}-6S_n=pn+q,\ n\in\mathbb{N}^*$,其中 $p,\ q$ 为常数。
\begin{enumerate}[label=(\arabic*)]
  \item 当 $p=0$ 时,若 $S_1=-2,\ S_2=-1$,求数列 $\{a_n\}$ 的通项公式;
  \item 若 $a_1=-\dfrac q5$。
  \begin{enumerate}[label=(\roman*)]
    \item 证明:数列 $\{a_n\}$ 为等差数列;
    \item 若 $8,\ \sqrt3a_3,\ a_{12}$ 成等比数列,当 $n$ 为何值时,$\displaystyle\sum_{k=1}^n a_ka_{k+1}a_{k+2}$ 取得最大值,请说明理由。
  \end{enumerate}
\end{enumerate}
\topics{判断数列的增减性、由递推关系证明数列是等差数列;等比中项的应用;利用 $a_n$ 与 $S_n$ 关系求通项或项}
\difficulty{0.4}
\explain{(1) 代入 $p=0$,再升次作差得 $(3n-2)a_{n+2}=(3n+1)a_{n+1}$,从而有 $\dfrac{a_{n+1}}{a_n}=\dfrac{3n-2}{3n-5}$,最后用累乘法即可得到答案:$a_n=3n-5\ (n\in\mathbb{N}^*)$;\\
(2) (i) 两次升次作差得 $a_{n+3}+a_{n+1}=2a_{n+2}$,再验证前 $3$ 项即可;(ii) 根据等比中项和等差数列通项公式得 $d<0$,得其单调性,再求出 $\dfrac{76}{5}\le n\le\dfrac{81}{5}$,再设 $T_n=\sum_{k=1}^n a_ka_{k+1}a_{k+2}$,得到其单调性即可得到最大值。}
\end{question}

\begin{question}
设 $a>0,\ b>0$,函数 $f(x)=a\ln x-\dfrac12\,b(x-1)^2$,函数 $g(x)=\dfrac{f(x)}x$。
\begin{enumerate}[label=(\arabic*)]
  \item 当 $b=2$ 时:
    \begin{enumerate}[label=(\roman*)]
      \item 讨论函数 $g(x)=\dfrac{f(x)}x$ 的单调性;
      \item 若 $f(x)\le0$ 在 $(0,+\infty)$ 上恒成立,求 $a$ 的值。
    \end{enumerate}
  \item 当 $a\ge b>0$ 时,证明:函数 $f(x)$ 有两个极值点 $x_1,\ x_2\,(x_1<x_2)$ 且 $x_2<\left(\dfrac{2ea}{b}\right)^2x_1$。
\end{enumerate}
\topics{利用导数证明不等式;利用导数研究不等式恒成立问题;利用导数求函数(含参)的单调区间}
\difficulty{0.15}
\explain{(1) 代入后求导得 $g'(x)=-\dfrac{x^2-ax+1}{x^2}$,再求出导函数为 $0$ 时的根,最后分析其单调性即可;(ii) 分 $a=0,\ a>0$ 以及 $a<0$ 讨论即可。\\
(2) 二次求导再其单调性,再利用零点存在性定理证明极值点所在区间,再利用切线不等式 $\ln x<x-1$ 得 $e^{-2}<x_1<1<x_2<\left(\dfrac{2a}{b}\right)^2$,从而证明原不等式结论。}
\end{question}
