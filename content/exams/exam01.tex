
% --- Make \choice immune to leading [ ... ] while keeping exam-zh layout ---
\makeatletter
\@ifundefined{choice}{\newcommand{\choice}{\item\relax}}{\renewcommand{\choice}{\item\relax}}
\makeatother
% --- end patch ---


% --- Safe fallbacks (do NOT override exam-zh; only if missing) ---
\makeatletter
\expandafter\ifx\csname tl_if_blank:NF\endcsname\relax
  \expandafter\def\csname tl_if_blank:NF\endcsname#1#2{%
    \edef\gpt@tmp{\detokenize{#1}}%
    \ifx\gpt@tmp\@empty\relax\else #2\fi
  }%
\fi
\expandafter\ifx\csname bool_if:NTF\endcsname\relax
  \expandafter\def\csname bool_if:NTF\endcsname#1#2#3{#2}% assume true
\fi
\makeatother
% --- end fallbacks ---

\section*{Ⅰ. 单选题(示例转换)}

\begin{question}
已知集合$A=\left\{x|-5<x^{3}<5\right\},B=\left\{-3,-1,0,2,3\right\}$ ,则$A\cap B= ($ )
\begin{choices}
\choice $\{-1,0\}$
\choice $\{2,3\}$
\choice $\{-3,-1,0\}$
\choice $\{-1,0,2\}$
\end{choices}
\topics{交集的概念及运算;由幂函数的单调性解不等式}
\difficulty{0.94}
\explain{
化简集合$A$ ,由交集的概念即可得解

因为$A=\left\{x\mid-\sqrt[3]{5}<x<\sqrt[3]{5}\right\},B=\left\{-3,-1,0,2,3\right\}$ $B=\left\{-3,-1,0,2,3\right\}$ $B=\left\{-3,-1,0,2,3\right\}$ ,且注意到1<\sqrt[3]{5}<2, 从而$A\cap B=\left\{-1,0\right\}$ 故选:A.

本题答案:A。
}\end{question}

\begin{question}
若$\frac z{z-1}=1+$i ,则$z= ($ )
\begin{choices}
\choice 1-i
\choice 1+i
\choice -1-i
\choice $-1+i$
\end{choices}
\topics{复数的乘方;复数的除法运算}
\difficulty{0.94}
\explain{
由复数四则运算法则直接运算即可求解

因为$\frac{z}{z-1}=\frac{z-1+1}{z-1}=1+\frac{1}{z-1}=1+$i ,所以$z=1+\frac{1}{\mathrm{i}}=1-$i 故选:C.

本题答案:C。
}\end{question}

\begin{question}
已知向量$\vec{a}=(0,1),\vec{b}=(2,x)$ ,若$\vec{b}\perp(\vec{b}-4\vec{a})$ ,则x=()
\begin{choices}
\choice 1
\choice -2
\choice -1
\choice 2
\end{choices}
\topics{平面向量线性运算的坐标表示;向量垂直的坐标表示}
\difficulty{0.85}
\explain{
根据向量垂直的坐标运算可求$x$ 的值


因为$\vec{b}\perp\left(\vec{b}-4\vec{a}\right)$ ,所以$\vec{b}\cdot\left(\vec{b}-4\vec{a}\right)=0$ 所以$\vec{b}^{2}-4\vec{a}\cdot\vec{b}=0$ 即$4+x^{2}-4x=0$ ,故$x=2$ 故选:D.

本题答案:D。
}\end{question}

\begin{question}
已知$\cos(\alpha+\beta)=m$ tan $\alpha$ tan $\beta=2$ ,则$\cos(\alpha-\beta)=($ )
\begin{choices}
\choice 3m
\choice 3m
\choice $-\frac{m}{3}$
\choice $\frac m3$
\end{choices}
\topics{三角函数的化简;求值一同角三角函数基本关系;用和;差角的余弦公式化简求值}
\difficulty{0.85}
\explain{
根据两角和的余弦可求cosαcos $\beta$ ,sin $\alpha$ sin $\beta$ 的关系,结合tan $\alpha$ tan $\beta$ 的值可求前者, 故可求$\cos(\alpha-\beta)$ 的值

因为$\cos\left(\alpha+\beta\right)=m$ ,所以cos $.a$ cos $\beta$ sin α sin $\beta=m$ 而tan $\alpha$ tan $\beta=2$ ,所以$\sin\alpha\sin\beta=2\cos\alpha\cos\beta$ 故co α $\alpha$ β $\beta$ ${\mathrm{s}}\alpha\cos\beta-2\cos\alpha\cos\beta=m$ β=m $\beta=m$ 即$\cos\alpha\cos\beta=-m$ $\beta=-m$ $\beta=-m$ 从而$\sin\alpha\sin\beta=-2m$ ,故$\cos(\alpha-\beta)=-3m$ 故选:A.

本题答案:A。
}\end{question}

\begin{question}
已知圆柱和圆锥的底面半径相等,侧面积相等,且它们的高均为$\sqrt{3}$ ,则圆锥的体积为()
\begin{choices}
\choice 3√3π
\choice 6√3π
\choice 2√3π
\choice 9√3π
\end{choices}
\topics{圆柱表面积的有关计算;圆锥表面积的有关计算;锥体体积的有关计算}
\difficulty{0.85}
\explain{
设圆柱的底面半径为r,根据圆锥和圆柱的侧面积相等可得半径$r$ 的方程,求出解后可求圆锥的体积.

设圆柱的底面半径为$r$ ,则圆锥的母线长为$\sqrt{r^2+3}$ 而它们的侧面积相等,所以$2\pi r\times\sqrt{3}=\pi r\times\sqrt{3+r^{2}}$ 即$2\sqrt{3}=\sqrt{3+r^2}$


故$r=3$ ,故圆锥的体积为$\frac{1}{3}\pi\times9\times\sqrt{3}=3\sqrt{3}\pi$ 故选:B.

本题答案:B。
}\end{question}

\begin{question}
已知函数$f(x)=\begin{cases}-x^{2}-2ax-a,x<0\\\mathrm{e}^{x}+\ln(x+1),x\geq0\end{cases}$ 在$R$ 上单调递增,则$a$ 的取值范围是()
\begin{choices}
\choice (∞0,0]
\choice $[0,+\infty)$
\choice [-1,0]
\choice [-1,1]
\end{choices}
\topics{判断指数函数的单调性;研究对数函数的单调性;根据分段函数的单调性求参数}
\difficulty{0.65}
\explain{
根据二次函数的性质和分界点的大小关系即可得到不等式组,解出即可

因为$f(x)$ 在R上单调递增,且$x\geq0$ 时, $f(x)=\mathrm{e}^{x}+\ln\left(x+1\right)$ 单调递增, 周质润足$\begin{cases}-\frac{-2a}{2\times\left(-1\right)}\geq0\end{cases}$ $\left\lfloor-a\leq\mathrm{e}^{0}+\ln1\right\rfloor$ 解拍$-1\leq a\leq0$ 即$a$ 的范围是[-1,0] 故选:B.

本题答案:B。
}\end{question}

\begin{question}
当$x\in[0,2\pi]$ 时,曲线$y=\sin x$ 与y=2sin3x-的交点个数为(
\begin{choices}
\choice 4
\choice 6
\choice 3
\choice 8
\end{choices}
\topics{正弦函数图象的应用;求函数零点或方程根的个数}
\difficulty{0.65}
\explain{
画出两函数在$\begin{bmatrix}0,2\pi\end{bmatrix}$ 上的图象,根据图象即可求解

因为函数$y=\sin x$ 的最小正周期为$T=2\pi$ 函数$y=2\sin\left(3x-\frac{\pi}{6}\right)$ 的最小正周期为$T=\frac{2\pi}{3}$ 所以在$x\in\left[0,2\pi\right]$ 上函数$y=2\sin\left(3x-\frac{\pi}{6}\right)$ 有三个周期的图象在坐标系中结合五点法画出两函数图象,如图所示:



由图可知,两函数图象有6个交点故选:C

本题答案:C。
}\end{question}

\begin{question}
已知函数$f(x)$ 的定义域为$R$ , $f(x)>f(x-1)+f(x-2)$ ,且当$x<3$ 时$f(x)=x$ ,则下列结论中一定正确的是()
\begin{choices}
\choice $f(20)>1000$
\choice $f(10)>100$
\choice $f(10)<1000$
\choice $f(20)<10000$
\end{choices}
\topics{求函数值;比较函数值的大小关系}
\difficulty{0.4}
\explain{
代入得到$f(1)=1,f(2)=2$ ,再利用函数性质和不等式的性质,逐渐递推即可判断

因为当$x<3$ 时$f(x)=x$ ,所以$f(1)=1,f(2)=2$ 又因为$f(x)>f(x-1)+f(x-2)$ , 则$f(3)>f(2)+f(1)=3,f(4)>f(3)+f(2)>5$ ,
$$f(5)>f(4)+f(3)>8,f(6)>f(5)+f(4)>13,f(7)>f(6)+f(5)>21\:,$$

$$f(8)>f(7)+f(6)>34,f(9)>f(8)+f(7)>55,f(10)>f(9)+f(8)>89\:,$$

$$\begin{aligned}
&f\left(0\right)>f\left(1\right)+f\left(0\right)>24,f\left(7\right)>f\left(0\right)<f\left(1\right)>22,f\left(10\right)>f\left(7\right)+f\left(0\right)>07, \\
&f(11)>f(10)+f(9)>144,f(12)>f(11)+f(10)>233,f(13)>f(12)+f(11)>377 \\
&f(14)>f(13)+f(12)>610,f(15)>f(14)+f(13)>987\:,
\end{aligned}$$
$f(16)>f(15)+f(14)>1597>1000$ ,则依次下去可知$f(20)>1000$ ,则B正确; 且无证据表明ACD一定正确故选:B.

本题答案:B。
}\end{question}

\begin{question}
随着“一带一路"国际合作的深入,某茶叶种植区多措并举推动茶叶出口.为了解推动出口后的亩收入(单位:万元)情况,从该种植区抽取样本,得到推动出口后亩收入的样本均值$\overline{x}=2.1$ ,样本方差$s^{2}=0.0$l ,已知该种植区以往的亩收入$X$ 服从正态分布$N\bigl(1.8,0.1^{2}\bigr)$ ,假设推动出口后的亩收入$Y$ 服从正态分布$N\left(\overline{x},s^{2}\right)$ ,则()(若随机变量$Z$ 服从正态分布
$$N\Big(\mu,\sigma^{2}\Big),\:P(Z<\mu+\sigma)\approx0.8413\:)$$
\begin{choices}
\choice $P(X>2)>0.2$
\choice $P(X>2)<0.5$
\choice $P(Y>2)>0.5$
\choice PY>20.8
\end{choices}
\topics{指定区间的概率;正态分布的实际应用}
\difficulty{0.85}
\explain{
根据正态分布的$3\sigma$ 原则以及正态分布的对称性即可解出,

依题可知, $\overline{x}=2.1,s^{2}=0.01$ ,所以$Y\sim N\Big(2.1,0.1^{2}\Big)$ 故$P\left(Y>2\right)=P\left(Y>2.1-0.1\right)=P\left(Y<2.1+0.1\right)\approx0.8413>0.5$ ,C正确,D错误; 因为$X\sim N\left(1.8,0.1^{2}\right)$ ,所以$P\big(X>2\big)=P\big(X>1.8+2\times0.1\big)$ 因为$P\big(X<1.8+0.1\big)\approx0.8413$ ,所以$P\big(X>1.8+0.1\big)\approx1-0.8413=0.1587<0.2$ 而$P\left(X>2\right)=P\left(X>1.8+2\times0.1\right)<P\left(X>1.8+0.1\right)<0.2$ ,B正确,A错误, 故选:BC.

本题答案:BC。
}\end{question}

\begin{question}
设函数$f(x)=(x-1)^{2}(x-4)$ ,则()
\begin{choices}
\choice 当$0<x<1$ 时, $f(x)<f\left(x^{2}\right)$
\choice $x=3$ 是$f(x)$ 的极小值点C.当$1<x<2$ 时, $-4<f(2x-1)<0$
\choice 当$-1<x<0$ 时, $f(2-x)>f(x)$
\end{choices}
\topics{利用导数求函数的单调区间(不含参);求已知函数的极值点}
\difficulty{0.65}
\explain{
求出函数$f(x)$ 的导数,得到极值点,即可判断A:利用函数的单调性可判断B; 根据函数$f(x)$ 在(1,3)上的值域即可判断C:直接作差可判断D


对A,因为函数$f(x)$ 的定义域为$R$ ,而

$$f'\left(x\right)=2\left(x-1\right)\left(x-4\right)+\left(x-1\right)^{2}=3\left(x-1\right)\left(x-3\right),$$

易知当$x\in(1,3)$ 时, $f'(x)<0$ ,当$x\in\left(-\infty,1\right)$ 或$x\in\left(3,+\infty\right)$ 时, $f'(x)>0$

函数$f(x)$ 在$\begin{pmatrix}-\infty,1\end{pmatrix}$ 上单调递增,在(1,3)上单调递减,在$\begin{pmatrix}3,+\infty\end{pmatrix}$ 上单调递增,故$x=3$ 是函数$f(x)$ 的极小值点,正确; 对B,当$0<x<1$ 时, $x-x^{2}=x\left(1-x\right)>0$ ,所以$1>x>x^{2}>0$ , 而由上可知,函数$f(x)$ 在(0,1)上单调递增,所以$f(x)>f\left(x^{2}\right)$ ,错误; 对C,当$1<x<2$ 时, $1<2x-1<3$ ,而由上可知,函数$f(x)$ 在(1,3)上单调递减所以$f(1)>f\left(2x-1\right)>f\left(3\right)$ ,即$-4<f\left(2x-1\right)<0$ ,正确;
$$f(2-x)-f(x)=\left(1-x\right)^2\left(-2-x\right)-\left(x-1\right)^2\left(x-4\right)=\left(x-1\right)^2\left(2-2x\right)>0\:,$$
所以$f(2-x)>f(x)$ ,正确: 故选:ACD.

本题答案:ACD。
}\end{question}

\begin{question}
设计一条美丽的丝带,其造型可以看作图中的曲线 $C$ 的一部分。已知 $C$ 过坐标原点且 $C$ 上的点满足:横坐标大于 $-2$,到点 $F(2,0)$ 的距离与到定直线 $x=a\ (a<0)$ 的距离之积为 $4$,则( )
\begin{center}
\IfFileExists{content/assets/q11.png}{\includegraphics[width=.62\linewidth]{content/assets/q11.png}}{\fbox{(图略)}}
\end{center}
\begin{choices}
\choice 点 $(2\sqrt{2},0)$ 在 $C$ 上
\choice $a=-2$
\choice $C$ 在第一象限的点的纵坐标的最大值为 $1$
\choice 当点 $(x_0,y_0)$ 在 $C$ 上时,$y\le 4$
\end{choices}
\topics{由方程研究曲线的性质;求平面轨迹方程}
\difficulty{0.65}
\explain{
根据题设将原点代入曲线方程可求 $a$,故可判断 A 的正误;结合曲线方程可判断 B 的正误;利用特例法可判断 C 的正误;将曲线方程化简后结合不等式的性质可判断 D 的正误。
\[ {\sqrt{(x-2)^2+y^2}}\,(x+2)=4. \]
当 $x=2\sqrt{2},y=0$ 时等式成立,故 A 正确;由 $x=0$ 得 $-a\le 1$,故 $a\le 0$,结合条件得 $a=-2$,故 B 正确;由上式解得 $y\le 4/(x+2)$,故 D 正确;综合可知本题答案:ABD。
}\end{question}

\begin{question}
设双曲线$C:\frac{x^{2}}{a^{2}}-\frac{y^{2}}{b^{2}}=1(a>0,b>0)$ 的左右焦点分别为$F_{1}$, $F_2$ ,过$F_{2}$ 作平行于$y$ 轴的直线交$C$ 于$A$ , $B$ 两点,若$\mid F_{1}A\mid=13,\mid AB\mid=10$ AB |=10 $AB\mid=10$ ,则$C$ 的离心率为
\topics{求双曲线的离心率或离心率的取值范围}
\difficulty{0.65}
\explain{
由题意画出双曲线大致图象,求出$\left|AF_2\right|$ ,结合双曲线第一定义求出$\left|AF_{1}\right|$ ,即可得到$a,b,c$ 的值,从而求出离心率

由题可知$A,B,F_{2}$ 三点横坐标相等,设$A$ 在第一象限,将$x=c$ 代入得$y=\pm\frac{b^2}a$ ,即$A\Bigg(c,\frac{b^2}a\Bigg),B\Bigg(c,-\frac{b^2}a\Bigg)$ ,故$\left|AB\right|=\frac{2b^2}a=10$ |Ar|=5又$\left|AF_{1}\right|-\left|AF_{2}\right|=2a$ ,得$\left|AF_{1}\right|=\left|AF_{2}\right|+2a=2a+5=13$ ,解得$a=4$ ,代入$\frac{b^2}a=5$ 得$b^2=20$ 故$c^{2}=a^{2}+b^{2}=36$ ,即$c=6$ ,所以$e=\frac{c}{a}=\frac{6}{4}=\frac{3}{2}$ 故答案为: $\frac{3}{2}$

}
\end{question}

\begin{question}
若曲线$y=\mathrm{e}^{x}+x$ 在点(0,1)处的切线也是曲线$y=\ln(x+1)+a$ 的切线,则$a=$
\topics{已知切线(斜率)求参数;两条切线平行;垂直;重合(公切线)问题}
\difficulty{0.65}
\explain{
先求出曲线$y=\mathrm{e}^{x}+x$ 在(0,1)的切线方程,再设曲线$y=\ln\left(x+1\right)+a$ 的切点为$\left(x_0,\ln\left(x_0+1\right)+a\right)$ ,求出$y^{\prime}$ ,利用公切线斜率相等求出$x_0$ ,表示出切线方程,结合两切线方程相同即可求解

由$y=\mathrm{e}^{x}+x$ 得$y^{\prime}=\mathrm{e}^{x}+1$ , $y^{\prime}\left|_{x=0}=\mathrm{e}^{0}+1=2\right.$ 故曲线$y=\mathrm{e}^{x}+x$ 在(0.1)处的切线方程为$y=2x+1$ 由$y=\ln\left(x+1\right)+a$ 得$y^{\prime}=\frac{1}{x+1}$ 设切线与曲线$y=\ln\left(x+1\right)+a$ 相切的切点为$\left(x_{0},\ln\left(x_{0}+1\right)+a\right)$ 由两曲线有公切线得$y'=\frac{1}{x_{0}+1}=2$ ,解得$x_{0}=-\frac{1}{2}$ ,则切点为$\cdot\left(-\frac{1}{2},a+\ln\frac{1}{2}\right)$


切线方程为$y= 2\left($ $x+ \frac 12\right) + a+ \ln \frac 12= 2x+ 1+ a- \ln 2$ . 根据两切线重合,所以$a-\ln2=0$ ,解得$a=\ln2$ 故答案为:ln2
}
\end{question}

\begin{question}
甲、乙两人各有四张卡片,每张卡片上标有一个数字,甲的卡片上分别标有数字1,35,7,乙的卡片上分别标有数字2,4,6,8,两人进行四轮比赛,在每轮比赛中,两人各自从自己持有的卡片中随机选一张,并比较所选卡片上数字的大小,数字大的人得1分,数字小的人得0分,然后各自弃置此轮所选的卡片(弃置的卡片在此后的轮次中不能使用) 则四轮比赛后,甲的总得分不小于2的概率为
\topics{计算古典概型问题的概率;求离散型随机变量的均值;均值的性质}
\difficulty{0.4}
\explain{
将每局的得分分别作为随机变量,然后分析其和随机变量即可,

设甲在四轮游戏中的得分分别为$X_{1},X_{2},X_{3},X_{4}$ ,四轮的总得分为$X$ 对于任意一轮,甲乙两人在该轮出示每张牌的概率都均等,其中使得甲得分的出牌组合有六种,从而甲在该轮得分的概率$P\big(X_{k}=1\big)=\frac{6}{4\times4}=\frac{3}{8}$ ,所以$E(X_{k})=\frac{3}{8}\Big(k=1,2,3,4\Big)$ 从而$E\left(X\right)=E\left(X_{1}+X_{2}+X_{3}+X_{4}\right)=\sum_{k=1}^{4}E\left(X_{k}\right)=\sum_{k=1}^{4}\frac{3}{8}=\frac{3}{2}.$ 记$p_{k}=P\Big(X=k\Big)\Big(k=0,1,2,3\Big)$ 如果甲得0分,则组合方式是唯一的:必定是甲出1,3,5,7分别对应乙出2,4,6,8所以$p_{0}=\frac{1}{\mathrm{A}_{4}^{4}}=\frac{1}{24}$ 如果甲得3分,则组合方式也是唯一的:必定是甲出1,3,5,7分别对应乙出8,2,46,所以$p_3=\frac1{\mathrm{A}_4^4}=\frac1{24}$ 而$X$ 的所有可能取值是0 ,1,2,3,故$p_{0}+p_{1}+p_{2}+p_{3}=1$ p+2p+3p=EX= 所以$p_1+p_2+\frac1{12}=1$ , $p_1+2p_2+\frac18=\frac32$ ,两式相减即得$p_{2}+\frac{1}{24}=\frac{1}{2}$ ,故$p_2+p_3=\frac12$ 所以甲的总得分不小于 2的概率为$p_2+p_3=\frac12$ 故答案为: $\frac12$
}
\end{question}

\begin{question}
记$\triangle ABC$ 的内角$A$ 、B、 $C$ 的对边分别为$a$ ,b,c,已知$\sin C=\sqrt{2}\cos B$
$$a^2+b^2-c^2=\sqrt{2}ab$$

(1)求$B$

(2)若$\triangle ABC$ 的面积为$3+\sqrt{3}$ ,求$c$
\topics{已知两角的正;余弦,求和;差角的正弦;正弦定理解三角形;三角形面积公式及其应用;余弦定理解三角形}
\difficulty{0.65}
\explain{
(1)由余弦定理、平方关系依次求出cos $C,\sin C$ ,最后结合已知$\sin C=\sqrt{2}\cos B$ 得cos $B$ 的值即可:

(2)首先求出$A,B,C$ ,然后由正弦定理可将$a,b$ 均用含有$c$ 的式子表示,结合三角形面积公式即可列方程求解

(1)由余弦定理有$a^{2}+b^{2}-c^{2}=2ab\cos C$ ,对比已知$a^{2}+b^{2}-c^{2}=\sqrt{2}ab$ 可得$\cos C=\frac{a^{2}+b^{2}-c^{2}}{2ab}=\frac{\sqrt{2}ab}{2ab}=\frac{\sqrt{2}}{2}$ 因为$C\in(0,\pi)$ ,所以sin $C>0$
$${\vec{\jmath}}\sin C={\sqrt{1-\cos^{2}C}}={\sqrt{1-\left({\frac{\sqrt{2}}{2}}\right)^{2}}}={\frac{\sqrt{2}}{2}}\:,$$
又因为sin $C=\sqrt{2}\cos B$ ,即$\cos B=\frac{1}{2}$ 注意到$B\in(0,\pi)$ 所以$B=\frac{\pi}{3}$

(2)由(1)可得$B=\frac{\pi}{3}$ $B=\frac{\pi}{3}$ $B= \frac \pi 3$, $\cos C= \frac {\sqrt {2}}2$ C= $.C=\frac{\sqrt{2}}{2}$ , $C\in(0,\pi)$ ,从而$C=\frac{\pi}{4}$ $C=\frac{\pi}{4}$ $C= \frac \pi 4$ , $A= \pi - \frac \pi 3- \frac \pi 4= \frac {5\pi }{12}$ Tisin $A= \sin \left( \frac {5\pi }{12}\right) = \sin \left( \frac \pi 4+ \frac \pi 6\right) = \frac {\sqrt {2}}2\times \frac {\sqrt {3}}2+ \frac {\sqrt {2}}2\times \frac 12= \frac {\sqrt {6}+ \sqrt {2}}4$ ,


由正张定理有$\frac{a}{\sin\frac{5\pi}{12}}=\frac{b}{\sin\frac{\pi}{3}}=\frac{c}{\sin\frac{\pi}{4}}$
$${\bar{1}}\:a={\frac{\sqrt{6}+\sqrt{2}}{4}}\cdot{\sqrt{2}}c={\frac{\sqrt{3}+1}{2}}c,b={\frac{\sqrt{3}}{2}}\cdot{\sqrt{2}}c={\frac{\sqrt{6}}{2}}c\:,$$
由三角形面积公式可知, $\triangle ABC$ 的面积可表示为
$$S_{_{s.ABC}}=\frac{1}{2}\:ab\sin C=\frac{1}{2}\cdot\frac{\sqrt{3}+1}{2}c\cdot\frac{\sqrt{6}}{2}c\cdot\frac{\sqrt{2}}{2}=\frac{3+\sqrt{3}}{8}c^{2}\:,$$
由已知$\triangle ABC$ 的面积为$3+\sqrt{3}$ ,可得$\frac{3+\sqrt{3}}{8}c^2=3+\sqrt{3}$ 所以$c=2\sqrt{2}$ 16.已知$A(0,3)$ 和P3,为椭圆$C:\frac{x^{2}}{a^{2}}+\frac{y^{2}}{b^{2}}=1(a>b>0)$ 上两点(1)求$C$ 的离心率: (2)若过$P$ 的直线1交$C$ 于另一点$B$ ,且$\Delta ABP$ 的面积为9,求7的方程,

本题答案:$(1)B=\frac{\pi}{3}$。
}\end{question}

\begin{question}
如图,四棱锥$P-ABCD$ 中, $PA\bot$ 底面ABCD, $PA=AC=2$ , $BC=1,AB=\sqrt{3}$


(1)若$AD\perp PB$ ,证明:AD//平面$PBC$ : (2)若$AD\perp DC$ ,且二面角$A-CP-D$ 的正弦值为$\frac{\sqrt{42}}7$ ,求$AD$
\topics{证明线面平行;证明面面垂直;由二面角大小求线段长度或距离}
\difficulty{0.65}
\explain{
(1)先证出$AD\bot$ 平面$PAB$ ,即可得$AD\perp AB$ ,由勾股定理逆定理可得$BC\perp AB$ 从而$AD//BC$ ,再根据线面平行的判定定理即可证出:

(2)过点$D$ 作$DE\perp AC$ 于$E$ ,再过点$E$ 作$EF\perp CP$ 于$F$ ,连接$DF$ ,根据三垂线法可知$\angle DFE$ 即为二面角$A-CP-D$ 的平面角,即可求得tan $\angle DFE=\sqrt{6}$ ,再分别用$AD$ 的长度表示出$DE,EF$ ,即可解方程求出$AD$

(1)因为$PA\bot$ 平面$ABCD$ ,而$AD\subset$ 平面$ABCD$ ,所以$PA\perp AD$ , 又$AD\perp PB$ , $PB\cap PA=P$ , $PB,PA\subset$ 平面$PAB$ ,所以$AD\bot$ 平面$PAB$ 而$AB\subset$ 平面$PAB$ ,所以$AD\perp AB$ 因为$BC^{2}+AB^{2}=AC^{2}$ ,所以$BC\perp AB$ ,根据平面知识可知$AD//BC$ ,

又$AD\not\subset$ 平面$PBC$ , $BC\subset$ 平面$PBC$ ,所以$AD//$ 平面$PBC$


(2)如图所示,过点$D$ 作$DE\perp AC$ 于$E$ ,再过点$E$ 作$EF\perp CP$ 于$F$ ,连接$DF$ 因为$PA\bot$ 平面$ABCD$ ,所以平面$PAC\bot$ 平面$ABCD$ ,而平面PAC门平面$ABCD=AC$ 所以$DE\bot$ 平面$PAC$ ,又$EF\bot CP$ ,所以$CP\bot$ 平面$DEF$ 根据二面角的定义可知, $\angle DFE$ 即为二面角$A-CP-D$ 的平面角, 即sin $\angle DFE=\frac{\sqrt{42}}{7}$ ,即tan $\angle DFE=\sqrt{6}$ 因为$AD\perp DC$ ,设$AD$ = $x$ ,则$CD=\sqrt{4-x^{2}}$ ,由等面积法可得,DE=x4-x² $.CE=\sqrt{\left(4-x^{2}\right)-\frac{x^{2}\left(4-x^{2}\right)}{4}}=\frac{4-x^{2}}{2}$ ,而$_{\Delta}EFC$ 为等腰直角三角形,所以$EF=\frac{4-x^2}{2\sqrt{2}}$ 成m $\angle DFE=\frac{\frac{x\sqrt{4-x^2}}{2}}{\frac{4-x^2}{2\sqrt{2}}}=\sqrt{6}$ 好$x=\sqrt{3}$ 肌$AD=\sqrt{3}$


本题答案:(1)证明见解析(2)√3。
}\end{question}

\begin{question}
已知函数$f(x)=\ln\frac{x}{2-x}+ax+b(x-1)^{3}$ (1)若$b=0$ ,且$f^{\prime}(x)\geq0$ ,求$a$ 的最小值: (2)证明:曲线$y=f(x)$ 是中心对称图形: (3)若$f(x)>-2$ 当且仅当$1<x<2$ ,求$b$ 的取值范围,
\topics{判断或证明函数的对称性;简单复合函数的导数;利用导数证明不等式;利用导数研究不等式恒成立问题}
\difficulty{0.4}
\explain{
(1)求出f(x)_=2+a后根据$f^{\prime}(x)\geq0$ 可求$a$ 的最小值:


(2)设$P(m,n)$ 为$y=f(x)$ 图象上任意一点,可证$P(m,n)$ 关于$(1,a)$ 的对称点为$Q(2-m,2a-n)$ 也在函数的图像上,从而可证对称性;

(3)根据题设可判断$f(1)=-2$ 即$a=-2$ ,再根据$f(x)>-2$ 在(1,2)上恒成立可求得$b\geq-\frac{2}{3}$

(1) $b=0$ 时, $f(x)=\ln\frac{x}{2-x}+ax$ ,其中$x\in(0,2)$ 则$f^{\prime}\left(x\right)=\frac{1}{x}+\frac{1}{2-x}+a=\frac{2}{x\left(2-x\right)}+a,x\in\left(0,2\right)$, 因为$x\left(2-x\right)\leq\left(\frac{2-x+x}{2}\right)^{2}=1$ ,当且仅当$x=1$ 时等号成立, 故$f^{\prime}(x)_{\mathrm{min}}=2+a$ ,而$f'(x)\geq0$ 成立,故$a+2\geq0$ 即$a\geq-2$ 所以$a$ 的最小值为-2 ·, (2) $f\left(x\right)=\ln\frac{x}{2-x}+ax+b\left(x-1\right)^{3}$ 的定义域为(0.2) 设$P(m,n)$ 为$y=f(x)$ 图象上任意一点, $P(m,n)$ 关于$(1,a)$ 的对称点为$Q(2-m,2a-n)$ 因为$P(m,n)$ 在$y=f(x)$ 图象上,故$n=\ln\frac{m}{2-m}+am+b\left(m-1\right)^{3}$
$${\vec{\jmath}}\:f\left(2-m\right)=\ln{\frac{2-m}{m}}+a\left(2-m\right)+b\left(2-m-1\right)^{3}=-\left[\ln{\frac{m}{2-m}}+am+b\left(m-1\right)^{3}\right]+2a\:,$$
$=-n+2a$ 所以$Q(2-m,2a-n)$ 也在$y=f(x)$ 图象上由$P$ 的任意性可得$y=f(x)$ 图象为中心对称图形,且对称中心为$(1,a)$ (3)因为$f(x)>-2$ 当且仅当$1<x<2$ ,故$x=1$ 为$f(x)=-2$ 的一个解,

所以$f(1)=-2$ 即$a=-2$ 先考虑$1<x<2$ 时, $f(x)>-2$ 恒成立此时$f(x)>-2$ 即为$\ln\frac{x}{2-x}+2\left(1-x\right)+b\left(x-1\right)^{3}>0$ 在(1,2)上恒成立, 设$t=x-1\in(0,1)$ ,则$\ln\frac{t+1}{1-t}-2t+bt^{3}>0$ 在(0.1)上恒成立设$g\left(t\right)=\ln\frac{t+1}{1-t}-2t+bt^{3},t\in\left(0,1\right)$,


$$\parallel g^{\prime}\left(t\right)={\frac{2}{1-t^{2}}}-2+3bt^{2}={\frac{t^{2}\left(-3bt^{2}+2+3b\right)}{1-t^{2}}}\:,$$
当$b\geq0$ , $-3bt^2+2+3b\geq-3b+2+3b=2>0$ 故$g'(t)>0$ 恒成立,故$g(t)$ 在(0,1)上为增函数故$g(t)>g(0)=0$ 即$f(x)>-2$ 在(1,2)上恒成立当$-\frac23\leq b<0$ 时, $-3bt^2+2+3b\geq2+3b\geq0$ 故$g'(t)\geq0$ 恒成立,故$g(t)$ 在(0.1)上为增函数故$g(t)>g(0)=0$ 即$f(x)>-2$ 在(1,2)上恒成立当$b<-\frac{2}{3}$ ,则当$0<t<\sqrt{1+\frac{2}{3b}}<1$ 时, $g'(t)<0$ 效在$:\left(0,\sqrt{1+\frac{2}{3b}}\right)$上$g\left(t\right)$ 为减函数,故$g(t)<g(0)=0$ ,不合题意,含综上, $f(x)>-2$ 在(1,2)上恒成立时$b\geq-\frac{2}{3}$ 而当$b\geq-\frac{2}{3}$ 时, 而$b\geq-\frac{2}{3}$ 时,由上述过程可得$g(t)$ 在(0.1)逆增,故$g(t)>0$ 的解为(0.1) 即$f(x)>-2$ 的解为(1,2) 综上, $b\geq-\frac{2}{3}$

本题答案:(1)-2 (2)证明见解析。
}\end{question}

\begin{question}
设$m$ 为正整数,数列$a_{1},a_{2},...,a_{4m+2}$ 是公差不为0的等差数列,若从中删去两项$a_{i}$ 和$a_{j}(i<j)$ 后剩余的$4m$ 项可被平均分为$m$ 组,且每组的4个数都能构成等差数列,则称数列$a_1,a_2,...,a_{4m+2}$ 是$(i,j)-$ 可分数列,

(1)写出所有的$(i,j)$ , $1\leq i<j\leq6$ ,使数列$a_{1},a_{2},...,a_{6}$ 是$(i,j)-$ 可分数列:

(2)当$m\geq3$ 时,证明:数列$a_{1},a_{2},...,a_{4m+2}$ 是(2,13)-可分数列


(3)从$1,2,...,4m+2$ 中任取两个数$i$ 和$j(i<j)$ ,记数列$a_{1},a_{2},...,a_{4m+2}$ 是$(i,j)-$ 可分数列的概率为$P_{m}$ ,证明: $P_{m}>\frac{1}{8}$
\topics{等差数列通项公式的基本量计算;数列新定义}
\difficulty{0.15}
\explain{
\textbf{(1)} 由定义,删去两项后剩余 $4$ 项须仍能构成一个等差数列。
对 $a_1,a_2,\ldots,a_6$ 而言,这等价于剩余四项的下标是四个连续整数。
逐一检查可得仅有 $(i,j)=(1,2),(1,6),(5,6)$ 三种情形满足题意。

\textbf{(2)} 当 $m\ge 3$ 时,删去第 $2$、$13$ 项:
将余下的 $4m$ 项按下标模 $4$ 的同余类分组并作适当“错位配对”,即可把它们分成 $m$ 组、每组 $4$ 项且成等差(公差与原数列相同)。
这给出 $(2,13)$ 为一组可行的 $(i,j)$。

\textbf{(3)} 设
\(
A=\{\,4k+1\mid k=0,1,\ldots,m\,\},\quad
B=\{\,4k+2\mid k=0,1,\ldots,m\,\}.
\)
若 $i\in A,\,j\in B$ 或 $i\in B,\,j\in A$ 且 $j-i\ne 3$,
则可按模 $4$ 的结构把余下各项划分为 $m$ 组,每组 $4$ 项且成等差;
因此这是一条\emph{充分条件}。

可行对数至少为
\((m+1)^2 - m\):其中 $(m+1)^2$ 来自 $A\times B$ 的全部配对,减去 $m$ 个满足 $j-i=3$ 的配对。
总配对数
\(\binom{4m+2}{2}=\dfrac{(4m+2)(4m+1)}2=8m^2+6m+1\)。
于是
\[
P_m \ge \frac{(m+1)^2-m}{\binom{4m+2}{2}}
= \frac{m^2+m+1}{8m^2+6m+1}
> \frac18,
\]
因为 \(8(m^2+m+1)-(8m^2+6m+1)=2m+7>0\) 恒成立。
}
\end{question}
