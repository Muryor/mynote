\examxtitle{nanjing_yancheng_2025_mock1_preprocessed}

\section{单选题}

\begin{question}
设集合\(A = \left\{ x \mid x^{2} - 4 \leq 0 \right\},B = \left\{ x \mid x + a \leq 0 \right\}\).若\(A \subseteq B\),则实数\(a\)的取值范围是(    )
\begin{choices}
  \(( - \infty,2)\)

\begin{enumerate}[label=(\arabic*)]
  \item \(( - \infty,2\rbrack\)
  \item \(( - \infty, - 2)\)
  \item \(( - \infty, - 2\rbrack\)
\end{choices}
\topics{根据集合的包含关系求参数;解不含参数的一元二次不等式}
\difficulty{0.85}
\answer{D}
\explain{由\(x^{2} - 4 \leq 0\)可得\(A = \lbrack - 2,2\rbrack\),由\(x + a \leq 0\)可得\(B = ( - \infty, - a\rbrack\),
又\(A \subseteq B\),所以\(2 \leq - a\),即\(a \leq - 2\),故D正确}
\end{enumerate}
\end{question}

\begin{question}
已知复数\(z\)满足\(\frac{1}{z + \text{i}} = \text{i}\)(\(\text{i}\)为虚数单位),则\(|z| =\)(    )
\begin{choices}
  4

\begin{enumerate}[label=(\arabic*)]
  \item 2
  \item 1
  \item \(\frac{1}{2}\)
\end{choices}
\topics{求复数的模;复数的除法运算}
\difficulty{0.85}
\answer{B}
\explain{因为\(\frac{1}{z + \text{i}} = \text{i}\),所以\(z + \text{i} = \frac{1}{\text{i}} = \frac{\text{i}}{\text{i}^{2}} = - \text{i}\),所以\(z = - 2\text{i}\),
%
所以\(|z| = 2\)}
\end{enumerate}
\end{question}

\begin{question}
已知\(\overrightarrow{a},\overrightarrow{b},\overrightarrow{c}\)均为单位向量.若\(\overrightarrow{a} = \overrightarrow{b} + \overrightarrow{c}\),则\(\overrightarrow{b}\)与\(\overrightarrow{c}\)夹角的大小是(    )
\begin{choices}
  \(\frac{\text{π}}{6}\)

\begin{enumerate}[label=(\arabic*)]
  \item \(\frac{\text{π}}{3}\)
  \item \(\frac{2\text{π}}{3}\)
  \item \(\frac{5\text{π}}{6}\)
\end{choices}
\topics{用定义求向量的数量积;数量积的运算律;向量夹角的计算}
\difficulty{0.65}
\answer{C}
\explain{已知\(\overrightarrow{a} = \overrightarrow{b} + \overrightarrow{c}\),两边平方可得\({\overrightarrow{a}}^{2} = {(\overrightarrow{b} + \overrightarrow{c})}^{2}\).
%
则\({(\overrightarrow{b} + \overrightarrow{c})}^{2} = {\overrightarrow{b}}^{2} + 2\overrightarrow{b} \cdot \overrightarrow{c} + {\overrightarrow{c}}^{2}\),所以\({\overrightarrow{a}}^{2} = {\overrightarrow{b}}^{2} + 2\overrightarrow{b} \cdot \overrightarrow{c} + {\overrightarrow{c}}^{2}\).
%
因为\(\overrightarrow{a},\overrightarrow{b},\overrightarrow{c}\)均为单位向量,所以\(|\overrightarrow{a}| = |\overrightarrow{b}| = |\overrightarrow{c}| = 1\).
%
根据\({\overrightarrow{a}}^{2} = |\overrightarrow{a}|^{2} = 1\),\({\overrightarrow{b}}^{2} = |\overrightarrow{b}|^{2} = 1\),\({\overrightarrow{c}}^{2} = |\overrightarrow{c}|^{2} = 1\).
%
将其代入\({\overrightarrow{a}}^{2} = {\overrightarrow{b}}^{2} + 2\overrightarrow{b} \cdot \overrightarrow{c} + {\overrightarrow{c}}^{2}\)可得:\(1 = 1 + 2\overrightarrow{b} \cdot \overrightarrow{c} + 1\).
则\(\overrightarrow{b} \cdot \overrightarrow{c} = - \frac{1}{2}\).
%
设\(\overrightarrow{b}\)与\(\overrightarrow{c}\)的夹角为\(\theta\),\(0 \leq \theta \leq \pi\),且\(|\overrightarrow{b}| = |\overrightarrow{c}| = 1\),\(\overrightarrow{b} \cdot \overrightarrow{c} = - \frac{1}{2}\),可得\(- \frac{1}{2} = 1 \times 1 \times \cos\theta\),即\(\cos\theta = - \frac{1}{2}\).
%
因为\(0 \leq \theta \leq \pi\),所以\(\theta = \frac{2\pi}{3}\).
%
则\(\overrightarrow{b}\)与\(\overrightarrow{c}\)夹角的大小是\(\frac{2\pi}{3}\)}
\end{enumerate}
\end{question}

\begin{question}
某项比赛共有10个评委评分,若去掉一个最高分与一个最低分,则与原始数据相比,一定不变的是(    )
\begin{choices}
  极差

\begin{enumerate}[label=(\arabic*)]
  \item 45百分位数
  \item 平均数
  \item 众数
\end{choices}
\topics{计算几个数的众数;计算几个数的平均数;计算几个数据的极差;方差;标准差;总体百分位数的估计}
\difficulty{0.94}
\answer{B}
\explain{对A,若每个数据都不相同,则极差一定变化,故A错误;
对B,由\(10 \times 0.45\text{=}4.5 < 5\),所以将10个数据从小到大排列,45百分位数为第5个数据,
从10个原始评分中去掉1个最高分、1个最低分,得到8个有效评分,\(8 \times 0.45\text{=}3.6 < 4\),
所以45百分位数为8个数据从小到大排列后第4个数据,即为原来的第5个数据.
对C,去掉一个最高分一个最低分,平均数可能变化,故C错误;
对D,去掉一个最高分一个最低分,众数可能变化,故D错误}
\end{enumerate}
\end{question}

\begin{question}
已知数列\(\left\{ a_{n} \right\}\)为等比数列,公比为2,且\(a_{1} + a_{2} = 3\).若\(a_{k} + a_{k + 1} + a_{k + 2} + \cdots + a_{k + 9} = 2^{14} - 2^{4}\),则正整数\(k\)的值是(    )
\begin{choices}
  4

\begin{enumerate}[label=(\arabic*)]
  \item 5
  \item 6
  \item 7
\end{choices}
\topics{等比数列通项公式的基本量计算;求等比数列前n项和}
\difficulty{0.85}
\answer{B}
\explain{因为数列\(\left\{ a_{n} \right\}\)为等比数列,公比为2,且\(a_{1} + a_{2} = 3\),所以\(a_{1} + 2a_{1} = 3\),解得\(a_{1} = 1\),
%
故\(a_{n} = 2^{n - 1}\),因为\(a_{k} + a_{k + 1} + a_{k + 2} + \cdots + a_{k + 9} = a_{k}(1 + 2 + 2^{2} + ... + 2^{9})= 2^{k - 1} \cdot \frac{1 - 2^{10}}{1 - 2} = 2^{k + 9} - 2^{k - 1} = 2^{14} - 2^{4}\),解得\(k = 5\)}
\end{enumerate}
\end{question}

\begin{question}
在锐角\(\bigtriangleup ABC\)中,角\(A,B,C\)所对的边分别为\(a,b,c\).若\(b - 2c = a\text{cos}C - 2a\text{cos}B\),则\(\frac{c}{b} =\)(    )
\begin{choices}
  \(\frac{1}{3}\)

\begin{enumerate}[label=(\arabic*)]
  \item \(\frac{1}{2}\)
  \item 1
  \item 2
\end{choices}
\topics{正弦定理边角互化的应用}
\difficulty{0.85}
\answer{D}
\explain{如图所示,过点\emph{A}作\(AD\bot BC\)于点\emph{D},
%

% IMAGE_TODO_START id=nanjing_yancheng_2025_mock1-Q6-img1 path=/Users/muryor/code/mynote/word\_to\_tex/output/figures/nanjing\_yancheng\_2025\_mock1/media/image2.png width=60% inline=true question_index=6 sub_index=1
% CONTEXT_BEFORE: 用射影定理即可化简求值. 【详解】如图所示,过点*A*作\(AD\bot BC\)于点*D*,
% CONTEXT_AFTER: 
\begin{tikzpicture}[scale=0.8,baseline=-0.5ex]
  % TODO: AI_AGENT_REPLACE_ME (id=nanjing_yancheng_2025_mock1-Q6-img1)
\end{tikzpicture}
% IMAGE_TODO_END id=nanjing_yancheng_2025_mock1-Q6-img1{
%
则\(BD = c\cos B,CD = b\cos C,\therefore a = BD + CD = c\cos B + b\cos C\),
%
同理可证\(b = a\cos C + c\cos A,c = a\cos B + b\cos A\),
%
因为\(b - 2c = a\text{cos}C - 2a\text{cos}B\),所以\(a\cos C + c\cos A - 2(a\cos B + b\cos A) = a\text{cos}C - 2a\text{cos}B\),
%
整理得\(c\cos A = 2b\cos A\),因为\(\bigtriangleup ABC\)为锐角三角形,所以\(\cos A \neq 0\),
%
所以\(c = 2b\),即\(\frac{c}{b} = 2\)}
\end{enumerate}
\end{question}
%
\begin{question}
已知双曲线\(C:\frac{x^{2}}{a^{2}} - \frac{y^{2}}{b^{2}} = 1(a > 0,b > 0)\)的左焦点、右顶点分别为\(F,A\),过点\(F\)倾斜角为\(\frac{\text{π}}{6}\)的直线交\(C\)的两条渐近线分别于点\(M,N\).若\(\bigtriangleup AMN\)为等边三角形,则双曲线\(C\)的渐近线方程是(    )
\begin{choices}
  \(y = \pm \frac{\sqrt{3}}{3}x\)

\begin{enumerate}[label=(\arabic*)]
  \item \(y = \pm \frac{2\sqrt{3}}{3}x\)
  \item \(y = \pm \sqrt{3}x\)
  \item \(y = \pm \frac{4\sqrt{3}}{3}x\)
\end{choices}
\topics{求直线交点坐标;根据a,b,c齐次式关系求渐近线方程}
\difficulty{0.4}
\answer{C}
\explain{由题意可得\(A(a,0),F( - c,0)\),所以直线\(MN\)的方程为\(y = \frac{\sqrt{3}}{3}(x + c)\),
%
由\(\left\{ \begin{array}{r}
y = \frac{\sqrt{3}}{3}(x + c) \\
y = \frac{b}{a}x
\end{array} \right.可得\)M\left( \frac{ac}{\sqrt{3}b - a},\frac{bc}{\sqrt{3}b - a} \right)\(,
%
由\)\left\{ \begin{array}{r}
y = \frac{\sqrt{3}}{3}(x + c) \\
y = - \frac{b}{a}x
\end{array} \right.可得\(N\left( \frac{- ac}{\sqrt{3}b + a},\frac{bc}{\sqrt{3}b + a} \right)\),
%
因为\(\bigtriangleup AMN\)为等边三角形,所以\(|AM| = |AN|\),
%
即\(\left( \frac{ac}{\sqrt{3}b - a} - a \right)^{2} + \left( \frac{bc}{\sqrt{3}b - a} \right)^{2} = \left( \frac{- ac}{\sqrt{3}b + a} - a \right)^{2} + \left( \frac{bc}{\sqrt{3}b + a} \right)^{2}\),
%
整理可得\(b^{2} = 3a^{2}\),所以\(\frac{b}{a} = \sqrt{3}\),
%
所以双曲线\(C\)的渐近线方程是\(y = \pm \sqrt{3}x\)}
\end{enumerate}
\end{question}
%
\begin{question}
已知函数\(f(x) = \frac{1}{a^{2x} + 1} - ax^{3},a > 1\),则关于\(x\)的不等式\(f\left( x^{2} \right) + f(5x - 6) > 1\)的解集是(    )
\begin{choices}
  \(( - 6,1)\)

\begin{enumerate}[label=(\arabic*)]
  \item \((2,3)\)
  \item \(( - \infty,1)\)
  \item \((2, + \infty)\)
\end{choices}
\topics{判断或证明函数的对称性;根据函数的单调性解不等式}
\difficulty{0.65}
\answer{A}
\explain{由\(1 - f( - x) = 1 - \frac{1}{a^{- 2x} + 1} + a( - x)^{3} = 1 - \frac{a^{2x}}{1 + a^{2x}} - ax^{3} = \frac{1}{a^{2x} + 1} - ax^{3} = f(x)\),
%
则\(1 - f(5x - 6) = f(6 - 5x)\),
%
由\(a > 1\),则函数\(y = a^{2x}\)在\(\text{R}\)上单调递增,易知函数\(f(x)\)在\(\text{R}\)上单调递减,
%
由\(f\left( x^{2} \right) + f(5x - 6) > 1\),则\(f\left( x^{2} \right) > 1 - f(5x - 6)\),即\(f\left( x^{2} \right) > f(6 - 5x)\),
%
可得\(x^{2} < 6 - 5x\),分解因式可得\((x + 6)(x - 1) < 0\),解得\(- 6 < x < 1\)}
\end{enumerate}
\end{question}
%
\section{多选题}
%
\begin{question}
已知\(\text{cos}\alpha\text{cos}\beta = \frac{1}{4},\text{cos}(\alpha + \beta) = \frac{1}{3}\),则(    )
\begin{choices}
  \(\text{sin}\alpha\text{sin}\beta = \frac{1}{12}\)

\begin{enumerate}[label=(\arabic*)]
  \item \(\text{cos}(\alpha - \beta) = \frac{1}{6}\)
  \item \(\text{tan}\alpha\text{tan}\beta = - \frac{1}{3}\)
  \item \(\text{sin}2\alpha\text{sin}2\beta = \frac{1}{12}\)
\end{choices}
\topics{三角函数的化简;求值------同角三角函数基本关系;用和;差角的正弦公式化简;求值;二倍角的正弦公式}
\difficulty{0.85}
\answer{BC}
\explain{由\(\cos(\alpha + \beta) = \cos\alpha\cos\beta - \sin\alpha\sin\beta = \frac{1}{3}\),且\(\cos\alpha\cos\beta = \frac{1}{4}\),则\(\sin\alpha\sin\beta = - \frac{1}{12}\),故A错误;
%
由\(\cos(\alpha - \beta) = \cos\alpha\cos\beta + \sin\alpha\sin\beta = \frac{1}{4} - \frac{1}{12} = \frac{1}{6}\),故B正确;
%
由\(\tan\alpha\tan\beta = \frac{\sin\alpha\sin\beta}{\cos\alpha\cos\beta} = \frac{- \frac{1}{12}}{\frac{1}{4}} = - \frac{1}{3}\),故C正确;
%
由\(\sin 2\alpha\sin 2\beta = 2\sin acos\alpha \cdot 2\sin\beta\cos\beta = 4\sin\alpha\sin\beta\cos\alpha\cos\beta = 4 \times \left( - \frac{1}{12} \right) \times \frac{1}{4} = - \frac{1}{12}\),故D错误}
\end{enumerate}
\end{question}
%
\begin{question}
在边长为2的菱形\(ABCD\)中,\(\angle BAD = \frac{\text{π}}{3}\),将菱形\(ABCD\)沿对角线\(BD\)折成四面体\(A'BCD\),使得\(\angle A'BC = \frac{\text{π}}{2}\),则(    )
\begin{choices}
  直线\(A'C\)与直线\(BD\)所成角为\(\frac{\text{π}}{2}\)

\begin{enumerate}[label=(\arabic*)]
  \item 直线\(A'C\)与平面\(BCD\)所成角的余弦值为\(\frac{\sqrt{6}}{3}\)
  \item 四面体\(A'BCD\)的体积为\(\frac{4\sqrt{2}}{3}\)
  \item 四面体\(A'BCD\)外接球的表面积为\(8\text{π}\)
\end{choices}
\topics{锥体体积的有关计算;多面体与球体内切外接问题;求异面直线所成的角;求线面角}
\difficulty{0.65}
\answer{ABD}
\explain{如图所示,取\(BD\)的中点,连接\(AE、CE\),
%
因\(\bigtriangleup A'BD\)和\(\bigtriangleup BDC\)为等边三角形,则\(AE\bot BD、CE\bot BD\),
%
因\(AE \cap CE = E,AE \subset\)平面\(AEC\),\(CE \subset\)平面\(AEC\),则\(BD\bot\)平面\(AEC\),
%
因\(A'C \subset\)平面\(AEC\),则\(BD\bot A'C\),故A正确;
%
因\(BD\bot\)平面\(AEC\),则\(A'\)在平面\(BCD\)内的投影落在直线\(EC\)上,
%
故\(\angle A'CE\)为直线\(A'C\)与平面\(BCD\)所成角,
%
因\(\angle A'BC = \frac{\text{π}}{2}\),\(\left| A'B \right| = |BC| = 2\),则\(\left| A'C \right| = 2\sqrt{2}\),
%
因\(\left| A'E \right| = |EC| = \sqrt{3}\),则在\(\bigtriangleup A'EC\)中边\(A'C\)上的高为\(1\),则\(\cos\angle A'CE = \frac{\sqrt{2}}{\sqrt{3}} = \frac{\sqrt{6}}{3}\),故B正确;
%
因\(S_{\bigtriangleup A'EC} = \frac{1}{2} \times 2\sqrt{2} \times 1 = \sqrt{2}\),\(BD\bot\)平面\(AEC\),则\(V_{三棱锥A' - BCD} = \frac{1}{3}S_{\bigtriangleup A'EC} \cdot |BD| = \frac{1}{3} \times \sqrt{2} \times 2 = \frac{2\sqrt{2}}{3}\),故C错误;
%
点\(O_{1}、O_{2}\)分别为\(\bigtriangleup A'BD\)和\(\bigtriangleup BDC\)的外心,过\(O_{1}、O_{2}\)分别作\(O_{1}O\bot\)平面\(A'BD\),\(O_{2}O\bot\)平面\(BDC\),\(O_{1}O \cap O_{2}O = O\),则点\(O\)为球心,
%
则\(\left| CO_{2} \right| = \frac{2\sqrt{3}}{3},\left| EO_{2} \right| = \left| EO_{1} \right| = \frac{\sqrt{3}}{3}\),
%
在\(\bigtriangleup A'EC\)中,\(\tan\frac{\angle A'EC}{2} = \sqrt{2}\),故\(\left| OO_{2} \right| = \left| EO_{2} \right|\tan\frac{\angle A'EC}{2} = \frac{\sqrt{3}}{3} \times \sqrt{2} = \frac{\sqrt{6}}{3}\),
%
则\(|OC|^{2} = \left| OO_{2} \right|^{2} + \left| CO_{2} \right|^{2} = \left( \frac{\sqrt{6}}{3} \right)^{2} + \left( \frac{2\sqrt{3}}{3} \right)^{2} = 2\),
%
则四面体\(A'BCD\)外接球的表面积为\(4\text{π} \times 2 = 8\text{π}\),故D正确.
%

% IMAGE_TODO_START id=nanjing_yancheng_2025_mock1-Q10-img1 path=/Users/muryor/code/mynote/word\_to\_tex/output/figures/nanjing\_yancheng\_2025\_mock1/media/image3.png width=60% inline=true question_index=10 sub_index=1
% CONTEXT_BEFORE: 外接球的表面积为\(4 \times 2 = 8\),故D正确.
% CONTEXT_AFTER: 
\begin{tikzpicture}[scale=0.8,baseline=-0.5ex]
  % TODO: AI_AGENT_REPLACE_ME (id=nanjing_yancheng_2025_mock1-Q10-img1)
\end{tikzpicture}
% IMAGE_TODO_END id=nanjing_yancheng_2025_mock1-Q10-img1{
}
\end{enumerate}
\end{question}
%
\begin{question}
已知函数\(f(x)\)满足:对任意\(x,y \in R,xf(y) + yf(x) = f(xy)\),且当\(0 < x < 1\)时,\(f(x) > 0\).下列说法正确的是(    )
\begin{choices}
  \(f(0) + f(1) = 0\)

\begin{enumerate}[label=(\arabic*)]
  \item \(f(x)\)为偶函数
  \item 当\(|x| > 1\)时,\(xf(x) < 0\)
  \item \(f(x)\)在\((1, + \infty)\)上单调递减
\end{choices}
\topics{求函数值;定义法判断或证明函数的单调性;函数奇偶性的定义与判断;用导数判断或证明已知函数的单调性}
\difficulty{0.4}
\answer{ACD}
\explain{因为\(x,y \in R,xf(y) + yf(x) = f(xy)\),
%
令\(x = 0\),\(y = 0\),可得\(0f(0) + 0f(0) = f(0 \times 0)\),
%
所以\(f(0) = 0\),
%
令\(x = 1\),\(y = 1\),可得\(1f(1) + 1f(1) = f(1 \times 1)\),
%
所以\(f(1) = 0\),
%
所以\(f(0) + f(1) = 0\),A正确;
%
由\(xf(y) + yf(x) = f(xy)\),
%
令\(y = - 1\)可得,\(xf( - 1) - f(x) = f( - x)\),
%
再将\(xf( - 1) - f(x) = f( - x)\)中的\(x\)替换为\(- 1\),可得\(- f( - 1) - f( - 1) = f(1)\),
%
所以\(f( - 1) = 0\),
%
所以\(- f(x) = f( - x)\),所以函数\(f(x)\)为奇函数,B错误;
%
当\(x \neq 0\)时,将\(xf(y) + yf(x) = f(xy)\)中的\(y\)用\(\frac{1}{x}\)替换,
%
可得\(xf\left( \frac{1}{x} \right) + \frac{1}{x}f(x) = f(1) = 0\),即\(xf(x) = - x^{3}f\left( \frac{1}{x} \right)\),
%
当\(x > 1\)时,\(0 < \frac{1}{x} < 1\),由已知可得\(f\left( \frac{1}{x} \right) > 0\),
%
所以\(xf(x) < 0\),\(f(x) < 0\),
%
又函数\(f(x)\)为奇函数,所以当\(x < - 1\)时,\(f(x) > 0\),\(xf(x) < 0\),
%
所以当\(|x| > 1\)时,\(xf(x) < 0\),C正确;
%
因为\(xf(y) + yf(x) = f(xy)\),
%
所以若\(xy \neq 0\),则\(\frac{f(y)}{y} + \frac{f(x)}{x} = \frac{f(xy)}{xy}\),
%
任取\(x_{1},x_{2} \in (1, + \infty)\),且\(x_{1} < x_{2}\),
%
则\(\frac{f\left( x_{2} \right)}{x_{2}} - \frac{f\left( x_{1} \right)}{x_{1}} = \frac{f\left( \frac{x_{2}}{x_{1}} \times x_{1} \right)}{\frac{x_{2}}{x_{1}} \times x_{1}} - \frac{f\left( x_{1} \right)}{x_{1}} = \frac{f\left( \frac{x_{2}}{x_{1}} \right)}{\frac{x_{2}}{x_{1}}} + \frac{f\left( x_{1} \right)}{x_{1}} - \frac{f\left( x_{1} \right)}{x_{1}} = \frac{x_{1}}{x_{2}}f\left( \frac{x_{2}}{x_{1}} \right)\),
%
因为\(x_{2} > x_{1} > 0\),所以\(\frac{x_{2}}{x_{1}} > 1\),\(0 < \frac{x_{1}}{x_{2}} < 1\),\(f\left( \frac{x_{2}}{x_{1}} \right) < 0\),
%
所以\(\frac{f\left( x_{2} \right)}{x_{2}} - \frac{f\left( x_{1} \right)}{x_{1}} = \frac{x_{1}}{x_{2}}f\left( \frac{x_{2}}{x_{1}} \right) < 0\),所以\(\frac{f\left( x_{2} \right)}{x_{2}} < \frac{f\left( x_{1} \right)}{x_{1}}\),
%
所以函数\(\frac{f(x)}{x}\)在\((1, + \infty)\)上单调递减,
%
设\(y = x \cdot \frac{f(x)}{x}\),
%
当\(x > 1\)时,\(y' = \frac{f(x)}{x} + x\left\lbrack \frac{f(x)}{x} \right\rbrack'\),
%
因为\(f(x) < 0\),所以\(\frac{f(x)}{x} < 0\),
%
因为函数\(\frac{f(x)}{x}\)在\((1, + \infty)\)上单调递减,所以\(\left\lbrack \frac{f(x)}{x} \right\rbrack' \leq 0\),
%
所以\(y' < 0\),
%
所以\(f(x)\)在\((1, + \infty)\)上单调递减}
\end{enumerate}
\end{question}
%
\section{填空题}
%
\begin{question}
若函数\(f(x) = \text{sin}x + a\text{cos}x\)的图象关于直线\(x = \frac{\text{π}}{6}\)对称,则实数\(a\)的值是
.
\topics{结合三角函数的图象变换求三角函数的性质;用和;差角的余弦公式化简;求值;用和;差角的正弦公式化简;求值}
\difficulty{0.65}
\answer{\(\sqrt{3}\)}
\explain{因为函数\(f(x) = \text{sin}x + a\text{cos}x\)的图象关于直线\(x = \frac{\text{π}}{6}\)对称,
%
所以\(f\left( \frac{\text{π}}{6} + x \right) = f\left( \frac{\text{π}}{6} - x \right)\),
%
所以\(\text{sin}\left( \frac{\text{π}}{6} + x \right) + a\text{cos}\left( \frac{\text{π}}{6} + x \right) = \text{sin}\left( \frac{\text{π}}{6} - x \right) + a\text{cos}\left( \frac{\text{π}}{6} - x \right)\),
%
所以\(\text{sin}\left( \frac{\text{π}}{6} + x \right) - \text{sin}\left( \frac{\text{π}}{6} - x \right) = a\left\lbrack \text{cos}\left( \frac{\text{π}}{6} - x \right) - \text{cos}\left( \frac{\text{π}}{6} + x \right) \right\rbrack\),
%
所以\(2\cos\frac{\text{π}}{6}\sin x = 2a\sin\frac{\text{π}}{6}\sin x\),因为\(\sin x\)不恒为\(0\),
%
所以\(\cos\frac{\text{π}}{6} = a\sin\frac{\text{π}}{6}\),所以\(a = \sqrt{3}\).\(\sqrt{3}\).}
\end{question}
%
\begin{question}
已知椭圆\(C:\frac{x^{2}}{2} + y^{2} = 1\)的上顶点为\(A\),直线\(l:y = kx + m\)交\(C\)于\(M,N\)两点.若\(\bigtriangleup AMN\)的重心为\(\left( \frac{1}{2},0 \right)\),则实数\(k\)的值是
.
\topics{根据直线与椭圆的位置关系求参数或范围;根据韦达定理求参数}
\difficulty{0.4}
\answer{\(\frac{3}{4}\)/0.75}
\explain{已知椭圆\(C:\frac{x^{2}}{2} + y^{2} = 1\),上顶点坐标为\(A(0,1)\).
%
设\(M(x_{1},y_{1})\),\(N(x_{2},y_{2})\),因为\(\bigtriangleup AMN\)的重心为\((\frac{1}{2},0)\),所以\(\frac{0 + x_{1} + x_{2}}{3} = \frac{1}{2}\),\(\frac{1 + y_{1} + y_{2}}{3} = 0\).
%
由\(\frac{0 + x_{1} + x_{2}}{3} = \frac{1}{2}\)可得\(x_{1} + x_{2} = \frac{3}{2}\);由\(\frac{1 + y_{1} + y_{2}}{3} = 0\)可得\(y_{1} + y_{2} = - 1\).
%
直曲联立,将直线\(l:y = kx + m\)代入椭圆\(C:\frac{x^{2}}{2} + y^{2} = 1\),可得\(\frac{x^{2}}{2} + {(kx + m)}^{2} = 1\).
%
展开并整理得\((1 + 2k^{2})x^{2} + 4kmx + 2m^{2} - 2 = 0\).
%
根据韦达定理可知\(x_{1} + x_{2} = - \frac{4km}{1 + 2k^{2}}\).
%
又因为\(y_{1} = kx_{1} + m\),\(y_{2} = kx_{2} + m\),所以\(y_{1} + y_{2} = k(x_{1} + x_{2}) + 2m = k \times ( - \frac{4km}{1 + 2k^{2}}) + 2m = \frac{2m}{1 + 2k^{2}}\).
%
由\(x_{1} + x_{2} = \frac{3}{2}\)可得\(- \frac{4km}{1 + 2k^{2}} = \frac{3}{2}\)
①;由\(y_{1} + y_{2} = - 1\)可得\(\frac{2m}{1 + 2k^{2}} = - 1\) ②.
%
由②可得\(m = - \frac{1 + 2k^{2}}{2}\),将其代入①可得:
%
\(- \frac{4k \times ( - \frac{1 + 2k^{2}}{2})}{1 + 2k^{2}} = \frac{3}{2}\),则\(2k = \frac{3}{2}\),解得\(k = \frac{3}{4}\).
%
当\(k = \frac{3}{4}\)时,代入②可得,
\(m = - \frac{17}{16}\),此时直线\(l:y = \frac{3}{4}x - \frac{17}{16}\)与椭圆\(\frac{x^{2}}{2} + y^{2} = 1\)有两个交点,符合题意.\(\frac{3}{4}\).
%

% IMAGE_TODO_START id=nanjing_yancheng_2025_mock1-Q13-img1 path=/Users/muryor/code/mynote/word\_to\_tex/output/figures/nanjing\_yancheng\_2025\_mock1/media/image4.png width=60% inline=true question_index=13 sub_index=1
% CONTEXT_BEFORE: + y^{2} = 1\(有两个交点,符合题意. \){4}\(.
% CONTEXT_AFTER: 
\begin{tikzpicture}[scale=0.8,baseline=-0.5ex]
  % TODO: AI_AGENT_REPLACE_ME (id=nanjing_yancheng_2025_mock1-Q13-img1)
\end{tikzpicture}
% IMAGE_TODO_END id=nanjing_yancheng_2025_mock1-Q13-img1{
}
\end{question}
%
\begin{question}
将9个互不相同的向量\({\overrightarrow{a}}_{i} = \left( x_{i},y_{i} \right),x_{i},y_{i} \in \left\{ - 1,0,1 \right\},i = 1,2,\cdots,9\),填入\(3 \times 3\)的方格中,使得每行、每列的三个向量的和都相等,则不同的填法种数是
.
\topics{分步乘法计数原理及简单应用;排列数的计算}
\difficulty{0.4}
\answer{72}
\explain{已知\({\overrightarrow{a}}_{i} = \left( x_{i},y_{i} \right),x_{i},y_{i} \in \left\{ - 1,0,1 \right\},i = 1,2,\cdots,9\),
%
那么向量\({\overrightarrow{a}}_{i}\)的所有可能情况有\(( - 1, - 1),( - 1,0),( - 1,1),(0, - 1),(0,0),(0,1),(1, - 1),(1,0),(1,1)\)共\(9\)种.
%
设每行、每列的三个向量的和为\(\overrightarrow{s} = (m,n)\),因为\(x_{i},y_{i} \in \{ - 1,0,1\}\),所以\(m,n \in \{ - 3, - 2, - 1,0,1,2,3\}\).
%
又因为三行向量和等于三列向量和,且所有向量和为\(\sum_{i = 1}^{9}{\overrightarrow{a}}_{i}\),所以\(3\overrightarrow{s} = \sum_{i = 1}^{9}{\overrightarrow{a}}_{i}\),
%
而\(\sum_{i = 1}^{9}{\overrightarrow{a}}_{i}\)的\(x\)分量和\(y\)分量都为\(0\)(\(x_{i},y_{i}\)取值\(- 1,0,1\)且各有\(3\)个),所以\(\overrightarrow{s} = (0,0)\).
%
要使每行、每列的三个向量和为\((0,0)\),则每行、每列的三个向量的\(x\)分量和\(y\)分量都分别为\(0\).
%
对于\(x\)分量,\(3\)个数的和为\(0\),有\(( - 1,0,1)\)这一种组合情况;
%
对于\(y\)分量,\(3\)个数的和为\(0\),也有\(( - 1,0,1)\)这一种组合情况.
%
先确定第一行的填法,第一行的\(3\)个向量的\(x\)分量和\(y\)分量都要满足\(( - 1,0,1)\)的组合,\(x\)分量的排列有\(\text{A}_{\text{3}}^{\text{3}}\text{=3!=6}\)种,\(y\)分量的排列也有\(\text{A}_{\text{3}}^{\text{3}}\text{=3!=6}\)种,所以第一行的填法有\(6 \times 6 = 36\)种.
%
当第一行确定后,第二行第一列的向量\(x\)分量要与第一行第一列和第三行第一列的\(x\)分量和为\(0\),\(y\)分量同理,所以第二行第一列的向量是唯一确定的,同理第二行第二列、第二行第三列的向量也唯一确定,第三行的向量也就随之确定.
%
因为第一行确定后,第二行和第三行可以交换位置.
%
所以不同的填法种数是\(36 \times 2 = 72\)种.72.}
\end{question}
%
\section{解答题}
%
\begin{question}
求证:平面\(BCE\bot\)平面\(ACD\);

\begin{enumerate}[label=(\arabic*)]
\item 若\(AB = \sqrt{5},BC = 1,CD = 2\sqrt{3}\),求平面\(BCE\)与平面\(BDE\)所成锐二面角的余弦值.
\topics{证明面面垂直;面面角的向量求法}
\difficulty{0.65}
\answer{(1)证明见解析
(2)\(\frac{1}{4}\)}
\explain{(1)因为\(\bigtriangleup ABC\)内接于圆\(O,AB\)为圆\(O\)的直径,所以\(AC\bot BC\).
%
因为\(CD\bot\)平面\(ABC,BC \subset\)平面\(ABC\),所以\(CD\bot BC\).
%
又\(AC,CD \subset\)平面\(ACD,AC \cap CD = C\),所以\(BC\bot\)平面\(ACD\).
%
因为\(BC \subset\)平面\(BCE\),所以平面\(BCE\bot\)平面\(ACD\).
%
(2)因为\(CD\bot\)平面\(ABC,AC,BC \subset\)平面\(ABC\),
%
所以\(CD\bot AC,CD\bot BC\).
%
以\(C\)为坐标原点建立如图所示的空间直角坐标系,
%

% IMAGE_TODO_START id=nanjing_yancheng_2025_mock1-Q15-img2 path=/Users/muryor/code/mynote/word\_to\_tex/output/figures/nanjing\_yancheng\_2025\_mock1/media/image6.png width=60% inline=true question_index=15 sub_index=1
% CONTEXT_BEFORE: \bot AC,CD\bot BC\). 以\(C\)为坐标原点建立如图所示的空间直角坐标系,
% CONTEXT_AFTER:  因
\begin{tikzpicture}[scale=0.8,baseline=-0.5ex]
  % TODO: AI_AGENT_REPLACE_ME (id=nanjing_yancheng_2025_mock1-Q15-img2)
\end{tikzpicture}
% IMAGE_TODO_END id=nanjing_yancheng_2025_mock1-Q15-img2{
%
因为\(AB = \sqrt{5},BC = 1\),所以\(AC = 2\),
%
则\(A(2,0,0),B(0,1,0),D\left( 0,0,2\sqrt{3} \right),E\left( 1,0,\sqrt{3} \right)\),
%
所以\(\overrightarrow{CE} = \left( 1,0,\sqrt{3} \right),\overrightarrow{CB} = (0,1,0)\).
%
设平面\(BCE\)的法向量\(\overrightarrow{m} = \left( x_{1},y_{1},z_{1} \right)\),
%
由\(\left\{ \begin{array}{r}
\overrightarrow{m} \cdot \overrightarrow{CE} = 0, \\
\overrightarrow{m} \cdot \overrightarrow{CB} = 0,
\end{array} \right.得\)\left\{ \begin{array}{r}
x_{1} + \sqrt{3}z_{1} = 0, \\
y_{1} = 0,
\end{array} \right.\ 不妨设\(z_{1} = 1\),则\(x_{1} = - \sqrt{3}\),
%
所以平面\(BCE\)的一个法向量\(\overrightarrow{m} = \left( - \sqrt{3},0,1 \right)\).
%
又\(\overrightarrow{BD} = \left( 0, - 1,2\sqrt{3} \right),\overrightarrow{DE} = \left( 1,0, - \sqrt{3} \right)\),
%
设平面\(BDE\)的法向量\(\overrightarrow{n} = \left( x_{2},y_{2},z_{2} \right)\),
%
由\(\left\{ \begin{array}{r}
\overrightarrow{n} \cdot \overrightarrow{BD} = 0, \\
\overrightarrow{n} \cdot \overrightarrow{DE} = 0,
\end{array} \right.得\)\left\{ \begin{array}{r}
 - y_{2} + 2\sqrt{3}z_{2} = 0, \\
x_{2} - \sqrt{3}z_{2} = 0,
\end{array} \right.\ 不妨设\(z_{2} = 1\),则\(x_{2} = \sqrt{3},y_{2} = 2\sqrt{3}\),
%
所以平面\(BDE\)的一个法向量\(\overrightarrow{n} = \left( \sqrt{3},2\sqrt{3},1 \right)\).
%
所以\(\text{cos} < \overset{\rightarrow}{m},\overset{\rightarrow}{n} > = \frac{\overset{\rightarrow}{m} \cdot \overset{\rightarrow}{n}}{\left| \overset{\rightarrow}{m} \right|\left| \overset{\rightarrow}{n} \right|} = \frac{2}{2 \times 4} = - \frac{1}{4}\),
%
即平面\(BCE\)与平面\(BDE\)所成锐二面角的余弦值为\(\frac{1}{4}\).}
\end{enumerate}
\end{question}
%
\begin{question}
求\(t\)的值;

\begin{enumerate}[label=(\arabic*)]
\item 证明:\(\left\{ a_{n} \right\}\)为等差数列;
\item 若\(n^{2} < S_{n} < {(n + 1)}^{2},n \in N^{\text{*}}\),求\(a_{1}\)的取值范围.
\topics{由递推关系证明数列是等差数列;求等差数列前n项和;利用an与sn关系求通项或项;数列不等式恒成立问题}
\difficulty{0.65}
\answer{(1)\(t = 1\)
(2)证明见解析
(3)\(1 < a_{1} \leq 3\)}
\explain{(1)因为\(\frac{S_{n}}{n} = a_{n} + (1 - n)t,n \in N^{\text{*}}\),
%
所以\(\frac{S_{2}}{2} = a_{2} - t\),又\(S_{2} = a_{1} + a_{2}\),
%
所以\(a_{2} - a_{1} = 2t\).
%
又\(a_{2} = a_{1} + 2\),所以\(t = 1\).
%
(2)由(1)可得\(\frac{S_{n}}{n} = a_{n} + 1 - n,n \in N^{\text{*}}\),所以\(S_{n} = na_{n} + n - n^{2}\),
%
因此\(S_{n + 1} = (n + 1)a_{n + 1} + n + 1 - {(n + 1)}^{2}\),
%
相减得\(a_{n + 1} = (n + 1)a_{n + 1} - na_{n} - 2n\),
%
得\(a_{n + 1} - a_{n} = 2,n \in N^{\text{*}}\),
%
所以\(\left\{ a_{n} \right\}\)为等差数列.
%
(3)由(2)得\(S_{n} = na_{1} + \frac{n(n - 1)}{2} \times 2 = n^{2} + \left( a_{1} - 1 \right)n\),
%
由\(n^{2} < S_{n} < {(n + 1)}^{2},n \in N^{\text{*}}\),得\(1 < a_{1} < 3 + \frac{1}{n}\).
%
因为\(1 < a_{1} < 3 + \frac{1}{n}\)对\(n \in N^{\text{*}}\)恒成立,
%
所以\(1 < a_{1} \leq 3\).}
\end{enumerate}
\end{question}
%
\begin{question}
已知甲先上场,\(p = \frac{1}{2},q = \frac{1}{3},n = 2\),

\begin{enumerate}[label=(\arabic*)]
%
①求挑战没有一关成功的概率;
%
②设\(X\)为挑战比赛结束时挑战成功的关卡数,求\(E(X)\);
\item 如果\(n\)关都挑战成功,那么比赛挑战成功.试判断甲先出场与乙先出场比赛挑战成功的概率是否相同,并说明理由.
\topics{利用对立事件的概率公式求概率;独立事件的乘法公式;求离散型随机变量的均值}
\difficulty{0.65}
\answer{(1)①\(\frac{1}{3}\);②\(\frac{19}{18}\)
(2)概率相同,理由见解析}
\explain{(1)①记甲先上场且挑战没有一关成功的概率为\(P\),则\(P = (1 - p)(1 - q) = \frac{1}{3}\).
②依题可知,\(X\)的可能取值为\(0,1,2\),则
\(P(X = 0) = \frac{1}{3};P(X = 1) = p(1 - p)(1 - q) + (1 - p)q(1 - q) = \frac{1}{2} \times \frac{1}{2} \times \left( 1 - \frac{1}{3} \right) + \frac{1}{2} \times \frac{1}{3} \times \left( 1 - \frac{1}{3} \right) = \frac{5}{18};P(X = 2) = 1 - \frac{1}{3} - \frac{5}{18} = \frac{7}{18}\),
所以\(E(X) = 0 \times \frac{1}{3} + 1 \times \frac{5}{18} + 2 \times \frac{7}{18} = \frac{19}{18}\).
(2)设甲先出场比赛挑战成功的概率为\(P_{1}\),乙先出场比赛挑战成功的概率为\(P_{2}\),
则\(P_{1} = p^{n} + p^{n - 1}(1 - p)q + p^{n - 2}(1 - p)q^{2} + \cdots + (1 - p)q^{n}= \left( p^{n} + p^{n - 1}q + p^{n - 2}q^{2} + \cdots + q^{n} \right) - \left( p^{n}q + p^{n - 1}q^{2} + p^{n - 2}q^{3} + \cdots + pq^{n} \right);P_{2} = q^{n} + q^{n - 1}(1 - q)p + q^{n - 2}(1 - q)p^{2} + \cdots + (1 - q)p^{n}= \left( q^{n} + q^{n - 1}p + q^{n - 2}p^{2} + \cdots + p^{n} \right) - \left( q^{n}p + q^{n - 1}p^{2} + q^{n - 2}p^{3} + \cdots + qp^{n} \right)\)
由\(p^{n} + p^{n - 1}q + p^{n - 2}q^{2} + \cdots + q^{n} = q^{n} + q^{n - 1}p + q^{n - 2}p^{2} + \cdots + p^{n}\),
\(p^{n}q + p^{n - 1}q^{2} + p^{n - 2}q^{3} + \cdots + pq^{n} = q^{n}p + q^{n - 1}p^{2} + q^{n - 2}p^{3} + \cdots + qp^{n},\)
得\(P_{1} = P_{2}\)
因此,甲先出场与乙先出场比赛挑战成功的概率相同.}
\end{enumerate}
\end{question}
%
\begin{question}
当\(a = 0\)时,求证:\(\frac{f(x)}{x} > x + 1\);

\begin{enumerate}[label=(\arabic*)]
\item 若\(f(x) > 0\)对于\(x \in \left( 0,\text{π} \right)\)恒成立,求\(a\)的取值范围;
\item 若存在\(x_{1},x_{2} \in \left( 0,\text{π} \right)\),使得\(f\left( x_{1} \right) = f'\left( x_{2} \right) = 0\),求证:\(x_{1} < 2x_{2}\).
\topics{利用导数证明不等式;利用导数研究不等式恒成立问题;利用导数研究函数的零点;导数中的极值偏移问题}
\difficulty{0.4}
\answer{(1)证明见解析
(2)\(\lbrack - 1, + \infty)\)
(3)证明见解析}
\explain{(1)由\(a = 0\),得\(f(x) = x\text{e}^{x}\).
%
要证\(\frac{f(x)}{x} > x + 1\),只需证\(\text{e}^{x} - x - 1 > 0\).
%
令\(g(x) = \text{e}^{x} - x - 1\),则\(g'(x) = \text{e}^{x} - 1\).
%
当\(x \in ( - \infty,0)\)时,\(g'(x) < 0\),则\(g(x)\)单调递减,
%
当\(x \in (0, + \infty)\)时,\(g'(x) > 0\),则\(g(x)\)单调递增,
%
所以\(g(x) > g(0) = 0\),故\(\text{e}^{x} > x + 1\),
%
因此\(\frac{f(x)}{x} > x + 1\).
%
(2)\(f'(x) = (x + 1)\text{e}^{x} + a\text{cos}x,\)
%
令\(m(x) = f'(x)\),则\(m'(x) = (x + 2)\text{e}^{x} - a\text{sin}x\)
%
①当\(a \geq 0\)时,由\(x \in \left( 0,\text{π} \right)\),得\(x\text{e}^{x} > 0,a\text{sin}x \geq 0\),
%
因此\(f(x) > 0\),满足题意.
%
②当\(a < 0\)时,由\(x \in \left( 0,\text{π} \right)\),得\((x + 2)\text{e}^{x} > 0, - a\text{sin}x > 0\),
%
因此\(m'(x) > 0\),则\(f'(x)\)在\(\left( 0,\text{π} \right)\)上单调递增.
%
\(1^{\circ}\)若\(- 1 \leq a < 0\),则\(f'\left( x) > f'(0) = 1 + a \geq 0 \right.,
%
则\)f(x)\(在\)\left( 0,\text{π} \right)\(上单调递增,
%
所以\)f(x) > f(0) = 0\(,满足题意;\)2^{\circ}\(若\)a < - 1\(,则\)f'(0) < 0,f'\left( \frac{\text{π}}{2} \right) > 0\(,
%
因此\)f'(x)\(在\)\left( 0,\text{π} \right)\(存在唯一的零点\)x_{0}\(,且\)x_{0} \in \left( 0,\frac{\text{π}}{2} \right)\(,
%
当\)0 < x < x_{0}\(时,\)f'(x) < 0,f(x)\(单调递减,
%
当\)x_{0} < x < \text{π}\(时,\)f'(x) > 0,f(x)\(单调递增,
%
所以\)f\left( x_{0} \right) < f(0) = 0\(,不合题意.
%
综上,\)a\(的取值范围为\)\lbrack - 1, + \infty)\(.
%
(3)由(2)知\)a < - 1\(,设\)x_{0} = x_{2}\(,
%
则\)f(x)\(在\)\left( 0,x_{2} \right)\(上单调递减,在\)\left( x_{2},\text{π} \right)\(上单调递增,
%
注意到\)f(0) = 0,f\left( x_{2} \right)\left\langle f(0) = 0,f\left( \text{π} \right) = \text{πe}^{\text{π}} \right\rangle 0\(,
%
故\)f(x)\(在\)\left( 0,\text{π} \right)\(上存在唯一的零点\)x_{1},x_{1} \in \left( x_{2},\text{π} \right)\(.
%
注意到\)x_{1},2x_{2} \in \left( x_{2},\text{π} \right)\(,且\)f(x)\(在\)\left( x_{2},\text{π} \right)\(上单调递增.
%
要证明\)x_{1} < 2x_{2}\(,只需证\)f\left( x_{1} \right) < f\left( 2x_{2} \right)\(,
%
因为\)f\left( x_{1} \right) = 0\(,所以只需证\)f\left( 2x_{2} \right) > 0\(,
%
即证\)2x_{2}\text{e}^{2x_{2}} + a\text{sin}2x_{2} > 0\(.
%
因为\)\left( x_{2} + 1 \right)\text{e}^{x_{2}} + a\text{cos}x_{2} = 0\(,即\)a = - \frac{\left( x_{2} + 1 \right)\text{e}^{x_{2}}}{\text{cos}x_{2}}\(,
%
所以,只需证\)2x_{2}\text{e}^{2x_{2}} - \frac{\left( x_{2} + 1 \right)\text{e}^{x_{2}}}{\text{cos}x_{2}}\text{sin}2x_{2} > 0\(,
%
只需证\)x_{2}\text{e}^{x_{2}} - \left( x_{2} + 1 \right)\text{sin}x_{2} > 0,x_{2} \in \left( 0,\frac{\text{π}}{2} \right)\((\*)
%
由(1)得\)\text{e}^{x_{2}} > x_{2} + 1\(,
%
因此\)x_{2}\text{e}^{x_{2}} - \left( x_{2} + 1 \right)\text{sin}x_{2} > {x_{2}}^{2} + x_{2} - \left( x_{2} + 1 \right)\text{sin}x_{2} = \left( x_{2} + 1 \right)\left( x_{2} - \text{sin}x_{2} \right)\(,
%
设\)h(x) = x - \text{sin}x,0 < x < \frac{\text{π}}{2}\(,
%
则\)h'(x) = 1 - \text{cos}x > 0\(,所以\)h(x)\(在\)\left( 0,\frac{\text{π}}{2} \right)\(上单调递增,
%
所以\)h(x) > h(0) = 0\(,
%
从而\)h\left( x_{2} \right) > 0\(,即\)x_{2} - \text{sin}x_{2} > 0\(,因此(\*)得证,
%
从而\)x_{1} < 2x_{2}\(.\)}
\end{enumerate}
\end{question}
%
\begin{question}
已知圆\(C_{1}:x^{2} + y^{2} = 1\)为\(M_{1}\)的包络曲线,判断直线\(l:x\text{sin}\theta - y\text{cos}\theta = 1\)(\(\theta\)为常数,\(\theta \in R\))与集合\(M_{1}\)的关系;

\begin{enumerate}[label=(\arabic*)]
\item 已知\(M_{2}\)的包络曲线为\(C_{2}:x^{2} = 4y\),直线\(l_{1},l_{2} \in M_{2}\).设\(l_{1},l_{2}\)与\(C_{2}\)的公共点分别为\(P,Q\),记\(l_{1} \cap l_{2} = A,C_{2}\)的焦点为\(F\).
%
①证明:\(|FA|\)是\(|FP|\)、\(|FQ|\)的等比中项;
%
②若点\(A\)在圆\(x^{2} + {(y + 1)}^{2} = 1\)上,求\(\frac{|FA|}{|FP|}\)的最大值.
\topics{求在曲线上一点处的切线方程(斜率);判断直线与圆的位置关系;求直线与抛物线相交所得弦的弦长;圆锥曲线新定义}
\difficulty{0.15}
\answer{(1)\(l \in M_{1}\)
(2)①证明见解析;②\(\frac{1 + \sqrt{5}}{2}\)}
\explain{(1)圆心\(C_{1}(0,0)\)到\(l\)的距离\(d = \frac{1}{\sqrt{\text{sin}^{2}\theta + {( - \text{cos}\theta)}^{2}}} = 1\),
%
即直线\(l\)与圆\(C_{1}\)相切,所以\(l \in M_{1}\).
%
(2)解法一:①证明:由\(y = \frac{1}{4}x^{2}\),知\(F(0,1),C_{2}\)的准线方程为\(y = - 1\),\(y' = \frac{1}{2}x\).
%
设\(A(s,t),P\left( x_{1},y_{1} \right),Q\left( x_{2},y_{2} \right)\).
%
因为\(l_{1} \in M_{2}\),且\(l_{1}\)与\(C_{2}\)的公共点为\(P\),
%
所以\(l_{1}\)是曲线\(C_{2}\)在点\(P\)处的切线,
%
其方程为\(PA:y = \frac{1}{2}x_{1}\left( x - x_{1} \right) + y_{1}\),即\(y = \frac{1}{2}x_{1}x - y_{1}\),
%
则\(t = \frac{1}{2}x_{1}s - y_{1}\)(\*),
%
同理,\(QA:y = \frac{1}{2}x_{2}x - y_{2}\),则\(t = \frac{1}{2}x_{2}s - y_{2}\)(\*\*),
%
由(\*)(\*\*)得直线\(PQ\)的方程为\(t = \frac{1}{2}xs - y\),即\(y = \frac{1}{2}sx - t\).
%
由\(\left\{ \begin{array}{r}
x^{2} = 4y \\
y = \frac{1}{2}sx - t
\end{array} \right.,消去\)x\(整理得\)y^{2} + \left( 2t - s^{2} \right)y + t^{2} = 0\(,则\)y_{1} + y_{2} = s^{2} - 2t,y_{1}y_{2} = t^{2}\(.
%
又因为\)|FP| = y_{1} + 1,|FQ| = y_{2} + 1\(,
%
则\)|FP| \cdot |FQ| = \left( y_{1} + 1 \right)\left( y_{2} + 1 \right) = y_{1} + y_{2} + y_{1}y_{2} + 1 = s^{2} - 2t + t^{2} + 1 = s^{2} + {(t - 1)}^{2}\(.
%
又因为\)|FA|^{2} = s^{2} + {(t - 1)}^{2}\(,所以\)|FA|^{2} = |FP| \cdot |FQ|\(,
%
故\)|FA|\(是\)|FP|\(、\)|FQ|\(的等比中项.
%
②解:由①知,\)\frac{|FA|^{2}}{|FP|^{2}} = \frac{|FP| \cdot |FQ|}{|FP|^{2}} = \frac{|FQ|}{|FP|}\(,
%
则\)\frac{|FQ|}{|FP|} + \frac{|FP|}{|FQ|} = \frac{y_{2} + 1}{y_{1} + 1} + \frac{y_{1} + 1}{y_{2} + 1}= 2 + \frac{\left( y_{1} - y_{2} \right)^{2}}{\left( y_{1} + 1 \right)\left( y_{2} + 1 \right)} = 2 + \frac{s^{2}\left( s^{2} - 4t \right)}{s^{2} + {(t - 1)}^{2}}\(.
%
因为\)s^{2} + {(t + 1)}^{2} = 1\(,所以\)s^{2} + {(t - 1)}^{2} = 1 - 4t\(,
%
则\)\frac{|FQ|}{|FP|} + \frac{|FP|}{|FQ|} = 2 + \frac{s^{2}\left( s^{2} - 4t \right)}{1 - 4t}\(,
%
又因为\)- 2 \leq t < 0,s^{2} \leq 1\(,则\)\frac{s^{2}\left( s^{2} - 4t \right)}{1 - 4t} \leq \frac{s^{2}(1 - 4t)}{1 - 4t} = s^{2} \leq 1\(,
%
从而可得\)\frac{|FA|^{2}}{|FP|^{2}} + \frac{|FP|^{2}}{|FA|^{2}} \leq 3\(,解得\)\frac{\sqrt{5} - 1}{2} \leq \frac{|FA|}{|FP|} \leq \frac{\sqrt{5} + 1}{2}\(,
%
当\)A(1, - 1),P\left( 1 - \sqrt{5},\frac{3 - \sqrt{5}}{2} \right)\(时等号成立,
%
故\)\frac{|FA|}{|FP|}\(的最大值为\)\frac{1 + \sqrt{5}}{2}\(.
%
解法2:①证明:由题意知\)F(0,1),y = \frac{x^{2}}{4}\(,则\)y' = \frac{x}{2}\(.
%
设\)P\left( 2m,m^{2} \right),Q\left( 2n,n^{2} \right),m \neq n\(.
%
因为\)l_{1} \in M_{2}\(,且\)l_{1}\(与\)C_{2}\(的公共点为\)P\(,所以\)l_{1}\(是曲线\)C_{2}\(在点\)P\(处的切线,
%
所以\)PA:y - m^{2} = m(x - 2m)\(,即\)y = mx - m^{2},\((\*)
%
同理\)QA:y = nx - n^{2},\((\*\*)
%
联立(\*)(\*\*)得\)x = m + n,y = mn\(,即\)A(m + n,mn)\(,
%
所以\)|FA|^{2} = {(m + n)}^{2} + {(mn - 1)}^{2} = m^{2} + n^{2} + m^{2}n^{2} + 1 = \left( m^{2} + 1 \right)\left( n^{2} + 1 \right)\(,
%
注意到\)|FP| = m^{2} + 1,|FQ| = n^{2} + 1\(,因此\)|FA|^{2} = |FP| \cdot |FQ|\(,
%
所以\)|FA|\(是\)|FP|,|FQ|\(的等比中项.
%
②解:由①知,\)\frac{|FA|^{2}}{|FP|^{2}} = \frac{n^{2} + 1}{m^{2} + 1}\(,设\)t = \frac{n^{2} + 1}{m^{2} + 1}\(,
%
则\)t + \frac{1}{t} = \frac{n^{2} + 1}{m^{2} + 1} + \frac{m^{2} + 1}{n^{2} + 1}= \frac{m^{4} + n^{4} + 2\left( m^{2} + n^{2} \right) + 2}{m^{2}n^{2} + m^{2} + n^{2} + 1} = 2 + \frac{m^{4} + n^{4} - 2m^{2}n^{2}}{m^{2}n^{2} + m^{2} + n^{2} + 1}= 2 + \frac{{(m - n)}^{2}{(m + n)}^{2}}{{(mn + 1)}^{2} + {(m - n)}^{2}} \leq 2 + {(m + n)}^{2}\(.
%
因为点\)A(m + n,mn)\(在圆\)x^{2} + {(y + 1)}^{2} = 1\(上,
%
所以\){(m + n)}^{2} + {(mn + 1)}^{2} = 1\(,于是\){(m + n)}^{2} \leq 1\(,
%
从而\)t + \frac{1}{t} \leq 3\(,
%
解得\)\frac{3 - \sqrt{5}}{2} \leq t \leq \frac{3 + \sqrt{5}}{2}\(,即\)\frac{\sqrt{5} - 1}{2} \leq \frac{|FA|}{|FP|} \leq \frac{\sqrt{5} + 1}{2}\(.
%
又当\)A(1, - 1),P\left( 1 - \sqrt{5},\frac{3 - \sqrt{5}}{2} \right)\(时,\)\frac{|FA|}{|FP|} = \frac{\sqrt{5} + 1}{2}\(,
%
故\)\frac{|FA|}{|FP|}\(的最大值为\)\frac{1 + \sqrt{5}}{2}\(.
%

% IMAGE_TODO_START id=nanjing_yancheng_2025_mock1-Q19-img1 path=/Users/muryor/code/mynote/word\_to\_tex/output/figures/nanjing\_yancheng\_2025\_mock1/media/image7.png width=60% inline=true question_index=19 sub_index=1
\begin{tikzpicture}[scale=0.8,baseline=-0.5ex]
  % TODO: AI_AGENT_REPLACE_ME (id=nanjing_yancheng_2025_mock1-Q19-img1)
\end{tikzpicture}
% IMAGE_TODO_END id=nanjing_yancheng_2025_mock1-Q19-img1\)}
\end{enumerate}
\end{question}
