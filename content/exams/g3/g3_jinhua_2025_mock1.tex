\examxtitle{浙江省金华十校 2025-2026 学年高三上学期一模考试数学试题}

\section{单选题}

\begin{question}
已知集合 \(U=\{1,2,3,4,5,6,7,8\}\) , \(A=\{2,3,4\}\) ,则集合 \(\complement_{U}A=\)()
\begin{choices}
  \item \(\{1\}\)
  \item \(\{2,3,4\}\)
  \item \(\{5,6,7,8\}\)
  \item \(\{1,5,6,7,8\}\)
\end{choices}
\topics{补集的概念及运算}
\difficulty{0.94}
\answer{D}
\explain{由\(U=\{1,2,3,4,5,6,7,8\}\),\(A=\{2,3,4\}\),则\(\complement_{U}A=\{1,5,6,7,8\}\).}
\end{question}

\begin{question}
已知等差数列\(\{a_{n}\}\)满足\(a_{1}=2\),\(a_{4}+a_{6}=20\),则\(a_{3}=\)()
\begin{choices}
  \item 4
  \item 6
  \item 8
  \item 10
\end{choices}
\topics{等差中项的应用}
\difficulty{0.85}
\answer{B}
\explain{由题设\(a_{4}+a_{6}=2a_{5}=20\Rightarrow a_{5}=10\),而\(a_{1}=2\), 所以\(a_{1}+a_{5}=2a_{3}=12\Rightarrow a_{3}=6\)。}
\end{question}

\begin{question}
\(\frac{5}{2+\mathrm{i}}=\) ( )
\begin{choices}
  \item \(2+\mathrm{i}\)
  \item \(2-\mathrm{i}\)
  \item \(-2+\mathrm{i}\)
  \item \(-2-\mathrm{i}\)
\end{choices}
\topics{复数代数形式的乘法运算;复数的除法运算}
\difficulty{0.85}
\answer{B}
\explain{\(\frac{5}{2+\mathrm{i}}=\frac{(2+\mathrm{i})(2-\mathrm{i})}{2+\mathrm{i}}=2-\mathrm{i}\).}
\end{question}

\begin{question}
已知\(\frac{1}{\log_{9}a}+\frac{1}{\log_{27}a}=\frac{5}{3}\),则\(a=\)()
\begin{choices}
  \item 3
  \item 9
  \item 27
  \item 81。
\end{choices}
\topics{指数幂的运算;指数式与对数式的互化;对数的运算;运用换底公式化简计算}
\difficulty{0.65}
\answer{C}
\explain{\(\frac{1}{\log_{9}a}+\frac{1}{\log_{27}a}=\log_{a}9+\log_{a}27=\log_{a}3^{5}=\frac{5}{3}\),所以\(a^{5}_{3}=3^{5}\),则\(a^{5}=(3^{5})^{3}=27^{5}\),解得\(a=27\)。}
\end{question}

\begin{question}
已知随机变量 \(X\sim N\left(2, \sigma^{2}\right)\) ,且 \(P\left(X<0\right)=0.3\) ,则 \(P\left(0<X<4\right)\) 的值为( )
\begin{choices}
  \item 0.2
  \item 0.4
  \item 0.7
  \item 0.35
\end{choices}
\topics{指定区间的概率}
\difficulty{0.65}
\answer{B}
\explain{由题设\(P(X<2)=0.5\),且\(P(X<0)=0.3\),则\(P(0<X<2)=0.2\),由正态分布曲线关于\(X=2\)对称,则\(P\big(0<X<4\big)=0.4\).}
\end{question}

\begin{question}
若圆\(C:(x-1)^{2}+(y+3)^{2}=1\)上存在两点\(A,B\),直线\(l:3x-4y+m=0\)上存在点\(P\),使得 \(\angle APB=60^\circ\) ,则实数\(m\)的取值范围为( )
\begin{choices}
  \item \([-25,-5]\)
  \item \((-\infty,-25)\cup[-5,+\infty)\)
  \item \([-35,5]\)
  \item \((-\infty,-35)\cup[5,+\infty)\)
\end{choices}
\begin{center}
% IMAGE_TODO: images/a99d3de1bdc13ded9e2189e89156e2e89443e5a247a242d5ba63439a49ef1a72_22.jpg (width=22%)
\begin{tikzpicture}[scale=1.05,>=Stealth,line cap=round,line join=round]
  % TODO: draw this figure in TikZ according to the original image.
\end{tikzpicture}
\end{center}
\begin{center}
% IMAGE_TODO: images/b0603473b53691a861bacdebfc070cb6d5c5a66ef1d6ad7bb7e674bbd1a5d041_22.jpg (width=22%)
\begin{tikzpicture}[scale=1.05,>=Stealth,line cap=round,line join=round]
  % TODO: draw this figure in TikZ according to the original image.
\end{tikzpicture}
\end{center}
\topics{求点到直线的距离;由直线与圆的位置关系求参数}
\difficulty{0.15}
\answer{A}
\explain{当直线与圆相交时,如图所示,若 \(A\)、\(B\) 离直线越近时,直至与直线和圆 \(C\) 的两交点重合,此时 \(\angle APB=\pi\),
若 \(A\)、\(B\) 相距越来越近时,直至 \(A\)、\(B\) 两点重合,此时 \(\angle APB=0^\circ\),所以一定存在 \(A\)、\(B\) 及 \(P\),使得 \(\angle APB=60^\circ\);
当直线与圆相切时,同直线与圆相交分析可知,一定存在 \(A\)、\(B\) 及 \(P\),使得 \(\angle APB=60^\circ\);
当直线与圆没有公共点时,对直线上的任一点 \(P\),若 \(A\)、\(B\) 相距越来越近时,直至 \(A\)、\(B\) 两点重合时,仍有 \(\angle APB=0^\circ\),
另一方面,若 \(PB\) 与圆 \(C\) 相切于 \(B\),\(PA\) 与圆 \(C\) 相切于 \(A\),此时 \(\angle APB\) 必为该 \(P\) 点所能达到的最大情况,如图所示,
由图可知\(\sin\angle CPA=\frac{r}{CP}\),\(\angle APB=2\angle CPA\),\(CP\)最短时,即等于圆心\(C\)到直线的距离\(d\),\(\sin\angle CPA\)最大,\(\angle CPA\)也最大,同时\(\angle APB\)最大,
所以若圆\(C\)上存在两点\(A,B\),直线\(l\)上存在点\(P\),使得\(\angle APB=60^\circ=\frac{\pi}{3}\),则必有\(\frac{r}{d}\geq\sin\frac{\pi}{6}=\frac{1}{2}\),解得\(d\leq2r\),
又因为圆\(C\)的半径\(r=1\),圆心\(C(1,-3)\)到直线\(3x-4y+m=0\)的距离\(d=\frac{|3\times1-4\times(-3)+m|}{\sqrt{3^{2}+(-4)^{2}}}=\frac{|m+15|}{5}\),
所以\(\frac{|m+15|}{5}\leq2\),解得\(-25\leq m\leq-5\)。}
\end{question}

\begin{question}
设\(\theta\)为两个非零向量\(\vec{a},\vec{b}\)所成的角,已知对任意\(t\in\mathbb{R}\),\(|\vec{a}-t\vec{b}|\)的最小值为\(\frac{1}{2}|\vec{a}|\),则\(\theta=\)()
\begin{choices}
  \item \(\frac{\pi}{6}\)
  \item \(\frac{\pi}{3}\)
  \item \(\frac{\pi}{6}\) 或 \(\frac{5\pi}{6}\)
  \item \(\frac{\pi}{3}\) 或 \(\frac{2\pi}{3}\)
\end{choices}
\begin{center}
% IMAGE_TODO: images/57749a757075489f16c159828273a5843679d59e072009e4c79f686c857cd060_18.jpg (width=18%)
\begin{tikzpicture}[scale=1.05,>=Stealth,line cap=round,line join=round]
  % TODO: draw this figure in TikZ according to the original image.
\end{tikzpicture}
\end{center}
\topics{向量减法法则的几何应用;向量与几何最值}
\difficulty{0.4}
\answer{C}
\explain{令\(\vec{a}=\overline{OA}\),\(\vec{b}=\overline{OB}\),\(t\vec{b}=\overline{OC}\),如下图示,\(|\vec{a}-t\vec{b}|\)即为线段AC的长度,由对任意\(t\in\mathbb{R}\),\(|\vec{a}-t\vec{b}|\)的最小值为\(\frac{1}{2}|\vec{a}|\),即\(|AC|_{\min}=\frac{1}{2}|\vec{a}|\),而\(\angle AOB=\theta\),显然\(AC\perp OB\)时,线段\(AC\)最短,此时\(|AC|_{\min}=|\overline{OA}|\sin\theta=|\vec{a}|\sin\theta=\frac{1}{2}|\vec{a}|\),所以\(\sin\theta=\frac{1}{2}\),又\(\theta\in[0,\pi]\),故\(\theta=\frac{\pi}{6}\)或\(\frac{5\pi}{6}\)。}
\end{question}

\begin{question}
若双曲线\(\frac{y^{2}}{a^{2}}-\frac{x^{2}}{b^{2}}=1(a>0,b>0)\)不存在以点\((a,2a)\)为中点的弦,则该双曲线离心率的取值范围为()
\begin{choices}
  \item \(\left(1,\frac{2\sqrt{3}}{3}\right]\)
  \item \(\left(1,\frac{\sqrt{5}}{2}\right]\)
  \item \(\left[\frac{\sqrt{5}}{2},\frac{2\sqrt{3}}{3}\right]\)
  \item \(\left[\frac{\sqrt{5}}{2},+\infty\right)\)      直线斜率 \(|k| \leq \frac{a}{b}\) 可得另一不等式,最后求解出 \(\frac{b}{a}\) 的范围,结合离心率等式即可求解.
\end{choices}
\topics{求双曲线的离心率或离心率的取值范围}
\difficulty{0.4}
\answer{C}
\explain{由题意得点\((a,2a)\)在双曲线外部或在双曲线上,则\(\frac{(2a)^2}{a^2}-\frac{a^2}{b^2}\leq1\),得\(\frac{b^2}{a^2}\leq\frac{1}{3}\), 假设存在以\((a,2a)\)为中点的弦,设弦与双曲线交于点\(A(x_1,y_1)\),\(B(x_2,y_2)\), 则\(\frac{x_1+x_2}{2}=a\), \(\frac{y_1+y_2}{2}=2a\), 由点\(A(x_1,y_1)\),\(B(x_2,y_2)\)在双曲线上,得\(\left(\frac{y_1^2}{a^2}-\frac{x_1^2}{b^2}=1 \right)\) , 由点\(A(x_1,y_1)\),\(B(x_2,y_2)\)在双曲线上,得 \(\left(\frac{y_2^2}{a^2}-\frac{x_2^2}{b^2}=1 \right)\) 两式作差得\(\frac{(y_1+y_2)(y_1-y_2)}{a^2}=\frac{(x_1+x_2)(x_1-x_2)}{b^2} \), 所以\(k_{AB}=\frac{y_1-y_2}{x_1-x_2}=\frac{a^2(x_1+x_2)}{b^2(y_1+y_2)}=\frac{a^2\cdot2a}{b^2\cdot4a}=\frac{a^2}{2b^2}\), 因为不存在该中点弦,所以直线\(AB\)与双曲线至多一个交点, 则\(k_{AB}=\frac{a^2}{2b^2}\leq\frac{a}{b}\),也即\(\frac{b}{a}\geq\frac{1}{2}\), 所以\(\frac{1}{4}\leq\frac{b^2}{a^2}\leq\frac{1}{3}\),则\(e=\frac{c}{a}=\sqrt{1+\frac{b^2}{a^2}}\in\left[\frac{\sqrt{5}}{2},\frac{2\sqrt{3}}{3}\right]\)。}
\end{question}

\section{多选题}

\begin{question}
已知圆锥的侧面展开图是半径等于2的半圆,则圆锥的()
\begin{choices}
  \item 底面半径为1
  \item 表面积为\(2\pi\)
  \item 体积为\(\frac{\sqrt{3}}{3}\pi\)
  \item 外接球与内切球半径比值为3       若底面半径为\(r\),则\(2\pi r=2\pi\Rightarrow r=1\),A对, 表面积为 \(\frac{1}{2}\pi\times2^{2}+\pi\times1^{2}=3\pi\) ,B错, 由上,圆锥的高 \(h=\sqrt{2^{2}-1^{2}}=\sqrt{3}\) ,则圆锥体积为 \(\frac{1}{3}h\pi r^{2}=\frac{1}{3}\times\sqrt{3}\pi=\frac{\sqrt{3}}{3}\pi\) ,C对, 由上,圆锥轴截面是边长为2的等边三角形,其外接圆和内切圆半径,分别为圆锥的外接球 和内切球半径, 所以圆锥的外接球半径为 \(\frac{2}{3}\times2\times\sin60^\circ=\frac{2}{\sqrt{3}}\) ,内切球半径为 \(\frac{1}{3}\times2\times\sin60^\circ=\frac{1}{\sqrt{3}}\) , 所以外接球与内切球半径比值为2,D错.
\end{choices}
\topics{圆锥中截面的有关计算;圆锥表面积的有关计算;锥体体积的有关计算;多面体与球体内切外接问题}
\difficulty{0.65}
\answer{AC}
\explain{由题意,圆锥的母线长为2,底面周长为\(2\pi\),}
\end{question}

\begin{question}
已知函数 \(f(x)=x^{2}(x-a)\) 在x=2处取得极小值,y=f'(x)为其导函数,则()
\begin{choices}
  \item \(a=3\)
  \item \(f'(\sqrt{3}+1)-f'(1-\sqrt{2})<0\)
  \item \(f(x)\geq-4\) 的解集为\(\left[-1,+\infty\right)\)
  \item \(\forall x>0,f\left(x+\frac{1}{x}\right)>f(-x-1)\)       因式分解得\(x^{3}-3x^{2}+4=(x+1)(x-2)^{2}\geq0\),解得\(x\geq-1\),故解集为\([-1,+\infty)\),故C正确; 对于D,对于\(\forall x>0\),有\(x+\frac{1}{x}\geq2\sqrt{x\cdot\frac{1}{x}}=2\),当且仅当\(x=1\)时取等号,同时 \(-x-1<-1\) , 由于\(f'(x)=3x^{2}-6x=3x(x-2)\),当\(x<0\)或\(x>2\)时, \(f'(x)>0\), \(f(x)\)单调递增; 当\(0<x<2\)时, \(f'(x)<0\), \(f(x)\)单调递减, 所以\(f\left(x+\frac{1}{x}\right)\geq f(2)=-4\), \(f(-x-1)<f(-1)=-4\),所以\(\forall x>0\), \(f\left(x+\frac{1}{x}\right)>f(-x-1)\), \(\forall x\)D正确。
\end{choices}
\topics{利用导数研究不等式恒成立问题;根据极值点求参数}
\difficulty{0.4}
\answer{ACD}
\explain{对于A, \(f'(x)=3x^{2}-2ax\) ,由题意可知 \(f'(2)=0\) ,解得a=3,此时 \(f(x)=x^{2}(x-3)\) ,故A正确; 对于B,由 \(f'(x)=3x^{2}-6x\) ,其为二次函数,开口向上,对称轴为x=\(\frac{6}{2\cdot3}=1\) , 则\(\sqrt{3}+1\)到对称轴的距离为 \(\left|\sqrt{3}+1-1\right|=\sqrt{3}\) , \(1-\sqrt{2}\)到对称轴的距离为 \(\left|1-\sqrt{2}-1\right|=\sqrt{2}<\sqrt{3}\) , 结合开口向上的二次函数图像特点可知,离对称轴较远的点函数值更大,也即 \(f'(\sqrt{3}+1)>f'(1-\sqrt{2})\) ,即 \(f'(\sqrt{3}+1)-f'(1-\sqrt{2})>0\) ,故B错误; 对于C,解不等式 \(f(x)\geq-4\) ,即 \(x^{3}-3x^{2}\geq-4\) ,整理为 \(x^{3}-3x^{2}+4\geq0\) ,}
\end{question}

\begin{question}
在\(\triangle ABC\)中,若\(A=\cos A\),\(B=\cos(\cos B)\),\(C=k\tan(\sin C)\),则(\quad)
\begin{choices}
  \item \(A=B\)
  \item \(B<C\)
  \item \(C<\frac{\pi}{2}\)
  \item \(k<2\)
\end{choices}
\begin{center}
% IMAGE_TODO: images/798d59a290c38ea32ad02467c98eb2c6849a9a1458dba0a4c3853218967d227d_20.jpg (width=20%)
\begin{tikzpicture}[scale=1.05,>=Stealth,line cap=round,line join=round]
  % TODO: draw this figure in TikZ according to the original image.
\end{tikzpicture}
\end{center}
\topics{用导数判断或证明已知函数的单调性;余弦函数图象的应用;由不等式的性质比较数(式)大小;求函数零点或方程根的个数}
\difficulty{0.15}
\answer{ABD}
\explain{根据 \(y=x\) 与 \(y=\cos x\) 在 \((0,\pi)\) 上的图象可知选项ABD正确。}
\end{question}

\clearpage
\section{填空题}

\begin{question}
\((1+x)^{5}\)的展开式中\(x^{3}\)项的系数为______.  
\topics{求指定项的系数}
\difficulty{0.94}
\answer{10}
\explain{\((1+x)^5\)的展开式中\(x^3\)项的系数为\(C_5^3=10\).}
\end{question}

\begin{question}
\(\triangle ABC\) 的三个内角\(A,B,C\) 的对边分别为\(a,b,c\),满足\(C=\frac{\pi}{4}\),且\(a^{2}+b^{2}-c^{2}=4\),则\(\triangle ABC\) 的面积为____.
\topics{三角形面积公式及其应用;余弦定理解三角形}
\difficulty{0.85}
\answer{1}
\explain{由余弦定理可得:\(c^{2}=a^{2}+b^{2}-2ab\cos\frac{\pi}{4}\),又\(a^{2}+b^{2}-c^{2}=4\),得\(c^{2}=c^{2}+4-\sqrt{2}ab\),解得\(ab=2\sqrt{2}\),所以\(\triangle ABC\)的面积为\(\frac{1}{2}ab\sin C=1\);故答案为:1}
\end{question}

\begin{question}
平面直角坐标系中,原点\(O\)处有一只蚂蚁,每过1秒,该蚂蚁都会随机地选择上、下、左、右四个方向之一移动一个单位长度,那么在6秒后,蚂蚁到原点\(O\)的距离等于\(\sqrt{2}\)的概率为____.  
\topics{实际问题中的组合计数问题;计算古典概型问题的概率}
\difficulty{0.15}
\answer{$\frac{75}{256}$}
\explain{总路径数\(4^6\),满足\(x^2+y^2=2\)即\(|x|=|y|=1\)的路径数为1200,概率为\(\frac{75}{256}\)。}
\end{question}

\section{解答题}

\begin{question}
如图, 长方体 \(ABCD-A_{1}B_{1}C_{1}D_{1}\) 中, \(AB=BC=2\), \(AA_{1}=3\), \(E\), \(F\) 三等分 \(CC_{1}\).  


(1)求证:\(D_{1}E\perp AF\);  

(2)求直线\(D_{1}E\)与平面\(AA_{1}C_{1}C\)所成角的大小.  


(2) \(30^\circ\)  




(2) 求解出平面 \(AA_{1}C_{1}C\) 的法向量, 再根据空间线面角公式求解即可.  


如图, 以 \(A_{1}\) 为坐标原点, 以 \(A_{1}B_{1}\), \(A_{1}D_{1}\), \(A_{1}A\) 分别为 \(x\), \(y\), \(z\) 轴正方向建立空间直角坐标系,

因为 \(AB=BC=2\), \(AA_{1}=3\), \(E\), \(F\) 为 \(CC_{1}\) 的三等分点,

得各点坐标 \(B_{1}\left(2,0,0\right)\), \(D_{1}\left(0,2,0\right)\), \(A\left(0,0,3\right)\), \(E\left(2,2,2\right)\), \(F\left(2,2,1\right)\),  


则 \(\overline{D_{1}E}=(2,0,2)\) , \(\overline{AF}=(2,2,-2)\) ,所以 \(\overline{D_{1}E}\cdot\overline{AF}=0\) ,即 \(D_{1}E\perp AF\).  

(2) 因为 \(AA_{1} \perp\) 平面 \(A_{1}B_{1}C_{1}D_{1}\), \(B_{1}D_{1} \subset\) 平面 \(A_{1}B_{1}C_{1}D_{1}\), 所以 \(B_{1}D_{1} \perp AA_{1}\), 因为 \(B_{1}D_{1} \perp A_{1}C_{1}\), \(AC / / A_{1}C_{1}\), 所以 \(B_{1}D_{1} \perp AC\),

又因为 \(AC \cap AA_{1}=A\), 所以 \(B_{1}D_{1} \perp\) 平面 \(AA_{1}C_{1}C\), 所以 \(\overline{B_{1}D_{1}}=(-2,2,0)\) 为平面 \(AA_{1}C_{1}C\) 的法向量, \(\overline{D_{1}E}=(2,0,2)\),  

设直线与平面所成角为\(\theta\),则\(\sin\theta=\frac{|\overline{B_{i}D_{1}}\cdot\overline{D_{1}E}|}{|\overline{B_{i}D}||\overline{D_{1}E}|}=\frac{-|4|}{2\sqrt{2}\times2\sqrt{2}}=\frac{1}{2}\),因为直线与平面所成角的范围为\(\left[0,\frac{\pi}{2}\right]\),

所以直线\(D_{1}E\)与平面\(AA_{1}C_{1}C\)所成角为30°。
\begin{center}
% IMAGE_TODO: images/47c06d92905fda7f06b480c21fe1303f613a5241885fa8969b429f3332f35d9c_17.jpg (width=17%)
\begin{tikzpicture}[scale=1.05,>=Stealth,line cap=round,line join=round]
  % TODO: draw this figure in TikZ according to the original image.
\end{tikzpicture}
\end{center}
\begin{center}
% IMAGE_TODO: images/b4ef0aaf22af206324f32508e33580e6516890b629cb7bc290249857569a9dc4_19.jpg (width=19%)
\begin{tikzpicture}[scale=1.05,>=Stealth,line cap=round,line join=round]
  % TODO: draw this figure in TikZ according to the original image.
\end{tikzpicture}
\end{center}
\topics{求空间图形上的点的坐标;空间位置关系的向量证明;线面角的向量求法}
\difficulty{0.65}
\answer{(1)证明见解析}
\explain{(1) 根据题意, 六面体 \(ABCD-A_{1}B_{1}C_{1}D_{1}\) 为长方体, 所以 \(AA_{1} \perp A_{1}D_{1}\), \(AA_{1} \perp A_{1}B_{1}\), \(A_{1}D_{1} \perp A_{1}B_{1}\),}
\end{question}

\begin{question}
近些年汽车市场发生了翻天覆地的变化,新能源汽车发展迅速,下表统计了2021 年到 2024 年某地新能源汽车销量(单位:千辆)  

| 年份 | 2021 | 2022 | 2023 | 2024 |
| --- | --- | --- | --- | --- |
| 年份代号x | 1 | 2 | 3 | 4 |
| 销量y | 33 | 69 | 93 | 129 |  

附:相关系数 \(r=\frac{\sum_{i=1}^{n}\left(x_{i}-\vec{x}\right)\left(y_{i}-\vec{y}\right)}{\sqrt{\sum_{i=1}^{n}\left(x_{i}-\vec{x}\right)^{2}}\sqrt{\sum_{i=1}^{n}\left(y_{i}-\vec{y}\right)^{2}}} }\) ; 回归方程 \(\hat{y}=\hat{b}x+\hat{a}\) 中斜率和截距的最小二乘法估计公式分别为 \(\hat{b}=\frac{\sum_{i=1}^{n}\left(x_{i}-\vec{x}\right)\left(y_{i}-\vec{y}\right)}{\sum_{i=1}^{n}\left(x_{i}-\vec{x}\right)^{2}}\) , \(\hat{a}=\vec{y}-\hat{b}\vec{x}\) , \(\sum_{i=1}^{4}x_{i}y_{i}=966\) , \(\sum_{i=1}^{4}\left(y_{i}-\vec{y}\right)^{2}=4896\) ,\(\sqrt{170}\approx13.04\) .  

(1)试根据样本相关系数\(r\)的值判断该地汽车销量\(y\)与年份代号\(x\)的线性相关性强弱

\((0.75\leq|r|\leq1)\), 则认为\(y\)与\(x\)的线性相关性较强, \(|r|<0.75\), 则认为\(y\)与\(x\)的线性相关性较弱); (精确到0.001)  

(2)建立 \(y\) 关于 \(x\) 的线性回归方程,并预测该地 2025 年的新能源汽车销量.  


(2)\(\hat{y}=31.2x+3\),159(千辆)  




(2)根据(1)算出的结果进一步算出\(\hat{b}\),再根据线性回归方程经过\((\bar{x},\bar{y})\)计算\(\hat{a}\),最后把x=5  

代入回归直线方程即可求解.
\topics{相关系数的计算;根据回归方程进行数据估计}
\difficulty{0.85}
\answer{(1)$y$与$x$具有较强的线性相关关系}
\explain{(1) 已知 \(n=4\), \(x_{1}=1, x_{2}=2, x_{3}=3, x_{4}=4\),则 \(\vec{x}=\frac{1+2+3+4}{4}=2.5\), \(y_{1}=33, y_{2}=69, y_{3}=93, y_{4}=129\),则 \(\vec{y}=\frac{33+69+93+129}{4}=81\), \(\sum_{i=1}^{4} x_{i}^{2}=1^{2}+2^{2}+3^{2}+4^{2}=1+4+9+16=30\), \(4 \vec{x}^{2}=4 \times 2.5^{2}=25\),所以 \(\sum_{i=1}^{4} \left(x_{i}-\vec{x}\right)^{2}=\sum_{i=1}^{4} x_{i}^{2}-4 \vec{x}^{2}=30-25=5\), \(\text { 已知 } \sum_{i=1}^{4} x_{i} y_{i}=966\) ,故 \(\sum_{i=1}^{4} \left(x_{i}-\vec{x}\right)\left(y_{i}-\vec{y}\right)=\sum_{i=1}^{4} x_{i} y_{i}-4 \vec{x} \cdot \vec{y}=\sum_{i=1}^{4} x_{i} y_{i}-4 \vec{x} \cdot \vec{y}=966-4 \times 2.5 \times 81=156\) , 又 \(\sum_{i=1}^{4} \left(y_{i}-\vec{y}\right)^{2}=4896\) ,代入相关系数公式, 可得 \(r=\frac{\sum_{i=1}^{n}\left(x_{i}-\vec{x}\right)\left(y_{i}-\vec{y}\right)}{\sqrt{\sum_{i=1}^{n}\left(x_{i}-\vec{x}\right)^{2}} \sqrt{\sum_{i=1}^{n}\left(y_{i}-\vec{y}\right)^{2}}=\frac{156}{\sqrt{5} \times 4896}=\frac{156}{12 \sqrt{170}} \approx \frac{13}{13.04} \approx 0.997\) , 因为 \(|r|=0.997 \geq 0.75\) ,所以 \(y\) 与 \(x\) 具有较强的线性相关关系. (2) 根据 \(\hat{b}=\frac{\sum_{i=1}^{4}\left(x_{i}-\vec{x}\right)\left(y_{i}-\vec{y}\right)}{\sum_{i=1}^{4}\left(x_{i}-\vec{x}\right)^{2}}\), \(\hat{a}=\vec{y}-\hat{b} \vec{x}\) , 由(1)可知 \(\sum_{i=1}^{4}\left(x_{i}-\vec{x}\right)\left(y_{i}-\vec{y}\right)=156\) , \(\sum_{i=1}^{4}\left(x_{i}-\vec{x}\right)^{2}=5\) , 所以 \(\hat{b}=\frac{156}{5}=31.2\) , 由 \(\hat{a}=\vec{y}-\hat{b} \vec{x}\) , 已知 \(\vec{x}=2.5\) , \(\vec{y}=81\) , \(\hat{b}=31.2\) ,则 \(\hat{a}=81-31.2 \times 2.5=81-78=3\) , 所以 \(y\) 关于 \(x\) 的线性回归方程为 \(\hat{y}=31.2 x+3\) ,将 \(x=5\) 代入线性回归方程 \(\hat{y}=31.2 \times 5+3=159\) (千辆).}
\end{question}

\begin{question}
已知数列\(\{a_{n}\}\),\(\{b_{n}\}\)满足\(\begin{cases}a_{n+1}=\frac{1}{2}a_{n}+\frac{3}{2}b_{n}\\b_{n+1}=\frac{1}{2}b_{n}+\frac{3}{2}a_{n}\end{cases}\)(\(n\in\mathbb{N}_{+}\)),且\(b_{1}=3a_{1}=\frac{3}{2}\).  

(1)证明:数列\(\{a_{n}+b_{n}\}\)与\(\{a_{n}-b_{n}\}\)均为等比数列;  

(2)求数列\(\left[\left\{a_{n}\right\}\right]\)的前25项和\(S_{25}\).(其中\(x\)表示不超过\(x\)的最大整数,如\([1.2]=1\))  


(2)\(2^{25}-14\).  




(2) 由 (1) 得 \(a_{n}+b_{n}=2^{n}\), \(a_{n}-b_{n}=(-1)^{n}\), 进而有 \(a_{n}=\frac{2^{n}+(-1)^{n}}{2}\), 根据新定义及分组求和、等比数列前 \(n\) 项和公式求 \(S_{25}\).  


(2) 由 (1) 知 \(a_{n}+b_{n}=2^{n}\), \(a_{n}-b_{n}=(-1)^{n}\), 所以 \(a_{n}=\frac{2^{n}+(-1)^{n}}{2}\), 则 \(\left[a_{2 n}\right]=\left[\frac{2^{2 n}+1}{2}\right]=2^{2 n-1}\), \(\left[a_{2 n-1}\right]=\left[\frac{2^{2 n-1}-1}{2}\right]=2^{2 n-2}-1\),

\(S_{25}=\left(2^{0}-1+2^{1}\right)+\left(2^{2}-1+2^{3}\right)+...+\left(2^{22}-1+2^{23}\right)+2^{24}-1=\frac{(1-2^{25})}{1-2}-13=2^{25}-14\).
\topics{由递推关系证明等比数列;求等比数列前n项和;分组(并项)法求和}
\difficulty{0.65}
\answer{(1)证明见解析;}
\explain{(1) 由 \(\left\{\begin{array}{l}a_{n+1}=\frac{1}{2}a_{n}+\frac{3}{2}b_{n} \\\ b_{n+1}=\frac{1}{2}b_{n}+\frac{3}{2}a_{n}\end{array}\right.\), 可得 \(\left\{\begin{array}{l}a_{n+1}+b_{n+1}=2\left(a_{n}+b_{n}\right) \\ a_{n+1}-b_{n+1}=-\left(a_{n}-b_{n}\right)\end{array}\right.\), 又 \(a_{1}+b_{1}=2\neq0, a_{1}-b_{1}=-1\neq0\) , 所以 \(\left\{a_{n}+b_{n}\right\}\) 与 \(\left\{a_{n}-b_{n}\right\}\) 均为等比数列;}
\end{question}

\begin{question}
已知函数 \(f(x)=e^{x}-x-1\).  

(1)求 \(y=f(x)\) 在 \(x=0\) 处的切线方程;

(2)若 \(f(\ln x)\geq kx-x\ln x-1\) 恒成立, 求实数 \(k\) 的取值范围;

(3)当 \(a\geq1\) 时, 讨论 \(g(x)=f(x)-ax\cos x\) 在区间 \(\left(-\pi,\frac{\pi}{2}\right)\) 上零点的个数.  


(2) \((-\infty,1]\);

(3) 3 个.  




(2)法一:应用分离参数法有\(k\leq1+\ln x-\frac{\ln x}{x}\),再应用导数研究右侧的单调性求最小值,即  

可得参数范围;法二:应用必要性探路,问题化为\(h(x)=(x-1)\ln x+(1-k)x\geq0\),令 \(h(1)\geq0\Rightarrow k\leq1\),再证明 \(k\leq1\), \(x>0\) 时,\(h(x)\geq0\) 恒成立,确保充分性成立,即可得:  

(3) 由题设得 \(x=0\) 是函数 \(f(x)\) 的一个零点, 讨论 \(x \in \left(-\frac{\pi}{2}, 0\right)\) 、 \(x \in \left(0, \frac{\pi}{2}\right)\) 、 \(x \in \left(-\pi, -\frac{\pi}{2}\right)\), 并利用导数研究函数的零点个数, 即可得.  


(2) \(f(x)=\mathrm{e}^{x}-x-1\),此时 \(f(\ln x)\geq kx-x\ln x-1\Leftrightarrow x-\ln x-1\geq kx-x\ln x-1\),  

(2) \(f(x)-e^{-x-1}\), 此时 \(f(\ln x)\leq kx-\ln x-1\) , 此时 \(f(\ln x)\leq kx-\ln x-1\leq kx-\ln x-1\)

法一: 分离参数法, 从而 \(kx\leq (x-1)\ln x+x\Rightarrow k\leq 1+\ln x-\frac{\ln x}{x}\) , 令 \(h(x)=1+\ln x-\frac{\ln x}{x}\) , 则 \(h'(x)=\frac{1}{x}-\frac{1-\ln x}{x^2}=\frac{x+\ln x-1}{x^2}\) , 所以 \(h'(x)>0\Rightarrow x>1\), \(h'(x)<0\Rightarrow 0<x<1\) , 所以 \(h(x)\) 在 \((0,1)\) 单调递减, 在 \((1,+\infty)\) 单调递增, 因此 \(h(x)_{\min}=h(1)=1\), 故 \(k\) 的取值范围为 \((-\infty,1]\) ;

法二: 必要性探路, \(x-\ln x-1\geq kx-x\ln x-1\Leftrightarrow (x-1)\ln x+(1-k)x\geq 0\) , 令 \(h(x)=(x-1)\ln x+(1-k)x\), \(h(1)=1-k\geq 0\Rightarrow k\leq 1\) , 下证: \(k\leq 1\), \(x>0\) 时, \(h(x)\geq 0\) 恒成立, 由一次函数 \(m(k)=(x-1)\ln x+x-kx\) 在 \((-\infty,1]\) 上递减, 则 \(m(k)\geq m(1)\Rightarrow (x-1)\ln x+x-kx\geq (x-1)\ln x\) , 在 \(x\in (0,1)\) 和 \(x\in (1,+\infty)\) 上 \((x-1)\ln x>0\) 恒成立, 且 \(x=1\) 时 \((x-1)\ln x=0\) , 所以 \(g(x)\geq 0\) 恒成立, 故 \(k\) 的取值范围为 \((-\infty,1]\) ;  

(3) \(g(x)\)在区间\(\left(-\pi,\frac{\pi}{2}\right)\)上有3个零点,理由如下:

由于\(f(0)=0\),所以\(x=0\)是函数\(f(x)\)的一个零点,\(g'(x)=e^{x}+a\left(x\sin x-\cos x\right)-1\),当\(x\in\left(-\frac{\pi}{2},0\right)\)时,此时\(-ax\cos x>0\)恒成立,又由(1)知\(e^{x}-x-1>0\)恒成立,  

从而 \(g(x)>0\) 恒成立, 所以 \(g(x)\) 在区间 \(x\left(-\frac{\pi}{2}, 0\right)\) 上没有零点;

当 \(x \in\left(0, \frac{\pi}{2}\right)\) 时, 此时 \(g^{\prime}(0)=-a<0\), \(g^{\prime}\left(\frac{\pi}{2}\right)=\mathrm{e}^{\frac{\pi}{2}}+\frac{\pi}{2} a-1>\frac{\pi}{2}-1>0\),

若 \(g^{\prime}(x)\) 是 \(g^{\prime}(x)\) 的导数, 则 \(g^{\prime}(x)=\mathrm{e}^{x}+a\left(2 \sin x+x \cos x\right)\),

由于 \(2 \sin x+x \cos x>0\) 恒成立, 所以 \(g^{\prime}\left(x\right)>0\), 即 \(g^{\prime}\left(x\right)\) 在 \(\left(0, \frac{\pi}{2}\right)\) 上单调递增,

从而存在 \(x_{1} \in\left(0, \frac{\pi}{2}\right)\) 使得 \(g^{\prime}\left(x_{1}\right)=0\), 且 \(g^{\prime}\left(x\right)>0 \Rightarrow x_{1}<x<\frac{\pi}{2},  g^{\prime}(x)<0 \Rightarrow 0<x<x_{1}\),

即 \(g\left(x\right)\) 在区间 \(\left(0, x_{1}\right)\) 上递减, 区间 \(\left(x_{1}, \frac{\pi}{2}\right)\) 上递增, 从而 \(g\left(x_{1}\right)<g\left(0\right)=0\),

又 \(g\left(\frac{\pi}{2}\right)=\mathrm{e}^{\frac{\pi}{2}}-\frac{\pi}{2}-1>0\), 所以 \(g\left(x\right)\) 在 \(\left(x_{1}, \frac{\pi}{2}\right)\) 有唯一零点, 即在 \(\left(0, \frac{\pi}{2}\right)\) 上有唯一零点;

当 \(x \in\left(-\pi,-\frac{\pi}{2}\right)\) 时, 此时 \(x \sin x-\cos x>0\), 从而

\(g^{\prime}\left(x\right)=\mathrm{e}^{\mathrm{x}}+a\left(x \sin x-\cos x\right)-1 \geq \mathrm{e}^{\mathrm{x}}+x \sin x-\cos x-1\),

由于 \(x \in\left(-\pi,-\frac{\pi}{2}\right)\) 时, \(x<\sin x\), 所以

\(\mathrm{e}^{\mathrm{x}}+x \sin x-\cos x-1>\mathrm{e}^{\mathrm{x}}+\sin ^{2} x-\cos x-1=\mathrm{e}^{\mathrm{x}}-\left(\cos ^{2} x+\cos x\right)\),

又 \(\cos ^{2} x+\cos x=\cos x \cdot\left(\cos x+1\right)<0\), 从而 \(\mathrm{e}^{\mathrm{x}}+x \sin x-\cos x-1>\mathrm{e}^{\mathrm{x}}-\left(\cos ^{2} x+\cos x\right)>0\) 恒成立,

即 \(g^{\prime}\left(x\right)>0\) 在 \(x \in\left(-\pi,-\frac{\pi}{2}\right)\) 上恒成立, 所以 \(g\left(x\right)\) 在区间 \(x \in\left(-\pi,-\frac{\pi}{2}\right)\) 上单调递增,

因为 \(g\left(-\frac{\pi}{2}\right)=\mathrm{e}^{\frac{\pi}{2}}+\frac{\pi}{2}-1>0\), \(g\left(-\pi\right)=\mathrm{e}^{\pi}-a \pi+\pi-1 \leq \mathrm{e}^{-\pi}-1<0\),

因此 \(g\left(x\right)\) 在区间 \(x \in\left(-\pi,-\frac{\pi}{2}\right)\) 上有一定零点,

综上所述, 函数 \(g\left(x\right)\) 在区间 \(\left(-\pi, \frac{\pi}{2}\right)\) 上有 3 个零点.
\topics{求在曲线上一点处的切线方程(斜率);利用导数研究不等式恒成立问题;利用导数研究函数的零点}
\difficulty{0.4}
\answer{(1) $y=0$;}
\explain{(1) 由 \(f(x)=e^{x}-x-1\),则 \(f'(x)=e^{x}-1\),显然 \(f'(0)=f(0)=0\),所以切线方程为 \(y=0\);}
\end{question}

\begin{question}
如图, 已知点 \(P\) 到两点 \(F_{1}(-2,0)\), \(F_{2}(2,0)\) 距离的乘积为 8, 点 \(P\) 的轨迹记为曲线 \(\Gamma\), \(\Gamma\) 与 \(x\) 轴交点分别记为 \(M, N\).  


(1)求曲线\(\Gamma\)的方程;  

(2)求\(\triangle PMN\)的周长的取值范围;  

(3)过\(P\)作直线分别交\(y=\pm x\)于两点\(A,B\),且\(\overrightarrow{AP}=\lambda\overrightarrow{PB}(\lambda>1)\),若\(\triangle OAB\)的面积为18,求\(\lambda\)的最小值.  


(2)\(\left[8\sqrt{3}, 4\sqrt{3}+2\sqrt{6+6\sqrt{3}}\right]\);

(3)\(2+\sqrt{3}\).  




(2) 令 \(t=x^{2}+y^{2}\), 且 \(M\left(2 \sqrt{3},0\right), N\left(-2 \sqrt{3},0\right), y \in[-2,2]\), 则 \(t \in[4,12]\), 进而得到 \(\left|PM\right|+\left|PN\right|\) 关于 \(t\) 的表达式, 应用导数研究单调性求值域, 即可得三角形周长的范围;

(3) 设 \(A\left(x_{1},x_{1}\right), B\left(x_{2},-x_{2}\right)\), 由已知得 \(x_{1} x_{2}=\pm 18\), 曲线 \(\Gamma\) 得 \(\left(\frac{\lambda^{2} x_{1}^{2}+x_{2}^{2}}{\left(1+\lambda\right)^{2}}\right)^{-\frac{8 \lambda x_{1} x_{2}}{\left(1+\lambda\right)^{2}}=12}\), 令 \(m=\frac{\lambda}{\left(1+\lambda\right)^{2}}\), 结合基本不等式及一元二次不等式的解法求参数范围, 即可得.  


令 \(f(t)=t+\sqrt{288-2t^{2}}\),则 \(f'(t)=\frac{\sqrt{288-2t^{2}}-2t}{\sqrt{288-2t^{2}}}\),其中 \(f'(4\sqrt{3})=\frac{\sqrt{192}-8\sqrt{3}}{\sqrt{192}}=0\),所以 \(4\leq t<4\sqrt{3}\) 时 \(f'(t)>0\),\(4\sqrt{3}<t\leq12\) 时 \(f'(t)<0\),则 \(f(t)\) 在 \(\left[4,4\sqrt{3}\right]\) 上单调递增,在 \(\left(4\sqrt{3},12\right]\) 上单调递减,所以 \(f(t)\in\left[12,12\sqrt{3}\right]\),即 \(|PM|+|PN|\in\left[4\sqrt{3},2\sqrt{6+6\sqrt{3}}\right]\),而 \(|MN|=4\sqrt{3}\),所以 \(\Delta PMN\) 的周长的取值范围为 \(\left[8\sqrt{3},4\sqrt{3}+2\sqrt{6+6\sqrt{3}}\right]\);  


(3) 设 \(A\left(x_{1}, x_{1}\right), B\left(x_{2},-\text{-}x_{2}\right)\), 则 \(18=\frac{1}{2}|OA|\cdot|OB|=|x_{1}x_{2}|\), 则 \(x_{1}x_{2}=\pm 18\),

由题知 \(\left\{\begin{array}{l}x-x_{1}=\lambda\left(x_{2}-x\right), \\ y-x_{1}=\lambda\left(-x_{2}-y\right)\end{array}\right.\), 则 \(\left\{\begin{array}{l}x=\frac{x_{1}+\lambda x_{2}}{1+\lambda}, \\ y=\frac{x_{1}-\lambda x_{2}}{1+\lambda}\end{array}\right.\), 代入曲线 \(\Gamma\) 得: \(\left(\frac{\lambda^{2}x_{2}^{2}+x_{1}^{2}}{\left(1+\lambda\right)^{2}}\right)^{2}-\frac{8 \lambda x_{1} x_{2}}{\left(1+\lambda\right)^{2}}=12\),

令 \(m=\frac{\lambda}{\left(1+\lambda\right)^{2}}\), 则

①当 \(x_{1}x_{2}=18\) 时, \(12=-\frac{144 \lambda}{\left(1+\lambda\right)^{2}}+\left(\frac{\lambda^{2} x_{2}^{2}+\frac{18^{2}}{x_{2}^{2}}}{\left(1+\lambda\right)^{2}}\right)^{2} \geq-144 m+36^{2} m^{2}\), 解得 \(m \leq \frac{1}{6}\), 则

\(\lambda \geq 2+\sqrt{3}\);

②当 \(x_{1}x_{2}=-18\) 时, \(12=\frac{144 \lambda}{\left(1+\lambda\right)^{2}}+\left(\frac{\lambda^{2} x_{2}^{2}+\frac{18^{2}}{x_{2}^{2}}}{\left(1+\lambda\right)^{2}}\right)^{2} \geq 144 m+36^{2} m^{2}\), 解得 \(m \leq \frac{1}{18}\), 则

\(\lambda \geq 8+3 \sqrt{7}\).  

综上所述:\(\lambda\)的最小值为\(2+\sqrt{3}\).
\begin{center}
% IMAGE_TODO: images/438ee7c79a7dc07fbb6fe1dcb2342e95434c0d40523d9dda252b0bca9f64ccd7_24.jpg (width=24%)
\begin{tikzpicture}[scale=1.05,>=Stealth,line cap=round,line join=round]
  % TODO: draw this figure in TikZ according to the original image.
\end{tikzpicture}
\end{center}
\begin{center}
% IMAGE_TODO: images/9d8aab6d9c57a6ab70bcb87ac2f14b0e0aa31358305c3e991e2bb54ada4f3bc7_23.jpg (width=23%)
\begin{tikzpicture}[scale=1.05,>=Stealth,line cap=round,line join=round]
  % TODO: draw this figure in TikZ according to the original image.
\end{tikzpicture}
\end{center}
\topics{利用导数求函数的单调区间(不含参);由方程研究曲线的性质;判断两曲线交点的个数;求平面轨迹方程}
\difficulty{0.15}
\answer{(1)$\left(x^{2}+y^{2}\right)^{2}-8\left(x^{2}-y^{2}\right)=48$;}
\explain{(1) 设 \(P(x,y)\), 则 \(\sqrt{(x+2)^2+y^2}\cdot\sqrt{(x-2)^2+y^2}=8\) , 得 \(x^4+y^4+2x^2y^2-8x^2+8y^2-48=0\) , 所以 \(\left(x^2+y^2\right)^2-8\left(x^2-y^2\right)=48\) ; (2) 由 (1) 知 \(M\left(2\sqrt{3},0\right)\), \(N\left(-2\sqrt{3},0\right)\) , 令 \(t=x^2+y^2\) , 由 (1) , 以 \(x^2\) 为主元直接求根公式知 \(x^2=4-y^2+4\sqrt{4-y^2}\) , 则 \(y\in[-2,2]\), 则 \(t=x^2+y^2=4+4\sqrt{4-y^2}\in[4,12]\) , 且 \(8\left(x^2-y^2\right)=\left(x^2+y^2\right)^2-48\), \(\left(|PM|+|PN|\right)^2=(\sqrt{(x+2\sqrt{3})^2+y^2}+\sqrt{(x-2\sqrt{3})^2+y^2})^2=2\left(x^2+y^2\right)+24+2\sqrt{\left(x^2+y^2\right)^2-24\left(x^2-y^2\right)+144}=2\left(t+\sqrt{288-2t^2}+12\right)\) ,}
\end{question}
