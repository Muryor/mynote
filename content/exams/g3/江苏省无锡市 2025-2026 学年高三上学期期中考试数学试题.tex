\examxtitle{江苏省无锡市 2025-2026 学年高三上学期期中考试数学试题}
\section{单选题}

\begin{question}
设复数 \(z=(1+\mathrm{i}a)(2-\mathrm{i})\),若 \(z\) 的实部与虚部相等,则实数 \(a\) 的值为(\quad)
\begin{choices}
  \item \(3\)
  \item \(1\)
  \item \(-1\)
  \item \(-3\)
\end{choices}
\topics{求复数的实部与虚部;复数代数式的乘法运算}
\difficulty{0.85}
\answer{A}
\explain{因为 \(z=(1+\mathrm{i}a)(2-\mathrm{i})=2+a+(2a-1)\mathrm{i}\),且 \(z\) 的实部与虚部相等,故 \(2+a=2a-1\),解得 \(a=3\),故选:A。}
\end{question}


\begin{question}
已知集合 \(M=\{\,x\mid x^2-2x-3\le 0\,\}\),\(N=\{\,x\mid y=\ln(x-4)\,\}\),则 \((\complement_{\mathbf{R}}M)\cap N\) 等于 (\quad)
\begin{choices}
  \item \((-1,4]\)
  \item \((-1,3]\)
  \item \((3,+\infty)\)
  \item \((4,+\infty)\)
\end{choices}
\topics{交集的概念及运算;补集的概念及运算;求对数型复合函数的定义域;解不含参数的一元二次不等式}
\difficulty{0.85}
\answer{D}
\explain{\(M=\{\,x\mid x^2-2x-3\le 0\,\}=\{\,x\mid -1\le x\le 3\,\}\),所以 \(\complement_{\mathbf{R}} M=\{\,x\mid x<-1\text{ 或 }x>3\,\}\);又 \(N=\{\,x\mid y=\ln(x-4)\,\}=\{\,x\mid x>4\,\}\),所以 \((\complement_{\mathbf{R}}M)\cap N=\{\,x\mid x>4\,\}=(4,+\infty)\)。故选:D。}
\end{question}

\begin{question}
已知 \(a\ne0\),命题 \(P:x=1\) 是一元二次方程 \(ax^2+bx+c=0\) 的一个根,命题 \(q:a+b+c=0\),则 \(P\) 是 \(q\) 的 (\quad)
\begin{choices}
  \item 充分不必要条件
  \item 必要不充分条件
  \item 充分必要条件
  \item 既不充分也不必要条件
\end{choices}

\topics{充分条件、必要条件、充要条件的证明}
\difficulty{0.85}
\answer{C}
\explain{对于命题 \(P\),\(x=1\) 为方程的根,则 \(a+b+c=0\),充分性成立;对命题 \(q\),\(a+b+c=0\) 且 \(a\ne0\),则 \(x=1\) 必是题设方程的一个根,必要性成立;所以 \(P\) 是 \(q\) 的充分必要条件。故选:C。}
\end{question}

\begin{question}
在 \(\triangle ABC\) 中,\(D\) 是 \(BC\) 的中点,\(E\) 是 \(AD\) 的中点。若 \(\overrightarrow{BE}=\lambda\overrightarrow{AB}+\mu\overrightarrow{AC}\),则 \(\dfrac{\lambda}{\mu}=\) (\quad)
\begin{choices}
  \item \(3\)
  \item \(-3\)
  \item \(2\)
  \item \(-2\)
\end{choices}

\topics{平面向量的混合运算;用基底表示向量}
\difficulty{0.85}
\answer{B}
\explain{
  \begin{center}
  \begin{tikzpicture}[scale=1.05,>=Stealth,line cap=round,line join=round]
    \coordinate (B) at (0,0);
    \coordinate (C) at (5.2,0.35);
    \coordinate (A) at (2.9,3.6);
    \path (B) -- (C) coordinate[pos=0.5] (D);
    \path (A) -- (D) coordinate[pos=0.5] (E);
    \draw[thick] (B)--(A)--(C)--cycle;
    \draw[thick] (A)--(D);
    \draw[thick] (B)--(E);
    \fill (A) circle (1.25pt) node[above] {\(A\)};
    \fill (B) circle (1.25pt) node[below left=-1pt] {\(B\)};
    \fill (C) circle (1.25pt) node[below right=-1pt] {\(C\)};
    \fill (D) circle (1.25pt) node[below=2pt] {\(D\)};
    \fill (E) circle (1.25pt) node[right=2pt] {\(E\)};
  \end{tikzpicture}
  \end{center}
  \(\overrightarrow{BE}=\dfrac12(\overrightarrow{BA}+\overrightarrow{BD})=-\dfrac12\overrightarrow{AB}+\dfrac14\overrightarrow{BC}=-\dfrac12\overrightarrow{AB}+\dfrac14(\overrightarrow{AC}-\overrightarrow{AB})=-\dfrac34\overrightarrow{AB}+\dfrac14\overrightarrow{AC}\),所以 \(\lambda=-\dfrac34,\ \mu=\dfrac14\),所以 \(\dfrac{\lambda}{\mu}=-3\)。故选:B。}
\end{question}

\begin{question}
已知函数 \(f(x)\) 是定义在 \(\mathbb{R}\) 上的奇函数,当 \(x<0\) 时,\(f(x)=10^x\),则 \(f(\lg3)=\) (\quad)
\begin{choices}
  \item \(-3\)
  \item \(3\)
  \item \(-\dfrac13\)
  \item \(\dfrac13\)
\end{choices}

\topics{函数奇偶性的应用;对数的运算}
\difficulty{0.85}
\answer{C}
\explain{由题意知函数 \(f(x)\) 是定义在 \(\mathbb{R}\) 上的奇函数,故 \(f(\lg3)=-f(-\lg3)=-f\!\left(\lg\dfrac13\right)\),而 \(\lg\dfrac13<0\),故 \(f\!\left(\lg\dfrac13\right)=10^{\lg(1/3)}=\dfrac13\),则 \(f(\lg3)=-\dfrac13\),故选:C。}
\end{question}

\begin{question}
在数列 \(\{a_n\}\) 中,\(a_1=1\),若数列 \(\{a_1,a_3,\dots,a_{2n-1}\}\) 是公比为 \(2\) 的等比数列,则 \(a_1+a_3+a_5+\cdots+a_9=\) (\quad)
\begin{choices}
  \item \(2048\)
  \item \(2047\)
  \item \(1024\)
  \item \(1023\)
\end{choices}
\end{question}
\topics{由定义判定等比数列;求等比数列前 \(n\) 项和}
\difficulty{0.65}
\answer{D}
\explain{数列 \(\{a_1,a_{n-1}\}\) 是公比为 \(2\) 的等比数列,则有 \(\dfrac{a_{n+1}\cdot a_{n+2}}{a_n\cdot a_{n+1}}=2\),所以 \(\dfrac{a_{n+2}}{a_n}=2\),因此数列 \(\{a_n\}\) 的奇数项是以 \(1\) 为首项、\(2\) 为公比的等比数列,所以 \(a_1+a_3+a_5+\cdots+a_9=\dfrac{1-2^{10}}{1-2}=1023\),故选:D。}

\begin{question}
在四边形 \(ABCD\) 中,\(\overrightarrow{AB}=\lambda\overrightarrow{DC}\),\(\overrightarrow{AB}=(1,-\sqrt2)\),\(\overrightarrow{AD}=(4,2\sqrt2)\)。若四边形 \(ABCD\) 的面积为 \(12\sqrt2\),则实数 \(\lambda\) 的值为(\quad)
\begin{choices}
  \item \(3\)
  \item \(\sqrt3\)
  \item \(\dfrac13\)
  \item \(\dfrac{\sqrt3}{3}\)
\end{choices}

\topics{向量的线性运算的几何应用;坐标计算向量的模;向量垂直的坐标表示}
\difficulty{0.65}
\answer{C}
\explain{因为 \(AB=\lambda DC\),\(|AB|=\sqrt{1+2}=\sqrt3\),\(|AD|=\sqrt{16+8}=2\sqrt6\),所以四边形 \(ABCD\) 为梯形;又 \(AB\cdot AD=1\times4+(-\sqrt2)\times2\sqrt2=0\),所以 \(AB\perp AD\),所以四边形 \(ABCD\) 为直角梯形,则 \(S_{\triangle ABD}=\dfrac12\sqrt3\times2\sqrt6=3\sqrt2\);又 \(S_{ABCD}=12\sqrt2\),得 \(S_{\triangle BCD}=9\sqrt2\),于是 \(\dfrac12|CD|\cdot|AD|=9\sqrt2\),即 \(\dfrac12\cdot\dfrac{\sqrt3}{\lambda}\cdot2\sqrt6=9\sqrt2\),故 \(\dfrac1\lambda=\dfrac13\),选 C。}
\end{question}

\begin{question}
已知函数 \(f(x)=\dfrac1x-\dfrac1{1-x}+x+k\) 的三个零点为 \(a,b,c\),且 \(a<b<c\),则下列结论不正确的是
\begin{choices}
  \item \(f(x)\) 在 \((0,1)\) 上单调递减
  \item 曲线 \(y=f(x)\) 是中心对称图形
  \item \(\forall k\in\mathbb{R}\),都有 \(c>b+1\)
  \item \(\forall k\in\mathbb{R}\),都有 \(c>a+3\)
\end{choices}

\topics{判断或证明函数的对称性;函数单调性的应用;用导数判断或证明已知函数的单调性;利用导数研究函数的零点}
\difficulty{0.4}
\answer{D}
\explain{对 A:由 \(f'(x)=-\dfrac1{x^2}-\dfrac1{(1-x)^2}-1\),当 \(x\in(-\infty,0)\cup(0,1)\cup(1,+\infty)\) 时 \(f'(x)<0\),故 \(f(x)\) 在上述区间单调递减,A 正确;\\
对 B:\(f(x)+f(-x+1)=\dfrac1x+\dfrac1{1-x}-x+k+\dfrac1{-x+1}+x-1+k=2k-1\),定义域为 \((-\infty,0)\cup(0,1)\cup(1,+\infty)\),故曲线 \(y=f(x)\) 关于点 \(\left(\dfrac12,k-\dfrac12\right)\) 对称,B 正确;\\
对 C:\(a\in(-\infty,0)\),\(b\in(0,1)\),\(c\in(1,+\infty)\),且 \(f\) 在 \((1,+\infty)\) 上单调递减,\(b+1\in(1,2)\),由 \(f(b)=0\) 得 \(k=b+\dfrac1{1-b}-\dfrac1b\),进而 \(f(b+1)=\dfrac2{1-b^2}-1>1\),故 \(c>b+1\) 恒成立,C 正确;\\
对 D:取 \(k=\dfrac12\),得 \(f(-1)=0\)、\(f(2)=0\),此时 \(a=-1, c=2\),\(c=a+3\),故"\(\forall k\) 都有 \(c>a+3\)"为假,D 错误。}
\end{question}

\section{多选题}

\begin{question}
若 \(\dfrac1a<\dfrac1b<0\),则下列不等式正确的是(\quad)。
\begin{choices}
  \item \(|a|>|b|\)
  \item \(\sqrt{-a}<\sqrt{-b}\)
  \item \(a+b>ab\)
  \item \(a^2-a<b^2-b\)
\end{choices}
\topics{由已知条件判断所给不等式是否正确;由不等式的性质比较数(式)大小;作差法比较代数式的大小}
\difficulty{0.65}
\answer{BD}
\explain{由 \(\dfrac1a<\dfrac1b<0\) 可知,\(b<a<0\),所以 \(-b>-a>0\),即 \(|b|>|a|\),故 A 错误;\(\sqrt{-b}>\sqrt{-a}\),故 B 正确;\(a+b<0,ab>0\),所以 \(a+b<ab\),故 C 错误;\(a^2-a-b^2+b=(a+b)(a-b)-(a-b)=(a-b)(a+b-1)\),由以上可知,\(a-b>0\),\(a+b-1<0\),所以 \(a^2-a-b^2+b<0\),即 \(a^2-a<b^2-b\),故 D 正确。}
\end{question}

\begin{question}
在直角坐标系 \(xOy\) 中,已知 \(O\) 是以 \(O\) 为圆心的单位圆,点 \(A\) 的坐标为 \((1,0)\),角 \(\theta\) 的始边为射线 \(OA\),终边 \(OB\) 交圆于点 \(B\),过点 \(B\) 作直线 \(OA\) 的垂线,垂足为 \(C\)。若将点 \(C\) 到直线 \(OB\) 的距离表示为 \(\theta\) 的函数 \(h(\theta)\),则(\quad)。
\begin{choices}
  \item \(h\!\left(\dfrac\pi{12}\right)=\dfrac14\)
  \item \(h(\theta)\) 的最小正周期为 \(\dfrac\pi4\)
  \item \(\left[\dfrac{3\pi}4,\pi\right]\) 是 \(h(\theta)\) 的一个单调减区间
  \item \(h(\theta)+h\!\left(\dfrac\pi4+\theta\right)\) 的最大值为 \(\dfrac{\sqrt2}{2}\)
\end{choices}
\topics{求含 \(\sin x(\cos x)\) 函数的值域和取值;求正弦(型)函数的最小正周期、二倍角的正弦公式;求点到直线的距离}
\difficulty{0.4}
\answer{ACD}
\explain{利用点到直线的距离公式及二倍角正弦公式得 \(h(\theta)=\dfrac12\lvert\sin2\theta\rvert\),代入求得 \(h\!\left(\dfrac\pi{12}\right)=\dfrac14\),正确;对 B,因为 \(h\!\left(\theta+\dfrac\pi4\right)=\dfrac12\lvert\sin(2\theta+\dfrac\pi2)\rvert\ne h(\theta)\),所以 \(\dfrac\pi4\) 不是 \(h(\theta)\) 的周期,错误;对 C,\(\theta\in\left[\dfrac{3\pi}4,\pi\right]\),所以 \(2\theta\in\left[\dfrac{3\pi}2,2\pi\right]\),所以 \(h(\theta)=-\dfrac12\sin2\theta\),因 \(y=\sin x\) 在 \(\left[\dfrac{3\pi}2,2\pi\right]\) 单调递增,所以 \(h(\theta)=-\dfrac12\sin2\theta\) 在 \(\theta\in\left[\dfrac{3\pi}4,\pi\right]\) 单调递减,正确;对 D,\(h(\theta)+h\!\left(\dfrac\pi4+\theta\right)=\dfrac12\left(|\sin2\theta|+|\cos2\theta|\right)\le\dfrac{\sqrt2}{2}\),当且仅当 \(|\sin2\theta|=|\cos2\theta|\) 时等号成立,正确。}
\end{question}

\begin{question}
设正项数列 \(\{a_{n}\}\) 的前 \(n\) 项和为 \(S_{n}\),若 \(a_{1}=1\),且对任意的正整数 \(n\) 都有 \(a_{n+1}\leq\lambda S_{n}\),\(\lambda\in\mathbb{N}^{*}\),称 \(\{a_{n}\}\) 是"\(\lambda\)-数列"。下列结论正确的是(\quad)。
\begin{choices}
  \item 若 \(\{b_{n}\}\) 是首项为 \(1\) 公差为 \(2\) 的等差数列,则 \(\{b_{n}\}\) 是"\(3\)-数列"
  \item 若 \(\{b_{n}\}\) 是"\(2\)-数列",则不可能存在正整数 \(n\geq2\),满足 \(b_{n}>2\cdot3^{n-2}\)
  \item 若 \(\{b_{n}\}\) 是"\(\lambda\)-数列",且 \(b_{n}=\frac{1}{2}\left(2^{n-1}+5^{n-1}\right)\),则 \(\lambda\) 的最小值是 \(4\)
  \item 任给 \(1<p<q\),若 \(b_{n}=\frac{1}{2}\left(p^{n-1}+q^{n-1}\right)\),且 \(1+\lambda\geq q\),则 \(\{b_{n}\}\) 是"\(\lambda\)-数列"
\end{choices}
\topics{数列新定义;数列不等式恒成立问题}
\difficulty{0.4}
\answer{ABC}
\explain{利用等差数列通项公式和前 \(n\) 项和公式得 \(S_{n}=n^{2}\),\(b_{n+1}=2n+1\),求得 \(b_{n+1}\leq3S_{n}\),根据数列新定义即可判断 A;根据数列新定义得 \(b_{n+1}\leq2S_{n}\),进而利用归纳法得 \(b_{n}\leq2\cdot3^{n-2}\),即可判断 B;根据数列新定义得 \((\lambda-1)2^{n}+\left(\frac{\lambda}{4}-1\right)5^{n}-\frac{5\lambda}{4}\geq0\) 恒成立,求解 \(\lambda\) 的最小值判断 C;结合数列的新定义举反例判断 D。对于 A,若 \(\{b_{n}\}\) 是首项为 \(1\) 公差为 \(2\) 的等差数列,则 \(b_{n}=2n-1\),所以 \(S_{n}=\dfrac{n(2n-1+1)}{2}=n^{2}\),\(b_{n+1}=2n+1\),\(b_{n+1}-3S_{n}=2n+1-3n^{2}=-(3n+1)(n-1)\leq0\),所以 \(b_{n+1}\leq3S_{n}\),所以 \(\{b_{n}\}\) 是"\(3\)-数列",A 正确;对于 B,若 \(\{b_{n}\}\) 是"\(2\)-数列",则 \(b_{n+1}\leq2S_{n}\),利用归纳法可得 \(b_{n}\leq2\cdot3^{n-2}\),B 正确;对于 C,通过计算验证 \(\lambda\geq4\) 时不等式恒成立,C 正确;对于 D,取 \(p=2, q=3, \lambda=2\) 时,\(b_{2}=\dfrac{5}{2}>2\),不满足,D 错误。故选:ABC。}
\end{question}

\section{填空题}

\begin{question}
已知向量 \(\vec a=(m+2n,2)\),\(\vec b=(1,2)\),其中 \(mn>0\)。若 \(\vec a\parallel\vec b\),则 \(mn\) 的最大值为\fillin{}。
\topics{由向量共线(平行)求参数;基本不等式求积的最大值}
\difficulty{0.65}
\answer{\(\dfrac{1}{8}\)}
\explain{\(\vec a=(m+2n,2)\),\(\vec b=(1,2)\),因为 \(\vec a\parallel\vec b\),所以 \(1\times2=2(m+2n)\),即 \(m+2n=1\),\(mn>0\)。所以 \(2mn\le\left(\dfrac{m+2n}{2}\right)^2=\dfrac18\),当且仅当 \(m=\dfrac12,\ n=\dfrac14\) 时等号成立。故 \(mn\) 的最大值为 \(\dfrac18\)。}
\end{question}

\begin{question}
在 2025 年江苏省"苏超"足球联赛的一场激烈比赛中,某城市队的 10 号球员从 \(A\) 出发,以 \(2.5\) 米/秒的速度做匀速直线运动,到达 \(B\) 点时,发现足球在点 \(C\) 处正以 \(3\) 倍于自己的速度向点 \(A\) 侧匀速直线运动。已知 \(AB=4\sqrt3\) 米,\(AC=20\) 米,\(\angle BAC=30^\circ\)。若忽略球员转身所需的时间,则该球员按原来的速度最快截住足球所用的时间为\fillin{}秒。
\topics{余弦定理;距离最短(时间最短)模型}
\difficulty{0.65}
\answer{\(\dfrac{8}{5}\)}
\explain{
  \begin{center}
\begin{tikzpicture}[scale=1.05,>=Stealth,line cap=round,line join=round]
  \coordinate (A) at (0,0);
  \coordinate (B) at (5.1,0);
  \coordinate (C) at (2.6,4.4);
  \path (A) -- (C) coordinate[pos=0.35] (D);
  \draw (A)--(B)--(C)--cycle;
  \draw (B)--(D);
  % Fix malformed coordinate math: use ($(A)+(0.7,0)$)
  \draw ($(A)+(0.7,0)$) arc[start angle=0,end angle=60,radius=0.7];
  \node at (0.95,0.35) {$(30^\circ)$};
  % Midpoint between A and B: ($(A)!0.5!(B)$)
  \node[below] at ($(A)!0.5!(B)$) {$(4\sqrt{3})$};
  \fill (A) circle (1.1pt) node[below left] {$(A)$};
  \fill (B) circle (1.1pt) node[below right] {$(B)$};
  \fill (C) circle (1.1pt) node[above] {$(C)$};
  \fill (D) circle (1.1pt) node[left] {$(D)$};
\end{tikzpicture}
\end{center}
  设在 \(D\) 处截住足球,时间设为 \(t\) 秒,则 \(BD=2.5t,\ CD=7.5t\),则 \(AD=20-7.5t\);又 \(AB=4\sqrt3,\ \angle BAC=30^\circ\),在 \(\triangle ABD\) 中,利用余弦定理可知,\(BD^2=AD^2+AB^2-2AD\cdot AB\cdot\sin\angle BAC\),则 \((2.5t)^2=(20-7.5t)^2+48-2\times4\sqrt3\times(20-7.5t)\times\dfrac{\sqrt3}{2}\),化简得 \(25t^2-110t+104=0\),解得 \(t=\dfrac85\) 或 \(t=\dfrac{13}5\),所以该球员按原来的速度最快截住足球所用的时间为 \(\dfrac85\) 秒。}
\end{question}

\begin{question}
设函数 \(f(x)=\sin x-kx-b,\ x\in\left[0,\dfrac\pi2\right)\),其中 \(k,\ b\in\mathbb{R}\)。若 \(k=\dfrac2\pi,\ b=0\),则 \(f(x)\) 的最小值为\fillin{};若 \(f'(x)\ge0\) 恒成立,则 \(-k^2+\pi k+b\) 的最大值为\fillin{}。

\topics{由导数求函数的最值(不含参);一元二次不等式在实数集上恒成立问题;函数不等式恒成立问题}
\difficulty{0.4}
\answer{\(0\);\(\dfrac{\pi^2}{16}+1\)}
\explain{
  \begin{center}
\begin{tikzpicture}[scale=1.05,>=Stealth,line cap=round,line join=round]
  \draw[->] (-0.5,0) -- (5.2,0) node[right] {$(x)$};
  \draw[->] (0,-2.2) -- (0,3.2) node[above] {$(y)$};
  \draw[thick,domain=0:4.5,samples=200] plot (\x,{1.6*sin(deg(\x/1.6))});
  \node at (2.2,1.8) {$(y=\sin x)$};
  \draw[thick] (-0.2,-1.4) -- (5,3);
  \node at (3.9,2.5) {$(y=kx+b)$};
  \draw[dashed] ({pi/2},-2.2) -- ({pi/2},3.2);
  \node[below] at ({pi/2},0) {$(\dfrac{\pi}{2})$};
  \fill (0,0) circle (1.1pt) node[below left] {$(O)$};
\end{tikzpicture}
\end{center}
若 \(k=\dfrac2\pi,\ b=0\),求导,利用导数判断 \(f(x)\) 的单调性和最值;若 \(f'(x)\ge0\) 恒成立,即 \(\sin x\ge kx+b\),结合图象可得 \(b\le0\),\(\pi k+2b-2\le0\);令 \(-k^2+\pi k+b=t\),消去 \(b\) 结合 \(a\) 判别式运算求解即可。\\
当 \(k=\dfrac2\pi,\ b=0\),则 \(f(x)=\sin x-\dfrac2\pi x,\ f'(x)=\cos x-\dfrac2\pi\),因为 \(f'(x)\) 在 \(\left(0,\dfrac\pi2\right)\) 内存在唯一零点 \(x_0\),当 \(0<x<x_0\) 时,\(f'(x)>0\);当 \(x_0<x<\dfrac\pi2\) 时,\(f'(x)<0\);可知 \(f(x)\) 在 \((0,x_0)\) 内单调递增,在 \(\left(x_0,\dfrac\pi2\right)\) 内单调递减,且 \(f(0)=f\!\left(\dfrac\pi2\right)=0\),所以 \(f(x)\) 的最小值为 \(0\);若 \(f(x)=\sin x-kx-b\ge0\),即 \(\sin x\ge kx+b\),可知当 \(x\in\left[0,\dfrac\pi2\right]\) 时,\(y=\sin x\) 在线段 \(y=kx+b\) 的上方,结合图象可得 \(\sin0\ge0\ge b\),即 \(b\le0\);\(\sin\dfrac\pi2=1\ge\dfrac\pi2k+b\),即 \(\pi k+2b-2\le0\),可知 \(k\in\mathbb{R}\),设 \(-k^2+\pi k+b=t\),则 \(b=k^2-\pi k+t\),可得 \(\begin{cases}k^2-\pi k+t\le0\\2k^2-\pi k+2t-2\le0\end{cases}\),因为该不等式组在 \(k\in\mathbb{R}\) 内有解,则 \(\begin{aligned}\Delta_1&=\pi^2-4t\ge0\\ \Delta_2&=\pi^2-8(2t-2)\ge0\end{aligned}\),解得 \(t\le\dfrac{\pi^2}{16}+1\),所以当 \(k=\dfrac\pi4,\ b=1-\dfrac{\pi^2}{8}\) 时,\(-k^2+\pi k+b\) 取得最大值 \(\dfrac{\pi^2}{16}+1\)。}
\end{question}

\section{解答题}

\begin{question}
在 \(\triangle ABC\) 中,角 \(A,B,C\) 所对的边为 \(a,b,c\),且 \(\dfrac{b}{\cos B}=\dfrac{a+b+c}{\cos A+\cos B+\cos C}\)。
\begin{enumerate}[label=(\arabic*)]
  \item 求角 \(B\) 的大小;
  \item 若 \(\cos A=\dfrac{2\sqrt7}{7}\),求 \(\tan(2A-B)\) 的值。
\end{enumerate}
\topics{用和、差的正弦公式化简、求值;用和、差的正切公式化简、求值;二倍角的正切公式;正弦定理边角互化的应用}
\difficulty{0.65}
\answer{(1) \(B=\dfrac{\pi}{3}\);(2) \(\dfrac{3\sqrt{3}}{13}\)}
\explain{(1) 由正弦定理,因 \(\dfrac{b}{\cos B}=\dfrac{a+b+c}{\cos A+\cos B+\cos C}\),得 \(\dfrac{\sin B}{\cos B}=\dfrac{\sin A+\sin B+\sin C}{\cos A+\cos B+\cos C}\)。两边交叉相乘并整理:\(\sin B\cos A-\sin A\cos B=\sin C\cos B-\sin B\cos C\),即 \(\sin(B-A)=\sin(C-B)\)。所以 \(B-A=C-B+2k\pi\) 或 \(B-A+C-B=\pi+2k\pi\),\(k\in\mathbb{Z}\)。又 \(A+C\in(0,\pi)\),故取 \(A+C=2B\),从而 \(B=\dfrac\pi3\)。\\
(2) 已知 \(\cos A=\dfrac{2\sqrt7}{7}\),则 \(\sin A=\sqrt{1-\cos^2 A}=\dfrac{\sqrt{21}}{7}\),故 \(\tan A=\dfrac{\sqrt{21}}{2\sqrt7}=\dfrac{\sqrt3}{2}\)。由二倍角公式 \(\tan 2A=\dfrac{2\tan A}{1-\tan^2 A}=\dfrac{\sqrt3}{1-\tfrac34}=4\sqrt3\)。因此
\[\tan(2A-B)=\frac{\tan2A-\tan B}{1+\tan2A\tan B}=\frac{4\sqrt3-\sqrt3}{1+4\cdot3}=\frac{3\sqrt3}{13}.\]}
\end{question}

\begin{question}
已知向量 \(\vec m=(\sin x,\sqrt3)\),\(\vec n=\!\left(\cos x,\dfrac12\cos2x\right)\),设函数 \(f(x)=\vec m\cdot\vec n\)。
\begin{enumerate}[label=(\arabic*)]
  \item 若 \(\left(\dfrac x2-\dfrac\pi{12}\right)=\dfrac35,\ \dfrac\pi3<x<\dfrac{5\pi}6\),求 \(\cos x\) 的值;
  \item 将函数 \(f(x)\) 的图像上所有的点向右平移 \(\varphi\ (0<\varphi<\dfrac\pi2)\) 个单位长度,再把所得各点的纵坐标缩短到原来的 \(\dfrac12\) 倍(纵坐标不变),得到函数 \(y=g(x)\) 的图像。若函数 \(y=g(x)\) 的图像关于直线 \(x=\dfrac\pi{12}\) 对称,求 \(g(x)\) 在 \(\left[0,\dfrac\pi4\right]\) 上的最大值和最小值。
\end{enumerate}
\topics{求含 \(\sin x(\cos x)\) 函数的值域和最值;做图象变换前(后)的解析式;给值求值型题;数量积的坐标表示}
\difficulty{0.65}
\answer{(1) \(\dfrac{3-4\sqrt{3}}{10}\);(2) 最大值为 \(1\),最小值为 \(-\dfrac{1}{2}\)}
\explain{(1) 由 \(\vec m=(\sin x,\sqrt3)\),\(\vec n=(\cos x,\tfrac12\cos2x)\),有
\[f(x)=\vec m\cdot\vec n=\sin x\cos x+\tfrac{\sqrt3}{2}\cos2x=\tfrac12\sin2x+\tfrac{\sqrt3}{2}\cos2x=\sin\!\left(2x+\tfrac\pi3\right).\]
于是 \(f\!\left(\tfrac{x}{2}-\tfrac\pi{12}\right)=\sin\!\left(x+\tfrac\pi6\right)=\tfrac35\)。由 \(\tfrac\pi3<x<\tfrac{5\pi}6\) 得 \(\tfrac\pi2<x+\tfrac\pi6<\pi\),故 \(\cos\!\left(x+\tfrac\pi6\right)=-\sqrt{1-(\tfrac35)^2}=-\tfrac45\)。因此
\[\cos x=\cos\!\left[\left(x+\tfrac\pi6\right)-\tfrac\pi6\right]=\cos\!\left(x+\tfrac\pi6\right)\cos\tfrac\pi6+\sin\!\left(x+\tfrac\pi6\right)\sin\tfrac\pi6=\frac{3-4\sqrt3}{10}.\]
(2) 由 (1) 知 \(f(x)=\sin\!\left(2x+\tfrac\pi3\right)\)。向右平移 \(\varphi\) 后得 \(y=\sin\!\left(2x-2\varphi+\tfrac\pi3\right)\),再将横坐标缩短到原来的 \(\tfrac12\) 倍得
\[g(x)=\sin\!\left(4x-2\varphi+\tfrac\pi3\right).\]
因 \(g(x)\) 关于直线 \(x=\tfrac\pi{12}\) 对称,故有 \(4\cdot\tfrac\pi{12}-2\varphi+\tfrac\pi3=\tfrac\pi2+k\pi\),解得 \(\varphi=\tfrac\pi{12}-\tfrac{k\pi}{2}\)。由 \(0<\varphi<\tfrac\pi2\) 得 \(\varphi=\tfrac\pi{12}\)。于是
\[g(x)=\sin\!\left(4x+\tfrac\pi6\right).\]
当 \(x\in[0,\tfrac\pi4]\) 时,\(4x+\tfrac\pi6\in[\tfrac\pi6,\tfrac{7\pi}6]\),故 \(g(x)\in\left[-\tfrac12,1\right]\)。从而最大值为 \(1\),最小值为 \(-\tfrac12\)。}
\end{question}

\begin{question}
已知函数 \(g(x)=\dfrac1x\)。
\begin{enumerate}[label=(\arabic*)]
  \item 求过点 \(A(-4,2)\) 且与曲线 \(y=g(x)\) 相切的直线方程;
  \item 令 \(f(x)=\dfrac{x^2-ax+1}{g(x)}\),若 \(f(x)\) 有两个极值点 \(x_1,\ x_2\),记过两点 \(P(x_1,f(x_1)),Q(x_2,f(x_2))\) 的直线斜率为 \(k\)。是否存在实数 \(a\),使得 \(k=\dfrac53-a\)?若存在,求 \(a\) 的值;若不存在,试说明理由。
\end{enumerate}
\topics{求过一点的切线方程;根据极值点求参数}
\difficulty{0.65}
\answer{(1) \(y=-\dfrac{1}{4}x+1\) 或 \(y=-x-2\);(2) \(a=3\)}
\explain{(1) 设切点为 \(\bigl(x_0,\tfrac1{x_0}\bigr)\)。由 \(g(x)=\tfrac1x\) 得 \(g'(x)=-\tfrac1{x^2}\),故切线斜率 \(k=-\tfrac1{x_0^2}\)。点斜式切线为
\[y-\tfrac1{x_0}= -\tfrac1{x_0^2}(x-x_0)\quad\Rightarrow\quad y=-\tfrac1{x_0^2}x+\tfrac2{x_0}.\]
代入点 \(A(-4,2)\):\(2=\tfrac4{x_0^2}+\tfrac2{x_0}\Rightarrow x_0^2-x_0-2=0\Rightarrow x_0=2\text{ 或 }-1\)。故切线为 \(y=-\tfrac14x+1\) 或 \(y=-x-2\)。\\
(2) \(f(x)=\dfrac{x^2-ax+1}{g(x)}=x^3-ax^2+x\),于是 \(f'(x)=3x^2-2ax+1\)。由题意有两个不等实根 \(x_1,x_2\),且 \(x_1+x_2=\tfrac{2a}{3},\ x_1x_2=\tfrac13\)。两点斜率
\[k=\frac{f(x_2)-f(x_1)}{x_2-x_1}=x_2^2+x_1x_2+x_1^2-a(x_1+x_2)+1=(x_1+x_2)^2-x_1x_2-a(x_1+x_2)+1.\]
代入得 \(k=\bigl(\tfrac{2a}{3}\bigr)^2-\tfrac13-a\cdot\tfrac{2a}{3}+1=-\tfrac{2a^2}{9}+\tfrac23\)。令 \(k=\tfrac53-a\) 得 \(\tfrac{2a^2}{9}-a+1=0\Rightarrow a=\tfrac32\) 或 \(a=3\)。又因判别式 \(4a^2-12>0\Rightarrow a^2>3\),故 \(a=3\)。}
\end{question}

\begin{question}
设数列 \(\{a_n\}\) 的前 \(n\) 项和为 \(S_n\),已知 \((3n-5)a_{n+1}-6S_n=pn+q,\ n\in\mathbb{N}^*\),其中 \(p,\ q\) 为常数。
\begin{enumerate}[label=(\arabic*)]
  \item 当 \(p=0\) 时,若 \(S_1=-2,\ S_2=-1\),求数列 \(\{a_n\}\) 的通项公式;
  \item 若 \(a_1=-\dfrac q5\)。
  \begin{enumerate}[label=(\roman*)]
    \item 证明:数列 \(\{a_n\}\) 为等差数列;
    \item 若 \(8,\ \sqrt3a_3,\ a_{12}\) 成等比数列,当 \(n\) 为何值时,\(\displaystyle\sum_{k=1}^n a_ka_{k+1}a_{k+2}\) 取得最大值,请说明理由。
  \end{enumerate}
\end{enumerate}
\topics{判断数列的增减性、由递推关系证明数列是等差数列;等比中项的应用;利用 \(a_n\) 与 \(S_n\) 关系求通项或项}
\difficulty{0.4}
\answer{(1) \(a_n=3n-5\);(2) (i) 证明略;(ii) \(n=16\) 时取最大值}
\explain{(1) 当 \(p=0\) 时,\((3n-5)a_{n+1}-6S_n=q\)。同理有 \((3n-2)a_{n+2}-6S_{n+1}=q\)。两式相减得 \((3n-2)a_{n+2}=(3n+1)a_{n+1}\)。由 \(S_1=-2,S_2=-1\) 得 \(a_1=-2,a_2=1\) 满足关系,递推可化为
\[\frac{a_{n+1}}{a_n}=\frac{3n-2}{3n-5}.\]
累乘得(当 \(n\ge2\)):\(\dfrac{a_n}{a_1}=-\dfrac{3n-5}{2}\)。由 \(a_1=-2\) 得 \(a_n=3n-5\),且 \(n=1\) 亦满足,故 \(a_n=3n-5\,(n\in\mathbb{N}^*)\)。\\
(2) (i) 由 \((3n-5)a_{n+1}-6S_n=pn+q\),得 \((3n-2)a_{n+2}-6S_{n+1}=p(n+1)+q\)。相减得 \((3n-2)a_{n+2}-(3n+1)a_{n+1}=p\),再对 \(n\mapsto n+1\) 相同处理并相减,得
\[a_{n+3}+a_{n+1}=2a_{n+2}\Rightarrow a_{n+3}-a_{n+2}=a_{n+2}-a_{n+1},\]
故 \(\{a_n\}\) 为等差数列。结合 \(a_1=-\tfrac q5\) 可解得前若干项一致,从而判定成立。\\
(ii) 设等差公差为 \(d\)。由 \(8,\sqrt3\,a_5,a_{12}\) 成等比,得 \(3a_5=8a_{12}>0\Rightarrow a_5=-\tfrac{56}{5}d>0\),故 \(d<0\)。进而 \(a_1=-\tfrac{76}{5}d\),\(a_n=(n-\tfrac{81}{5})d\)。于是 \(a_1>a_2>\cdots>a_{16}>0>a_{17}>\cdots\)。令 \(T_n=\sum_{k=1}^n a_ka_{k+1}a_{k+2}\),可见
\[a_1a_2a_3>\cdots>a_{14}a_{15}a_{16}>0>a_{17}a_{18}a_{19}>\cdots\]
且 \(a_{15}a_{16}a_{17}<0\)、\(a_{16}a_{17}a_{18}>0\)。比较序列 \(T_n\) 的增长:\(T_{14}>\cdots>T_1\),\(T_{14}>T_{15}<T_{16}>T_{17}>\cdots\),并由
\[T_{16}-T_{14}=a_{16}a_{17}(a_{15}+a_{18})=a_{16}a_{17}\cdot\tfrac{3}{5}d>0\]
可知 \(T_{16}\) 为最大值,故当 \(n=16\) 时 \(\sum_{k=1}^n a_ka_{k+1}a_{k+2}\) 取最大。}
\end{question}

\begin{question}
设 \(a>0,\ b>0\),函数 \(f(x)=a\ln x-\dfrac12\,b(x-1)^2\),函数 \(g(x)=\dfrac{f(x)}x\)。
\begin{enumerate}[label=(\arabic*)]
  \item 当 \(b=2\) 时:
    \begin{enumerate}[label=(\roman*)]
      \item 讨论函数 \(g(x)=\dfrac{f(x)}x\) 的单调性;
      \item 若 \(f(x)\le0\) 在 \((0,+\infty)\) 上恒成立,求 \(a\) 的值。
    \end{enumerate}
  \item 当 \(a\ge b>0\) 时,证明:函数 \(f(x)\) 有两个极值点 \(x_1,\ x_2\,(x_1<x_2)\) 且 \(x_2<\left(\dfrac{2ea}{b}\right)^2x_1\)。
\end{enumerate}
\topics{利用导数证明不等式;利用导数研究不等式恒成立问题;利用导数求函数(含参)的单调区间}
\difficulty{0.15}
\answer{(1) (i) 单调性见详解;(ii) \(a=0\);(2) 证明略}
\explain{(1) (i) 当 \(b=2\) 时,\(g(x)=\dfrac{f(x)}x=a\ln x-\Bigl(x+\dfrac1x-2\Bigr)\)。故
\[g'(x)=\frac{a}{x}-1+\frac{1}{x^2}=\frac{-x^2+ax+1}{x^2}.\]
记 \(m(x)=-x^2+ax+1\) 为开口向下二次函数。由 \(\Delta=a^2+4>0\) 且 \(m(0)=1>0\),存在唯一正根 \(x_0=\dfrac{a+\sqrt{a^2+4}}{2}\)。于是 \(x\in(0,x_0)\) 时 \(m(x)>0\Rightarrow g'(x)>0\),\(g\) 递增;\(x\in(x_0,+\infty)\) 时 \(g'(x)<0\),\(g\) 递减。\\
(ii) \(f(x)\le0\) 在 \((0,+\infty)\) 恒成立等价于 \(g(x)\le0\) 在 \((0,+\infty)\) 恒成立。注意到 \(g(1)=0\)。若 \(a>0\),则 \(x_0>1\),在 \((1,x_0)\) 上 \(g\) 递增,故 \(g(x)>g(1)=0\),矛盾;若 \(a<0\),则 \(0<x_0<1\),在 \((x_0,1)\) 上 \(g\) 递减,亦有 \(g(x)>g(1)=0\),矛盾。唯有 \(a=0\) 时成立。\\
(2) \(f'(x)=a\ln x+a-b(x-1)\)。记 \(H(x)=f'(x)\),则 \(H'(x)=\dfrac{a}{x}-b\)。令 \(H'(x)=0\) 得 \(x=\dfrac{a}{b}\)。故 \(0<x<\dfrac{a}{b}\) 时 \(f'\) 递增,\(x>\dfrac{a}{b}\) 时 \(f'\) 递减。又 \(f'(1)=a>0\),从而 \(f'\bigl(\tfrac{a}{b}\bigr)\ge f'(1)>0\)。当 \(0<x<1\) 时,\(f'(x)<a\ln x+a+b\le a(\ln x+2)\),故 \(f'(e^{-2})<0\),存在 \(x_1\in(e^{-2},1)\) 使 \(f'(x_1)=0\)。当 \(x>1\) 时,由 \(\ln x<x-1\) 得 \(\ln\sqrt{x}<\sqrt{x}-1\Rightarrow \ln x<2\sqrt{x}-2\)。于是
\[f'(x)<a(2\sqrt{x}-2)-b(x-1)=-bx+2a\sqrt{x}-1,\]
从而 \(f'\bigl((\tfrac{2a}{b})^2\bigr)<0\)。故存在 \(x_2\in\bigl(1,(\tfrac{2a}{b})^2\bigr)\) 使 \(f'(x_2)=0\)。据此:\(x\in(0,x_1)\) 递减,\(x\in(x_1,x_2)\) 递增,\(x\in(x_2,+\infty)\) 递减,\(f\) 有两极值点 \(x_1<x_2\)。又有 \(e^{-2}<x_1<1<x_2<(\tfrac{2a}{b})^2\),故
\[\frac{x_2}{x_1}<\frac{(\tfrac{2a}{b})^2}{e^{-2}}=\left(\frac{2ea}{b}\right)^2,\]
即 \(x_2<\left(\tfrac{2ea}{b}\right)^2x_1\)。}
\end{question}
