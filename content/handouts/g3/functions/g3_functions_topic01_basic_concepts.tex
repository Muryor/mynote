%% content/handouts/g3/functions/g3_functions_topic01_basic_concepts.tex
%% 高三函数专题(一):基本概念与性质

\chapter{函数的基本概念与性质}

\begin{definitionx}{函数的定义}{def:function}
设 \(A\)、\(B\) 是非空数集,如果按某种对应关系 \(f\),使得对于集合 \(A\) 中的\textbf{任意}一个数 \(x\),
在集合 \(B\) 中都有\textbf{唯一确定}的数 \(f(x)\) 与之对应,那么就称 \(f: A \to B\) 为从集合 \(A\) 到集合 \(B\) 的一个\textbf{函数},
记作 \(y=f(x),\, x\in A\)。

其中 \(x\) 叫做\textbf{自变量},\(A\) 叫做函数的\textbf{定义域},集合 \(\{y\mid y=f(x),\, x\in A\}\) 叫做函数的\textbf{值域}。
\end{definitionx}

\begin{propertyx}{函数的单调性}{prop:monotonicity}
设函数 \(f(x)\) 的定义域为 \(D\),区间 \(I \subseteq D\)。

\textbf{单调递增}:如果对于区间 \(I\) 内的任意两个自变量 \(x_1, x_2\),当 \(x_1 < x_2\) 时,都有 \(f(x_1) < f(x_2)\),
那么就说 \(f(x)\) 在区间 \(I\) 上是\textbf{增函数},\(I\) 叫做 \(f(x)\) 的\textbf{单调递增区间}。

\textbf{单调递减}:如果对于区间 \(I\) 内的任意两个自变量 \(x_1, x_2\),当 \(x_1 < x_2\) 时,都有 \(f(x_1) > f(x_2)\),
那么就说 \(f(x)\) 在区间 \(I\) 上是\textbf{减函数},\(I\) 叫做 \(f(x)\) 的\textbf{单调递减区间}。
\end{propertyx}

\begin{examplex}{例1}{ex:monotonicity-01}
判断函数 \(f(x) = x^2 - 2x + 3\) 在区间 \((-\infty, 1]\) 和 \([1, +\infty)\) 上的单调性。

\topics{二次函数;单调性;对称轴}
\difficulty{0.5}
\answer{在 \((-\infty, 1]\) 上单调递减,在 \([1, +\infty)\) 上单调递增}
\explain{配方得 \(f(x) = (x-1)^2 + 2\),对称轴为 \(x=1\)。当 \(x < 1\) 时,\(f(x)\) 随 \(x\) 增大而减小,即在 \((-\infty, 1]\) 上单调递减;当 \(x > 1\) 时,\(f(x)\) 随 \(x\) 增大而增大,即在 \([1, +\infty)\) 上单调递增。}
\end{examplex}

\begin{propertyx}{函数的奇偶性}{prop:parity}
设函数 \(f(x)\) 的定义域 \(D\) 关于原点对称。

\textbf{偶函数}:如果对于任意 \(x \in D\),都有 \(f(-x) = f(x)\),那么函数 \(f(x)\) 就叫做\textbf{偶函数}。
偶函数的图像关于 \(y\) 轴对称。

\textbf{奇函数}:如果对于任意 \(x \in D\),都有 \(f(-x) = -f(x)\),那么函数 \(f(x)\) 就叫做\textbf{奇函数}。
奇函数的图像关于原点对称。
\end{propertyx}

\begin{examplex}{例2}{ex:parity-01}
判断下列函数的奇偶性:(1) \(f(x) = x^3 - x\);(2) \(g(x) = x^2 + 1\);(3) \(h(x) = \dfrac{1}{x}\)。

\topics{函数的奇偶性;定义域对称性}
\difficulty{0.4}
\answer{(1) 奇函数;(2) 偶函数;(3) 奇函数}
\explain{(1) \(f(-x) = (-x)^3 - (-x) = -x^3 + x = -f(x)\),故为奇函数。(2) \(g(-x) = (-x)^2 + 1 = x^2 + 1 = g(x)\),故为偶函数。(3) \(h(-x) = \dfrac{1}{-x} = -\dfrac{1}{x} = -h(x)\),故为奇函数。}
\end{examplex}

\begin{notebox}{关键提示}
判断函数奇偶性的步骤:
\begin{enumerate}
  \item 首先检查定义域是否关于原点对称;
  \item 若对称,再计算 \(f(-x)\) 并与 \(f(x)\) 比较;
  \item 若 \(f(-x) = f(x)\) 则为偶函数,若 \(f(-x) = -f(x)\) 则为奇函数。
\end{enumerate}
\end{notebox}
