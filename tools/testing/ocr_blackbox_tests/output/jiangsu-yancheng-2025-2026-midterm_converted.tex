\examxtitle{测试试卷 - jiangsu-yancheng-2025-2026-midterm}

\section{单选题}

\begin{question}
若集合\(P = \left\{ x|y = \sqrt{1 - x} \right\}\),
\(Q = \left\{ y|y = e^{x} \right\}\),
则\(P \cap Q =\)(    )
\begin{choices}
  \item \(\lbrack 0, + \infty)\)
  \item \((0, + \infty)\)
  \item \((0,1)\)
  \item \((0,1\rbrack\)
\end{choices}
\topics{交集的概念及运算;具体函数的定义域;求指数函数在区间内的值域}
\difficulty{0.65}
\answer{D}
\explain{由题可知\(P = \left\{ x|x \leq 1 \right\}\),
\(Q = \left\{ y|y > 0 \right\}\),
所以\(P \cap Q = \left\{ x|0 < x \leq 1 \right\} = (0,1\rbrack\)}
\end{question}

\begin{question}
已知复数\(z = \left( \sqrt{3} - \text{i} \right)^{2}\),则\(|z| =\)(    )
\begin{choices}
  \item 2
  \item \(2\sqrt{7}\)
  \item \(4\sqrt{3}\)
  \item 4
\end{choices}
\topics{求复数的模;复数的乘方}
\difficulty{0.85}
\answer{D}
\explain{因为\(z = \left( \sqrt{3} - \text{i} \right)^{2} = 3 - 2\sqrt{3}\text{i} + \text{i}^{2} = 2 - 2\sqrt{3}\text{i}\),
所以\(|z| = \sqrt{2^{2} + \left( - 2\sqrt{3} \right)^{2}} = 4\)}
\end{question}

\begin{question}
已知向量\(\overrightarrow{a}\),
\(\overrightarrow{b}\)满足\(\left| \overrightarrow{a} \right| = 2\),
\(\left| \overrightarrow{b} \right| = 1\),
若\(\overrightarrow{b}\bot\left( \overrightarrow{b} - \overrightarrow{a} \right)\),
则\(\overrightarrow{a}\)与\(\overrightarrow{b}\)的夹角为(    )
\begin{choices}
  \item \(\frac{\pi}{6}\)
  \item \(\frac{\pi}{3}\)
  \item \(\frac{2\pi}{3}\)
  \item \(\frac{5\pi}{6}\)
\end{choices}
\topics{向量夹角的计算;垂直关系的向量表示}
\difficulty{0.85}
\answer{B}
\explain{因为\(\overrightarrow{b}\bot\left( \overrightarrow{b} - \overrightarrow{a} \right)\),
所以\(\overrightarrow{b} \cdot \left( \overrightarrow{b} - \overrightarrow{a} \right) = 0\),
即\(\overrightarrow{b} \cdot \overrightarrow{b} - \overrightarrow{a} \cdot \overrightarrow{b} = 0\),
所以\(\overrightarrow{a} \cdot \overrightarrow{b} = \overrightarrow{b} \cdot \overrightarrow{b} = \left| \overrightarrow{b} \right|^{2} = 1\)\par
所以\(\overrightarrow{a}\)与\(\overrightarrow{b}\)的夹角的余弦值为\(\cos\left\langle \overrightarrow{a},\overrightarrow{b} \right\rangle = \frac{\overrightarrow{a} \cdot \overrightarrow{b}}{\left| \overrightarrow{a} \right| \cdot \left| \overrightarrow{b} \right|} = \frac{1}{2}\)
,
所以\(\overrightarrow{a}\)与\(\overrightarrow{b}\)的夹角为\(\frac{\pi}{3}\)}
\end{question}

\begin{question}
设\(S_{n}\)是等差数列\(\left\{ a_{n} \right\}\)的前\(n\)项和,
若\(a_{3} + a_{4} = 50\),
则\(S_{6} =\)(    )
\begin{choices}
  \item 50
  \item 100
  \item 150
  \item 200
\end{choices}
\topics{利用等差数列的性质计算;求等差数列前n项和}
\difficulty{0.85}
\answer{C}
\explain{因为\(\left\{ a_{n} \right\}\)为等差数列,
且\(a_{3} + a_{4} = 50\),
所以\(a_{1} + a_{6} = a_{3} + a_{4} = 50\),\par
所以\(S_{6} = \frac{6\left( a_{1} + a_{6} \right)}{2} = 3\left( a_{1} + a_{6} \right) = 150\)}
\end{question}

\begin{question}
已知\(\bigtriangleup ABC\)中,"\(\sin A > \sin B\)"是"\(\cos A < \cos B\)"成立的(    )
\begin{choices}
  \item 充分不必要条件
  \item 必要不充分条件
  \item 充分必要条件
  \item 既不充分也不必要条件
\end{choices}
\topics{充要条件的证明;比较余弦值的大小;正弦定理及辨析}
\difficulty{0.85}
\answer{C}
\explain{由正弦定理以及三角形大边对大角可得:\par
\(\sin A > \sin B \Leftrightarrow a > b \Leftrightarrow A > B\),\par
又\(A,B \in (0,\pi)\),
\(y = \cos x\)在\((0,\pi)\)上单调递减,\par
\(\therefore A > B \Leftrightarrow \cos A < \cos B\),
即\(\sin A > \sin B \Leftrightarrow \cos A < \cos B\),\par
\(\therefore\)"\(\sin A > \sin B\)"是"\(\cos A < \cos B\)"成立的充分必要条件}
\end{question}

\begin{question}
已知函数\(f(x) = x^{3} - ax\),
若\(\lim_{\Delta x arrow 0}\frac{f(1 + 2\Delta x) - f(1)}{\Delta x} = 1\),
则实数\(a =\)(    )
\begin{choices}
  \item \(\frac{5}{2}\)
  \item \(2\)
  \item \(\frac{3}{2}\)
  \item \(1\)
\end{choices}
\topics{导数定义中极限的简单计算;导数的加减法}
\difficulty{0.85}
\answer{A}
\explain{因为\(f(x) = x^{3} - ax\),
故\(f'(x) = 3x^{2} - a\),\par
所以\(\lim_{\Delta x arrow 0}\frac{f(1 + 2\Delta x) - f(1)}{\Delta x} = 2\lim_{\Delta x arrow 0}\frac{f(1 + 2\Delta x) - f(1)}{2\Delta x} = 2f'(1) = 1\),\par
可得\(f'(1) = 3 - a = \frac{1}{2}\),
解得\(a = \frac{5}{2}\)}
\end{question}

\begin{question}
已知\(\sin\alpha - \cos\left( \beta + \frac{\pi}{4} \right) = 2\),
则\(\tan(\beta - \alpha) =\)(    )
\begin{choices}
  \item \(0\)
  \item \(- 1\)
  \item \(1\)
  \item \(\pm 1\)
\end{choices}
\topics{诱导公式一;由正弦(型)函数的值域(最值)求参数;由cosx(型)函数的值域(最值)求参数}
\difficulty{0.65}
\answer{C}
\explain{若\(\sin\alpha - \cos\left( \beta + \frac{\pi}{4} \right) = 2\),
且\(- 1 \leq \sin\alpha \leq 1\),
\(- 1 \leq \cos\left( \beta + \frac{\pi}{4} \right) \leq 1\),\par
所以\(\sin\alpha = 1\),
\(\cos\left( \beta + \frac{\pi}{4} \right) = - 1\),\par
所以\(\alpha = \frac{\pi}{2} + 2k\pi(k \in Z)\),
\(\beta + \frac{\pi}{4} = \pi + 2n\pi(n \in Z)\),
则\(\beta = \frac{3\pi}{4} + 2n\pi(n \in Z)\),\par
故\(\tan(\beta - \alpha) = \tan\left( \frac{3\pi}{4} + 2n\pi - \frac{\pi}{2} - 2k\pi \right) = \tan\left( \frac{\pi}{4} + 2n\pi - 2k\pi \right) = \tan\frac{\pi}{4} = 1\),
其中\(k\)、\(n \in Z\)}
\end{question}

\begin{question}
对于问题:若正数\(a\)、\(b\)满足\(a + b = 1\),
求\(\frac{1}{a} + \frac{2}{b}\)的最小值.有一种常规解法:\(\frac{1}{a} + \frac{2}{b} = (a + b)\left( \frac{1}{a} + \frac{2}{b} \right) = 1 + \frac{b}{a} + \frac{2a}{b} + 2 \geq 3 + 2\sqrt{2}\),
当且仅当\(\frac{b}{a} = \frac{2a}{b}\)且\(a + b = 1\)时,
即\(a = \sqrt{2} - 1\)且\(b = 2 - \sqrt{2}\)时,
等号成立.请运用上述方法,
解决下列问题:若实数\(a\)、\(b\)、\(x\)、\(y\)满足\(\frac{x^{2}}{a^{2}} - \frac{y^{2}}{4b^{2}} = 1\left( a^{2} > b^{2} \right)\),
设\(M = a^{2} - b^{2}\),
\(N = \left( x + \frac{y}{2} \right)^{2}\),
则\(M\)、\(N\)的大小关系为(    )
\begin{choices}
  \item \(M = N\)
  \item \(M \leq N\)
  \item \(M \geq N\)
  \item 不确定
\end{choices}
\topics{由基本不等式比较大小}
\difficulty{0.65}
\answer{B}
\explain{因为\(M = a^{2} - b^{2} = \left( a^{2} - b^{2} \right)\left( \frac{x^{2}}{a^{2}} - \frac{y^{2}}{4b^{2}} \right) = x^{2} + \frac{y^{2}}{4} - \left( \frac{b^{2}x^{2}}{a^{2}} + \frac{a^{2}y^{2}}{4b^{2}} \right)\leq x^{2} + \frac{y^{2}}{4} - 2\sqrt{\frac{b^{2}x^{2}}{a^{2}} \cdot \frac{a^{2}y^{2}}{4b^{2}}}= x^{2} + \frac{y^{2}}{4} - |xy| \leq x^{2} + \frac{y^{2}}{4} + xy = \left( x + \frac{y}{2} \right)^{2} = N\),
即\(M \leq N\),\par
当且仅当\(\frac{b^{2}x^{2}}{a^{2}} = \frac{a^{2}y^{2}}{4b^{2}}\)且\(xy \leq 0\)时,
上述不等式中的两个等号同时成立,故\(M \leq N\)}
\end{question}

\section{多选题}

\begin{question}
若\(\overrightarrow{a}\)、\(\overrightarrow{b}\)、\(\overrightarrow{c}\)是非零向量,
则下列说法正确的是(    )
\begin{choices}
  \item \(\left( \overrightarrow{a} + \overrightarrow{b} \right) \cdot \overrightarrow{c} = \overrightarrow{a} \cdot \overrightarrow{c} + \overrightarrow{b} \cdot \overrightarrow{c}\)
  \item \(\left| \overrightarrow{a} \right|^{2} = {\overrightarrow{a}}^{2}\)
  \item 若\(\overrightarrow{a} \cdot \overrightarrow{b} = \overrightarrow{a} \cdot \overrightarrow{c}\),则\(\overrightarrow{b} = \overrightarrow{c}\)
  \item \(\left| \overrightarrow{a} - \overrightarrow{b} \right| \leq \left| \overrightarrow{a} \right| + \left| \overrightarrow{b} \right|\)
\end{choices}
\topics{向量减法法则的几何应用;用定义求向量的数量积;数量积的运算律;垂直关系的向量表示}
\difficulty{0.85}
\answer{ABD}
\explain{对于A选项,
\(\left( \overrightarrow{a} + \overrightarrow{b} \right) \cdot \overrightarrow{c} = \overrightarrow{a} \cdot \overrightarrow{c} + \overrightarrow{b} \cdot \overrightarrow{c}\),
A对;\par
对于B选项,
\(\left| \overrightarrow{a} \right|^{2} = {\overrightarrow{a}}^{2}\),
B对;\par
对于C选项,
若\(\overrightarrow{a} \cdot \overrightarrow{b} = \overrightarrow{a} \cdot \overrightarrow{c}\),
则\(\overrightarrow{a} \cdot \left( \overrightarrow{b} - \overrightarrow{c} \right) = 0\),\par
所以\(\overrightarrow{b} = \overrightarrow{c}\)或当\(\overrightarrow{b} \neq \overrightarrow{c}\)时,
\(\overrightarrow{a}\bot\left( \overrightarrow{b} - \overrightarrow{c} \right)\),
C错;\par
对于D选项,
\(\left| \overrightarrow{a} - \overrightarrow{b} \right| \leq \left| \overrightarrow{a} \right| + \left| \overrightarrow{b} \right|\),
当且仅当\(\overrightarrow{a}\)、\(\overrightarrow{b}\)方向相反时,
等号成立,D对}
\end{question}

\begin{question}
若数列\(\left\{ a_{n} \right\}\)的首项\(a_{1} = \frac{2}{3}\),
且\(a_{n + 1} = \frac{2a_{n}}{a_{n} + 1}\),
则(    )
\begin{choices}
  \item 数列\(\left\{ \frac{1}{a_{n}} - 1 \right\}\)为等比数列
  \item \(a_{n} = \frac{2^{n}}{2n + 1}\)
  \item 数列\(\left\{ a_{n} \right\}\)为递增数列
  \item 存在正整数\(m\)、\(n\)、\(l\),使得\(a_{m} + a_{n} = a_{l}\)
\end{choices}
\topics{判断数列的增减性;写出等比数列的通项公式;由定义判定等比数列}
\difficulty{0.65}
\answer{AC}
\explain{对于A选项,因为\(a_{1} = \frac{2}{3}\),
由\(a_{n + 1} = \frac{2a_{n}}{a_{n} + 1}\)可知\(a_{2} > 0\),
\(a_{3} > 0\),\(\cdots\),\par
以此类推可知,对任意的\(n \in N^{\ast}\),
\(a_{n} > 0\),\par
在等式\(a_{n + 1} = \frac{2a_{n}}{a_{n} + 1}\)两边同时取倒数得\(\frac{1}{a_{n + 1}} = \frac{1 + a_{n}}{2a_{n}} = \frac{1}{2a_{n}} + \frac{1}{2}\),\par
所以\(\frac{1}{a_{n + 1}} - 1 = \frac{1}{2}\left( \frac{1}{a_{n}} - 1 \right)\),
且\(\frac{1}{a_{1}} - 1 = \frac{3}{2} - 1 = \frac{1}{2}\),\par
所以数列\(\left\{ \frac{1}{a_{n}} - 1 \right\}\)是首项为\(\frac{1}{2}\),
公比为\(\frac{1}{2}\)的等比数列,A对;\par
对于B选项,
由A选项可知\(\frac{1}{a_{n}} - 1 = \frac{1}{2} \times \left( \frac{1}{2} \right)^{n - 1} = \frac{1}{2^{n}}\),
故\(a_{n} = \frac{2^{n}}{2^{n} + 1}\),B错;\par
对于C选项,
\(a_{n + 1} - a_{n} = \frac{2^{n + 1}}{2^{n + 1} + 1} - \frac{2^{n}}{2^{n} + 1} = \frac{2^{n + 1}\left( 2^{n} + 1 \right) - 2^{n}\left( 2^{n + 1} + 1 \right)}{\left( 2^{n} + 1 \right)\left( 2^{n + 1} + 1 \right)} = \frac{2^{n}}{\left( 2^{n} + 1 \right)\left( 2^{n + 1} + 1 \right)} > 0\),\par
即\(a_{n + 1} > a_{n}\),
故数列\(\left\{ a_{n} \right\}\)为递增数列,C对;\par
对于D选项,假设正整数\(m\)、\(n\)、\(l\),
使得\(a_{m} + a_{n} = a_{l}\),\par
由于数列\(\left\{ a_{n} \right\}\)单调递增,
不妨设\(m \leq n < l\),
则\(\frac{2^{m}}{2^{m} + 1} + \frac{2^{n}}{2^{n} + 1} = \frac{2^{l}}{2^{l} + 1}\),\par
所以\(2^{m}\left( 2^{n} + 1 \right)\left( 2^{l} + 1 \right) + 2^{n}\left( 2^{m} + 1 \right)\left( 2^{l} + 1 \right) = 2^{l}\left( 2^{n} + 1 \right)\left( 2^{m} + 1 \right)\),\par
所以\(\left( 2^{n} + 1 \right)\left( 2^{l} + 1 \right) + 2^{n - m}\left( 2^{m} + 1 \right)\left( 2^{l} + 1 \right) = 2^{l - m}\left( 2^{n} + 1 \right)\left( 2^{m} + 1 \right)\)①,\par
若\(m = n\),
则\(\left( 2^{n} + 1 \right)\left( 2^{l} + 1 \right) + \left( 2^{m} + 1 \right)\left( 2^{l} + 1 \right) = 2^{l - m}\left( 2^{n} + 1 \right)\left( 2^{m} + 1 \right)\),\par
所以\(2 \times 2^{l} + 2 = 2^{l} + 2^{l - m}\),
所以\(2^{l} + 2 = 2^{l - m}\),
即\(2^{l - 1} + 1 = 2^{l - m - 1}\),\par
若\(l = m + 1\),则\(2^{l - 1} + 1 = 1\),
即\(2^{l - 1} = 0\)矛盾,\par
若\(l - m - 1 \geq 1\),
则\(2^{l - 1} + 1\)为奇数,
\(2^{l - m - 1}\)为偶数,
等式\(2^{l - 1} + 1 = 2^{l - m - 1}\)不成立,\par
若\(m < n\),等式①的左边为奇数,右边为偶数,不成立,\par
综上所述,不存在\(m\)、\(n\)、\(l\),
使得\(a_{m} + a_{n} = a_{l}\),D错}
\end{question}

\begin{question}
若\(\bigtriangleup ABC\)的外接圆半径为2,
且\(\sin A\sin B\sin C = \frac{\sqrt{3}}{8}\),
则(    )
\begin{choices}
  \item \(abc = \sqrt{3}\)
  \item \(\bigtriangleup ABC\)的面积为\(\sqrt{3}\)
  \item 当\(B = \frac{\pi}{3}\)时,则\(|A - C| = \frac{\pi}{2}\)
  \item \(\bigtriangleup ABC\)可能是等腰三角形
\end{choices}
\topics{用和;差角的余弦公式化简;求值;正弦定理解三角形;三角形面积公式及其应用}
\difficulty{0.4}
\answer{BCD}
\explain{由\(\sin A\sin B\sin C = \frac{\sqrt{3}}{8}\),
得\((2R)^{3}\sin A\sin B\sin C = 4^{3} \times \frac{\sqrt{3}}{8}\),\par
所以\(abc = 8\sqrt{3}\),故A错误;\par
\(\bigtriangleup ABC\)的面积为\(\frac{1}{2}ab\sin C = \frac{1}{2} \times 2R\sin A \times 2R\sin B \times \sin C = 8 \times \frac{\sqrt{3}}{8} = \sqrt{3}\),
故B正确;\par
当\(B = \frac{\pi}{3}\)时,
由已知得\(\sin A\sin\frac{\pi}{3}\sin C = \frac{\sqrt{3}}{8}\),\par
所以\(\sin A\sin C = \frac{1}{4}\),
所以\(\frac{1}{2}\left\lbrack \cos(A - C) - \cos(A + C) \right\rbrack = \frac{1}{4}\),\par
所以\(\frac{1}{2}\left\lbrack \cos(A - C) - \cos\frac{2\pi}{3} \right\rbrack = \frac{1}{4}\),
所以\(\cos(A - C) = 0\),\par
又因为\(- \frac{2\pi}{3} < A - C < \frac{2\pi}{3}\),
所以\(A - C = \pm \frac{\pi}{2}\),
所以\(|A - C| = \frac{\pi}{2}\),故C正确;\par
取\(A = \frac{2\pi}{3},B = C = \frac{\pi}{6}\),
满足\(\sin A\sin B\sin C = \frac{\sqrt{3}}{8}\),
所以\(\bigtriangleup ABC\)可能是等腰三角形,故D正确}
\end{question}

\section{填空题}

\begin{question}
写出一个相邻对称轴间距离为\(2\)的函数\(f(x) =\) .
\topics{利用正弦函数的对称性求参数}
\difficulty{0.85}
\answer{\(\sin\frac{\pix}{2}\)(答案不唯一).}
\explain{不妨取函数\(f(x) = \sin\omega x(\omega > 0)\),
由题意可知函数\(f(x)\)的最小正周期为\(4\),\par
故\(\omega = \frac{2\pi}{4} = \frac{\pi}{2}\),
故\(f(x) = \sin\frac{\pix}{2}\).\(f(x) = \sin\frac{\pix}{2}\)(答案不唯一).}
\end{question}

\begin{question}
设等比数列\(\left\{ a_{n} \right\}\)的前\(n\)项和为\(S_{n}\),
若公比\(q = 2\),
则\(\frac{S_{9} - S_{6}}{S_{3}} =\)
.
\topics{求等比数列前n项和;等比数列片段和性质及应用}
\difficulty{0.65}
\answer{64}
\explain{由等比数列的性质得\(\frac{S_{9} - S_{6}}{S_{3}} = \frac{a_{7} + a_{8} + a_{9}}{a_{1} + a_{2} + a_{3}} = \frac{a_{1}q^{6} + a_{2}q^{6} + a_{3}q^{6}}{a_{1} + a_{2} + a_{3}} = q^{6} = 64\).64.}
\end{question}

\begin{question}
已知函数\(f(x) = \left\{ \begin{array}{r}
x + \mathrm{e}^{b},x \leq 0 \\
x\ln x + 2bx,x > 0
\end{array} \right.\).若存在实数\(b\),使得\(f(x + m) \geq f(x)\)对任意实数\(x\)恒成立,则正实数\(m\)的最小值为
.

\begin{center}
% IMAGE_TODO_START id=jiangsu-yancheng-2025-2026-midterm-Q14-img1 path=/Users/muryor/code/mynote/word\\_to\\_tex/output/figures/jiangsu-yancheng-2025-2026-midterm/media/image2.png width=60% inline=false question_index=14 sub_index=1
% CONTEXT_BEFORE: = ^{- 2b}$$时,$$y_{2} = 0$$,可作出函数的大致图象如下:
% CONTEXT_AFTER: 显然要满足题意需整个函数图象向左平移后完全在原函数图象上方即可, 考虑临界情况,即左移后的右半
\begin{tikzpicture}[scale=1.05,>=Stealth,line cap=round,line join=round]
  % TODO: AI_AGENT_REPLACE_ME (id=jiangsu-yancheng-2025-2026-midterm-Q14-img1)
\end{tikzpicture}
% IMAGE_TODO_END id=jiangsu-yancheng-2025-2026-midterm-Q14-img
1
\end{center}

\topics{分段函数的性质及应用;利用导数研究不等式恒成立问题}
\difficulty{0.15}
\answer{\(\frac{3\sqrt[3]{2}}{2}\)}
\explain{易知\(y_{1} = x + \mathrm{e}^{b},x \leq 0\)时单调递增,\par
对于\(y_{2} = x\ln x + 2bx,x > 0\),
\({y'}_{2} = \ln x + 2b + 1\),\par
易得\(x \in \left( 0,\mathrm{e}^{- 2b - 1} \right)\)时\({y'}_{2} < 0\),
\(x \in \left( \mathrm{e}^{- 2b - 1}, + \infty \right)\)时\({y'}_{2} > 0\),\par
即\(y_{2}\)在\(\left( 0,\mathrm{e}^{- 2b - 1} \right)\)上单调递减,
在\(\left( \mathrm{e}^{- 2b - 1}, + \infty \right)\)上单调递增,
即\(y_{2} \geq - \mathrm{e}^{- 2b - 1}\),\par
又\(x = \mathrm{e}^{- 2b}\)时,
\(y_{2} = 0\),可作出函数的大致图象如下:\par
显然要满足题意需整个函数图象向左平移后完全在原函数图象上方即可,\par
考虑临界情况,
即左移后的右半段函数\(y = (x + m)\ln(x + m) + 2b(x + m)\)与平移前的左半段函数\(y_{1} = x + \mathrm{e}^{b}\)相切,
此时平移距离最短,\par
为方便计算,
可转化为\(y_{2} = x\ln x + 2bx\)与\(y_{3} = x - m + \mathrm{e}^{b}\)相切,\par
不妨设切点为\(\left( x_{0},y_{0} \right)\),\par
由上可知\(\ln x_{0} + 2b + 1 = 1 \Rightarrow x_{0} = \mathrm{e}^{- 2b}\),
即切点为\(\left( \mathrm{e}^{- 2b},0 \right)\),\par
则\(m = \mathrm{e}^{- 2b} - \left( - \mathrm{e}^{b} \right) = \mathrm{e}^{- 2b} + \frac{\mathrm{e}^{b}}{2} + \frac{\mathrm{e}^{b}}{2} \geq 3\sqrt[3]{\mathrm{e}^{- 2b} \cdot \frac{\mathrm{e}^{b}}{2} \cdot \frac{\mathrm{e}^{b}}{2}} = \frac{3\sqrt[3]{2}}{2}\),\par
当且仅当\(\mathrm{e}^{- 2b} = \frac{\mathrm{e}^{b}}{2}\)即\(b = \frac{\ln 2}{3}\)时,
\emph{m}取得最小值\(\frac{3\sqrt[3]{2}}{2}\).\par
下证:\(b = \frac{\ln 2}{3}\),
且\(m = \frac{3\sqrt[3]{2}}{2}\)时,
\(x\ln x + 2bx \geq x - m + \mathrm{e}^{b}\)恒成立,\par
令\(g(x) = x\ln x + 2bx - x + m - \mathrm{e}^{b}\),
则\(g'(x) = \ln x + 2b\),\par
易得\(x \in \left( 0,\mathrm{e}^{- 2b} \right)\)时\(g'(x) < 0\),
\(x \in \left( \mathrm{e}^{- 2b}, + \infty \right)\)时\(g'(x) > 0\),\par
即\(g(x)\)在\(\left( 0,\mathrm{e}^{- 2b} \right)\)上单调递减,
在\(\left( \mathrm{e}^{- 2b}, + \infty \right)\)上单调递增,\par
即\(g(x) \geq g\left( \mathrm{e}^{- 2b} \right) = - 2b\mathrm{e}^{- 2b} + 2b\mathrm{e}^{- 2b} - \mathrm{e}^{- 2b} + m - \mathrm{e}^{b} = - 2^{- \frac{2}{3}} + \frac{3}{2} \times 2^{\frac{1}{3}} - 2^{\frac{1}{3}} = 0\),\par
所以\(y_{2} = x\ln x + 2bx\)的图象恒在\(y_{3} = x - m + \mathrm{e}^{b}\)的图象上方(除切点处有交点),\par
即从临界处分析平移距离符合要求.\(\frac{3\sqrt[3]{2}}{2}\).}
\end{question}

\section{解答题}

\begin{question}
已知二次函数\(f(x) = 4x^{2} + bx + c\),
且关于\(x\)的不等式\(f(x) \leq 0\)的解集为\(\left\lbrack \frac{1}{4},1 \right\rbrack\).
\begin{enumerate}[label=(\arabic*)]
  \item 求实数\(b\),\(c\)的值;
  \item 若关于\(x\)的方程\(f\left( 3^{x} \right) = m \cdot 3^{x}\)有解,
\item 求实数\(m\)的取值范围.
\end{enumerate}
\topics{函数与方程的综合应用;由一元二次不等式的解确定参数}
\difficulty{0.65}
\answer{(1)实数\(b\),\(c\)的值分别为\(- 5,1\)
(2)\(\lbrack - 1, + \infty)\)}
\explain{(1)因为关于\(x\)的不等式\(f(x) \leq 0\)的解集为\(\left\lbrack \frac{1}{4},1 \right\rbrack\),
所以\(\frac{1}{4},1\)是方程\(4x^{2} + bx + c = 0\)的两根,\par
由根与系数的关系得\(\left\{ \begin{array}{r}
\frac{1}{4} + 1 = - \frac{b}{4} \\
\frac{1}{4} \times 1 = \frac{c}{4}
\end{array} \right.\),解得\(\left\{ \begin{array}{r}
b = - 5 \\
c = 1
\end{array} \right.\),\par
所以实数\(b\),\(c\)的值分别为\(- 5,1\);\par
(2)由(1)可得\(f(x) = 4x^{2} - 5x + 1\),结合\(f\left( 3^{x} \right) = m \cdot 3^{x}\),\par
可得\(4\left( 3^{x} \right)^{2} - 5\left( 3^{x} \right) + 1 = m \cdot 3^{x}\),所以\(m = 4 \times 3^{x} - 5 + \frac{1}{3^{x}}\),\par
所以\(m = 4 \times 3^{x} - 5 + \frac{1}{3^{x}} \geq 2\sqrt{4 \times 3^{x} \times \frac{1}{3^{x}}} - 5 = - 1\),\par
当且仅当\(4 \times 3^{x} = \frac{1}{3^{x}}\),即\(3^{x} = \frac{1}{2}\)时取等号,\par
所以实数\(m\)的取值范围为\(\lbrack - 1, + \infty)\).}
\end{question}

\begin{question}
已知向量\(\overrightarrow{a} = \left( \sin 2x,\cos 2x \right)\),
\(\overrightarrow{b} = \left( 1,\sqrt{3} \right)\),
记函数\(f(x) = \overrightarrow{a} \cdot \overrightarrow{b}\).
\begin{enumerate}[label=(\arabic*)]
  \item 求函数\(f(x)\)的对称中心;
  \item 将函数\(f(x)\)图象上的所有点的纵坐标保持不变,
\item 横坐标伸长为原来的2倍,得到函数\(g(x)\)的图象,
\item 若\(g\left( x_{0} \right) = \frac{4}{5}\),
\item 求\(f\left( \frac{\pi}{4} - x_{0} \right)\)的值.
\end{enumerate}
\topics{求正弦(型)函数的对称轴及对称中心;求图象变化前(后)的解析式;二倍角的余弦公式;数量积的坐标表示}
\difficulty{0.4}
\answer{(1)\(\left( - \frac{\pi}{6} + \frac{k\pi}{2},0 \right)k \in Z\)
(2)\(- \frac{34}{25}\)}
\explain{(1)因为\(f(x) = \overrightarrow{a} \cdot \overrightarrow{b} = \sin 2x + \sqrt{3}\cos 2x = 2\sin\left( 2x + \frac{\pi}{3} \right)\)\par
令\(2x + \frac{\pi}{3} = k\pi,k \in Z\),
解得\(x = - \frac{\pi}{6} + \frac{k\pi}{2},k \in Z\),\par
所以函数\(f(x)\)的对称中心为\(\left( - \frac{\pi}{6} + \frac{k\pi}{2},0 \right),k \in Z\);\par
(2)因为将函数\(f(x)\)图象上的所有点的纵坐标保持不变,
横坐标伸长为原来的2倍,得到函数\(g(x)\)的图象,
所以\(g(x) = 2\sin\left( x + \frac{\pi}{3} \right)\);\par
又因为\(g\left( x_{0} \right) = \frac{4}{5}\),
所以\(2\sin\left( x_{0} + \frac{\pi}{3} \right) = \frac{4}{5}\),
所以\(\sin\left( x_{0} + \frac{\pi}{3} \right) = \frac{2}{5}\),\par
所以\par
\(\begin{array}{r}
f\left( \frac{\pi}{4} - x_{0} \right) = 2\sin\left\lbrack 2\left( \frac{\pi}{4} - x_{0} \right) + \frac{\pi}{3} \right\rbrack = 2\sin\left( \frac{5\pi}{6} - 2x_{0} \right) = 2\sin\left\lbrack \pi - \left( \frac{\pi}{6} + 2x_{0} \right) \right\rbrack \\
 = 2\sin\left( \frac{\pi}{6} + 2x_{0} \right) = 2\sin\left\lbrack 2\left( \frac{\pi}{3} + x_{0} \right) - \frac{\pi}{2} \right\rbrack = - 2\cos\left\lbrack 2\left( x_{0} + \frac{\pi}{3} \right) \right\rbrack \\
 = - 2\left\lbrack 1 - 2\sin^{2}\left( x_{0} + \frac{\pi}{3} \right) \right\rbrack = - 2 \times \left\lbrack 1 - 2 \times \left( \frac{2}{5} \right)^{2} \right\rbrack = - \frac{34}{25}
\end{array}\)}
\end{question}

\begin{question}
在\(\bigtriangleup ABC\)中,
内角\(A\)、\(B\)、\(C\)所对的边分别是\(a\)、\(b\)、\(c\),
\(\sqrt{3}b\cos A + a\sin B = \sqrt{3}c\),
\(\angle ABC\)的角平分线\(BT\)交\(AC\)于点\(T\),
\(BT = 2\).
\begin{enumerate}[label=(\arabic*)]
  \item 求\(B\);
  \item 若\(AT = 2TC\),
\item 求\(\bigtriangleup ABC\)的面积;
  \item 若\(\bigtriangleup ABC\)为锐角三角形,
\item 求\(AT + AB\)的取值范围.
\end{enumerate}
\topics{用和;差角的正弦公式化简;求值;正弦定理边角互化的应用;三角形面积公式及其应用;求三角形中的边长或周长的最值或范围}
\difficulty{0.65}
\answer{(1)\(B = \frac{\pi}{3}\)
(2)\(\frac{3\sqrt{3}}{2}\)
(3)\(\left( 1 + \sqrt{3},2 + 2\sqrt{3} \right)\)}
\explain{(1)因为\(\sqrt{3}b\cos A + a\sin B = \sqrt{3}c\),
由正弦定理得\(\sin B\left( \sqrt{3}\cos A + \sin A \right) = \sqrt{3}\sin C\),\par
即\(\sin B\left( \sqrt{3}\cos A + \sin A \right) = \sqrt{3}\sin(A + B) = \sqrt{3}\left( \sin A\cos B + \cos A\sin B \right)\),\par
整理得\(\sin A\sin B = \sqrt{3}\sin A\cos B\),\par
因为\(A \in \left( 0,\pi \right)\),
所以\(\sin A > 0\),
故\(\sin B = \sqrt{3}\cos B > 0\),
可得\(\tan B = \sqrt{3}\),\par
因为\(B \in (0,\pi)\),
所以\(B = \frac{\pi}{3}\).\par
(2)因为\(\angle ABC\)的角平分线\(BT\)交\(AC\)于点\(T\),
且\(AT = 2TC\),\par
由角平分线定理可得\(\frac{AT}{CT} = \frac{AB}{BC} = \frac{c}{a} = 2\),\par
又因为\(B = \frac{\pi}{3}\),
由余弦定理可得\(b^{2} = a^{2} + c^{2} - 2ac\cos B = 5a^{2} - 2 \times a \times 2a \times \frac{1}{2} = 3a^{2}\),\par
所以\(a^{2} + b^{2} = c^{2}\),
故\(C = \frac{\pi}{2}\),\(A = \frac{\pi}{6}\),\par
因为\(S_{\bigtriangleup ABC} = S_{\bigtriangleup ABT} + S_{\bigtriangleup BCT}\),
则\(\frac{1}{2}ac\sin\frac{\pi}{3} = \frac{1}{2}c \cdot BT\sin\frac{\pi}{6} + \frac{1}{2}a \cdot BT\sin\frac{\pi}{6}\),\par
可得\(BT = \frac{\sqrt{3}ac}{a + c} = \frac{2\sqrt{3}a}{3} = 2\),
故\(a = \sqrt{3}\),\(c = 2a = 2\sqrt{3}\),
\(b = \sqrt{3}a = 3\),\par
因此\(S_{\bigtriangleup ABC} = \frac{1}{2}ab = \frac{1}{2} \times \sqrt{3} \times 3 = \frac{3\sqrt{3}}{2}\).\par
(3)因为\(\bigtriangleup ABC\)为锐角三角形,
且\(B = \frac{\pi}{3}\),则\(\left\{ \begin{array}{r}
0 < A < \frac{\pi}{2} \\
0 < C = \frac{2\pi}{3} - A < \frac{\pi}{2}
\end{array} \right.\),解得\(\frac{\pi}{6} < A < \frac{\pi}{2}\),\par
由角平分线定理可得\(\frac{AT}{CT} = \frac{AB}{BC}\),即\(\frac{AT}{AC - AT} = \frac{c}{a}\),解得\(AT = \frac{bc}{a + c}\),\par
故\(AT + AB = \frac{bc}{a + c} + c = \frac{c(a + b + c)}{a + c}\)①,\par
又因为\(BT = \frac{\sqrt{3}ac}{a + c} = 2\)②,\par
①\(\div\)②得\(\frac{AT + AB}{2} = \frac{a + b + c}{\sqrt{3}a}\),故\(AT + AB = \frac{2(a + b + c)}{\sqrt{3}a} = \frac{2\sqrt{3}}{3} \cdot \frac{\sin A + \sin B + \sin C}{\sin A} = \frac{2\sqrt{3}}{3}\left\lbrack 1 + \frac{\sin B + \sin(A + B)}{\sin A} \right\rbrack= \frac{2\sqrt{3}}{3}\left( 1 + \frac{\sin B + \sin A\cos B + \cos A\sin B}{\sin A} \right) = \frac{2\sqrt{3}}{3}\left\lbrack 1 + \frac{\frac{\sqrt{3}}{2}\left( 1 + \cos A \right) + \frac{1}{2}\sin A}{\sin A} \right\rbrack= \sqrt{3} + \frac{1 + \cos A}{\sin A} = \sqrt{3} + \frac{1 + 2\cos^{2}\frac{A}{2} - 1}{2\sin\frac{A}{2}\cos\frac{A}{2}} = \sqrt{3} + \frac{1}{\tan\frac{A}{2}}\),\par
因为\(A \in \left( \frac{\pi}{6},\frac{\pi}{2} \right)\),则\(\frac{A}{2} \in \left( \frac{\pi}{12},\frac{\pi}{4} \right)\),\par
因为\(\tan\frac{\pi}{12} = \tan\left( \frac{\pi}{3} - \frac{\pi}{4} \right) = \frac{\tan\frac{\pi}{3} - \tan\frac{\pi}{4}}{1 + \tan\frac{\pi}{3}\tan\frac{\pi}{4}} = \frac{\sqrt{3} - 1}{1 + \sqrt{3}} = 2 - \sqrt{3}\),故\(\tan\frac{A}{2} \in \left( 2 - \sqrt{3},1 \right)\),\par
所以\(\frac{1}{\tan\frac{A}{2}} \in \left( 1,2 + \sqrt{3} \right)\),因此\(AT + AB = \sqrt{3} + \frac{1}{\tan\frac{A}{2}} \in \left( 1 + \sqrt{3},2 + 2\sqrt{3} \right)\).}
\end{question}

\begin{question}
已知数列\(\left\{ a_{n} \right\}\)满足\(a_{1} + \frac{a_{2}}{2} + \cdots + \frac{a_{n}}{n} = 2n^{2} + 2n\).
\begin{enumerate}[label=(\arabic*)]
  \item 求数列\(\left\{ a_{n} \right\}\)的通项公式;
  \item 设\(b_{n} = \frac{(2n - 3) \cdot 2^{n}}{1 - a_{n}}\),
\item 求数列\(\left\{ b_{n} \right\}\)的前\(n\)项和\(S_{n}\);
  \item 设\(T_{n}\)是数列\(\left\{ a_{n} \right\}\)的前\(n\)项积,
\item 求证:\(T_{n} \leq 4^{n} \cdot \mathrm{e}^{n^{2} - n}\).
\end{enumerate}
\topics{利用导数证明不等式;裂项相消法求和;利用an与sn关系求通项或项}
\difficulty{0.4}
\answer{(1)\(a_{n} = 4n^{2}\)
(2)\(S_{n} = 2 - \frac{2^{n + 1}}{2n + 1}\)
(3)证明见解析}
\explain{(1)\(a_{1} + \frac{a_{2}}{2} + \cdots + \frac{a_{n}}{n} = 2n^{2} + 2n\)\par
当\(n \geq 2\)时,
得\(a_{1} + \frac{a_{2}}{2} + \cdots + \frac{a_{n - 1}}{n - 1} = 2(n - 1)^{2} + 2(n - 1)\),\par
两式相减得\(\frac{a_{n}}{n} = 2n^{2} + 2n - \left\lbrack 2(n - 1)^{2} + 2(n - 1) \right\rbrack = 4n\),
所以\(a_{n} = 4n^{2}\),当\par
当\(n = 1\)时,
\(S_{1} = a_{1} = 2 \times 1^{2} + 2 \times 1 = 4\),
适合上式,\par
所以数列\(\left\{ a_{n} \right\}\)的通项公式为\(a_{n} = 4n^{2}\);\par
(2)\(b_{n} = \frac{(2n - 3) \cdot 2^{n}}{1 - a_{n}} = \frac{(2n - 3) \cdot 2^{n}}{1 - 4n^{2}} = \frac{2^{n}}{2n - 1} - \frac{2^{n + 1}}{2n + 1}\),\par
\(S_{n} = \left( \frac{2^{1}}{1} - \frac{2^{2}}{3} \right) + \left( \frac{2^{2}}{3} - \frac{2^{3}}{5} \right) + \cdots + \left( \frac{2^{n}}{2n - 1} - \frac{2^{n + 1}}{2n + 1} \right) = 2 - \frac{2^{n + 1}}{2n + 1}\),\par
所以数列\(\left\{ b_{n} \right\}\)的前\(n\)项和\(S_{n} = 2 - \frac{2^{n + 1}}{2n + 1}\);\par
(3)令\(f(x) = \ln x - x + 1\),
则\(f'(x) = \frac{1}{x} - 1 = \frac{1 - x}{x}\),\par
当\(x \geq 1\)时,\(f'(x) \leq 0\),
所以\(f(x)\)在\(\lbrack 1, + \infty)\)上单调递减,\par
所以\(f(x) \leq f(1) = \ln 1 - 1 + 1 = 0\),
即\(f(x) = \ln x - x + 1 \leq 0\),\par
即\(\ln x \leq x - 1\)对\(\lbrack 1, + \infty)\)恒成立,
当且仅当\(x = 1\)时取等号,\par
所以\(\ln 1 \leq 1 - 1 = 0\),
\(\ln 2 < 2 - 1 = 1\),
\(\ln 3 < 3 - 1 = 2\),
\(\ln 4 < 4 - 1 = 3\),\(\ldots\),
\(\ln n < n - 1\),\par
两边分别相加得\(\ln 1 + \ln 2 + \ln 3 + \ln 4 + \cdots + \ln n \leq 0 + 1 + 2 + 3 + \cdots + (n - 1)\),\par
所以\(\ln(1 \times 2 \times 3 \times 4 \times \cdots \times n) \leq \frac{(1 + n - 1)(n - 1)}{2} = \frac{n(n - 1)}{2} = \frac{n^{2} - n}{2}\),\par
所以\(2\ln(1 \times 2 \times 3 \times 4 \times \cdots \times n) \leq n^{2} - n\),\par
所以\(\ln(1 \times 2 \times 3 \times 4 \times \cdots \times n)^{2} \leq n^{2} - n\),\par
所以\((1 \times 2 \times 3 \times 4 \times \cdots \times n)^{2} \leq \mathrm{e}^{n^{2} - n}\),\par
所以\(4^{n}(1 \times 2 \times 3 \times 4 \times \cdots \times n)^{2} \leq 4^{n} \cdot \mathrm{e}^{n^{2} - n}\),\par
所以\(\left( 4 \times 1^{2} \right) \times \left( 4 \times 2^{2} \right) \times \left( 4 \times 3^{2} \right) \times \cdots \times 4n^{2} \leq 4^{n} \cdot \mathrm{e}^{n^{2} - n}\),\par
所以\(T_{n} \leq 4^{n} \cdot \mathrm{e}^{n^{2} - n}\).}
\end{question}

\begin{question}
已知函数\(f(x) = e^{x} - mx - n\sin x(m,n \in R)\).
\begin{enumerate}[label=(\arabic*)]
  \item 当\(n = 0\)时,讨论\(f(x)\)的单调性;
  \item 当\(m = n\)时,
\item 若\(f(x) \geq 0\)在\(\left( 0,\pi \right)\)上恒成立,
\item 求正整数\(m\)的最大值;
  \item 若\(f(x)\)在\((0, + \infty)\)上有零点,
\item 求证:\(m^{2} + n^{2} > \frac{1}{2}e^{2}\).

\item (参考数据:\(e^{\frac{\pi}{4}} \approx 2.2\),
\item \(e^{\frac{\pi}{2}} \approx 4.8\),
\item \(e^{\pi} \approx 23.1\))
\end{enumerate}
\topics{用导数判断或证明已知函数的单调性;利用导数证明不等式;利用导数研究不等式恒成立问题;利用导数研究函数的零点}
\difficulty{0.4}
\answer{(1)答案见解析;
(2)\(1\);
(3)证明见解析.}
\explain{(1)当\(n = 0\)时,\par
①当\(m \leq 0\)时,
\(f'(x) > 0,f(x)\)在\(( - \infty, + \infty)\)上单调递增;\par
②当\(m > 0\)时,由\(f'(x) = 0\),
得\(x = \ln m\),\par
\(x \in ( - \infty,\ln m)\)时,
\(f'(x) < 0,f(x)\)单调递减.\par
\(x \in (\ln m, + \infty)\)时,
\(f'(x) > 0,f(x)\)单调递增.\par
综上,\par
\(m \leq 0\)时,
\(f(x)\)在\(( - \infty, + \infty)\)上为增函数;\par
\(m > 0\)时,
\(f(x)\)在\(( - \infty,\ln m)\)上为减函数,
在\((\ln m, + \infty)\)上为增函数.\par
(2)当\(m = n\)时,
\(f(x) = \mathrm{e}^{x} - m(x + \sin x)\),\par
因\(x \in (0,\pi),f(x) \geq 0\)恒成立,
所以\(f\left( \frac{\pi}{2} \right) \geq 0\),\par
即\(f\left( \frac{\pi}{2} \right) = e^{\frac{\pi}{2}} - m\left( \frac{\pi}{2} + 1 \right) \geq 0,m \leq \frac{e^{\frac{\pi}{2}}}{\frac{\pi}{2} + 1} \approx 1.87\),\par
所以正整数\(m\)的最大值为1.\par
下证\(m = 1\)时,
\(f(x) = \mathrm{e}^{x} - x - \sin x \geq 0\)在\((0,\pi)\)上恒成立.\par
设\(h(x) = \mathrm{e}^{x} - x - 1,x \in (0,\pi)\),\par
则\(h'(x) = e^{x} - 1 > 0,h(x)\)在\((0,\pi)\)上单调递增,
\(h(x) > h(0) = 0\),
即\(\mathrm{e}^{x} - x > 1\),\par
所以\(f(x) = \mathrm{e}^{x} - x - \sin x > 1 - \sin x\),
又\(\sin x \leq 1\),\par
所以\(f(x) > 1 - \sin x \geq 0\),
即\(f(x) = \mathrm{e}^{x} - x - \sin x > 0\)恒成立.\par
所以正整数\(m\)的最大值为1.\par
(3)由题意设\(x_{0}\)为\(f(x)\)的零点\(\left( x_{0} > 0 \right)\),
则\(e^{x_{0}} - mx_{0} - n\sin x_{0} = 0\),\par
即\(mx_{0} + n\sin x_{0} - e^{x_{0}} = 0\),
则点\(M(m,n)\)在直线\(xx_{0} + y\sin x_{0} - e^{x_{0}} = 0\)上,\par
所以\(\sqrt{m^{2} + n^{2}} \geq \frac{e^{x_{0}}}{\sqrt{x_{0}^{2} + \sin^{2}x_{0}}}\),
即\(m^{2} + n^{2} \geq \frac{e^{2x_{0}}}{x_{0}^{2} + \sin^{2}x_{0}}\),\par
当\(x \in (0,1\rbrack\)时,
设\(g(x) = x - \sin x\),
所以\(g'(x) = 1 - \cos x \geq 0\),
则\(g(x)\)在\((0,1\rbrack\)上单调递增,\par
所以\(g(x) > g(0) = 0\),
所以\(x > \sin x > 0\),
又\(x \in (1, + \infty)\)时,
\(\sin^{2}x \leq 1 < x^{2}\),\par
所以\(x_{0} > 0\)时,
\(\sin^{2}x_{0} < x_{0}^{2}\),
则\(m^{2} + n^{2} \geq \frac{e^{2x_{0}}}{x_{0}^{2} + \sin^{2}x_{0}} > \frac{e^{2x_{0}}}{2x_{0}^{2}} = \frac{1}{2}\left( \frac{e^{x_{0}}}{x_{0}} \right)^{2}\),\par
令\(k(x) = \frac{e^{x}}{x},x \in (0, + \infty)\),
则\(k'(x) = \frac{e^{x}(x - 1)}{x^{2}}\),\par
\(x \in (0,1)\)时,\(k'(x) < 0,k(x)\)单调递减;
\(x \in (1, + \infty)\)时,
\(k'(x) > 0,k(x)\)单调递增,\par
所以\(k(x) \geq k(1) = e\),
即\(\frac{e^{x}}{x} \geq e\),
所以\(m^{2} + n^{2} > \frac{1}{2}\left( \frac{e^{x_{0}}}{x_{0}} \right)^{2} \geq \frac{1}{2}e^{2}\).}
\end{question}
