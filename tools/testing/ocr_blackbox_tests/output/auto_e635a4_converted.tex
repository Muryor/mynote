\examxtitle{测试试卷 - auto_e635a4}

\section{单选题}

\begin{question}
已知\(\text{i}\)为虚数单位,则\(\frac{2 + \text{i}}{\text{i}} =\)(    )
\begin{choices}
  \item \(1 + 2\text{i}\)
  \item \(1 - 2\text{i}\)
  \item \(2 - \text{i}\)
  \item \(- 1 + 2\text{i}\)
\end{choices}
\topics{复数的除法运算}
\difficulty{0.94}
\answer{B}
\explain{\(\frac{2 + \text{i}}{\text{i}} = \frac{\text{i}\left( 2 + \text{i} \right)}{\text{i}^{2}} = 1 - 2\text{i}\)}
\end{question}

\begin{question}
设集合\(M = \left\{ 1,2 \right\},N = \left\{ x \in N^{\text{*}}\left| \frac{6}{x} \in N^{\text{*}} \right.\  \right\}\),
则\(M \cap N =\)(    )
\begin{choices}
  \item \(\left\{ 1 \right\}\)
  \item \(\left\{ 1,2 \right\}\)
  \item \(\left\{ 2,3 \right\}\)
  \item \(\left\{ 1,2,3 \right\}\)
\end{choices}
\topics{交集的概念及运算}
\difficulty{0.94}
\answer{B}
\explain{\(N = \left\{ x \in N^{\text{*}}\left| \frac{6}{x} \in N^{\text{*}} \right.\  \right\} = \left\{ 1,2,3,6 \right\}\),\par
因为\(M = \left\{ 1,2 \right\}\),
所以\(M \cap N =\left\{ 1,2 \right\}\)}
\end{question}

\begin{question}
设向量\(\overrightarrow{a} = (2,x),\overrightarrow{b} = (2 + x,2x)\).若\(\overrightarrow{a} \cdot \left( 2\overrightarrow{a} - \overrightarrow{b} \right) = 0\),
则\(x =\)(    )
\begin{choices}
  \item 2
  \item 3
  \item 4
  \item 5
\end{choices}
\topics{数量积的坐标表示}
\difficulty{0.65}
\answer{A}
\explain{\(2\overrightarrow{a} - \overrightarrow{b} = (2 - x,0)\),\par
\(\therefore\) \(\overrightarrow{a} \cdot \left( 2\overrightarrow{a} - \overrightarrow{b} \right) = 4 - 2x + 0 = 0\),\par
解得\(x = 2\)}
\end{question}

\begin{question}
《算经十书》是中国古代数学典籍的合集.书中记载(用现代文表达):今有牛、羊、猪各数头(各有至少1头),
已知猪的数量多于羊,羊的数量多于牛,牛的数量的3倍多于猪、羊数量之和,
则牛、羊、猪的总头数至少为(    )
\begin{choices}
  \item 12
  \item 15
  \item 18
  \item 21
\end{choices}
\topics{利用不等式求值或取值范围}
\difficulty{0.65}
\answer{B}
\explain{设牛、羊、猪分别为\(x,y,z\)
头,则根据题意有\(\left\{ \begin{array}{r}
3x > y + z \\
y \geq x + 1 \\
z \geq x + 2 \\
x,y,z \in N^{\ast}
\end{array} \right.\),则\(x > 3\),\par
则\(x_{\min} = 4\) ,则 \(\left\{ \begin{array}{r}
12 > y + z \\
y \geq 5 \\
z \geq 6
\end{array} \right.\),则\((x + y + z)_{\min} = 4 + 5 + 6 = 15\)}
\end{question}

\begin{question}
已知函数\(f(x)(x \in R)\).若对于任意的等差数列\(\left\{ a_{n} \right\}\),
总有\(\left\{ f\left( a_{n} \right) \right\}\)是等差数列,
则称函数\(f(x)\)具有"保等差性".函数\(f(x)\)可能是(    )
\begin{choices}
  \item \(f(x) = 2^{x}\)
  \item \(f(x) = x^{2}\)
  \item \(f(x) = \sinx\)
  \item \(f(x) = 2x + 1\)
\end{choices}
\topics{判断等差数列}
\difficulty{0.65}
\answer{D}
\explain{\(\left\{ f\left( a_{n} \right) \right\}\)是等差数列,
则需要满足\(2f\left( a_{n + 1} \right) = f\left( a_{n} \right) + f\left( a_{n + 2} \right)\),\par
对于A,取等差数列\(a_{n} = n\),
则\(f\left( a_{n} \right) = 2^{n}\),\(f\left( a_{n + 1} \right) = 2^{n + 1}\),
\(f\left( a_{n + 2} \right) = 2^{n + 2}\),则\(2f\left( a_{n + 1} \right) \neq f\left( a_{n} \right) + f\left( a_{n + 2} \right)\),
故A不正确;\par
对于B,取等差数列\(a_{n} = n\),
则\(f\left( a_{n} \right) = n^{2}\),\(f\left( a_{n + 1} \right) = {(n + 1)}^{2}\),\(f\left( a_{n + 2} \right) = {(n + 2)}^{2}\),
则\(2f\left( a_{n + 1} \right) \neq f\left( a_{n} \right) + f\left( a_{n + 2} \right)\),
故B不正确;\par
对于C,取等差数列\(a_{n} = n\),
则\(f\left( a_{n} \right) = \sin n\),\(f\left( a_{n + 1} \right) = \sin(n + 1)\),\(f\left( a_{n + 2} \right) = \sin(n + 2)\),则\(2f\left( a_{n + 1} \right) \neq f\left( a_{n} \right) + f\left( a_{n + 2} \right)\),
故C不正确;\par
对于D,\(f(a_{n}) = 2a_{n} + 1f(a_{n + 1}) = 2a_{n + 1} + 1\),\(f(a_{n + 2}) = 2a_{n + 2} + 1\),\par
所以\(2f(a_{n + 1}) = 4a_{n + 1} + 2\),
\(f(a_{n}) + f(a_{n + 2}) = 2a_{n} + 2a_{n + 2} + 2\),\par
由于\(\left\{ a_{n} \right\}\)为等差数列,
则\(2a_{n + 1} = a_{n} + a_{n + 2}\),
所以\(2f\left( a_{n + 1} \right) = f\left( a_{n} \right) + f\left( a_{n + 2} \right)\),
故D正确}
\end{question}

\begin{question}
设样本数据\(x_{1},x_{2},\cdots,x_{2025}\)的平均数,
中位数,
众数和标准差分别为\(a,b,c,d\).当\(\sum_{i = 1}^{2025}\left( x_{i} - k \right)^{2}\)取得最小值时,
\(k =\)(    )
\begin{choices}
  \item \(a\)
  \item \(b\)
  \item \(c\)
  \item \(d\)
\end{choices}
\topics{计算几个数的平均数}
\difficulty{0.65}
\answer{A}
\explain{令\(f(k) = \sum_{i = 1}^{2025}\left( x_{i} - k \right)^{2} = 2025k^{2} - 2\sum_{i = 1}^{2025}x_{i}k + \sum_{i = 1}^{2025}x_{i}^{2}\),\par
\(f(k)\)是一个开口向上的关于\(k\)的二次函数,
故函数在对称轴处取得最小值,\par
即\(k = \frac{2\sum_{i = 1}^{2025}x_{i}}{2 \times 2025} = \frac{\sum_{i = 1}^{2025}x_{i}}{2025} = \overline{x} = a\)}
\end{question}

\begin{question}
若圆\(C\)经过\(A(1,1),B(2, - 2)\),圆心在直线\(x - y + 1 = 0\)上,则圆\(C\)的面积为(    )
\begin{choices}
  \item \(16\pi\)
  \item \(25\pi\)
  \item \(36\pi\)
  \item \(49\pi\)
\end{choices}
\topics{由圆心(或半径)求圆的方程}
\difficulty{0.65}
\answer{B}
\explain{设圆的方程为:\((x - a)^{2} + {(y - b)}^{2} = r^{2}\),\par
所以\(\left\{ \begin{array}{r}
(1 - a)^{2} + {(1 - b)}^{2} = r^{2} \\
(2 - a)^{2} + {( - 2 - b)}^{2} = r^{2} \\
a - b + 1 = 0
\end{array} \right.\),解得:\(\left\{ \begin{array}{r}
a = - 3 \\
b = - 2 \\
r = 5
\end{array} \right.\),\par
所以圆\(C\)的面积为\(\pir^{2} = 25\pi\)}
\end{question}

\begin{question}
设函数\(f(x) = x^{3} + 3x^{2} + 6x + 5\),
若\(f(a) = 15,f(b) = - 13\),
则\(a + b =\)(    )
\begin{choices}
  \item 2
  \item 1
  \item -1
  \item -2
\end{choices}
\topics{函数奇偶性的应用;由函数的单调区间求参数}
\difficulty{0.65}
\answer{D}
\explain{因为\(f(x) = (x + 1)^{3} + 3(x + 1) + 1\),
所以\(\left\{ \begin{array}{r}
{(a + 1)}^{3} + 3(a + 1) + 1 = 15 \\
{(b + 1)}^{3} + 3(b + 1) + 1 = - 13
\end{array} \right.\),\par
即\(\left\{ \begin{array}{r}
{(a + 1)}^{3} + 3(a + 1) = 14 \\
{(b + 1)}^{3} + 3(b + 1) = - 14
\end{array} \right.\),\par
令\(g(t) = t^{3} + 3t\),\(g'(t) = 3t^{2} + 3 > 0\),所以\(g(t)\)在\(\mathbb{R}\)上为单调递增的奇函数,\par
由于\(g(a + 1) = 14\),\(g(b + 1) = - 14\),\par
所以\(a + 1 + b + 1 = 0\),则\(a + b = - 2\)}
\end{question}

\section{多选题}

\begin{question}
在\(\left( 2x + \frac{1}{x} \right)^{5}\)的展开式中,(    )
\begin{choices}
  \item 常数项为20
  \item 含\(x\)的项的系数为80
  \item 各项系数的和为32
  \item 各项系数中的最大值为80
\end{choices}
\topics{求系数最大(小)的项}
\difficulty{0.65}
\answer{BD}
\explain{2\emph{x}和\(\frac{1}{x}\)只有分得的次数相同才能得到常数项,
5次方无法均分,因此没有常数项,故A不正确;\par
含\emph{x}的项为\(\mathbb{C}_{5}^{2}{(2x)}^{3}\left( \frac{1}{x} \right)^{2} = 80x\),
故\emph{x}的系数是80,所以B正确;\par
各项系数的和是令\(x = 1\)时得到,即\(3^{5}\),故C错误.\par
\(\left( 2x + \frac{1}{x} \right)^{5}\)的展开式的通项公式为:\(T_{r + 1} = \mathbb{C}_{5}^{r}{(2x)}^{5 - r}\left( x^{- 1} \right)^{r} = \mathbb{C}_{5}^{r}2^{5 - r}x^{5 - 2r}\),\par
设第\(r + 1\)项的系数最大,
系数为\(\mathbb{C}_{5}^{r}2^{5 - r}\),则\(\left\{ \begin{array}{r}
\mathbb{C}_{5}^{r}2^{5 - r} \geq \mathbb{C}_{5}^{r - 1}2^{6 - r} \\
\mathbb{C}_{5}^{r}2^{5 - r} \geq \mathbb{C}_{5}^{r + 1}2^{4 - r}
\end{array} \right.\),\par
解得:\(r = 1\)或\(r = 2\),此时系数为\(\mathbb{C}_{5}^{1}2^{4} = \mathbb{C}_{5}^{2}2^{3} = 80\),故D正确;}
\end{question}

\begin{question}
设函数\(f(x) = 2\cosx\left( \sqrt{3}\sinx + \cosx \right)\),则(    )
\begin{choices}
  \item \(f\left( \frac{\pi}{3} \right) = 2\)
  \item \(f(x)\)的最小正周期是\(\pi\)
  \item \(f(x)\)的值域是\(\lbrack - 1,3\rbrack\)
  \item \(f(x)\)在区间\(\left( \frac{\pi}{3},\frac{\pi}{2} \right)\)上单调递增
\end{choices}
\topics{求含sinx(型)函数的值域和最值;求正弦(型)函数的最小正周期;三角恒等变换的化简问题;求sinx型三角函数的单调性}
\difficulty{0.85}
\answer{ABC}
\explain{\(f(x) = 2\cosx\left( \sqrt{3}\sinx + \cosx \right) = 2\sqrt{3}\cosx\sinx + 2\cos^{2}x - 1 + 1\),\par
\(= \sqrt{3}\text{sin2}x + \cos2x + 1 = 2\sin\left( 2x + \frac{\pi}{6} \right) + 1\),\par
\(\therefore\) \(f\left( \frac{\pi}{3} \right) = 2\sin\left( \frac{2\pi}{3} + \frac{\pi}{6} \right) + 1 = 2\sin\frac{5\pi}{6} + 1 = 2\),
故A正确;\par
函数\(f(x)\)的最小正周期\(T = \frac{2\pi}{2} = \pi\),
故B正确;\par
因\(- 1 \leq \sin\left( 2x + \frac{\pi}{6} \right)\text{≤1}\),
则函数\(f(x)\)的值域是\(\lbrack - 1,3\rbrack\),
故C正确;\par
当\(x \in \left( \frac{\pi}{3},\frac{\pi}{2} \right)\)时,
\(2x + \frac{\pi}{6} \in \left( \frac{5\pi}{6},\frac{7\pi}{6} \right)\),
此时函数\(y = \sin\left( 2x + \frac{\pi}{6} \right)\)单调递减,
则函数\(f(x)\)也单调递减,故D错误}
\end{question}

\begin{question}
已知函数\(y = f(n)\left( n \in N^{\text{*}} \right)\)的函数值等于\(n\)的正因数的个数.例如\(f(1) = 1,f(4) = 3\).则下列选项正确的是(    )
\begin{choices}
  \item \(f(6) = 4\)
  \item \(f(2025) = 20\)
  \item \(\sum_{k = 1}^{2025}\frac{1}{f\left( 6^{k} \right)} < 1\)
  \item 设\(b_{n} = \left( \sqrt{2} \right)^{n}\),则\(\sum_{k = 1}^{2025}\frac{{( - 1)}^{f(k + 1)}}{b_{2k - 1}b_{2k}} \leq \frac{5\sqrt{2}}{16}\)
\end{choices}
\topics{求函数值;求等比数列前n项和;裂项相消法求和}
\difficulty{0.4}
\answer{ACD}
\explain{对于A,6的正因数为\(1,2,3,6\)共4个,所以\(f(6) = 4\),
故A正确;\par
对于B,\(2025 = 3^{4} \times 5^{2}\),
它的因数形如\(3^{i} \times 5^{j}\),
其中\(i \in \left\{ 0,1,2,3,4 \right\},j \in \left\{ 0,1,2 \right\}\),\par
所以不同的因数有\(5 \times 3 = 15\)个,
即\(f(2025) = 15\),故B不正确.\par
对于C,
因为\(6^{k} = (2 \times 3)^{k} = 2^{k} \times 3^{k}\),
所以\(f\left( 6^{k} \right) = (k + 1)^{2}\),\par
所以\(S_{2025} = \sum_{k = 1}^{2025}\frac{1}{f\left( 6^{k} \right)} = \sum_{k = 1}^{2025}\frac{1}{(k + 1)^{2}} < \sum_{k = 1}^{2025}\frac{1}{k(k + 1)} = \sum_{k = 1}^{2025}\left( \frac{1}{k} - \frac{1}{k + 1} \right)= \left( 1 - \frac{1}{2} \right) + \left( \frac{1}{2} - \frac{1}{3} \right) + \cdots + \left( \frac{1}{2025} - \frac{1}{2026} \right) = 1 - \frac{1}{2026} < 1\),
故C正确;\par
对于D,
\(f(2) = 2,f(3) = 2,f(4) = 3,f(5) = 2\),
则\par
\(\sum_{k = 1}^{2025}\frac{( - 1)^{f(k + 1)}}{b_{2k - 1}b_{2k}} = \frac{1}{b_{1}b_{2}} + \frac{1}{b_{3}b_{4}} - \frac{1}{b_{5}b_{6}} + \left\lbrack \frac{1}{b_{7}b_{8}} + \cdots + \frac{( - 1)^{f(2026)}}{b_{4049}b_{4050}} \right\rbrack= \frac{1}{2\sqrt{2}} + \frac{1}{2\sqrt{2} \times 4} - \frac{1}{b_{5}b_{6}} + \left\lbrack \frac{1}{b_{7}b_{8}} + \cdots + \frac{( - 1)^{f(2026)}}{b_{4049}b_{4050}} \right\rbrack\leq \frac{5\sqrt{2}}{16} - \frac{1}{b_{5}b_{6}} + \left( \frac{1}{b_{7}b_{8}} + \cdots + \frac{1}{b_{4049}b_{4050}} \right)= \frac{5\sqrt{2}}{16} - \frac{\sqrt{2}}{4^{3}} + \sqrt{2} \cdot \frac{\frac{1}{4^{4}}\left\lbrack 1 - \left( \frac{1}{4} \right)^{2022} \right\rbrack}{1 - \frac{1}{4}} = \frac{5\sqrt{2}}{16} - \left( \frac{\sqrt{2}}{96} + \frac{\sqrt{2}}{3 \times 4^{2025}} \right) < \frac{5\sqrt{2}}{16}\),
故D正确}
\end{question}

\section{填空题}

\begin{question}
已知随机变量\(X\)服从正态分布\(N\left( 3,\sigma^{2} \right)\).若\(P(X > 4) = 0.36\),
则\(P(2 \leqslant X \leqslant 4)\)=
.
\topics{指定区间的概率}
\difficulty{0.65}
\answer{0.28}
\explain{由题可得:\(P(2 \leq X \leq 4) = 1 - 2P(X > 4) = 1 - 2 \times 0.36 = 0.28\);\(0.28\)}
\end{question}

\begin{question}
函数\(f(x) = \frac{f(2)}{4}x^{2} + \frac{f(1)}{x} - 1\)在\(\left\lbrack \frac{1}{2},2 \right\rbrack\)上的最小值为
.
\topics{由导数求函数的最值(不含参)}
\difficulty{0.65}
\answer{2}
\explain{由题可得:\(\left\{ \begin{array}{r}
f(1) = \frac{f(2)}{4} + f(1) - 1 \\
f(2) = \frac{f(2)}{4} \times 4 + \frac{f(1)}{2} - 1
\end{array} \right.\),解得:\(\left\{ \begin{array}{r}
f(1) = 2 \\
f(2) = 4
\end{array} \right.\),\par
所以\(f(x) = x^{2} + \frac{2}{x} - 1\),则\(f'(x) = 2x - \frac{2}{x^{2}} = \frac{2(x^{3} - 1)}{x^{2}}\),令\(f'(x) = \frac{2(x^{3} - 1)}{x^{2}} = 0\),解得:\(x = 1\),\par
令\(f'(x) < 0\),解得:\(\frac{1}{2} < x < 1\),令\(f'(x) > 0\),解得:\(1 < x < 2\),\par
所以\(f(x)\)在\(\left( \frac{1}{2},1 \right)\)上单调递减,在\((1,2)\)上单调递增,\par
所以\(f(x)_{\min} = f(1) = 1 + 2 - 1 = 2\)\(2\)}
\end{question}

\begin{question}
过点\(F\left( - \sqrt{2},\sqrt{2} \right)\)的直线\(l\)与圆\(O:x^{2} + y^{2} = 1\)相切于点\(M\),
与曲线\(y = - \frac{1}{x}(x > 0)\)交于点\emph{R}.若\(FR\)的中点为\(N\),
则\(|ON| - |MN| =\)
.

\begin{center}
% IMAGE_TODO_START id=auto_e635a4-Q14-img1 path=/Users/muryor/code/mynote/word\\_to\\_tex/output/figures/auto\\_e635a4/media/image2.png width=60% inline=false question_index=14 sub_index=1
% CONTEXT_BEFORE: $$2$$.即$$|FR| - |RK| = 2$$,如图所示.
% CONTEXT_AFTER: 因为点$$O,N$$分别是$$FK$$和$$FR$$的中点,故$$|FN| - |ON| = \
\begin{tikzpicture}[scale=1.05,>=Stealth,line cap=round,line join=round]
  % TODO: AI_AGENT_REPLACE_ME (id=auto_e635a4-Q14-img1)
\end{tikzpicture}
% IMAGE_TODO_END id=auto_e635a4-Q14-img
1
\end{center}

\topics{等轴双曲线}
\difficulty{0.4}
\answer{\(\sqrt{3} - \sqrt{2}\)}
\explain{设\(P(x,y)\)为\(y = - \frac{1}{x}(x > 0)\)上的点,
将点\(P(x,y)\)绕原点逆时针旋转\(45^{{^\circ}}\)到\(Q(\mu,\nu)\),\par
则\(\left\{ \begin{array}{r}
x = \frac{\sqrt{2}}{2}\mu + \frac{\sqrt{2}}{2}\nu \\
y = - \frac{\sqrt{2}}{2}\mu + \frac{\sqrt{2}}{2}\nu
\end{array} \right.\),由于\(xy = - 1\),则\(\left( \frac{\sqrt{2}}{2}\mu + \frac{\sqrt{2}}{2}\nu \right)\left( - \frac{\sqrt{2}}{2}\mu + \frac{\sqrt{2}}{2}\nu \right) = - 1\),\par
化简可得:\(\frac{\mu^{2}}{2} - \frac{\nu^{2}}{2} = 1\),则点\(Q\)的轨迹为等轴双曲线,其焦点为\(F_{1}( - 2,0)\),\(F_{2}(2,0)\),且\(\left| \left| QF_{1} \right| - \left| QF_{2} \right| \right| = 2\sqrt{2}\);\par
所以曲线\(y = - \frac{1}{x}\)也是等轴双曲线,其焦点为\(F( - \sqrt{2},\sqrt{2})\),\(K(\sqrt{2}, - \sqrt{2})\),故点\(R\)到焦点\(F\left( - \sqrt{2},\sqrt{2} \right),K\left( \sqrt{2}, - \sqrt{2} \right)\)距离之差为常数\(2\sqrt{2}\).即\(|FR| - |RK| = 2\sqrt{2}\),如图所示.\par
因为点\(O,N\)分别是\(FK\)和\(FR\)的中点,故\(|FN| - |ON| = \sqrt{2}\),\par
而\(|FN| = |FM| + |MN|\),由于\(|FM| = \sqrt{|OF|^{2} - |OM|^{2}} = \sqrt{3}\),\par
所以\(|ON| - |MN| = |FN| - \sqrt{2} + |FM| - |FN| = \sqrt{3} - \sqrt{2}\).\(\sqrt{3} - \sqrt{2}\)}
\end{question}

\section{解答题}

\begin{question}
已知等差数列\(\left\{ a_{n} \right\}\)满足\(a_{4} = 7,a_{6} = 11\).
\begin{enumerate}[label=(\arabic*)]
  \item 求\(\left\{ a_{n} \right\}\)的通项公式;
  \item 设等比数列\(\left\{ b_{n} \right\}\)的前\(n\)项和为\(S_{n}\),
\item 且\(b_{n + 1} = S_{n} + 2\).令\(c_{n} = a_{n} + b_{n}\),
\item 求数列\(\left\{ c_{n} \right\}\)的前\(n\)项和\(T_{n}\).
\end{enumerate}
\topics{等差数列通项公式的基本量计算;等比数列通项公式的基本量计算;求等比数列前n项和;分组(并项)法求和}
\difficulty{0.65}
\answer{(1)\(a_{n} = 2n - 1\);
(2)\(T_{n} = 2^{n + 1} + n^{2} - 2\).}
\explain{(1)设等差数列\(\left\{ a_{n} \right\}\)的公差为\(d\),
则\(\left\{ \begin{array}{r}
a_{4} = a_{1} + 3d = 7, \\
a_{6} = a_{1} + 5d = 11,
\end{array} \right.\)解得\(\left\{ \begin{array}{r}
a_{1} = 1 \\
d = 2
\end{array} \right.\),\par
所以\(\left\{ a_{n} \right\}\)的通项公式为\(a_{n} = 2n - 1\).\par
(2)设等比数列\(\left\{ b_{n} \right\}\)的公比是\(q\),\par
由\(\left\{ \begin{array}{r}
b_{2} = S_{1} + 2 \\
b_{3} = S_{2} + 2
\end{array} \right.\),得\(\left\{ \begin{array}{r}
b_{1}q = b_{1} + 2 \\
b_{1}q^{2} = b_{1} + b_{1}q + 2
\end{array} \right.\),解得\(\left\{ \begin{array}{r}
b_{1} = 2 \\
q = 2
\end{array} \right.\),\par
所以\(\left\{ b_{n} \right\}\)的通项公式为\(b_{n} = 2^{n}\),此时\(S_{n} = \frac{2\left( 1 - 2^{n} \right)}{1 - 2} = 2^{n + 1} - 2\),\(b_{n + 1} = 2^{n + 1}\),\par
满足\(b_{n + 1} = S_{n} + 2\),故\(b_{n} = 2^{n}\).\par
结合(1)知\(c_{n} = a_{n} + b_{n} = 2^{n} + 2n - 1\),\par
所以数列\(\left\{ c_{n} \right\}\)的前\(n\)项和\(T_{n} = \frac{2\left( 1 - 2^{n} \right)}{1 - 2} + n^{2} = 2^{n + 1} + n^{2} - 2\).}
\end{question}

\begin{question}
设\(\bigtriangleup ABC\)的内角\(A,B,C\)的对边分别为\(a,b,c\),
已知\(2\sin(A - C) = \sinB\).
\begin{enumerate}[label=(\arabic*)]
  \item 若\(C = \frac{\pi}{4},c = 1\).

\item (i)求\(\tanA\);

\item (ii)求\(b\);
  \item 求\(\tan(A - C)\)的最大值.
\end{enumerate}
\topics{用和;差角的正弦公式化简;求值;用和;差角的正切公式化简;求值;正弦定理解三角形;基本不等式求和的最小值}
\difficulty{0.65}
\answer{(1)(i)3;(ii)\(b = \frac{2\sqrt{10}}{5}\)
(2)\(\frac{\sqrt{3}}{3}\)}
\explain{(1)(i)\(2\sin(A - C) = \sin(A + C)\),
展开化简得:\(\sinA\cosC = 3\cosA\sinC\)\par
所以\(\tanA = 3\tanC = 3\);\par
(ii)由\(\tanA = 3\),而\(A\)为三角形内角,
故\(\sinA = \frac{3\sqrt{10}}{10},\cosA = \frac{\sqrt{10}}{10}\),\par
所以\(\sinB = \sin(A + C) = \sin\left( A + \frac{\pi}{4} \right) = \sin A\cos\frac{\pi}{4} + \cos A\sin\frac{\pi}{4} = \frac{2\sqrt{5}}{5}\),\par
由正弦定理\(\frac{b}{\sinB} = \frac{c}{\sinC}\),
得\(b = \frac{2\sqrt{10}}{5}\).\par
(2)由(1)可得\(\tanA = 3\tanC\),故\(A,C\)均为锐角,\par
所以\(\tan(A - C) = \frac{3\tanC - \tanC}{1 + 3\tan^{2}C} = \frac{2\tanC}{1 + 3\tan^{2}C} \leq \frac{2\tanC}{2\sqrt{3}\tanC} = \frac{\sqrt{3}}{3}\),\par
当且仅当\(\tanC = \frac{\sqrt{3}}{3}\)时,
\(\tan(A - C)\)取到最大值\(\frac{\sqrt{3}}{3}\).}
\end{question}

\begin{question}
已知函数\(f(x) = \frac{\mathrm{e}^{x} + \mathrm{e}^{- x}}{2}\),
\(f'(x)\)为\(f(x)\)的导数,
其中\(\text{e}\)为自然对数的底数.
\begin{enumerate}[label=(\arabic*)]
  \item 求\({\lbrack f(x)\rbrack}^{2} - \left\lbrack f'(x) \right\rbrack^{2}\);
  \item 证明:当\(x \in (0, + \infty)\)时,
\item \(f'(x) > x\);
  \item 设\(n \in N^{\text{*}}\),
\item 对任意的\(x_{i} > 1(i = 1,2,\cdots,n)\),
\item 若\(x_{1}x_{2}\cdots x_{n} = \mathrm{e}^{2}\),
\item 求证:\(x_{1} + x_{2} + \cdots + x_{n} - \left( \frac{1}{x_{1}} + \frac{1}{x_{2}} + \cdots + \frac{1}{x_{n}} \right) > 4\).
\end{enumerate}
\topics{基本初等函数的导数公式;利用导数证明不等式}
\difficulty{0.4}
\answer{(1)1
(2)证明见解析
(3)证明见解析}
\explain{(1)\(f'(x) = \frac{\mathrm{e}^{x} - \mathrm{e}^{- x}}{2}\),\par
所以\({\lbrack f(x)\rbrack}^{2} - \left\lbrack f'(x) \right\rbrack^{2} = \left( \frac{\mathrm{e}^{x} + \mathrm{e}^{- x}}{2} \right)^{2} - \left( \frac{\mathrm{e}^{x} - \mathrm{e}^{- x}}{2} \right)^{2} = \left( \frac{\mathrm{e}^{2x} + \mathrm{e}^{- 2x} + 2}{4} \right) - \left( \frac{\mathrm{e}^{2x}\text{+e}^{- 2x} - 2}{4} \right) = 1\);\par
(2)设\(g(x) = f'(x) - x = \frac{\mathrm{e}^{x} - \mathrm{e}^{- x}}{2} - x\),\par
\(g'(x) = \frac{\mathrm{e}^{x} + \mathrm{e}^{- x}}{2} - 1 \geq \frac{2\sqrt{\mathrm{e}^{x}\text{·e}^{- x}}}{2} - 1 = 0\),\par
所以\(g(x)\)在\((0, + \infty)\)上单调递增,
当\(x > 0\)时,\(g(x) > g(0) = 0\),\par
所以,当\(x \in (0, + \infty)\)时,
\(f'(x) > x\)成立.\par
(3)因为\(x_{i} > 1\),则\(\lnx_{i} > 0\),\par
由(2)知\(f'(x) > x\),
即\(\frac{\mathrm{e}^{x} - \mathrm{e}^{- x}}{2} > x\),\par
\(\therefore\) \(\mathrm{e}^{x} - \frac{1}{\mathrm{e}^{x}} > 2x\)\par
所以\(x_{i} - \frac{1}{x_{i}} = \mathrm{e}^{\lnx_{i}} - \frac{1}{\mathrm{e}^{\lnx_{i}}} > 2\lnx_{i}\).\par
原式\(= \sum_{i = 1}^{n}\left( x_{i} - \frac{1}{x_{i}} \right) > \sum_{i = 1}^{n}{}2\ln x_{i} = 2\ln\left( x_{1}x_{2}\cdots x_{n} \right) = 2{lne}^{2} = 4\).}
\end{question}

\begin{question}
已知\(F(1,0)\)是椭圆\(C:\frac{x^{2}}{a^{2}} + \frac{y^{2}}{b^{2}} = 1(a > b > 0)\)的右焦点,
过\(F\)作直线\(l\)交椭圆于\(A,B\)两点,
其中\(A\)在\(x\)轴上方.当\(AB\bot x\)轴时,
\(|AB| = 3\).
\begin{enumerate}[label=(\arabic*)]
  \item 求椭圆\(C\)的标准方程;
  \item 设\(P(4,0)\),

\item (i)求证:\(\angle APF = \angle BPF\);

\item (ii)设点\(M\)在椭圆\(C\)上,
\item 点\(N\)是\(\bigtriangleup FMP\)的外接圆与椭圆\(C\)的另一个交点(异于\(M\)),
\item 若\(MF\)平分\(\angle AMB\),
\item 且\(\frac{1}{|NA|} + \frac{1}{|NB|} = \frac{\sqrt{3}}{|NF|}\),
\item 求\(\cos\angle ANB\)的值.
\end{enumerate}

\begin{center}
% IMAGE_TODO_START id=auto_e635a4-Q18-img1 path=/Users/muryor/code/mynote/word\\_to\\_tex/output/figures/auto\\_e635a4/media/image3.png width=60% inline=false question_index=18 sub_index=1
% CONTEXT_AFTER: 设$$AP$$与$$BP$$的斜率分别为$$k_{1},k_{2}$$,则 $$k_{1} +
\begin{tikzpicture}[scale=1.05,>=Stealth,line cap=round,line join=round]
  % TODO: AI_AGENT_REPLACE_ME (id=auto_e635a4-Q18-img1)
\end{tikzpicture}
% IMAGE_TODO_END id=auto_e635a4-Q18-img
1
\end{center}


\begin{center}
% IMAGE_TODO_START id=auto_e635a4-Q18-img2 path=/Users/muryor/code/mynote/word\\_to\\_tex/output/figures/auto\\_e635a4/media/image4.png width=60% inline=false question_index=18 sub_index=1
% CONTEXT_AFTER: 将$$x_{1} = my_{1} + 1,x_{2} = my_{2} + 1$$代入②,并化
\begin{tikzpicture}[scale=1.05,>=Stealth,line cap=round,line join=round]
  % TODO: AI_AGENT_REPLACE_ME (id=auto_e635a4-Q18-img2)
\end{tikzpicture}
% IMAGE_TODO_END id=auto_e635a4-Q18-img
2
\end{center}

\topics{根据a;b;c求椭圆标准方程;根据直线与椭圆的位置关系求参数或范围;椭圆中的定值问题}
\difficulty{0.15}
\answer{(1)\(\frac{x^{2}}{4} + \frac{y^{2}}{3} = 1\)
(2)(i)证明见解析;(ii)\(\cos\angle ANB = \frac{1}{2}\).}
\explain{(1)由题知,\(AB = \frac{2b^{2}}{a} = 3\),
又\(a^{2} = b^{2} + 1\),解得\(a = 2\).\par
故椭圆\(C\)的方程为\(\frac{x^{2}}{4} + \frac{y^{2}}{3} = 1\).\par
(2)(i)记\(A\left( x_{1},y_{1} \right),B\left( x_{2},y_{2} \right)\),
由题意知\(y_{1} > 0,y_{2} < 0\).\par
设直线\(AB\)的方程为\(l:x = my + 1\),
代入椭圆得:\(\left( 3m^{2} + 4 \right)y^{2} + 6my - 9 = 0\).\par
则有\(y_{1} + y_{2} = - \frac{6m}{3m^{2} + 4},y_{1}y_{2} = - \frac{9}{3m^{2} + 4}\),
①\par
设\(AP\)与\(BP\)的斜率分别为\(k_{1},k_{2}\),则\par
\(k_{1} + k_{2} = \frac{y_{1}}{x_{1} - 4} + \frac{y_{2}}{x_{2} - 4} = \frac{2my_{1}y_{2} - 3\left( y_{1} + y_{2} \right)}{\left( my_{1} - 3 \right)\left( my_{2} - 3 \right)} = \frac{\frac{- 18m}{3m^{2} + 4} + \frac{18m}{3m^{2} + 4_{1}}}{\left( my_{1} - 3 \right)\left( my_{2} - 3 \right)} = 0\),\par
所以\(\angle APF = \angle BPF\).\par
(ii)设\(T\left( x_{0},y_{0} \right)\)满足\(\frac{|AM|}{|BM|} = \frac{|AF|}{|BF|} = \frac{|AT|}{|BT|}\),
则\par
\(\begin{array}{r}
\frac{\left( x_{0} - x_{1} \right)^{2} + \left( y_{0} - y_{1} \right)^{2}}{\left( x_{0} - x_{2} \right)^{2} + \left( y_{0} - y_{2} \right)^{2}} = \frac{y_{1}^{2}}{y_{2}^{2}}
\end{array}\)②\par
将\(x_{1} = my_{1} + 1,x_{2} = my_{2} + 1\)代入②,
并化简得\par
\(x_{0}^{2} + y_{0}^{2} - 2x_{0}\left( 1 + m\frac{y_{1}y_{2}}{y_{1} + y_{2}} \right) - 2y_{0}\frac{y_{1}y_{2}}{y_{1} + y_{2}} + 2m\frac{y_{1}y_{2}}{y_{1} + y_{2}} + 1 = 0\),
③\par
将(2)中①代入③得:\(x_{0}^{2} + y_{0}^{2} - 5x_{0} - \frac{3}{m}y_{0} + 4 = 0\),\par
即\(\left( x_{0} - \frac{5}{2} \right)^{2} + \left( y_{0} - \frac{3}{2m} \right)^{2} = \frac{9}{4m^{2}} + \frac{9}{4}\).\par
又因为直线\(AB\)和直线\(x = 4\)的交点为\(R\left( 4,\frac{3}{m} \right)\).\par
故满足\(\frac{|AM|}{|BM|} = \frac{|AF|}{|BF|} = \frac{|AT|}{|BT|}\)的\(T\)点都在以\(FR\)为直径的圆上.\par
因为\(R,P,F,M,N\)都在以\(FR\)为直径的圆上,\par
故\(\frac{|AT|}{|BT|} = \frac{|AF|}{|BF|} = \frac{|AN|}{|BN|}\),
所以\(NF\)是\(\angle ANB\)的角平分线.\par
则\(S_{\bigtriangleup ANF} + S_{\bigtriangleup BNF} = S_{\bigtriangleup ABN}\),\par
所以\(|NA| \cdot |NF| \cdot \sin\angle ANF + |NB| \cdot |NF| \cdot \sin\angle BNF = |NA| \cdot |NB| \cdot \sin\angle ANB\),\par
即\(\frac{\sin\angle ANF}{|NB|} + \frac{\sin\angle BNF}{|NA|} = \frac{\sin\angle ANB}{|NF|} = \frac{2\sin\angle ANF\cos\angle ANF}{|NF|}\).\par
所以\(\frac{1}{|NA|} + \frac{1}{|NB|} = \frac{2\cos\angle ANF}{|NF|} = \frac{\sqrt{3}}{|NF|}\),
解得\(\cos\angle ANF = \frac{\sqrt{3}}{2}\),\par
所以\(\cos\angle ANB = \frac{1}{2}\).}
\end{question}

\begin{question}
现有一款益智棋类游戏,棋盘由全等的正三角形组成(如图所示),
假设棋盘足够大.一颗质地均匀的正方体骰子,
六个面分别以\(1 \sim 6\)标号.在棋盘上,
以\(O\)为原点建立平面直角坐标系,
设点\(A\)的坐标为\((1,0)\).棋子初始位置为坐标原点,
投掷骰子\(n\)次,
用\(X_{n}\)表示第\(n\)次投掷后棋子的位置(\(X_{0}\)为坐标原点),
规定:\(\overrightarrow{OX_{n}} = \left\{ \begin{array}{r}
\overrightarrow{OX_{n - 1}} + \overrightarrow{u_{k}},第n次掷得奇数\text{,} \\
\overrightarrow{OX_{n - 1}},第n次掷得偶数\text{,}
\end{array} \right.\)其中向量\(\overrightarrow{u_{k}} = \left( \cos\frac{2k\pi}{3},\sin\frac{2k\pi}{3} \right)(k \in Z),k\)为前\(n\)次投掷过程中,掷得偶数的总次数.
\begin{enumerate}[label=(\arabic*)]
  \item 求点\(X_{2}\)所有可能的坐标;
  \item 求投掷骰子8次后棋子在原点的概率;
  \item 投掷骰子80次,记棋子在原点且投掷过程中掷得奇数的次数恰为\(r(0 \leq r \leq 80)\)的概率为\(p(r)\),求\(p(r)\)的表达式,并指出当\(r\)为何值时,\(p(r)\)取得最大值.
\end{enumerate}

\begin{center}
% IMAGE_TODO_START id=auto_e635a4-Q19-img1 path=/Users/muryor/code/mynote/word\\_to\\_tex/output/figures/auto\\_e635a4/media/image5.png width=60% inline=false question_index=19 sub_index=1
% CONTEXT_AFTER: (1)求点$$X_{2}$$所有可能的坐标; (2)求投掷骰子8次后棋子在原点的概率; (3
\begin{tikzpicture}[scale=1.05,>=Stealth,line cap=round,line join=round]
  % TODO: AI_AGENT_REPLACE_ME (id=auto_e635a4-Q19-img1)
\end{tikzpicture}
% IMAGE_TODO_END id=auto_e635a4-Q19-img
1
\end{center}


\begin{center}
% IMAGE_TODO_START id=auto_e635a4-Q19-img2 path=/Users/muryor/code/mynote/word\\_to\\_tex/output/figures/auto\\_e635a4/media/image6.png width=60% inline=false question_index=19 sub_index=1
% CONTEXT_BEFORE: $$,每两个操作小节也由操作$$T$$连接,所以共有$$27 - m$$个操作小节,如下图所示:
\begin{tikzpicture}[scale=1.05,>=Stealth,line cap=round,line join=round]
  % TODO: AI_AGENT_REPLACE_ME (id=auto_e635a4-Q19-img2)
\end{tikzpicture}
% IMAGE_TODO_END id=auto_e635a4-Q19-img
2
\end{center}

\topics{平面向量线性运算的坐标表示;x+y+z=n的整数解的个数;二项式系数的增减性和最值;计算古典概型问题的概率}
\difficulty{0.15}
\answer{(1)\((0,0),\left( - \frac{1}{2},\frac{\sqrt{3}}{2} \right),(1,0),(2,0)\);
(2)\(\frac{5}{128}\);
(3)\(p(r) = \left\{ \begin{matrix}
0, & r不是3的倍数\text{,} \\
\frac{1}{2^{80}}\left( \mathbb{C}_{26}^{\frac{r}{3}} \right)^{3}, & r是3的倍数\text{,}
\end{matrix} \right.\),\(r = 39\)时,\(p(r)\)取得最大值.}
\explain{(1)由题意,
点\(X_{2}\)可能的坐标为\((0,0),\left( - \frac{1}{2},\frac{\sqrt{3}}{2} \right),(1,0),(2,0)\).\par
(2)令向量\(\overrightarrow{a} = (1,0),\overrightarrow{b} = \left( - \frac{1}{2},\frac{\sqrt{3}}{2} \right),\overrightarrow{c} = \left( - \frac{1}{2}, - \frac{\sqrt{3}}{2} \right)\),\par
则当\(k = 3m\)时,
\(u_{k} = \overrightarrow{a}\);当\(k = 3m + 1\)时,
\(u_{k} = \overrightarrow{b}\);\par
当\(k = 3m + 2\)时\(u_{k} = \overrightarrow{c}\),
其中\(m \in N\),
且\(\overrightarrow{a} + \overrightarrow{b} + \overrightarrow{c} = \overrightarrow{0}\).\par
要保证\(X_{8}\)为原点,则在8次投掷过程中,
掷得奇数的次数\(r\)应为\(0,3,6\).\par
①若\(r = 0\),即8次投掷全部为偶数,共1种情况:偶偶偶偶偶偶偶偶;\par
②若\(r = 3\),即8次投掷过程中有5次偶数,3次奇数,则共8种情况:\par
奇偶奇偶奇偶偶偶,奇偶奇偶偶偶偶奇,奇偶偶奇偶偶奇偶,奇偶偶偶偶奇偶奇,\par
偶奇偶奇偶奇偶偶,偶奇偶偶奇偶偶奇,偶偶奇偶奇偶奇偶,偶偶偶奇偶奇偶奇;\par
③若\(r = 6\),即6次奇数,仅有1种情况:奇奇偶奇奇偶奇奇.\par
故\(X_{8}\)为坐标原点的概率\(p = \frac{10}{2^{8}} = \frac{5}{128}\).\par
(3)当\(r\)不是3的倍数时,显然有\(p(r) = 0\).\par
以下讨论当\(r\)是3的倍数的情况.不妨设\(r = 3m\),
则掷得偶数的次数为\(80 - 3m\)次.\par
记进行加向量\(\overrightarrow{a}\)为操作\(A\),
加向量\(\overrightarrow{b}\)为操作\(B\),
加向量\(\overrightarrow{c}\)为操作\(C\),不做任何操作记为操作\(T\).\par
定义操作小结:\(\underset{x次}{\overset{A\cdots A}{︸}}T\underset{y次}{\overset{B\cdots B}{︸}}T\underset{z次}{\overset{C\cdots C}{︸}}\),
其中\(x,y,z\)可以为0.\par
在80次投掷产生的操作过程,
可分为若干操作小结.注意到1个操作小节中有2次操作\(T\),
每两个操作小节也由操作\(T\)连接,所以共有\(27 - m\)个操作小节,
如下图所示:\par
所以有\(\left\{ \begin{array}{r}
x_{1} + x_{2} + \cdots + x_{27 - m} = m, \\
y_{1} + y_{2} + \cdots + y_{27 - m} = m, \\
z_{1} + z_{2} + \cdots + z_{27 - m} = m,
\end{array} \right.\)其中\(x_{i},y_{i},z_{i} \in N,1 \leq i \leq 27 - m\).\par
由隔板法可知,上述不定方程共有\(\left( \mathbb{C}_{26}^{m} \right)^{3}\)组解,而每一组解对应着一种满足题意的投掷,于是有\(p(r) = \frac{\left( \mathbb{C}_{26}^{m} \right)^{3}}{2^{80}} = \frac{1}{2^{80}}\left( \mathbb{C}_{26}^{\frac{r}{3}} \right)^{3}\)\par
.综上,有\(p(r) = \left\{ \begin{matrix}
0, & r不是3的倍数\text{,} \\
\frac{1}{2^{80}}\left( \mathbb{C}_{26}^{\frac{r}{3}} \right)^{3}, & r是3的倍数\text{,}
\end{matrix} \right.\)\par
因此,当\(\frac{r}{3} = 13\),即\(r = 39\)时,\(p(r)\)取得最大值.}
\end{question}
