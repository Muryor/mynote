\examxtitle{测试试卷 - auto_137268}

\section{单选题}

\begin{question}
已知集合\(U = \{ 1,2,3,4,5,6,7,8\}\),
\(A = \{ 2,3,4\}\),
则集合\(\complement_{U}A =\)(    )
\begin{choices}
  \item \(\{ 1\}\)
  \item \(\{ 2,3,4\}\)
  \item \(\{ 5,6,7,8\}\)
  \item \(\{ 1,5,6,7,8\}\)
\end{choices}
\topics{补集的概念及运算}
\difficulty{0.94}
\answer{D}
\explain{由\(U = \{ 1,2,3,4,5,6,7,8\}\),
\(A = \{ 2,3,4\}\),
则\(\complement_{U}A =\{ 1,5,6,7,8\}\)}
\end{question}

\begin{question}
已知等差数列\(\left\{ a_{n} \right\}\)满足\(a_{1} = 2,a_{4} + a_{6} = 20\),
则\(a_{3} =\)(    )
\begin{choices}
  \item 4
  \item 6
  \item 8
  \item 10
\end{choices}
\topics{等差中项的应用}
\difficulty{0.85}
\answer{B}
\explain{由题设\(a_{4} + a_{6} = 2a_{5} = 20 \Rightarrow a_{5} = 10\),
而\(a_{1} = 2\),\par
所以\(a_{1} + a_{5} = 2a_{3} = 12 \Rightarrow a_{3} = 6\)}
\end{question}

\begin{question}
\(\frac{5}{2 + \text{i}} =\)(   )
\begin{choices}
  \item \(2 + \text{i}\)
  \item \(2 - \text{i}\)
  \item \(- 2 + \text{i}\)
  \item \(- 2 - \text{i}\)
\end{choices}
\topics{复数代数形式的乘法运算;复数的除法运算}
\difficulty{0.85}
\answer{B}
\explain{\(\frac{5}{2 + \text{i}} = \frac{\left( 2 + \text{i} \right)\left( 2 - \text{i} \right)}{2 + \text{i}} = 2 - \text{i}\)}
\end{question}

\begin{question}
已知\(\frac{1}{\log_{9}a} + \frac{1}{\log_{27}a} = \frac{5}{3}\),则\(a =\)(    )
\begin{choices}
  \item 3
  \item 9
  \item 27
  \item 81
\end{choices}
\topics{指数幂的运算;指数式与对数式的互化;对数的运算;运用换底公式化简计算}
\difficulty{0.65}
\answer{C}
\explain{\(\frac{1}{\log_{9}a} + \frac{1}{\log_{27}a} = \log_{a}9 + \log_{a}27 = \log_{a}3^{5} = \frac{5}{3}\),\par
所以\(a^{\frac{5}{3}} = 3^{5}\),
则\(a^{5} = \left( 3^{5} \right)^{3} = 27^{5}\),
解得\(a = 27\)}
\end{question}

\begin{question}
已知随机变量\(X \sim N\left( 2,\sigma^{2} \right)\),
且\(P(X < 0) = 0.3\),
则\(P(0 < X < 4)\)的值为(    )
\begin{choices}
  \item 0.2
  \item 0.4
  \item 0.7
  \item 0.35
\end{choices}
\topics{指定区间的概率}
\difficulty{0.65}
\answer{B}
\explain{由题设\(P(X < 2) = 0.5\),
且\(P(X < 0) = 0.3\),
则\(P(0 < X < 2) = 0.2\),\par
由正态分布曲线关于\(X = 2\)对称,
则\(P(0 < X < 4) = 0.4\)}
\end{question}

\begin{question}
若圆\(C:(x - 1)^{2} + (y + 3)^{2} = 1\)上存在两点\(A,B\),
直线\(l:3x - 4y + m = 0\)上存在点\(P\),
使得\(\angle APB = 60{^\circ}\),
则实数\(m\)的取值范围为(    )
\begin{choices}
  \item \(\lbrack - 25, - 5\rbrack\)
  \item \(( - \infty, - 25\rbrack \cup \lbrack - 5, + \infty)\)
  \item \(\lbrack - 35,5\rbrack\)
  \item \(( - \infty, - 35\rbrack \cup \lbrack 5, + \infty)\)
\end{choices}

\begin{center}
% IMAGE_TODO_START id=auto_137268-Q6-img1 path=/Users/muryor/code/mynote/word\\_to\\_tex/output/figures/auto\\_137268/media/image2.png width=60% inline=false question_index=6 sub_index=1
% CONTEXT_BEFORE: 所以一定存在*A*、*B*及*P*,使得$$\angle APB = 60{^\circ}$$;
% CONTEXT_AFTER: 当直线与圆相切时,同直线与圆相交分析可知,一定存在*A*、*B*及*P*,使得$$\angle
\begin{tikzpicture}[scale=1.05,>=Stealth,line cap=round,line join=round]
  % TODO: AI_AGENT_REPLACE_ME (id=auto_137268-Q6-img1)
\end{tikzpicture}
% IMAGE_TODO_END id=auto_137268-Q6-img
1
\end{center}


\begin{center}
% IMAGE_TODO_START id=auto_137268-Q6-img2 path=/Users/muryor/code/mynote/word\\_to\\_tex/output/figures/auto\\_137268/media/image3.png width=60% inline=false question_index=6 sub_index=1
% CONTEXT_BEFORE: *C*相切于*A*,此时$$\angle APB$$必为该*P*点所能达到的最大情况,如图所示,
% CONTEXT_AFTER: 由图可知$$\sin\angle CPA = {CP}$$,$$\angle A
\begin{tikzpicture}[scale=1.05,>=Stealth,line cap=round,line join=round]
  % TODO: AI_AGENT_REPLACE_ME (id=auto_137268-Q6-img2)
\end{tikzpicture}
% IMAGE_TODO_END id=auto_137268-Q6-img
2
\end{center}

\topics{求点到直线的距离;由直线与圆的位置关系求参数}
\difficulty{0.15}
\answer{A}
\explain{当直线与圆相交时,如图所示,若\emph{A}、\emph{B}离直线越近时,
直至与直线和圆\emph{C}的两交点重合,
此时\(\angle APB = \pi\),\par
若\emph{A}、\emph{B}相距越来越近时,
直至\emph{A}、\emph{B}两点重合,此时\(\angle APB = 0{^\circ}\),\par
所以一定存在\emph{A}、\emph{B}及\emph{P},
使得\(\angle APB = 60{^\circ}\);\par
当直线与圆相切时,同直线与圆相交分析可知,
一定存在\emph{A}、\emph{B}及\emph{P},
使得\(\angle APB = 60{^\circ}\);\par
当直线与圆没有公共点时,对直线上的任一点\emph{P},
若\emph{A}、\emph{B}相距越来越近时,直至\emph{A}、\emph{B}两点重合时,
仍有\(\angle APB = 0{^\circ}\),\par
另一方面,若*PB*与圆\emph{C}相切于\emph{B},
*PA*与圆\emph{C}相切于\emph{A},
此时\(\angle APB\)必为该\emph{P}点所能达到的最大情况,如图所示,\par
由图可知\(\sin\angle CPA = \frac{r}{CP}\),
\(\angle APB = 2\angle CPA\),*CP*最短时,\par
即等于圆心\emph{C}到直线的距离\emph{d},
\(\sin\angle CPA\)最大,\(\angle CPA\)也最大,
同时\(\angle APB\)最大,\par
所以若圆\(C\)上存在两点\(A,B\),直线\(l\)上存在点\(P\),
使得\(\angle APB = 60{^\circ} = \frac{\pi}{3}\),\par
则必有\(\frac{r}{d} \geq \sin\frac{\pi}{6} = \frac{1}{2}\),
解得\(d \leq 2r\),又因为圆\(C\)的半径\(r = 1\),\par
圆心\(C(1, - 3)\)到直线\(3x - 4y + m = 0\)的距离\(d = \frac{\left| 3 \times 1 - 4 \times ( - 3) + m \right|}{\sqrt{3^{2} + ( - 4)^{2}}} = \frac{|m + 15|}{5}\),\par
所以\(\frac{|m + 15|}{5} \leq 2\),
解得\(- 25 \leq m \leq - 5\).}
\end{question}

\begin{question}
设\(\theta\)为两个非零向量\(\overrightarrow{a},\overrightarrow{b}\)所成的角,
已知对任意\(t \in \mathbb{R}\),
\(|\overrightarrow{a} - t\overrightarrow{b}|\)的最小值为\(\frac{1}{2}|\overrightarrow{a}|\),
则\(\theta =\)(    )
\begin{choices}
  \item \(\frac{\pi}{6}\)
  \item \(\frac{\pi}{3}\)
  \item \(\frac{\pi}{6}\)或\(\frac{5\pi}{6}\)
  \item \(\frac{\pi}{3}\)或\(\frac{2\pi}{3}\)
\end{choices}

\begin{center}
% IMAGE_TODO_START id=auto_137268-Q7-img1 path=/Users/muryor/code/mynote/word\\_to\\_tex/output/figures/auto\\_137268/media/image4.png width=60% inline=false question_index=7 sub_index=1
% CONTEXT_BEFORE: tarrow{a} - t|$$即为线段$$AC$$的长度,
% CONTEXT_AFTER: 由对任意$$t \in $$,$$| - t
\begin{tikzpicture}[scale=1.05,>=Stealth,line cap=round,line join=round]
  % TODO: AI_AGENT_REPLACE_ME (id=auto_137268-Q7-img1)
\end{tikzpicture}
% IMAGE_TODO_END id=auto_137268-Q7-img
1
\end{center}

\topics{向量减法法则的几何应用;向量与几何最值}
\difficulty{0.4}
\answer{C}
\explain{令\(\overrightarrow{a} = \overrightarrow{OA},\overrightarrow{b} = \overrightarrow{OB},t\overrightarrow{b} = \overrightarrow{OC}\),
如下图示,
\(|\overrightarrow{a} - t\overrightarrow{b}|\)即为线段\(AC\)的长度,\par
由对任意\(t \in \mathbb{R}\),
\(|\overrightarrow{a} - t\overrightarrow{b}|\)的最小值为\(\frac{1}{2}|\overrightarrow{a}|\),
即\(|AC|_{\min} = \frac{1}{2}|\overrightarrow{a}|\),
而\(\angle AOB = \theta\),\par
显然\(AC\bot OB\)时,线段\(AC\)最短,
此时\(|AC|_{\min} = |\overrightarrow{OA}|\sin\theta = |\overrightarrow{a}|\sin\theta = \frac{1}{2}|\overrightarrow{a}|\),\par
所以\(\sin\theta = \frac{1}{2}\),
又\(\theta \in \lbrack 0,\pi\rbrack\),
故\(\theta =\frac{\pi}{6}\)或\(\frac{5\pi}{6}\)}
\end{question}

\begin{question}
若双曲线\(\frac{y^{2}}{a^{2}} - \frac{x^{2}}{b^{2}} = 1(a > 0,b > 0)\)不存在以点\((a,2a)\)为中点的弦,
则该双曲线离心率的取值范围为(    )
\begin{choices}
  \item \(\left( 1,\frac{2\sqrt{3}}{3} \right\rbrack\)
  \item \(\left( 1,\frac{\text{\,}\sqrt{5}}{2} \right\rbrack\)
  \item \(\left\lbrack \frac{\sqrt{5}}{2},\frac{2\sqrt{3}}{3} \right\rbrack\)
  \item \(\left\lbrack \frac{\sqrt{5}}{2}, + \infty \right)\)
\end{choices}
\topics{求双曲线的离心率或离心率的取值范围}
\difficulty{0.4}
\answer{C}
\explain{由题意得点\((a,2a)\)在双曲线外部或在双曲线上,
则\(\frac{(2a)^{2}}{a^{2}} - \frac{a^{2}}{b^{2}} \leq 1\),
得\(\frac{b^{2}}{a^{2}} \leq \frac{1}{3}\),\par
假设存在以\((a,2a)\)为中点的弦,
设弦与双曲线交于点\(A\left( x_{1},y_{1} \right)\),
\(B\left( x_{2},y_{2} \right)\),\par
则\(\frac{x_{1} + x_{2}}{2} = a\),
\(\frac{y_{1} + y_{2}}{2} = 2a\),\par
由点\(A\left( x_{1},y_{1} \right)\),
\(B\left( x_{2},y_{2} \right)\)在双曲线上,得\(\left\{ \begin{array}{r}
\frac{y_{1}^{2}}{a^{2}} - \frac{x_{1}^{2}}{b^{2}} = 1 \\
\frac{y_{2}^{2}}{a^{2}} - \frac{x_{2}^{2}}{b^{2}} = 1
\end{array} \right.\),\par
两式作差得\(\frac{\left( y_{1} + y_{2} \right)\left( y_{1} - y_{2} \right)}{a^{2}} = \frac{\left( x_{1} + x_{2} \right)\left( x_{1} - x_{2} \right)}{b^{2}}\),\par
所以\(k_{AB} = \frac{y_{1} - y_{2}}{x_{1} - x_{2}} = \frac{a^{2}\left( x_{1} + x_{2} \right)}{b^{2}\left( y_{1} + y_{2} \right)} = \frac{a^{2} \cdot 2a}{b^{2} \cdot 4a} = \frac{a^{2}}{2b^{2}}\),\par
因为不存在该中点弦,所以直线*AB*与双曲线至多一个交点,\par
则\(k_{AB} = \frac{a^{2}}{2b^{2}} \leq \frac{a}{b}\),也即\(\frac{b}{a} \geq \frac{1}{2}\),\par
所以\(\frac{1}{4} \leq \frac{b^{2}}{a^{2}} \leq \frac{1}{3}\),则\(e = \frac{c}{a} = \sqrt{1 + \frac{b^{2}}{a^{2}}} \in \left\lbrack \frac{\sqrt{5}}{2},\frac{2\sqrt{3}}{3} \right\rbrack\).}
\end{question}

\section{多选题}

\begin{question}
已知圆锥的侧面展开图是半径等于2的半圆,则圆锥的(    )
\begin{choices}
  \item 底面半径为1
  \item 表面积为\(2\pi\)
  \item 体积为\(\frac{\sqrt{3}}{3}\pi\)
  \item 外接球与内切球半径比值为3
\end{choices}
\topics{圆锥中截面的有关计算;圆锥表面积的有关计算;锥体体积的有关计算;多面体与球体内切外接问题}
\difficulty{0.65}
\answer{AC}
\explain{由题意,圆锥的母线长为2,底面周长为\(2\pi\),\par
若底面半径为\(r\),
则\(2\pir = 2\pi \Rightarrow r = 1\),A对,\par
表面积为\(\frac{1}{2}\pi \times 2^{2} + \pi \times 1^{2} = 3\pi\),
B错,\par
由上,
圆锥的高\(h = \sqrt{2^{2} - 1^{2}} = \sqrt{3}\),
则圆锥体积为\(\frac{1}{3}h\pir^{2} = \frac{1}{3} \times \sqrt{3}\pi = \frac{\sqrt{3}}{3}\pi\),
C对,\par
由上,圆锥轴截面是边长为2的等边三角形,其外接圆和内切圆半径,
分别为圆锥的外接球和内切球半径,\par
所以圆锥的外接球半径为\(\frac{2}{3} \times 2 \times \sin 60{^\circ} = \frac{2}{\sqrt{3}}\),
内切球半径为\(\frac{1}{3} \times 2 \times \sin 60{^\circ} = \frac{1}{\sqrt{3}}\),\par
所以外接球与内切球半径比值为2,D错}
\end{question}

\begin{question}
已知函数\(f(x) = x^{2}(x - a)\)在\(x = 2\)处取得极小值,\(y = f'(x)\)为其导函数,则(    )
\begin{choices}
  \item \(a = 3\)
  \item \(f'\left( \sqrt{3} + 1 \right) - f'\left( 1 - \sqrt{2} \right) < 0\)
  \item \(f(x) \geq - 4\)的解集为\(\lbrack - 1, + \infty)\)
  \item \(\forall x > 0,f\left( x + \frac{1}{x} \right) > f( - x - 1)\)
\end{choices}
\topics{利用导数研究不等式恒成立问题;根据极值点求参数}
\difficulty{0.4}
\answer{ACD}
\explain{对于A,\(f'(x) = 3x^{2} - 2ax\),
由题意可知\(f'(2) = 0\),解得\(a = 3\),
此时\(f(x) = x^{2}(x - 3)\),故A正确;\par
对于B,由\(f'(x) = 3x^{2} - 6x\),其为二次函数,开口向上,
对称轴为\(x = \frac{6}{2 \cdot 3} = 1\),\par
则\(\sqrt{3} + 1\)到对称轴的距离为\(\left| \sqrt{3} + 1 - 1 \right| = \sqrt{3}\),
\(1 - \sqrt{2}\)到对称轴的距离为\(\left| 1 - \sqrt{2} - 1 \right| = \sqrt{2} < \sqrt{3}\),\par
结合开口向上的二次函数图像特点可知,离对称轴较远的点函数值更大,
也即\(f'\left( \sqrt{3} + 1 \right) > f'\left( 1 - \sqrt{2} \right)\),
即\(f'\left( \sqrt{3} + 1 \right) - f'\left( 1 - \sqrt{2} \right) > 0\),
故B错误;\par
对于C,解不等式\(f(x) \geq - 4\),
即\(x^{3} - 3x^{2} \geq - 4\),
整理为\(x^{3} - 3x^{2} + 4 \geq 0\),\par
因式分解得\(x^{3} - 3x^{2} + 4 = (x + 1)(x - 2)^{2} \geq 0\),
解得\(x \geq - 1\),
故解集为\(\lbrack - 1, + \infty)\),故C正确;\par
对于D,对于\(\forall x > 0\),
有\(x + \frac{1}{x} \geq 2\sqrt{x \cdot \frac{1}{x}} = 2\),
当且仅当\(x = 1\)时取等号,同时\(- x - 1 < - 1\),\par
由于\(f'(x) = 3x^{2} - 6x = 3x(x - 2)\),
当\(x < 0\)或\(x > 2\)时,\(f'(x) > 0\),
\(f(x)\)单调递增;\par
当\(0 < x < 2\)时,\(f'(x) < 0\),
\(f(x)\)单调递减,\par
所以\(f\left( x + \frac{1}{x} \right) \geq f(2) = - 4\),
\(f( - x - 1) < f( - 1) = - 4\),
所以\(\forall x > 0,f\left( x + \frac{1}{x} \right) > f( - x - 1)\),
故D正确.}
\end{question}

\begin{question}
在\(\bigtriangleup ABC\)中,若\(A = cosA,B = cos(cosB),C = ktan(sinC)\),则(    )
\begin{choices}
  \item \(A = B\)
  \item \(B < C\)
  \item \(C < \frac{\pi}{2}\)
  \item \(k < 2\)
\end{choices}

\begin{center}
% IMAGE_TODO_START id=auto_137268-Q11-img1 path=/Users/muryor/code/mynote/word\\_to\\_tex/output/figures/auto\\_137268/media/image5.png width=60% inline=false question_index=11 sub_index=1
% CONTEXT_BEFORE: = x$$与$$y = \cos x$$在$$(0,)$$上的图象,如下图示,
% CONTEXT_AFTER: 显然$$y = x$$与$$y = \cos x$$在$$(0,)$$上有且仅有
\begin{tikzpicture}[scale=1.05,>=Stealth,line cap=round,line join=round]
  % TODO: AI_AGENT_REPLACE_ME (id=auto_137268-Q11-img1)
\end{tikzpicture}
% IMAGE_TODO_END id=auto_137268-Q11-img
1
\end{center}

\topics{用导数判断或证明已知函数的单调性;余弦函数图象的应用;由不等式的性质比较数(式)大小;求函数零点或方程根的个数}
\difficulty{0.15}
\answer{ABD}
\explain{根据\(y = x\)与\(y = \cos x\)在\((0,\pi)\)上的图象,
如下图示,\par
显然\(y = x\)与\(y = \cos x\)在\((0,\pi)\)上有且仅有唯一交点,\par
即\(x = \cos x\)在\((0,\pi)\)上有且仅有一个根,
而\(A = \cos A\),\(B = \cos(\cos B)\),\par
由\(A,B \in (0,\pi)\),所以\(A = \cos B\),
且\(A = B\),A对,\par
又\(\frac{\pi}{6} < \cos\frac{\pi}{6} = \frac{\sqrt{3}}{2}\),
\(\frac{\pi}{4} > \cos\frac{\pi}{4} = \frac{\sqrt{2}}{2}\),
则\(\frac{\pi}{6} < A = B < \frac{\pi}{4}\)\par
所以\(C = \pi - (A + B) \in (\frac{\pi}{2},\frac{2\pi}{3})\),
即\(B < C\),B对,C错,\par
由\(C \in (\frac{\pi}{2},\frac{2\pi}{3})\),
则\(\sin C \in (\frac{\sqrt{3}}{2},1)\),
而\(C = k\tan(\sin C)\)中\(k > 0\),\par
由\(y = \frac{x}{k}\)在区间\((\frac{\pi}{2},\frac{2\pi}{3})\)上单调递增,
\(y = \tan(\sin x)\)在区间\((\frac{\pi}{2},\frac{2\pi}{3})\)上单调递减,\par
所以,只需\(\left\{ \begin{array}{r}
\frac{\pi}{2k} < \tan 1 \\
\frac{2\pi}{3k} > \tan\frac{\sqrt{3}}{2}
\end{array} \right.\),可得\(\frac{\pi}{2\tan 1} < k < \frac{2\pi}{3\tan\frac{\sqrt{3}}{2}}\),\par
令\(f(x) = \tan x - x - \frac{x^{3}}{3}\)且\(0 < x < \frac{\pi}{2}\),则\(f'(x) = \frac{1}{\cos^{2}x} - 1 - x^{2} = 1 + \tan^{2}x - 1 - x^{2} = \tan^{2}x - x^{2}\),\par
对于\(g(x) = \tan x - x\)且\(0 < x < \frac{\pi}{2}\),则\(g'(x) = \frac{1 - \cos^{2}x}{\cos^{2}x} > 0\),故\(g(x)\)在\((0,\frac{\pi}{2})\)上单调递增,\par
所以\(g(x) > g(0) = 0 \Rightarrow \tan x > x\),即\(\tan^{2}x > x^{2}\),则\(f'(x) > 0\),\par
所以\(f(x)\)在\((0,\frac{\pi}{2})\)上单调递增,故\(f(x) > f(0) = 0\),即\(f(\frac{\sqrt{3}}{2}) = \tan\frac{\sqrt{3}}{2} - \frac{\sqrt{3}}{2} - \frac{\sqrt{3}}{8} > 0\),\par
所以\(\tan\frac{\sqrt{3}}{2} > \frac{5\sqrt{3}}{8}\),而\({(15\sqrt{3})}^{2} = 675 > 64\pi^{2} = {(8\pi)}^{2}\),则\(15\sqrt{3} > 8\pi\),即\(\frac{5\sqrt{3}}{8} > \frac{\pi}{3}\),\par
所以\(\tan\frac{\sqrt{3}}{2} > \frac{\pi}{3}\),故\(\frac{2\pi}{3\tan\frac{\sqrt{3}}{2}} < 2\),即\(k < 2\),D对}
\end{question}

\section{填空题}

\begin{question}
\({(1 + x)}^{5}\)的展开式中\(x^{3}\)项的系数为 .
\topics{求指定项的系数}
\difficulty{0.94}
\answer{\(10\)}
\explain{\({(1 + x)}^{5}\)的展开式中\(x^{3}\)项的系数为\(\mathbb{C}_{5}^{3} = 10\).\(10\)}
\end{question}

\begin{question}
\(\bigtriangleup ABC\)的三个内角\(A,B,C\)的对边分别为\(a,b,c\),
满足\(C = \frac{\pi}{4}\),
且\(a^{2} + b^{2} - c^{2} = 4\),
则\(\bigtriangleup ABC\)的面积为
.
\topics{三角形面积公式及其应用;余弦定理解三角形}
\difficulty{0.85}
\answer{1}
\explain{由余弦定理可得:\(c^{2} = a^{2} + b^{2} - 2ab\cos\frac{\pi}{4}\),
又\(a^{2} + b^{2} - c^{2} = 4\),\par
得\(c^{2} = c^{2} + 4 - \sqrt{2}ab\),
解得\(ab = 2\sqrt{2}\),
所以\(\bigtriangleup ABC\)的面积为\(\frac{1}{2}ab\sin C = 1\);
\(1\)}
\end{question}

\begin{question}
平面直角坐标系中,原点\(O\)处有一只蚂蚁,每过1秒,
该蚂蚁都会随机地选择上、下、左、右四个方向之一移动一个单位长度,那么在6秒后,
蚂蚁到原点\(O\)的距离等于\(\sqrt{2}\)的概率为
.
\topics{实际问题中的组合计数问题;计算古典概型问题的概率}
\difficulty{0.15}
\answer{\(\frac{75}{256}\)}
\explain{蚂蚁每一秒有4种移动方向,共移动6秒,根据分步乘法计数原理,
总路径数为\(4^{6}\),\par
若蚂蚁到原点\emph{O}的距离为\(\sqrt{2}\),
原点\((0,0)\)到该点\((x,y)\)的距离满足\(x^{2} + y^{2} = 2\),\par
设蚂蚁右移\emph{a}次、左移\emph{b}次,
则\(x = a - b\),上移\emph{c}次、下移\emph{d}次,
则\(y = c - d\),总步数\(a + b + c + d = 6\),\par
要满足\(x^{2} + y^{2} = 2\),
即\((a - b)^{2} + (c - d)^{2} = 2\),
由于\(a,b,c,d\)都是非负整数,可能的组合必须满足\(|x| = 1\),
\(|y| = 1\),此时,\(a - b = \pm 1\),
\(c - d = \pm 1\),
且\((a + b) + (c + d) = 6\),\par
设左右移动总次数为\(m = a + b\),
上下移动总次数为\(n = c + d\),则\(m + n = 6\),\par
由于\(a = \frac{m + (a - b)}{2}\),
\(b = \frac{m - (a - b)}{2}\)需为整数,
则\emph{m}与\(|a - b|\)同奇偶,所以\emph{m}为奇数,同理,
\emph{n}也为奇数,\par
又\(m + n = 6\),
可能的组合有\(m = 1,n = 5\)、\(m = 3,n = 3\)、\(m = 5,n = 1\),\par
当\(m = 1\),\(n = 5\)时,左右移动1次,
满足\(|x| = 1\)的方式有2种,即左或右,\par
上下移动5次,满足\(|y| = 1\)的方式有\(c = 3\),
\(d = 2\)或\(c = 2\),\(d = 3\),
共\(2 \times \mathbb{C}_{5}^{3}\)种,即选3次上移,
剩下2次下移,或选2次上移,剩下3次下移,\par
其次,在6步中选1步用于左右移动,其余5步用于上下移动,
\(\mathbb{C}_{6}^{1}\)种,因此,
此情况的路径数为\(\mathbb{C}_{6}^{1} \times 2 \times 2 \times \mathbb{C}_{5}^{3} = 240\);\par
当\(m = 3\),\(n = 3\)时,左右移动3次,
满足\(|x| = 1\)的方式有\(a = 2\),
\(b = 1\)或\(a = 1\),\(b = 2\),
共\(2 \times \mathbb{C}_{3}^{2}\)种,上下移动3次,
满足\(|y| = 1\)的方式同理,
共\(2 \times \mathbb{C}_{3}^{2}\)种,\par
此外,6步中选择3步左右移动,剩余上下移动,
共\(\mathbb{C}_{6}^{3}\)种,\par
因此,
此情况的路径数为\(\mathbb{C}_{6}^{3} \times 2 \times \mathbb{C}_{3}^{2} \times 2 \times \mathbb{C}_{3}^{2} = 20 \times 2 \times 3 \times 2 \times 3 = 720\);\par
当\(m = 5\),\(n = 1\)时,与\(m = 1\),
\(n = 5\)对称,路径数为240;\par
满足条件的总路径数有\(240 + 720 + 240 = 1200\),
概率为\(\frac{1200}{4^{6}} = \frac{75}{256}\).\(\frac{75}{256}\).}
\end{question}

\section{解答题}

\begin{question}
如图,长方体\(ABCD - A_{1}B_{1}C_{1}D_{1}\)中,
\(AB = BC = 2\),\(AA_{1} = 3\),\(E\),
\(F\)三等分\(CC_{1}\).
\begin{enumerate}[label=(\arabic*)]
  \item 求证:\(D_{1}E\bot AF\);
  \item 求直线\(D_{1}E\)与平面\(AA_{1}C_{1}C\)所成角的大小.
\end{enumerate}

\begin{center}
% IMAGE_TODO_START id=auto_137268-Q15-img1 path=/Users/muryor/code/mynote/word\\_to\\_tex/output/figures/auto\\_137268/media/image6.png width=60% inline=false question_index=15 sub_index=1
% CONTEXT_BEFORE: C = 2$$,$$AA_{1} = 3$$,$$E$$,$$F$$三等分$$CC_{1}$$.
% CONTEXT_AFTER: (1)求证:$$D_{1}E\bot AF$$; (2)求直线$$D_{1}E$$与平面$$A
\begin{tikzpicture}[scale=1.05,>=Stealth,line cap=round,line join=round]
  % TODO: AI_AGENT_REPLACE_ME (id=auto_137268-Q15-img1)
\end{tikzpicture}
% IMAGE_TODO_END id=auto_137268-Q15-img
1
\end{center}


\begin{center}
% IMAGE_TODO_START id=auto_137268-Q15-img2 path=/Users/muryor/code/mynote/word\\_to\\_tex/output/figures/auto\\_137268/media/image7.png width=60% inline=false question_index=15 sub_index=1
% CONTEXT_BEFORE: 0,2,0)$$,$$A(0,0,3)$$,$$E(2,2,2)$$,$$F(2,2,1)$$,
\begin{tikzpicture}[scale=1.05,>=Stealth,line cap=round,line join=round]
  % TODO: AI_AGENT_REPLACE_ME (id=auto_137268-Q15-img2)
\end{tikzpicture}
% IMAGE_TODO_END id=auto_137268-Q15-img
2
\end{center}

\topics{空间位置关系的向量证明;线面角的向量求法}
\difficulty{0.65}
\answer{(1)证明见解析
(2)\(\frac{\pi}{6}\)}
\explain{(1)根据题意,
六面体\(ABCD - A_{1}B_{1}C_{1}D_{1}\)为长方体,
所以\(AA_{1}\bot A_{1}D_{1}\),
\(AA_{1}\bot A_{1}B_{1}\),
\(A_{1}D_{1}\bot A_{1}B_{1}\),\par
如图,以\(A_{1}\)为坐标原点,以\(A_{1}B_{1}\),
\(A_{1}D_{1}\),\(A_{1}A\)分别为\(x\),\(y\),
\(z\)轴建立空间直角坐标系,\par
因为\(AB = BC = 2\),\(AA_{1} = 3\),\(E\),
\(F\)为\(CC_{1}\)的三等分点,\par
得各点坐标\(B_{1}(2,0,0)\),\(D_{1}(0,2,0)\),
\(A(0,0,3)\),\(E(2,2,2)\),\(F(2,2,1)\),\par
则\(\overrightarrow{D_{1}E} = (2,0,2)\),
\(\overrightarrow{AF} = (2,2, - 2)\),
所以\(\overrightarrow{D_{1}E} \cdot \overrightarrow{AF} = 0\),
即\(D_{1}E\bot AF\).\par
(2)因为\(AA_{1}\bot\)平面\(A_{1}B_{1}C_{1}D_{1}\),
\(B_{1}D_{1} \subset\)平面\(A_{1}B_{1}C_{1}D_{1}\),
所以\(B_{1}D_{1}\bot AA_{1}\),\par
又因为\(B_{1}D_{1}\bot A_{1}C_{1}\),
\(A_{1}C_{1} \cap AA_{1} = A_{1}\),
\(A_{1}C_{1},AA_{1} \subset\)平面\(AA_{1}C_{1}C\),\par
所以\(B_{1}D_{1}\bot\)平面\(AA_{1}C_{1}C\),
所以\({\overrightarrow{B_{1}D}}_{1} = ( - 2,2,0)\)为平面\(AA_{1}C_{1}C\)的一个法向量,\par
结合\(\overrightarrow{D_{1}E} = (2,0,2)\),\par
设直线\(D_{1}E\)与平面\(AA_{1}C_{1}C\)所成角为\(\theta\),\(\theta \in \left\lbrack 0,\frac{\pi}{2} \right\rbrack\),
则\(\sin\theta = \frac{\left| \overrightarrow{B_{1}D_{1}} \cdot \overrightarrow{D_{1}E} \right|}{\left| \overrightarrow{B_{1}D_{1}} \right| \cdot \left| \overrightarrow{D_{1}E} \right|} = \frac{| - 4|}{2\sqrt{2} \times 2\sqrt{2}} = \frac{1}{2}\),\par
所以直线\(D_{1}E\)与平面\(AA_{1}C_{1}C\)所成角为\(\frac{\pi}{6}\).}
\end{question}

\begin{question}
近些年汽车市场发生了翻天覆地的变化,新能源汽车发展迅速,
下表统计了2021年到2024年某地新能源汽车销量(单位:千辆)

\begin{center}
\begin{tabular}{ccccc}
\hline
年份 & 2021 & 2022 & 2023 & 2024 \\
\hline
年份代号\(x\) & 1 & 2 & 3 & 4 \\
销量\(y\) & 33 & 69 & 93 & 129 \\
\hline
\end{tabular
\end{center}
\topics{相关系数的计算;根据回归方程进行数据估计}
\difficulty{0.85}
\answer{(1)\(y\)与\(x\)具有较强的线性相关关系
(2)\(\widehat{y} = 31.2x + 3\),\(159\)(千辆)}
\explain{(1)已知\(n = 4\),
\(x_{1} = 1,x_{2} = 2,x_{3} = 3,x_{4} = 4\),
则\(\overline{x} = \frac{1 + 2 + 3 + 4}{4} = 2.5\),\par
\(y_{1} = 33,y_{2} = 69,y_{3} = 93,y_{4} = 129\),
则\(\overline{y} = \frac{33 + 69 + 93 + 129}{4} = 81\),\par
\(\sum_{i = 1}^{4}x_{i}^{2} = 1^{2} + 2^{2} + 3^{2} + 4^{2} = 1 + 4 + 9 + 16 = 30\),
\(4{\overline{x}}^{2} = 4 \times {2.5}^{2} = 25\),
所以\(\sum_{i = 1}^{4}\left( x_{i} - \overline{x} \right)^{2} = \sum_{i = 1}^{4}x_{i}^{2} - 4{\overline{x}}^{2} = 30 - 25 = 5\),\par
已知\(\sum_{i = 1}^{4}x_{i}y_{i} = 966\),
故\(\sum_{i = 1}^{4}{\left( x_{i} - \overline{x} \right)\left( y_{i} - \overline{y} \right) = \sum_{i = 1}^{4}{x_{i}y_{i}} - 4\overline{x} \cdot \overline{y} = \sum_{i = 1}^{4}{x_{i}y_{i}} - 4\overline{x} \cdot \overline{y} = 966 - 4 \times 2.5 \times 81 = 156}\),\par
又\(\sum_{i = 1}^{4}\left( y_{i} - \overline{y} \right)^{2} = 4896\),
代入相关系数公式,\par
可得\(r = \frac{\sum_{i = 1}^{n}\left( x_{i} - \overline{x} \right)\left( y_{i} - \overline{y} \right)}{\sqrt{\sum_{i = 1}^{n}\left( x_{i} - \overline{x} \right)^{2}}\sqrt{\sum_{i = 1}^{n}\left( y_{i} - \overline{y} \right)^{2}}} = \frac{156}{\sqrt{5 \times 4896}} = \frac{156}{12\sqrt{170}} \approx \frac{13}{13.04} \approx 0.997\),\par
因为\(|r| = 0.997 \geq 0.75\),
所以\(y\)与\(x\)具有较强的线性相关关系.\par
(2)根据\(\widehat{b} = \frac{\sum_{i = 1}^{4}{\left( x_{i} - \overline{x} \right)\left( y_{i} - \overline{y} \right)}}{\sum_{i = 1}^{4}\left( x_{i} - \overline{x} \right)^{2}},\widehat{a} = \overline{y} - \widehat{b}\overline{x}\),\par
由(1)可知\(\sum_{i = 1}^{4}{\left( x_{i} - \overline{x} \right)\left( y_{i} - \overline{y} \right) = 156}\),
\(\sum_{i = 1}^{4}\left( x_{i} - \overline{x} \right)^{2} = 5\),
所以\(\widehat{b} = \frac{156}{5} = 31.2\),\par
由\(\widehat{a} = \overline{y} - \widehat{b}\overline{x}\),
已知\(\overline{x} = 2.5\),
\(\overline{y} = 81\),\(\widehat{b} = 31.2\),
则\(\widehat{a} = 81 - 31.2 \times 2.5 = 81 - 78 = 3\),\par
所以\(y\)关于\(x\)的线性回归方程为\(\widehat{y} = 31.2x + 3\),
将\(x = 5\)代入线性回归方程\(\widehat{y} = 31.2 \times 5 + 3 = 159\)(千辆).}

\vspace{1em}
\textbf{附:}

附:相关系数\(r = \frac{\sum_{i = 1}^{n}{\left( x_{i} - \overline{x} \right)\left( y_{i} - \overline{y} \right)}}{\sqrt{\sum_{i = 1}^{n}\left( x_{i} - \overline{x} \right)^{2}}\sqrt{\sum_{i = 1}^{n}\left( y_{i} - \overline{y} \right)^{2}}}\);

回归方程\(\widehat{y} = \widehat{b}x + \widehat{a}\)中斜率和截距的最小二乘法估计公式分别为\(\widehat{b} = \frac{\sum_{i = 1}^{n}{\left( x_{i} - \overline{x} \right)\left( y_{i} - \overline{y} \right)}}{\sum_{i = 1}^{n}\left( x_{i} - \overline{x} \right)^{2}},\widehat{a} = \overline{y} - \widehat{b}\overline{x}\),\(\sum_{i = 1}^{4}{x_{i}y_{i}} = 966,\sum_{i = 1}^{4}\left( y_{i} - \overline{y} \right)^{2} = 4896,\sqrt{170} \approx 13.04.\)

(1)试根据样本相关系数\(r\)的值判断该地汽车销量\(y\)与年份代号\(x\)的线性相关性强弱(\(0.75 \leq |r| \leq 1\),则认为\(y\)与\(x\)的线性相关性较强,\(|r| < 0.75\),则认为\(y\)与\(x\)的线性相关性较弱);(精确到0.001)

(2)建立\(y\)关于\(x\)的线性回归方程,并预测该地2025年的新能源汽车销量.

\end{question}

\begin{question}
已知数列\(\left\{ a_{n} \right\}\),
\(\left\{ b_{n} \right\}\)满足\(\left\{ \begin{array}{r}
a_{n + 1} = \frac{1}{2}a_{n} + \frac{3}{2}b_{n} \\
b_{n + 1} = \frac{1}{2}b_{n} + \frac{3}{2}a_{n}
\end{array} \right.)(n \in \mathbb{N}_{+}\)),且\(b_{1} = 3a_{1} = \frac{3}{2}\).
\begin{enumerate}[label=(\arabic*)]
  \item 证明:数列\(\left\{ a_{n} + b_{n} \right\}\)与\(\left\{ a_{n} - b_{n} \right\}\)均为等比数列;
  \item 求数列\(\left\{ \left\lbrack a_{n} \right\rbrack \right\}\)的前25项和\(S_{25}\).(其中\(\lbrack x\rbrack\)表示不超过\(x\)的最大整数,如\(\lbrack 1.2\rbrack = 1\))
\end{enumerate}
\topics{由递推关系证明等比数列;求等比数列前n项和;分组(并项)法求和}
\difficulty{0.65}
\answer{(1)证明见解析;
(2)\(2^{25} - 14\).}
\explain{(1)由\(\left\{ \begin{array}{r}
a_{n + 1} = \frac{1}{2}a_{n} + \frac{3}{2}b_{n} \\
b_{n + 1} = \frac{1}{2}b_{n} + \frac{3}{2}a_{n}
\end{array} \right.\),可得\(\{ \begin{array}{r}
a_{n + 1} + b_{n + 1} = 2\left( a_{n} + b_{n} \right\),\par
又\(a_{1} + b_{1} = 2 \neq 0,a_{1} - b_{1} = - 1 \neq 0\),\par
所以\(\left\{ a_{n} + b_{n} \right\}\)与\(\left\{ a_{n} - b_{n} \right\}\)均为等比数列;\par
(2)由(1)知\(a_{n} + b_{n} = 2^{n}\),\(a_{n} - b_{n} = ( - 1)^{n}\),所以\(a_{n} = \frac{2^{n} + ( - 1)^{n}}{2}\),\par
则\(\left\lbrack a_{2n} \right\rbrack = \left\lbrack \frac{2^{2n} + 1}{2} \right\rbrack = 2^{2n - 1}\),\(\left\lbrack a_{2n - 1} \right\rbrack = \left\lbrack \frac{2^{2n - 1} - 1}{2} \right\rbrack = 2^{2n - 2} - 1\),\par
\(S_{25} = \left( 2^{0} - 1 + 2^{1} \right) + \left( 2^{2} - 1 + 2^{3} \right) + ... + \left( 2^{22} - 1 + 2^{23} \right) + 2^{24} - 1 = \frac{(1 - 2^{25})}{1 - 2} - 13 = 2^{25} - 14\).}
\end{question}

\begin{question}
已知函数\(f(x) = \mathrm{e}^{x} - x - 1\).
\begin{enumerate}[label=(\arabic*)]
  \item 求\(y = f(x)\)在\(x = 0\)处的切线方程;
  \item 若\(f\left( \ln x \right) \geq kx - x\ln x - 1\)恒成立,
\item 求实数\(k\)的取值范围;
  \item 当\(a \geq 1\)时,
\item 讨论\(g(x) = f(x) - ax\cos x\)在区间\(\left( - \pi,\frac{\pi}{2} \right)\)上零点的个数.
\end{enumerate}
\topics{求在曲线上一点处的切线方程(斜率);利用导数研究不等式恒成立问题;利用导数研究函数的零点}
\difficulty{0.4}
\answer{(1)\(y = 0\);
(2)\(( - \infty,1\rbrack\);
(3)3个.}
\explain{(1)由\(f(x) = \mathrm{e}^{x} - x - 1\),
则\(f'(x) = \mathrm{e}^{x} - 1\),
显然\(f'(0) = f(0) = 0\),所以切线方程为\(y = 0\);\par
(2)\(f(x) = \mathrm{e}^{x} - x - 1\),
此时\(f(lnx) \geq kx - xlnx - 1 \Leftrightarrow x - lnx - 1 \geq kx - xlnx - 1\),\par
**法一:**分离参数法,
从而\(kx \leq (x - 1)lnx + x \Rightarrow k \leq 1 + lnx - \frac{lnx}{x}\),\par
令\(h(x) = 1 + lnx - \frac{lnx}{x}\),
则\({h'}^{}(x) = \frac{1}{x} - \frac{1 - lnx}{x^{2}} = \frac{x + lnx - 1}{x^{2}}\),\par
所以\(h'(x) > 0 \Rightarrow x > 1\),
\({h'}^{}(x) < 0 \Rightarrow 0 < x < 1\),\par
所以\(h(x)\)在\((0,1)\)单调递减,
在\((1, + \infty)\)单调递增,\par
因此\(h(x)_{min} = h(1) = 1\),
故\(k\)的取值范围为\(( - \infty,1\rbrack\);\par
**法二:**必要性探路,
\(x - lnx - 1 \geq kx - xlnx - 1 \Leftrightarrow (x - 1)lnx + (1 - k)x \geq 0\),\par
令\(h(x) = (x - 1)lnx + (1 - k)x\),
\(h(1) = 1 - k \geq 0 \Rightarrow k \leq 1\),\par
下证:\(k \leq 1\),\(x > 0\)时,
\(h(x) \geq 0\)恒成立,\par
由一次函数\(m(k) = (x - 1)lnx + x - kx\)在\(( - \infty,1\rbrack\)上递减,\par
则\(m(k) \geq m(1) \Rightarrow (x - 1)lnx + x - kx \geq (x - 1)lnx\),\par
在\(x \in (0,1)\)和\(x \in (1, + \infty)\)上\((x - 1)lnx > 0\)恒成立,
且\(x = 1\)时\((x - 1)lnx = 0\),\par
所以\(g(x) \geq 0\)恒成立,
故\(k\)的取值范围为\(( - \infty,1\rbrack\);\par
(3)\(g(x)\)在区间\(\left( - \pi,\frac{\pi}{2} \right)\)上有3个零点,
理由如下:\par
由于\(f(0) = 0\),
所以\(x = 0\)是函数\(f(x)\)的一个零点,
\(g'(x) = e^{x} + a(xsinx - cosx) - 1\),\par
当\(x \in \left( - \frac{\pi}{2},0 \right)\)时,
此时\(- axcosx > 0\)恒成立,
又由(1)知\(\mathrm{e}^{x} - x - 1 > 0\)恒成立,\par
从而\(g(x) > 0\)恒成立,
所以\(g(x)\)在区间\(x \in \left( - \frac{\pi}{2},0 \right)\)上没有零点;\par
当\(x \in \left( 0,\frac{\pi}{2} \right)\)时,
此时\(g'(0) = - a < 0\),
\(g'\left( \frac{\pi}{2} \right) = e^{- \frac{\pi}{2}} + \frac{\pi}{2}a - 1 > \frac{\pi}{2} - 1 > 0\),\par
若\(g''(x)\)是\(g'(x)\)的导数,
则\(g''(x) = e^{x} + a(2sinx + xcosx)\),\par
由于\(2\sin x + x\cos x > 0\)恒成立,
所以\(g''(x) > 0\),
即\(g'(x)\)在\(\left( 0,\frac{\pi}{2} \right)\)上单调递增,\par
从而存在\(x_{1} \in \left( 0,\frac{\pi}{2} \right)\)使得\(g'\left( x_{1} \right) = 0\),
且\(g'(x) > 0 \Rightarrow x_{1} < x < \frac{\pi}{2}\),
\(g'(x) < 0 \Rightarrow 0 < x < x_{1}\),\par
即\(g(x)\)在区间\(\left( 0,x_{1} \right)\)上递减,
区间\(\left( x_{1},\frac{\pi}{2} \right)\)上递增,
从而\(g\left( x_{1} \right) < g(0) = 0\),\par
又\(g\left( \frac{\pi}{2} \right) = e^{\frac{\pi}{2}} - \frac{\pi}{2} - 1 > 0\),
所以\(g(x)\)在\(\left( x_{1},\frac{\pi}{2} \right)\)有唯一零点,
即在\(\left( 0,\frac{\pi}{2} \right)\)上有唯一零点;\par
当\(x \in \left( - \pi, - \frac{\pi}{2} \right)\)时,
此时\(xsinx - cosx > 0\),
从而\(g'(x) = e^{x} + a(xsinx - cosx) - 1 \geq e^{x} + xsinx - cosx - 1\),\par
由于\(x \in \left( - \pi, - \frac{\pi}{2} \right)\)时,
\(x < \sin x\),
所以\(e^{x} + xsinx - cosx - 1 > e^{x} + {sin}^{2}x - cosx - 1 = e^{x} - \left( {cos}^{2}x + cosx \right)\),\par
又\({cos}^{2}x + cosx = cosx \cdot (cosx + 1) < 0\),
从而\(e^{x} + xsinx - cosx - 1 > e^{x} - \left( {cos}^{2}x + cosx \right) > 0\)恒成立,\par
即\(g'(x) > 0\)在\(x \in \left( - \pi, - \frac{\pi}{2} \right)\)上恒成立,
所以\(g(x)\)在区间\(x \in \left( - \pi, - \frac{\pi}{2} \right)\)上单调递增,\par
因为\(g\left( - \frac{\pi}{2} \right) = e^{- \frac{\pi}{2}} + \frac{\pi}{2} - 1 > 0\),
\(g( - \pi) = e^{- \pi} - a\pi + \pi - 1 \leq e^{- \pi} - 1 < 0\),\par
因此\(g(x)\)在区间\(x \in \left( - \pi, - \frac{\pi}{2} \right)\)上有唯一零点,\par
综上所述,
函数\(g(x)\)在区间\(\left( - \pi,\frac{\pi}{2} \right)\)上有3个零点.}
\end{question}

\begin{question}
如图,已知点\(P\)到两点\(F_{1}( - 2,0)\),
\(F_{2}(2,0)\)距离的乘积为8,
点\(P\)的轨迹记为曲线\(\Gamma\),
\(\Gamma\)与\(x\)轴交点分别记为\(M,N\).
\begin{enumerate}[label=(\arabic*)]
  \item 求曲线\(\Gamma\)的方程;
  \item 求\(\bigtriangleup PMN\)的周长的取值范围;
  \item 过\(P\)作直线分别交\(y = \pm x\)于两点\(A,B\),
\item 且\(\overrightarrow{AP} = \lambda\overrightarrow{PB}(\lambda > 1)\),
\item 若\(\bigtriangleup OAB\)的面积为18,
\item 求\(\lambda\)的最小值.
\end{enumerate}

\begin{center}
% IMAGE_TODO_START id=auto_137268-Q19-img1 path=/Users/muryor/code/mynote/word\\_to\\_tex/output/figures/auto\\_137268/media/image8.png width=60% inline=false question_index=19 sub_index=1
% CONTEXT_BEFORE: 轨迹记为曲线$$\Gamma$$,$$\Gamma$$与$$x$$轴交点分别记为$$M,N$$.
% CONTEXT_AFTER: (1)求曲线$$\Gamma$$的方程; (2)求$$\bigtriangleup PMN$$
\begin{tikzpicture}[scale=1.05,>=Stealth,line cap=round,line join=round]
  % TODO: AI_AGENT_REPLACE_ME (id=auto_137268-Q19-img1)
\end{tikzpicture}
% IMAGE_TODO_END id=auto_137268-Q19-img
1
\end{center}


\begin{center}
% IMAGE_TODO_START id=auto_137268-Q19-img2 path=/Users/muryor/code/mynote/word\\_to\\_tex/output/figures/auto\\_137268/media/image9.png width=60% inline=false question_index=19 sub_index=1
% CONTEXT_AFTER: (3)设$$A\left( x_{1},x_{1} \right),B( x_{2},
\begin{tikzpicture}[scale=1.05,>=Stealth,line cap=round,line join=round]
  % TODO: AI_AGENT_REPLACE_ME (id=auto_137268-Q19-img2)
\end{tikzpicture}
% IMAGE_TODO_END id=auto_137268-Q19-img
2
\end{center}

\topics{利用导数求函数的单调区间(不含参);由方程研究曲线的性质;判断两曲线交点的个数;求平面轨迹方程}
\difficulty{0.15}
\answer{(1)\(\left( x^{2} + y^{2} \right)^{2} - 8\left( x^{2} - y^{2} \right) = 48\);
(2)\(\left\lbrack 8\sqrt{3},4\sqrt{3} + 2\sqrt{6 + 6\sqrt{3}} \right\rbrack\);
(3)\(2 + \sqrt{3}\).}
\explain{(1)设\(P(x,y)\),
则\(\sqrt{(x + 2)^{2} + y^{2}} \cdot \sqrt{(x - 2)^{2} + y^{2}} = 8\),
得\(x^{4} + y^{4} + 2x^{2}y^{2} - 8x^{2} + 8y^{2} - 48 = 0\),\par
所以\(\left( x^{2} + y^{2} \right)^{2} - 8\left( x^{2} - y^{2} \right) = 48\);\par
(2)由(1)知\(M\left( 2\sqrt{3},0 \right),N\left( - 2\sqrt{3},0 \right)\),
令\(t = x^{2} + y^{2}\),\par
由(1),
以\(x^{2}\)为主元直接求根公式知\(x^{2} = 4 - y^{2} + 4\sqrt{4 - y^{2}}\),
则\(y \in \lbrack - 2,2\rbrack\),\par
则\(t = x^{2} + y^{2} = 4 + 4\sqrt{4 - y^{2}} \in \lbrack 4,12\rbrack\),
且\(8\left( x^{2} - y^{2} \right) = \left( x^{2} + y^{2} \right)^{2} - 48\),\par
\(\left( |PM| + |PN| \right)^{2} ={(\sqrt{{(x + 2\sqrt{3})}^{2} + y^{2}} + \sqrt{{(x - 2\sqrt{3})}^{2} + y^{2}})}^{2}= 2\left( x^{2} + y^{2} \right) + 24 + 2\sqrt{\left( x^{2} + y^{2} \right)^{2} - 24\left( x^{2} - y^{2} \right) + 144} = 2\left( t + \sqrt{288 - 2t^{2}} + 12 \right)\),\par
令\(f(t) = t + \sqrt{288 - 2t^{2}}\),
则\(f'(t) = \frac{\sqrt{288 - 2t^{2}} - 2t}{\sqrt{288 - 2t^{2}}}\),
其中\(f'\left( 4\sqrt{3} \right) = \frac{\sqrt{192} - 8\sqrt{3}}{\sqrt{192}} = 0\),\par
所以\(4 \leq t < 4\sqrt{3}\)时\(f'(t) > 0\),
\(4\sqrt{3} < t \leq 12\)时\(f'(t) < 0\),\par
则\(f(t)\)在\(\left\lbrack 4,4\sqrt{3} \right\rbrack\)上单调递增,
在\(\left( 4\sqrt{3},12 \right\rbrack\)上单调递减,\par
所以\(f(t) \in \lbrack 12,12\sqrt{3}\rbrack\),
即\(|PM| + |PN| \in \left\lbrack 4\sqrt{3},2\sqrt{6 + 6\sqrt{3}} \right\rbrack\),
而\(|MN| = 4\sqrt{3}\),\par
所以\(\bigtriangleup PMN\)的周长的取值范围为\(\left\lbrack 8\sqrt{3},4\sqrt{3} + 2\sqrt{6 + 6\sqrt{3}} \right\rbrack\);\par
(3)设\(A\left( x_{1},x_{1} \right),B\left( x_{2}, - x_{2} \right)\),
则\(18 = \frac{1}{2}|OA| \cdot |OB| = \left| x_{1}x_{2} \right|\),
则\(x_{1}x_{2} = \pm 18\),\par
由题知\(\{ \begin{array}{r}
x - x_{1} = \lambda\left( x_{2} - x \right\),则\(\left\{ \begin{array}{r}
x = \frac{x_{1} + \lambda x_{2}}{1 + \lambda} \\
y = \frac{x_{1} - \lambda x_{2}}{1 + \lambda}
\end{array} \right.\),代入曲线\(\Gamma\)得:\(\left( \frac{\lambda^{2}x_{2}^{2} + x_{1}^{2}}{\left( 1 + \lambda \right)^{2}} \right)^{2} - \frac{8\lambda x_{1}x_{2}}{\left( 1 + \lambda \right)^{2}} = 12\),\par
令\(m = \frac{\lambda}{(1 + \lambda)^{2}}\),则\par
①当\(x_{1}x_{2} = 18\)时,\(12 = - \frac{144\lambda}{\left( 1 + \lambda \right)^{2}} + \left( \frac{\lambda^{2}x_{2}^{2} + \frac{18^{2}}{x_{2}^{2}}}{\left( 1 + \lambda \right)^{2}} \right)^{2} \geq - 144m + 36^{2}m^{2}\),解得\(m \leq \frac{1}{6}\),则\(\lambda \geq 2 + \sqrt{3}\);\par
②当\(x_{1}x_{2} = - 18\)时,\(12 = \frac{144\lambda}{\left( 1 + \lambda \right)^{2}} + \left( \frac{\lambda^{2}x_{2}^{2} + \frac{18^{2}}{x_{2}^{2}}}{\left( 1 + \lambda \right)^{2}} \right)^{2} \geq 144m + 36^{2}m^{2}\),解得\(m \leq \frac{1}{18}\),则\(\lambda \geq 8 + 3\sqrt{7}\).\par
综上所述:\(\lambda\)的最小值为\(2 + \sqrt{3}\).}
\end{question}
