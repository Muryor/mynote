\examxtitle{测试试卷 - jiangsu-rugao-2025-2026-mock1}

\section{单选题}

\begin{question}
已知集合\(A = \left\{ x|\ln(x - 1) \leq 0 \right\}\),
集合\(B = \left\{ x|x^{2} - 5x + 6 \leq 0 \right\}\),
则\(A \cup B =\)
(  )
\begin{choices}
  \item \((1,3\rbrack\)
  \item \(( - \infty,3\rbrack\)
  \item \(( - \infty,6\rbrack\)
  \item \(\left\{ 2 \right\}\)
\end{choices}
\topics{并集的概念及运算;解不含参数的一元二次不等式;由对数函数的单调性解不等式}
\difficulty{0.85}
\answer{A}
\explain{\(\because\) \(A = \left\{ x|\ln(x - 1) \leq 0 \right\} = \left\{ x|0 < x - 1 \leq 1 \right\} = \left\{ x|1 < x \leq 2 \right\}\),
\(B = \left\{ x|x^{2} - 5x + 6 \leq 0 \right\} = \left\{ x|(x - 2)(x - 3) \leq 0 \right\} = \left\{ x|2 \leq x \leq 3 \right\}\),\par
\(\therefore\) \(A \cup B = \left\{ x|1 < x \leq 3 \right\}\)}
\end{question}

\begin{question}
设\(z = \frac{2}{\text{i}^{5} + \text{i}^{10}}\),则\(\overline{z} =\)(  )
\begin{choices}
  \item \(- 1 - \text{i}\)
  \item \(- 1 + \text{i}\)
  \item \(1 + \text{i}\)
  \item \(1 - \text{i}\)
\end{choices}
\topics{复数的除法运算;共轭复数的概念及计算}
\difficulty{0.85}
\answer{B}
\explain{\(z = \frac{2}{\text{i}^{5} + \text{i}^{10}}= \frac{2}{\text{i} + \text{i}^{2}}\) \(= \frac{2}{- 1 + \text{i}}= \frac{2\left( - 1 - \text{i} \right)}{\left( - 1 + \text{i} \right)\left( - 1 - \text{i} \right)}= \frac{2\left( - 1 - \text{i} \right)}{2}\) \(= - 1 - \text{i}\),\par
所以\(\overline{z} =\) \(- 1 + \text{i}\)}
\end{question}

\begin{question}
"双曲线\(C:\frac{x^{2}}{a^{2}} - \frac{y^{2}}{3} = 1(a > 0)\)的两渐近线夹角为\(\frac{\pi}{3}\)"是"\(a = 3\)"的(  )
\begin{choices}
  \item 充要条件
  \item 充分不必要条件
  \item 必要不充分条件
  \item 既不充分也不必要条件
\end{choices}
\topics{判断命题的必要不充分条件;已知方程求双曲线的渐近线}
\difficulty{0.65}
\answer{C}
\explain{双曲线\(C:\frac{x^{2}}{a^{2}} - \frac{y^{2}}{3} = 1(a > 0)\)的两渐近线夹角为\(\frac{\pi}{3}\),\par
所以\(\frac{\sqrt{3}}{a} = \tan\frac{\pi}{6} = \frac{\sqrt{3}}{3}\)或\(\frac{\sqrt{3}}{a} = \tan\frac{\pi}{3} = \sqrt{3}\),
所以\(a = 3\)或\(a = 1\).\par
"\(a = 3\)或\(a = 1\)"是"\(a = 3\)"的必要不充分条件}
\end{question}

\begin{question}
已知随机事件\(A,B\)互相独立,
满足\(P(A + B) = \frac{3}{4}\),
\(P\left( A|B \right) = \frac{1}{5}\),
则\(P\left( \overline{B}|A \right) =\)(  )
\begin{choices}
  \item \(\frac{5}{16}\)
  \item \(\frac{11}{16}\)
  \item \(\frac{11}{20}\)
  \item \(\frac{9}{20}\)
\end{choices}
\topics{利用对立事件的概率公式求概率;计算条件概率;独立事件的乘法公式}
\difficulty{0.65}
\answer{A}
\explain{因为随机事件\(A,B\)互相独立,所以\(P(AB) = P(A)P(B)\),\par
则\(P\left( A|B \right) = \frac{P(AB)}{P(B)} = \frac{P(A)P(B)}{P(B)} = P(A) = \frac{1}{5}\),\par
\(P(A + B) = P(A) + P(B) - P(AB) = P(A) + P(B) - P(A)P(B)= \frac{1}{5} + P(B) - \frac{1}{5}P(B) = \frac{3}{4}\),\par
解得\(P(B) = \frac{11}{16}\),
\(P\left( \overline{B} \right) = \frac{5}{16}\),
\(P\left( A\overline{B} \right) = P(A)P\left( \overline{B} \right) = \frac{1}{5} \times \frac{5}{16} = \frac{1}{16}\),\par
\(P\left( \overline{B}|A \right) = \frac{P\left( A\overline{B} \right)}{P(A)} = \frac{\frac{1}{16}}{\frac{1}{5}} = \frac{5}{16}\).}
\end{question}

\begin{question}
三棱锥\(A - BCD\)中,平面\(ABD\bot\)平面\(BCD\),
\(\bigtriangleup ABD\)和\(\bigtriangleup BCD\)均为等边三角形,
则二面角\(A - BC - D\)的余弦值是(  )
\begin{choices}
  \item \(\frac{1}{2}\)
  \item \(\frac{\sqrt{3}}{2}\)
  \item \(\frac{\sqrt{5}}{5}\)
  \item \(\frac{2\sqrt{5}}{5}\)
\end{choices}

\begin{center}
% IMAGE_TODO_START id=jiangsu-rugao-2025-2026-mock1-Q5-img1 path=/Users/muryor/code/mynote/word\\_to\\_tex/output/figures/jiangsu-rugao-2025-2026-mock1/media/image2.png width=60% inline=false question_index=5 sub_index=1
% CONTEXT_BEFORE: . 【详解】如图,作出符合题意的图形,取$$BD$$的中点$$O$$,连接$$AO,CO$$,
% CONTEXT_AFTER: 因为$$\bigtriangleup ABD$$和$$\bigtriangleup BCD$$均
\begin{tikzpicture}[scale=1.05,>=Stealth,line cap=round,line join=round]
  % TODO: AI_AGENT_REPLACE_ME (id=jiangsu-rugao-2025-2026-mock1-Q5-img1)
\end{tikzpicture}
% IMAGE_TODO_END id=jiangsu-rugao-2025-2026-mock1-Q5-img
1
\end{center}

\topics{面面角的向量求法}
\difficulty{0.65}
\answer{C}
\explain{如图,作出符合题意的图形,取\(BD\)的中点\(O\),连接\(AO,CO\),\par
因为\(\bigtriangleup ABD\)和\(\bigtriangleup BCD\)均为等边三角形,
所以\(BD\bot AO\),\(BD\bot CO\),\par
因为平面\(ABD\bot\)平面\(BCD\),
且\(AO \subset\)面\(ABD\),所以\(AO\bot\)面\(BCD\),\par
则以\(O\)为原点建立空间直角坐标系,
设\(\bigtriangleup ABD\)和\(\bigtriangleup BCD\)的边长为\(2\),\par
可得\(A(0,0,\sqrt{3})\),\(B(0, - 1,0)\),
\(C(\sqrt{3},0,0)\),\par
得到\(\overrightarrow{AB} = (0, - 1, - \sqrt{3})\),
\(\overrightarrow{AC} = (\sqrt{3},0, - \sqrt{3})\),\par
设面\(ABC\)的法向量为\(\overrightarrow{n} = (x,y,z)\),
可得\(\left\{ \begin{array}{r}
\overset{\right\),\par
令\(x = 1\),解得\(y = - \sqrt{3},z = 1\),故\(\overrightarrow{n} = (1, - \sqrt{3},1)\),\par
易得面\(BCD\)的法向量为\(\overrightarrow{m} = (0,0,1)\),\par
设二面角\(A - BC - D\)为\(\alpha\),由图可知\(\alpha\)为锐角,\par
则\(\cos\alpha = \frac{\left| \overrightarrow{m} \cdot \overrightarrow{n} \right|}{\left| \overrightarrow{m} \right| \cdot \left| \overrightarrow{n} \right|} = \frac{1}{\sqrt{5} \cdot 1} = \frac{\sqrt{5}}{5}\),故C正确}
\end{question}
%
\begin{question}
函数\(f(x) = - \frac{1}{2}x^{2} + kx - \ln x\)在区间\(\left\lbrack \frac{1}{2},3 \right\rbrack\)上存在单调增区间,
则\(k\)的取值范围是(  )
\begin{choices}
  \item \(\left\lbrack 2,\frac{10}{3} \right\rbrack\)
  \item \((2, + \infty)\)
  \item \(\lbrack 2, + \infty)\)
  \item \(\left\lbrack \frac{10}{3}, + \infty \right)\)
\end{choices}
\topics{由函数在区间上的单调性求参数}
\difficulty{0.65}
\answer{B}
\explain{函数\(f(x)\)定义域为\((0, + \infty)\),
\(f'(x) = - x + k - \frac{1}{x} = \frac{- x^{2} + kx - 1}{x}\),\par
因为函数\(f(x)\)在区间\(\left\lbrack \frac{1}{2},3 \right\rbrack\)上存在单调增区间,\par
所以\(f'(x) = \frac{- x^{2} + kx - 1}{x} > 0\)在区间\(\left\lbrack \frac{1}{2},3 \right\rbrack\)有解,\par
即\(- x^{2} + kx - 1 > 0\)在区间\(\left\lbrack \frac{1}{2},3 \right\rbrack\)有解,\par
所以\(k > x + \frac{1}{x}\)在区间\(\left\lbrack \frac{1}{2},3 \right\rbrack\)上能成立,
故\(k > {(x + \frac{1}{x})}_{\min}\),\par
又\(y = x + \frac{1}{x} \geq 2\),
当且仅当\(x = 1\)时取等,所以\(k > 2\).}
\end{question}
%
\begin{question}
已知圆\(C:(x + 4)^{2} + (y + 3)^{2} = 1\)及\(A(0,a),B(0, - a)\)两点,
\(\left( a \in \mathbb{R}^{+} \right)\),
若圆\(\mathbb{C}\)上任一点\(M\),
都满足\(\angle AMB>\frac{\pi}{2}\),
则\(a\)的取值范围是(  )
\begin{choices}
  \item \((0,4)\)
  \item \(\lbrack 4,6\rbrack\)
  \item \((4, + \infty)\)
  \item \((6, + \infty)\)
\end{choices}
%
\begin{center}
% IMAGE_TODO_START id=jiangsu-rugao-2025-2026-mock1-Q7-img1 path=/Users/muryor/code/mynote/word\\_to\\_tex/output/figures/jiangsu-rugao-2025-2026-mock1/media/image3.png width=60% inline=false question_index=7 sub_index=1
% CONTEXT_BEFORE: a$$的取值范围是$$(6, + \infty)$$. } > 36$$
\begin{tikzpicture}[scale=1.05,>=Stealth,line cap=round,line join=round]
  % TODO: AI_AGENT_REPLACE_ME (id=jiangsu-rugao-2025-2026-mock1-Q7-img1)
\end{tikzpicture}
% IMAGE_TODO_END id=jiangsu-rugao-2025-2026-mock1-Q7-img
1
\end{center}
%
\topics{定点到圆上点的最值(范围);利用数量积求参数}
\difficulty{0.4}
\answer{D}
\explain{设点\(M(x,y)\),
则\(\overset{arrow}{MA} = ( - x,a - y)\),
\(\overset{arrow}{MB} = ( - x, - a - y)\).\par
若满足\(\angle AMB>\frac{\pi}{2}\),
则\(\overset{arrow}{MA} \cdot \overset{arrow}{MB} < 0\),
即\(x^{2} + y^{2} - a^{2} < 0\),
即\(a^{2} > x^{2} + y^{2}\),
所以\(a > \sqrt{x^{2} + y^{2}}\).\par
令\(t = \sqrt{x^{2} + y^{2}}\),
则\(t\)表示点\(M\)到坐标原点\(O\)的距离\(|OM|\).\par
如图,当线段\(OM\)过圆心\(C\)时,\(|OM|\)最大,
最大值为\(|OC| + 1 = 6\).\par
所以\(a\)的取值范围是\((6, + \infty)\).\par
} > 36}
\end{question}
%
\begin{question}
已知定义在\(\lbrack 0,1\rbrack\)上的函数\(f(x)\)满足:\(\forall x \in \lbrack 0,1\rbrack\),
都有\(f(1 - x) + f(x) = 2\),
且\(f\left( \frac{x}{5} \right) = \frac{1}{2}f(x)\),
当\(0 \leq x_{1} < x_{2} \leq 1\)时,
有\(f\left( x_{1} \right) \leq f\left( x_{2} \right)\),
则\(f\left( \frac{1}{2025} \right)\)的值为(  )
\begin{choices}
  \item \(\frac{1}{8}\)
  \item \(\frac{1}{16}\)
  \item \(\frac{1}{32}\)
  \item \(\frac{1}{64}\)
\end{choices}
\topics{求函数值}
\difficulty{0.4}
\answer{B}
\explain{定义在\(\lbrack 0,1\rbrack\)上的函数\(f(x)\)满足:\(\forall x \in \lbrack 0,1\rbrack\),
都有\(f(1 - x) + f(x) = 2\),
且\(f\left( \frac{x}{5} \right) = \frac{1}{2}f(x)\),\par
所以\(f\left( \frac{0}{5} \right) = \frac{1}{2}f(0)\),
故\(f(0) = 0\),\par
在等式\(f(1 - x) + f(x) = 2\)中,
令\(x = 1\)可得\(f(1) + f(0) = 2\),
所以\(f(1) = 2\),\par
所以\(f\left( \frac{1}{5} \right) = \frac{1}{2}f(1) = 1\),
\(f\left( \frac{1}{25} \right) = \frac{1}{2}f\left( \frac{1}{5} \right) = \frac{1}{2}\),
\(f\left( \frac{1}{125} \right) = \frac{1}{2}f\left( \frac{1}{25} \right) = \frac{1}{4}\),\par
\(f\left( \frac{1}{625} \right) = \frac{1}{2}f\left( \frac{1}{125} \right) = \frac{1}{8}\),
\(f\left( \frac{1}{3125} \right) = \frac{1}{2}f\left( \frac{1}{625} \right) = \frac{1}{16}\),\par
在等式\(f(1 - x) + f(x) = 2\)中,
令\(x = \frac{1}{2}\)可得\(2f\left( \frac{1}{2} \right) = 2\),
所以\(f\left( \frac{1}{2} \right) = 1\),\par
所以\(f\left( \frac{1}{10} \right) = \frac{1}{2}f\left( \frac{1}{2} \right) = \frac{1}{2}\),
\(f\left( \frac{1}{50} \right) = \frac{1}{2}f\left( \frac{1}{10} \right) = \frac{1}{4}\),
\(f\left( \frac{1}{250} \right) = \frac{1}{2}f\left( \frac{1}{50} \right) = \frac{1}{8}\),\par
\(f\left( \frac{1}{1250} \right) = \frac{1}{2}f\left( \frac{1}{250} \right) = \frac{1}{16}\),\par
当\(0 \leq x_{1} < x_{2} \leq 1\)时,
有\(f\left( x_{1} \right) \leq f\left( x_{2} \right)\),\par
又因为\(\frac{1}{3125} < \frac{1}{2025} < \frac{1}{1250}\),
且\(f\left( \frac{1}{3125} \right) = f\left( \frac{1}{1250} \right) = \frac{1}{16}\),
故\(f\left( \frac{1}{2025} \right) = \frac{1}{16}\)}
\end{question}
%
\section{多选题}
%
\begin{question}
已知函数\(f(x) = \left\{ \begin{array}{r}
x^{2},x \in Z \\
1,x \notin Z
\end{array} \right.\),下列说法正确的是(  )
\begin{choices}
  \item \(f\left( \pi \right) + f(0) = \pi^{2}\)
  \item \(f(x)\)为偶函数
  \item 当\(x \in N\)时,\(f(x) < f(x + 1)\)
  \item 若\(x \notin Z\),则\(f\left( f(x) \right) \neq 1\)
\end{choices}
\topics{求分段函数解析式或求函数的值;函数奇偶性的定义与判断;作差法比较代数式的大小}
\difficulty{0.65}
\answer{BC}
\explain{因为\(f(x) = \left\{ \begin{array}{r}
x^{2},x \in Z \\
1,x \notin Z
\end{array} \right.\),\par
对于A选项,\(f\left( \pi \right) + f(0) = 1 + 0^{2} = 1\),A错;\par
对于B选项,当\(x \in Z\)时,\(- x \in Z\),则\(f( - x) = ( - x)^{2} = x^{2} = f(x)\),\par
当\(x \notin Z\)时,\(- x \notin Z\),则\(f( - x) = 1 = f(x)\),\par
所以,对任意的\(x \in R\),\(f( - x) = f(x)\),故函数\(f(x)\)为偶函数,B对;\par
对于C选项,当\(x \in N\)时,\(x + 1 \in N\),\par
所以\(f(x + 1) - f(x) = (x + 1)^{2} - x^{2} = 2x + 1 \geq 1 > 0\),故\(f(x) < f(x + 1)\),C对;\par
对于D选项,若\(x \notin Z\),则\(f(x) = 1\),此时\(f\left( f(x) \right) = f(1) = 1^{2} = 1\),D错}
\end{question}
%
\begin{question}
已知二项展开式\((1 - 2x)^{2025} = a_{0} + a_{1}x + a_{2}x^{2} + \cdots + a_{2025}x^{2025}\),
下列说法正确的是(  )
\begin{choices}
  \item \(a_{1} = - 4050\)
  \item \(a_{1} + a_{2} + a_{3} + \cdots + a_{2025} = - 1\)
  \item \(a_{0} + a_{2} + a_{4} + a_{6} + \cdots + a_{2024} = \frac{3^{2025} - 1}{2}\)
  \item \(2a_{2} + 3a_{3} + 4a_{4}\cdots + 2025a_{2025} = 0\)
\end{choices}
\topics{简单复合函数的导数;求指定项的系数;二项展开式各项的系数和;奇次项与偶次项的系数和}
\difficulty{0.65}
\answer{ACD}
\explain{令\(f(x) = (1 - 2x)^{2025} = a_{0} + a_{1}x + a_{2}x^{2} + \cdots + a_{2025}x^{2025}\),\par
对于A选项,
\((1 - 2x)^{2025}\)的展开式通项为\(T_{k + 1} = \mathbb{C}_{2025}^{k} \cdot 1^{2025 - k} \cdot ( - 2x)^{k} = \mathbb{C}_{2025}^{k} \cdot ( - 2)^{k} \cdot x^{k}\),\par
其中\(0 \leq k \leq 2025\),\(k \in N\),
所以\(a_{1} = - 2\mathbb{C}_{2025}^{1} = - 2 \times 2025 = - 4050\),
A对;\par
对于B选项,\(a_{0} = f(0) = 1\),\par
所以\(a_{1} + a_{2} + a_{3} + \cdots + a_{2025} = a_{0} + a_{1} + a_{2} + a_{3} + \cdots + a_{2025} - a_{0} = f(1) - 1 = ( - 1)^{2025} - 1 = - 2\),
B错;\par
对于C选项,\(\left\{ \begin{array}{r}
f(1) = a_{0} + a_{1} + a_{2} + a_{3} + \cdots + a_{2025} = (1 - 2)^{2025} = - 1 \\
f( - 1) = a_{0} - a_{1} + a_{2} - a_{3} + \cdots - a_{2025} = (1 + 2)^{2025} = 3^{2025}
\end{array} \right.\),\par
所以\(a_{0} + a_{2} + a_{4} + a_{6} + \cdots + a_{2024} = \frac{f(1) + f( - 1)}{2} = \frac{3^{2025} - 1}{2}\),C对;\par
对于D选项,\(f'(x) = - 4050(1 - 2x)^{2024} = a_{1} + 2a_{2}x + 3a_{3}x^{2} + \cdots + 2025a_{2025}x^{2024}\),\par
故\(2a_{2} + 3a_{3} + 4a_{4}\cdots + 2025a_{2025} = a_{1} + 2a_{2} + 3a_{3} + 4a_{4}\cdots + 2025a_{2025} - a_{1} = f'(1) + 4050 = - 4050 + 4050 = 0\),D对}
\end{question}
%
\begin{question}
一个封闭的直三棱柱容器\(ABC - A_{1}B_{1}C_{1}\)内装有高度为3的水(如图所示,
底面处于水平状态).记水面为\(\alpha\),\(AC = BC = 2\),
\(AC\bot BC,CC_{1} = 4\),
现以\(AB\)所在直线为旋转轴,
将容器逆时针旋转\(90{^\circ}\)的过程中,下列说法正确的是(  )
\begin{choices}
  \item 水面形状的变化依次为三角形,等腰梯形,矩形
  \item 水面可能是正三角形
  \item 当\(\alpha\)经过\(C\)时,\(\alpha\)与面\(A_{1}B_{1}C_{1}\)的交线长为\(\sqrt{6}\)
  \item 当逆时针旋转\(90{^\circ}\)时,水面的面积为\(4\sqrt{2}\)
\end{choices}
%
\begin{center}
% IMAGE_TODO_START id=jiangsu-rugao-2025-2026-mock1-Q11-img1 path=/Users/muryor/code/mynote/word\\_to\\_tex/output/figures/jiangsu-rugao-2025-2026-mock1/media/image4.png width=60% inline=false question_index=11 sub_index=1
% CONTEXT_BEFORE: 所在直线为旋转轴,将容器逆时针旋转$$90{^\circ}$$的过程中,下列说法正确的是( )
% CONTEXT_AFTER: > A.水面形状的变化依次为三角形,等腰梯形,矩形 > > B.水面可能是正三角形 > > C.
\begin{tikzpicture}[scale=1.05,>=Stealth,line cap=round,line join=round]
  % TODO: AI_AGENT_REPLACE_ME (id=jiangsu-rugao-2025-2026-mock1-Q11-img1)
\end{tikzpicture}
% IMAGE_TODO_END id=jiangsu-rugao-2025-2026-mock1-Q11-img
1
\end{center}
%

\begin{center}
% IMAGE_TODO_START id=jiangsu-rugao-2025-2026-mock1-Q11-img2 path=/Users/muryor/code/mynote/word\\_to\\_tex/output/figures/jiangsu-rugao-2025-2026-mock1/media/image5.png width=60% inline=false question_index=11 sub_index=1
% CONTEXT_BEFORE: 为$$S_{▭DEFG} = DE \cdot DG = 4$$,故D选项正确.
\begin{tikzpicture}[scale=1.05,>=Stealth,line cap=round,line join=round]
  % TODO: AI_AGENT_REPLACE_ME (id=jiangsu-rugao-2025-2026-mock1-Q11-img2)
\end{tikzpicture}
% IMAGE_TODO_END id=jiangsu-rugao-2025-2026-mock1-Q11-img
2
\end{center}
%
\topics{柱体体积的有关计算;锥体体积的有关计算}
\difficulty{0.4}
\answer{ABCD}
\explain{A选项,
水的体积\(V_{水} = \frac{3}{4}V_{ABC - A_{1}B_{1}C_{1}} = \frac{3}{4}S_{\bigtriangleup ABC} \cdot CC_{1} = 6\),\par
\(V_{C - ABB_{1}A_{1}} = V_{ABC - A_{1}B_{1}C_{1}} - V_{C - A_{1}B_{1}C_{1}} = \frac{2}{3}V_{ABC - A_{1}B_{1}C_{1}} = \frac{16}{3}\),\par
所以当水面经过\(A_{1}B_{1}\)时,水面与棱\(CC_{1}\)相交,
如图3,\par
当水面经过点\(C\)时,水面与面\(A_{1}B_{1}C_{1}\)相交,
如图4,\par
则在此之前水面形状均为三角形,\par
继续旋转直至\(90{^\circ}\)之前,水面形状为等腰梯形,如图5,\par
转至\(90{^\circ}\)时,水面形状为矩形,如图6,故A选项正确;\par
B选项,初始位置,如图1,
\(DE = 2\sqrt{2},DF = EF = 2\),\par
当水面经过\(A_{1}B_{1}\)时,如图3,
此时\(V_{F - A_{1}B_{1}C_{1}} = \frac{1}{4}V_{ABC - A_{1}B_{1}C_{1}}\text{=}\frac{3}{4}V_{C - A_{1}B_{1}C_{1}}\),\par
所以\(C_{1}F = \frac{3}{4}CC_{1} = 3\),
\(DF = EF = B_{1}F = \sqrt{B_{1}{C_{1}}^{2} + C_{1}F^{2}} = \sqrt{13}\),\par
所以在转动过程中,存在\(DE = DF = EF\),使得水面是正三角形,
故B选项正确;\par
C选项,如图4,
\(V_{C - DEC_{1}} = \frac{3}{4}V_{C - A_{1}B_{1}C_{1}}\),
且由于\(\bigtriangleup DEC_{1}\)与\(\bigtriangleup A_{1}B_{1}C_{1}\)相似,\par
则\(\frac{S_{\bigtriangleup DEC_{1}}}{S_{\bigtriangleup A_{1}B_{1}C_{1}}} = \left( \frac{DE}{A_{1}B_{1}} \right)^{2} = \frac{3}{4}\),\par
\(DE = \frac{\sqrt{3}}{2}A_{1}B_{1} = \sqrt{6}\),
故C选项正确;\par
D选项,当逆时针旋转\(90{^\circ}\)时,如图6,
\(V_{DEC_{1} - GFC} = \frac{1}{4}V_{ABC - A_{1}B_{1}C_{1}}\),\par
且由于\(\bigtriangleup DEC_{1}\)与\(\bigtriangleup A_{1}B_{1}C_{1}\)相似,
则\(\frac{S_{\bigtriangleup DEC_{1}}}{S_{\bigtriangleup A_{1}B_{1}C_{1}}} = \left( \frac{DE}{A_{1}B_{1}} \right)^{2} = \frac{1}{4}\),
则\(DE = \frac{1}{2}A_{1}B_{1} = \sqrt{2}\),\par
则水面的面积为\(S_{▭DEFG} = DE \cdot DG = 4\sqrt{2}\),
故D选项正确.}
\end{question}
%
\section{填空题}
%
\begin{question}
已知\(x>0,y>0\),\(\frac{2}{x} + y = 1\),则\(x + \frac{2}{y}\)的最小值为
.
\topics{基本不等式"1"的妙用求最值}
\difficulty{0.85}
\answer{\(8\)}
\explain{\(x>0,y>0\),且\(\frac{2}{x} + y = 1\),\par
\(x + \frac{2}{y} = \left( x + \frac{2}{y} \right)\left( y + \frac{2}{x} \right) = xy + \frac{4}{xy} + 4 \geq 2\sqrt{xy \cdot \frac{4}{xy}} + 4 = 8\),\par
当且仅当\(xy = \frac{4}{xy}\),即\(xy = 2\)时,
等号成立,又\(\frac{2}{x} + y = 1\),\par
故\(x = 4,y = \frac{1}{2}\)时,等号成立,
所以\(x + \frac{2}{y}\)的最小值为8.\(8\)}
\end{question}
%
\begin{question}
已知\(O\)是坐标原点,
抛物线\(C:y^{2} = 4x\)的焦点是\(F\),
过\(F\)的直线与\(C\)交于\(M,N\)两点,
现将抛物线沿\(x\)轴翻折,则三棱锥\(M - ONF\)体积的最大值为
.
\topics{锥体体积的有关计算;直线与抛物线交点相关问题}
\difficulty{0.65}
\answer{\(\frac{2}{3}\)}
\explain{抛物线\(C:y^{2} = 4x\)的焦点为\(F(1,0)\).\par
易知直线\(MN\)的斜率必不为0,故设直线\(MN:x = my + 1\).\par
联立方程组\(\left\{ \begin{array}{r}
y^{2} = 4x \\
x = my + 1
\end{array} \right.\),消去\(x\)并整理得\(y^{2} - 4my - 4 = 0\).\par
设\(M\left( x_{1},y_{1} \right)\),\(N\left( x_{2},y_{2} \right)\),则\(y_{1} + y_{2} = 4m\),\(y_{1} \cdot y_{2} = - 4\).\par
设抛物线沿\(x\)轴翻折后点\(M\)到平面\(OFN\)的距离为\(h\),则\(h \leq \left| y_{1} \right|\),\par
\(\therefore\) \(V_{M - ONF} = \frac{1}{3}h \times \frac{1}{2} \times 1 \times \left| y_{2} \right| \leq \frac{1}{3}\left| y_{1} \right| \times \frac{1}{2} \times 1 \times \left| y_{2} \right| = \frac{1}{6}\left| y_{1}y_{2} \right| = \frac{2}{3}\).\(\frac{2}{3}\).}
\end{question}
%
\begin{question}
小明同学有一个质地均匀的正四面体玩具,四个面分别标有数字1,2,3,4,
现随机抛掷,记录每次朝下的面上的数字,如果是数字4就停止,否则继续抛掷,
至多抛3次.设这几次记录的最大数字为\(X\),则\(P(X = 2) =\)
;\(E(X) =\) .
\topics{实际问题中的组合计数问题;计算古典概型问题的概率;求离散型随机变量的均值}
\difficulty{0.65}
\answer{\(\frac{7}{64}\) \(\frac{55}{16}\)}
\explain{\(X\)的可能取值为1,2,3,4,\par
\(X = 1\),即抛掷3次,朝下的面上的数字均为1,\par
抛掷3次,朝下的面上的数字共有\(4^{3} = 64\)种情况,\par
故\(P(X = 1) = \frac{1}{64}\),\par
\(X = 2\),即抛掷3次,朝下的面上的数字中,最大数字为2,\par
分有1个2,2个2和3个2三种情况,\par
故\(P(X = 2) = \frac{\mathbb{C}_{3}^{1} + \mathbb{C}_{3}^{2} + \mathbb{C}_{3}^{3}}{64} = \frac{7}{64}\);\par
\(X = 3\),即抛掷3次,朝下的面上的数字中,最大数字为3,\par
分有1个3,2个3和3个3三种情况,\par
故\(P(X = 3) = \frac{2^{2}\mathbb{C}_{3}^{1} + 2\mathbb{C}_{3}^{2} + \mathbb{C}_{3}^{3}}{64} = \frac{19}{64}\);\par
\(X = 4\),抛掷1次,朝下的面上的数字为4,
此时概率为\(\frac{1}{4}\),\par
或抛掷2次,第二次朝下的面上的数字为4,
此时概率为\(\frac{3}{4} \times \frac{1}{4} = \frac{3}{16}\),\par
抛掷3次,第三次朝下的面上的数字为4,
此时概率为\(\left( \frac{3}{4} \right)^{2} \times \frac{1}{4} = \frac{9}{64}\),\par
故\(P(X = 4) = \frac{1}{4} + \frac{3}{16} + \frac{9}{64} = \frac{37}{64}\);\par
故\(EX = 1 \times \frac{1}{64} + 2 \times \frac{7}{64} + 3 \times \frac{19}{64} + 4 \times \frac{37}{64} = \frac{55}{16}\)\(\frac{7}{64}\),
\(\frac{55}{16}\).}
\end{question}
%
\section{解答题}
%
\begin{question}
为促进消费,扩大内需,江苏省体育局主办了\(2025\)年城市足球联赛,
简称"苏超".随着赛事的进行,引发全省乃至全国人民的关注,
城市旅游人数显著提升.下表是比赛五个月来的某城市旅游人数\(y\)(百万)与第\(x\)个月的数据:
%
  --------------- ----------- ----------- ----------- ----------- -----------
   \(x\)(月份)     \(1\)       \(2\)       \(3\)       \(4\)       \(5y\)(人数)     \(2\)       \(3\)       \(5\)       \(7\)       \(8\)
  --------------- ----------- ----------- ----------- ----------- -----------
\begin{enumerate}[label=(\arabic*)]
  \item 已知可用线性回归模型拟合\(y\)与\(x\)的关系,
\item 请建立\(y\)关于\(x\)的线性回归方程;
  \item 该市随机抽取了部分市民及游客,调查他们对赛事的关注情况,得到如下列联表:
%
\item \begin{center}
\item \begin{tabular}{ccc}
\item \hline
\item 性别 & 不关注赛事 & 关注赛事 \\
\item \hline
\item 男性 & \(120\) & \(380\) \\
\item 女性 & \(80\) & \(420\) \\
\item \hline
\item \end{tabular
\item \end{center}
%
\item 请依据小概率值\(\alpha = 0.010\)的独立性检验,能否认为关注"苏超"赛事与性别有关.
%
\item 参考公式:\(\widehat{b} = \frac{\sum_{i = 1}^{n}{\left( x_{i} - \overline{x} \right)\left( y_{i} - \overline{y} \right)}}{\sum_{i = 1}^{n}\left( x_{i} - \overline{x} \right)^{2}}\),\(\widehat{a} = \overline{y} - \widehat{b}\overline{x}\),\(\chi^{2} = \frac{n(ad - bc)^{2}}{(a + b)(c + d)(a + c)(b + d)}\),其中\(n = a + b + c + d\).
%
  \item ---------------- ---------------- --------------- ---------------
     \item \(\alpha\)       \(0.050\)        \(0.010\)       \(0.001x_{\alpha}\)     \(3.841\)        \(6.635\)      \(10.828\)
  \item ---------------- ---------------- --------------- ---------------
\end{enumerate}
\topics{求回归直线方程;独立性检验解决实际问题}
\difficulty{0.65}
\answer{(1)\(\widehat{y} = 1.6x + 0.2\)
(2)能,理由见解析}
\explain{(1)由表格中的数据可得\(\overline{x} = \frac{1 + 2 + 3 + 4 + 5}{5} = 3\),
\(\overline{y} = \frac{2 + 3 + 5 + 7 + 8}{5} = 5\),\par
所以\(\widehat{b} = \frac{\sum_{i = 1}^{5}{\left( x_{i} - \overline{x} \right)\left( y_{i} - \overline{y} \right)}}{\sum_{i = 1}^{5}\left( x_{i} - \overline{x} \right)^{2}} = \frac{(1 - 3)(2 - 5) + (2 - 3)(3 - 5) + (3 - 3)(5 - 5) + (4 - 3)(7 - 5) + (5 - 3)(8 - 5)}{(1 - 3)^{2} + (2 - 3)^{2} + (3 - 3)^{2} + (4 - 3)^{2} + (5 - 3)^{2}} = 1.6\),\par
\(\widehat{a} = \overline{y} - \widehat{b}\overline{x} = 5 - 3 \times 1.6 = 0.2\),\par
故\(y\)关于\(x\)的线性回归方程为\(\widehat{y} = 1.6x + 0.2\).\par
(2)零假设\(H_{0}\):关注"苏超"赛事与性别无关,\par
由表格中的数据可得\(\chi^{2} = \frac{1000 \times (120 \times 420 - 80 \times 380)^{2}}{200 \times 800 \times 500 \times 500} = 10 > x_{0.010} = 6.635\),\par
依据小概率值\(\alpha = 0.010\)的独立性检验,
能认为关注"苏超"赛事与性别有关.}
\end{question}
%
\begin{question}
已知函数\(f(x) = \left( x^{2} - kx - k^{2} \right)\mathrm{e}^{\frac{x}{k}}\).
\begin{enumerate}[label=(\arabic*)]
  \item 求\(f(x)\)的极值;
  \item 若\(\forall x \in (0, + \infty)\),
\item 都有\(f(x) \leq \mathrm{e}^{- 2}\),
\item 求\(k\)的取值范围.
\end{enumerate}
\topics{求已知函数的极值;利用导数研究不等式恒成立问题}
\difficulty{0.65}
\answer{(1)极小值为\(f(k) = - k^{2}\text{e}\),极大值为\(f( - 2k) = 5k^{2}\mathrm{e}^{- 2}\)
(2)\(\left\lbrack - \frac{\sqrt{5}}{5},0 \right)\)}
\explain{(1)由题意可知\(k \neq 0\),
且\(f(x) = \left( x^{2} - kx - k^{2} \right)\mathrm{e}^{\frac{x}{k}}\),\par
所以\(f'(x) = (2x - k)\mathrm{e}^{\frac{x}{k}} + \frac{x^{2} - kx - k^{2}}{k}\mathrm{e}^{\frac{x}{k}} = \frac{x^{2} + kx - 2k^{2}}{k}\mathrm{e}^{\frac{x}{k}} = \frac{(x - k)(x + 2k)}{k}\mathrm{e}^{\frac{x}{k}}\),\par
当\(k < 0\)时,\(- 2k > 0\),列表如下:\par
  ----------- ------------------- ------------ --------------- ------------ -----------------------
     \(x\)     \(( - \infty,k)\)     \(k\)      \((k, - 2k)\)    \(- 2k\)    \(( - 2k, + \infty)f'(x)\)         \(-\)           \(0\)          \(+\)         \(0\)              \(-f(x)\)           减             极小值          增           极大值              减
  ----------- ------------------- ------------ --------------- ------------ -----------------------\par
此时,
函数\(f(x)\)的极小值为\(f(k) = - k^{2}\text{e}\),
极大值为\(f( - 2k) = 5k^{2}\mathrm{e}^{- 2}\);\par
当\(k > 0\)时,\(- 2k < 0\),列表如下:\par
  ----------- ----------------------- ------------ --------------- ------------ -------------------
     \(x\)     \(( - \infty, - 2k)\)    \(- 2k\)    \(( - 2k,k)\)     \(k\)      \((k, + \infty)f'(x)\)           \(+\)             \(0\)          \(-\)         \(0\)            \(+f(x)\)             增               极大值          减           极小值            增
  ----------- ----------------------- ------------ --------------- ------------ -------------------\par
此时,
函数\(f(x)\)的极小值为\(f(k) = - k^{2}\text{e}\),
极大值为\(f( - 2k) = 5k^{2}\mathrm{e}^{- 2}\).\par
综上所述,
函数\(f(x)\)的极小值为\(f(k) = - k^{2}\text{e}\),
极大值为\(f( - 2k) = 5k^{2}\mathrm{e}^{- 2}\).\par
(2)当\(k < 0\)时,
由(1)可知函数\(f(x)\)在\((0, - 2k)\)上单调递增,
在\(( - 2k, + \infty)\)上单调递减,\par
若\(\forall x \in (0, + \infty)\),
都有\(f(x) \leq \mathrm{e}^{- 2}\),
只需\(f( - 2k) = 5k^{2}\mathrm{e}^{- 2} \leq \mathrm{e}^{- 2}\),
即\(5k^{2} \leq 1\),
解得\(- \frac{\sqrt{5}}{5} \leq k \leq \frac{\sqrt{5}}{5}\),\par
此时\(- \frac{\sqrt{5}}{5} \leq k < 0\);\par
当\(k > 0\)时,
由(1)可知函数\(f(x)\)在\((0,k)\)上单调递减,
在\((k, + \infty)\)上单调递增,\par
因为\(f(2k + 1) = \left\lbrack (2k + 1)^{2} - k(2k + 1) - k^{2} \right\rbrack\mathrm{e}^{\frac{2k + 1}{k}} = \left( k^{2} + 3k + 1 \right)\mathrm{e}^{\frac{2k + 1}{k}} > \mathrm{e}^{2} > \mathrm{e}^{- 2}\),\par
与题设条件矛盾.\par
综上所述,
实数\(k\)的取值范围是\(\left\lbrack - \frac{\sqrt{5}}{5},0 \right)\).}
\end{question}
%
\begin{question}
如图,在斜三棱柱\(ABC - A_{1}B_{1}C_{1}\)中,
\(AB = AC\),侧面\(B_{1}BCC_{1}\)为矩形,
\(A_{1}\)在底面\(ABC\)内的射影为\(O\).
\begin{enumerate}[label=(\arabic*)]
  \item 求证:\(AO\bot BC\)且\(OB = OC\);
  \item 若\(BC = OA = \sqrt{2}OB\),
\item \(AA_{1} = 2\sqrt{3}\),
\item \(AA_{1}\)与底面所成角的正切值为\(\frac{\sqrt{2}}{2}\),
\item 求直线\(AA_{1}\)到平面\(B_{1}BCC_{1}\)的距离.
\end{enumerate}
%
\begin{center}
% IMAGE_TODO_START id=jiangsu-rugao-2025-2026-mock1-Q17-img1 path=/Users/muryor/code/mynote/word\\_to\\_tex/output/figures/jiangsu-rugao-2025-2026-mock1/media/image6.png width=60% inline=false question_index=17 sub_index=1
% CONTEXT_BEFORE: B_{1}BCC_{1}$$为矩形,$$A_{1}$$在底面$$ABC$$内的射影为$$O$$.
% CONTEXT_AFTER: (1)求证:$$AO\bot BC$$且$$OB = OC$$; (2)若$$BC = OA
\begin{tikzpicture}[scale=1.05,>=Stealth,line cap=round,line join=round]
  % TODO: AI_AGENT_REPLACE_ME (id=jiangsu-rugao-2025-2026-mock1-Q17-img1)
\end{tikzpicture}
% IMAGE_TODO_END id=jiangsu-rugao-2025-2026-mock1-Q17-img
1
\end{center}
%

\begin{center}
% IMAGE_TODO_START id=jiangsu-rugao-2025-2026-mock1-Q17-img2 path=/Users/muryor/code/mynote/word\\_to\\_tex/output/figures/jiangsu-rugao-2025-2026-mock1/media/image7.png width=60% inline=false question_index=17 sub_index=1
% CONTEXT_BEFORE: O$$并延长交$$BC$$于点$$D$$,连接$$A_{1}B,A_{1}C,A_{1}D$$,
% CONTEXT_AFTER: 因为$$A_{1}$$在底面$$ABC$$内的射影为$$O$$, 所以$$A_{1}O\bot
\begin{tikzpicture}[scale=1.05,>=Stealth,line cap=round,line join=round]
  % TODO: AI_AGENT_REPLACE_ME (id=jiangsu-rugao-2025-2026-mock1-Q17-img2)
\end{tikzpicture}
% IMAGE_TODO_END id=jiangsu-rugao-2025-2026-mock1-Q17-img
2
\end{center}
%
\topics{证明线面垂直;求点面距离;线面垂直证明线线垂直}
\difficulty{0.65}
\answer{(1)见解析
(2)\(\sqrt{6}\)}
\explain{(1)连接\(AO\)并延长交\(BC\)于点\(D\),
连接\(A_{1}B,A_{1}C,A_{1}D\),\par
因为\(A_{1}\)在底面\(ABC\)内的射影为\(O\),\par
所以\(A_{1}O\bot\)平面\(ABC\),
则\(A_{1}O\bot BC\),\par
又因为侧面\(B_{1}BCC_{1}\)为矩形,\par
所以\(BB_{1}\bot BC\),
而\(BB_{1}\text{//}AA_{1}\),所以\(AA_{1}\bot BC\),\par
由于\(AO \cap AA_{1} = A_{1},AO,AA_{1} \subset\)平面\(AA_{1}O\),\par
所以\(BC\bot\)平面\(AA_{1}O\),\par
又因为\(AO \subset\)平面\(AA_{1}O\),
所以\(AO\bot BC\),即\(AD\bot BC\),\par
因为\(AB = AC\),\(AD\bot BC\),
所以\emph{D}为\(BC\)中点,\par
则\(AD\)为\(BC\)的垂直平分线,\par
所以\(OB = OC\),\par
因此,\(AO\bot BC\)且\(OB = OC\)得证;\par
(2)由(1)知\(A_{1}O\bot\)平面\(ABC\),
已知\(BC = OA = \sqrt{2}OB\),
\(AA_{1} = 2\sqrt{3}\),\par
则\(\angle A_{1}AO\)就是\(AA_{1}\)与底面所成角,
其正切值为\(\frac{\sqrt{2}}{2}\),
余弦值为\(\frac{\sqrt{6}}{3}\),\par
\(\cos\angle A_{1}AO = \frac{AO}{AA_{1}} = \frac{AO}{2\sqrt{3}} = \frac{\sqrt{6}}{3}\),
解得\(AO = 2\sqrt{2}\),\par
则\(A_{1}O = 2,BC = 2\sqrt{2},OB = 2\),\par
\(\therefore OD = \sqrt{OB^{2} - BD^{2}} = \sqrt{2},AD = 3\sqrt{2}\),\par
\(V_{B_{1} - ABC} = V_{A_{1} - ABC} = \frac{1}{3} \times \frac{1}{2} \times 2\sqrt{2} \times 3\sqrt{2} \times 2 = 4\),\par
设点\(A\)到平面\(B_{1}BCC_{1}\)的距离为\(h\),\par
\(V_{A - BB_{1}C} = \frac{1}{3}S_{\bigtriangleup BB_{1}C}h = \frac{1}{3} \times \frac{1}{2} \times 2\sqrt{2} \times 2\sqrt{3}h = V_{A_{1} - ABC} = 4\),\par
解得\(h = \sqrt{6}\),\par
又易得\(AA_{1}\text{//}\)平面\(B_{1}BCC_{1}\),\par
所以\(AA_{1}\)到平面\(B_{1}BCC_{1}\)的距离为\(\sqrt{6}\).}
\end{question}
%
\begin{question}
某班准备在周六和周日两天分别进行一次环保志愿活动,分别由李老师和王老师负责通知,
已知该班共60名学生,每次活动需40人参加,
假设李老师和王老师通过"家校通"平台分别将通知独立、随机地发给60位学生家长中的40人,
且保证所发通知都能收到.
\begin{enumerate}[label=(\arabic*)]
  \item 求该班甲同学家长收到李老师或王老师通知的概率;
  \item 设该班乙同学家长收到通知的次数为\(X\),求\(X\)的分布列及数学期望;
  \item 设两次都收到通知的人数为变量\(Y\),则\(Y\)的可能取值有哪些?
\item 并求出\(Y\)取到其中哪一个值的可能性最大?请说明理由.
\end{enumerate}
\topics{计算古典概型问题的概率;写出简单离散型随机变量分布列;服从二项分布的随机变量概率最大问题}
\difficulty{0.4}
\answer{(1)\(1 - \frac{1}{9} = \frac{8}{9}\)
(2)分布列见解析,\(E(X) = \frac{4}{3}\)
(3)\(Y\)取到27的可能性最大}
\explain{(1)李老师通知40人,
甲同学家长未收到李老师通知的概率为\(\frac{20}{60} = \frac{1}{3}\),\par
王老师通知40人,
甲同学家长未收到王老师通知的概率也为\(\frac{20}{60} = \frac{1}{3}\),\par
因为李老师和王老师发通知是独立事件,\par
所以甲同学家长未收到李老师和王老师通知的概率为\(\frac{1}{3} \times \frac{1}{3} = \frac{1}{9}\),\par
所以甲同学家长收到李老师或王老师通知的概率为\(1 - \frac{1}{9} = \frac{8}{9}\);\par
(2)\(X\)表示乙同学家长收到通知的次数,\(X\)的可能取值为0,1,2,\par
\(P(X = 0) = \frac{1}{3} \times \frac{1}{3} = \frac{1}{9}\),\par
\(P(X = 1) = 2 \times \frac{1}{3} \times \frac{2}{3} = \frac{4}{9}\),\par
\(P(X = 2) = \frac{2}{3} \times \frac{2}{3} = \frac{4}{9}\),\par
所以分布列为:\par
\begin{center}
\begin{tabular}{cccc}
\hline
\(X\) & 0 & 1 & 2 \\
\hline
\(P\) & \(\frac{1}{9}\) & \(\frac{4}{9}\) & \(\frac{4}{9}\) \\
\hline
\end{tabular
\end{center}\par
期望\(E(X) = 0 \times \frac{1}{9} + 1 \times \frac{4}{9} + 2 \times \frac{4}{9} = \frac{4}{3}\);\par
(3)\(Y\)表示两次都收到通知的人数,\(Y\)的可能取值为20,21,22,...,40,\par
设\(Y = k\),则\(P(Y = k) = \frac{\mathbb{C}_{40}^{k}\mathbb{C}_{20}^{40 - k}}{\mathbb{C}_{60}^{40}}\),\par
所以\(\frac{P(Y = k + 1)}{P(Y = k)} = \frac{\frac{\mathbb{C}_{40}^{k + 1}\mathbb{C}_{20}^{39 - k}}{\mathbb{C}_{60}^{40}}}{\frac{\mathbb{C}_{40}^{k}\mathbb{C}_{20}^{40 - k}}{\mathbb{C}_{60}^{40}}} = \frac{(40 - k)(40 - k)}{(k + 1)(k - 19)}\),\par
令\(\frac{P(Y = k + 1)}{P(Y = k)} > 1\),解得\(k < \frac{1619}{62} \approx 26.11\),\par
所以\(k \leq 26\)时,\(P(Y = k) = \frac{\mathbb{C}_{40}^{k}\mathbb{C}_{20}^{40 - k}}{\mathbb{C}_{60}^{40}}\)单调递增,\par
\(k \geq 27\)时,\(P(Y = k) = \frac{\mathbb{C}_{40}^{k}\mathbb{C}_{20}^{40 - k}}{\mathbb{C}_{60}^{40}}\)单调递减,\par
又\(\frac{P(Y = 27)}{P(Y = 26)} = \frac{(40 - 26)^{2}}{(26 + 1)(26 - 19)} = \frac{196}{189} > 1\),\par
则\(P(Y = 27) > P(Y = 26)\),\par
所以\(k = 27\)时概率最大,\par
则\(Y\)取到27的可能性最大.}
\end{question}
%
\begin{question}
已知椭圆\(G:\frac{x^{2}}{a^{2}} + \frac{y^{2}}{b^{2}} = 1(a > b > 0)\)的离心率为\(\frac{\sqrt{2}}{2}\),
短轴长为2,椭圆\(G\)上有两点\(A,B\)关于原点对称,
动点\(P\)与\(A,B\)两点的连线分别交椭圆\(G\)于点\(C,D\),
满足\(\overrightarrow{CA} = 2\overrightarrow{PC}\),
\(\overrightarrow{DB} = 2\overrightarrow{PD}\).
\begin{enumerate}[label=(\arabic*)]
  \item 求椭圆\(G\)的方程;
  \item 求动点\(P\)的轨迹方程;
  \item 过\(P\)点作椭圆\(G\)的两条切线(与坐标轴不垂直),
\item 试探究两切线斜率乘积是否为定值?
\end{enumerate}
%
\begin{center}
% IMAGE_TODO_START id=jiangsu-rugao-2025-2026-mock1-Q19-img1 path=/Users/muryor/code/mynote/word\\_to\\_tex/output/figures/jiangsu-rugao-2025-2026-mock1/media/image8.png width=60% inline=false question_index=19 sub_index=1
% CONTEXT_AFTER: 由题设,切线的斜率必定存在,设斜率为$$k$$,得到切线方程为$$y - y_{0} = k(x
\begin{tikzpicture}[scale=1.05,>=Stealth,line cap=round,line join=round]
  % TODO: AI_AGENT_REPLACE_ME (id=jiangsu-rugao-2025-2026-mock1-Q19-img1)
\end{tikzpicture}
% IMAGE_TODO_END id=jiangsu-rugao-2025-2026-mock1-Q19-img
1
\end{center}
%
\topics{根据a;b;c求椭圆标准方程;根据离心率求椭圆的标准方程;椭圆中的定值问题;椭圆中向量共线比例问题}
\difficulty{0.15}
\answer{(1)\(\frac{x^{2}}{2} + y^{2} = 1\)
(2)\(\frac{x^{2}}{4} + \frac{y^{2}}{2} = 1\)
(3)为定值,证明见解析}
\explain{(1)因为椭圆的短轴长为2,离心率为\(\frac{\sqrt{2}}{2}\),
所以\(b = 1\),
\(\frac{c}{a} = \frac{\sqrt{2}}{2}\),\par
由椭圆的性质得\(a^{2} - c^{2} = b^{2} = 1\),
且\(2c = \sqrt{2}a\),解得\(a = \sqrt{2}\),
\(c = 1\),\par
则椭圆\(G\)的方程为\(\frac{x^{2}}{2} + y^{2} = 1\).\par
(2)设\(A(x_{1},y_{1}),C(x_{2},y_{2}),P(x,y),D(x_{3},y_{3})\),\par
而\(A,B\)关于原点对称,则\(B( - x_{1}, - y_{1})\),
可得\(\overrightarrow{CA} = (x_{1} - x_{2},y_{1} - y_{2})\),
\(\overrightarrow{PC} = (x_{2} - x,y_{2} - y)\),\par
因为\(\overrightarrow{CA} = 2\overrightarrow{PC}\),
所以\(\left\{ \begin{array}{r}
x_{1} - x_{2} = 2(x_{2} - x) \\
y_{1} - y_{2} = 2(y_{2} - y)
\end{array} \right.\),解得\(x_{2} = \frac{2x + x_{1}}{3},y_{2} = \frac{2y + y_{1}}{3}\),\par
可得\(C(\frac{2x + x_{1}}{3},\frac{2y + y_{1}}{3})\),因为\(C\)在椭圆上,所以其坐标满足\(\frac{x^{2}}{2} + y^{2} = 1\),\par
则\(\frac{{(\frac{2x + x_{1}}{3})}^{2}}{2} + {(\frac{y_{1} + 2y}{3})}^{2} = 1\),化简得\(x^{2} + 2y^{2} + xx_{1} + 2yy_{1} = 4\),\par
而\(\overrightarrow{DB} = ( - x_{1} - x_{3}, - y_{1} - y_{3})\),\(\overrightarrow{PD} = (x_{3} - x,y_{3} - y)\),\par
因为\(\overrightarrow{DB} = 2\overrightarrow{PD}\),所以\(\left\{ \begin{array}{r}
 - x_{1} - x_{3} = 2(x_{3} - x) \\
 - y_{1} - y_{3} = 2(y_{3} - y)
\end{array} \right.\),\par
解得\(x_{3} = \frac{2x - x_{1}}{3},y_{3} = \frac{2y - y_{1}}{3}\),则\(D(\frac{2x - x_{1}}{3},\frac{2y - y_{1}}{3})\),\par
因为\(D\)在椭圆上,所以其坐标满足\(\frac{x^{2}}{2} + y^{2} = 1\),\par
则\(\frac{{(\frac{2x - x_{1}}{3})}^{2}}{2} + {(\frac{2y - y_{1}}{3})}^{2} = 1\),化简得\(x^{2} + 2y^{2} - xx_{1} - 2yy_{1} = 4\),\par
两式相加可得\(2x^{2} + 4y^{2} = 8\),即\(\frac{x^{2}}{4} + \frac{y^{2}}{2} = 1\).\par
(3)如图,作出符合题意的图形,\par
由题设,切线的斜率必定存在,设斜率为\(k\),得到切线方程为\(y - y_{0} = k(x - x_{0})\),\par
联立方程组\(\left\{ \begin{array}{r}
y - y_{0} = k(x - x_{0}) \\
\frac{x^{2}}{2} + y^{2} = 1
\end{array} \right.\),\par
得到\((\frac{1}{2} + k^{2})x^{2} + 2k(y_{0} - kx_{0})x + \left\lbrack {(y_{0} - kx_{0})}^{2} - 1 \right\rbrack = 0\),\par
因为直线与椭圆相切,所以\(\Delta = 0\),\par
可得\(\left\lbrack 2k(y_{0} - kx_{0}) \right\rbrack^{2} - 4 \times (\frac{1}{2} + k^{2}) \times \left\lbrack {(y_{0} - kx_{0})}^{2} - 1 \right\rbrack = 0\),\par
化简得\((x_{0}^{2} - 2)k^{2} - 2x_{0}y_{0}k + y_{0}^{2} - 1 = 0\),\par
设过\(P(x_{0},y_{0})\)的两条切线的斜率分别为\(k_{1},k_{2}\),\par
因为\(P\)的轨迹方程为\(\frac{x_{0}^{2}}{4} + \frac{y_{0}^{2}}{2} = 1\),所以解得\(x_{0}^{2} = 4 - 2y_{0}^{2}\),\par
由韦达定理得\(k_{1}k_{2} = \frac{y_{0}^{2} - 1}{x_{0}^{2} - 2} = \frac{y_{0}^{2} - 1}{4 - 2y_{0}^{2} - 2} = \frac{y_{0}^{2} - 1}{2 - 2y_{0}^{2}} = \frac{y_{0}^{2} - 1}{- 2(y_{0}^{2} - 1)} = - \frac{1}{2}\).}
\end{question}
