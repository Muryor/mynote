\examxtitle{测试试卷 - auto_0c1b2d}

\section{单选题}

\begin{question}
样本数据2,8,14,16,20的平均数为(   )
\begin{choices}
  \item 8
  \item 9
  \item 12
  \item 18
\end{choices}
\topics{计算几个数的平均数}
\difficulty{0.94}
\answer{C}
\explain{样本数据\(2,8,14,16,20\)的平均数为\(\frac{2 + 8 + 14 + 16 + 20}{5} = \frac{60}{5} = 12\)}
\end{question}

\begin{question}
已知\(z = 1 + \text{i}\),则\(\frac{1}{z - 1} =\)(   )
\begin{choices}
  \item \(- \text{i}\)
  \item \(\text{i}\)
  \item \(- 1\)
  \item 1
\end{choices}
\topics{复数的除法运算}
\difficulty{0.94}
\answer{A}
\explain{因为\(z = 1 + \text{i}\),
所以\(\frac{1}{z - 1} = \frac{1}{1 + \text{i} - 1} = \frac{1}{\text{i}} = \frac{\text{i}}{\text{i}^{2}} = - \text{i}\)}
\end{question}

\begin{question}
已知集合\(A = \{ - 4,0,1,2,8\},B = \left\{ x \mid x^{3} = x \right\}\),则\(A \cap B =\)(   )
\begin{choices}
  \item \(\{ 0,1,2\}\)
  \item \(\{ 1,2,8\}\)
  \item \(\{ 2,8\}\)
  \item \(\{ 0,1\}\)
\end{choices}
\topics{交集的概念及运算}
\difficulty{0.94}
\answer{D}
\explain{\(B = \left\{ x|x^{3} = x \right\} = \left\{ 0, - 1,1 \right\}\),
故\(A \cap B = \left\{ 0,1 \right\}\)}
\end{question}

\begin{question}
不等式\(\frac{x - 4}{x - 1} \geq 2\)的解集是(   )
\begin{choices}
  \item \(\{ x \mid - 2 \leq x \leq 1\}\)
  \item \(\{ x \mid x \leq - 2\}\)
  \item \(\{ x \mid - 2 \leq x < 1\}\)
  \item \(\{ x \mid x > 1\}\)
\end{choices}
\topics{分式不等式}
\difficulty{0.94}
\answer{C}
\explain{\(\frac{x - 4}{x - 1} \geq 2\)即为\(\frac{x + 2}{x - 1} \leq 0\)即\(\left\{ \begin{array}{r}
(x + 2)(x - 1) \leq 0 \\
x - 1 \neq 0
\end{array} \right.\),故\(- 2 \leq x < 1\),\par
故解集为\(\{ x \mid - 2 \leq x < 1\}\)}
\end{question}

\begin{question}
在\(\bigtriangleup ABC\)中,\(BC = 2\),
\(AC = 1 + \sqrt{3}\),\(AB = \sqrt{6}\),
则\(A =\)(   )
\begin{choices}
  \item \(45{^\circ}\)
  \item \(60{^\circ}\)
  \item \(120{^\circ}\)
  \item \(135{^\circ}\)
\end{choices}
\topics{余弦定理解三角形}
\difficulty{0.94}
\answer{A}
\explain{由题意得\(\cos A = \frac{AB^{2} + AC^{2} - BC^{2}}{2AB \cdot AC} = \frac{\left( \sqrt{6} \right)^{2} + \left( 1 + \sqrt{3} \right)^{2} - 2^{2}}{2 \times \sqrt{6} \times \left( 1 + \sqrt{3} \right)} = \frac{\sqrt{2}}{2}\),\par
又\(0^{\circ} < A < 180^{\circ}\),
所以\(A = 45^{\circ}\)}
\end{question}

\begin{question}
设抛物线\(C:y^{2} = 2px(p > 0)\)的焦点为\(F\),点\emph{A}在\emph{C}上,
过\emph{A}作\(C\)的准线的垂线,
垂足为*B.*若直线*BF*的方程为\(y = - 2x + 2\),
则\(|AF| =\)(   )
\begin{choices}
  \item 3
  \item 4
  \item 5
  \item 6
\end{choices}

\begin{center}
% IMAGE_TODO_START id=auto_0c1b2d-Q6-img1 path=/Users/muryor/code/mynote/word\\_to\\_tex/output/figures/auto\\_0c1b2d/media/image2.png width=60% inline=false question_index=6 sub_index=1
% CONTEXT_BEFORE: |AB| = x_{A} + {2} = 4 + 1 = 5$$. 
\begin{tikzpicture}[scale=1.05,>=Stealth,line cap=round,line join=round]
  % TODO: AI_AGENT_REPLACE_ME (id=auto_0c1b2d-Q6-img1)
\end{tikzpicture}
% IMAGE_TODO_END id=auto_0c1b2d-Q6-img
1
\end{center}

\topics{根据抛物线方程求焦点或准线;抛物线的焦半径公式}
\difficulty{0.85}
\answer{C}
\explain{对\(l_{BF}:y = - 2x + 2\),令\(y = 0\),
则\(x = 1\),\par
所以\(F(1,0)\),
\(p = 2\)即抛物线\(C:y^{2} = 4x\),
故抛物线的准线方程为\(x = - 1\),\par
故\(B( - 1,4)\),则\(y_{A} = 4\),
代入抛物线\(C:y^{2} = 4x\)得\(x_{A} = 4\).\par
所以\(|AF| = |AB| = x_{A} + \frac{p}{2} = 4 + 1 = 5\)}
\end{question}

\begin{question}
记\(S_{n}\)为等差数列\(\left\{ a_{n} \right\}\)的前\emph{n}项和.若\(S_{3} = 6,S_{5} = - 5\),则\(S_{6} =\)(   )
\begin{choices}
  \item \(- 20\)
  \item \(- 15\)
  \item \(- 10\)
  \item \(- 5\)
\end{choices}
\topics{求等差数列前n项和;等差数列前n项和的基本量计算}
\difficulty{0.85}
\answer{B}
\explain{设等差数列\(\left\{ a_{n} \right\}\)的公差为\emph{d},
则由题可得
\(\left\{ \begin{array}{r}
3a_{1} + 3d = 6 \\
5a_{1} + 10d = - 5
\end{array} \right.\),\par
所以\(S_{6} = 6a_{1} + 15d = 6 \times 5 + 15 \times ( - 3) = - 15\)}
\end{question}

\begin{question}
已知\(0 < \alpha < \pi\),
\(\cos\frac{\alpha}{2} = \frac{\sqrt{5}}{5}\),
则\(\sin\left( \alpha - \frac{\pi}{4} \right) =\)(   )
\begin{choices}
  \item \(\frac{\sqrt{2}}{10}\)
  \item \(\frac{\sqrt{2}}{5}\)
  \item \(\frac{3\sqrt{2}}{10}\)
  \item \(\frac{7\sqrt{2}}{10}\)
\end{choices}
\topics{已知正(余)弦求余(正)弦;用和;差角的正弦公式化简;求值;二倍角的余弦公式}
\difficulty{0.65}
\answer{D}
\explain{\(\cos\alpha = 2\cos^{2}\frac{\alpha}{2} - 1 = 2 \times \left( \frac{\sqrt{5}}{5} \right)^{2} - 1 = - \frac{3}{5}\),\par
因为\(0 < \alpha < \pi\),
则\(\frac{\pi}{2} < \alpha < \pi\),
则\(\sin\alpha = \sqrt{1 - \cos^{2}\alpha} = \sqrt{1 - \left( - \frac{3}{5} \right)^{2}} = \frac{4}{5}\),\par
则\(\sin\left( \alpha - \frac{\pi}{4} \right) = \sin\alpha\cos\frac{\pi}{4} - \cos\alpha\sin\frac{\pi}{4} = \frac{4}{5} \times \frac{\sqrt{2}}{2} - \left( - \frac{3}{5} \right) \times \frac{\sqrt{2}}{2} = \frac{7\sqrt{2}}{10}\)}
\end{question}

\section{多选题}

\begin{question}
记\(S_{n}\)为等比数列\(\left\{ a_{n} \right\}\)的前\emph{n}项和,
\(q\)为\(\left\{ a_{n} \right\}\)的公比,
\(q > 0\).若\(S_{3} = 7,a_{3} = 1\),
则(   )
\begin{choices}
  \item \(q = \frac{1}{2}\)
  \item \(a_{5} = \frac{1}{9}\)
  \item \(S_{5} = 8\)
  \item \(a_{n} + \text{S}_{n} = 8\)
\end{choices}
\topics{等比数列通项公式的基本量计算;求等比数列前n项和;等比数列前n项和的基本量计算}
\difficulty{0.85}
\answer{AD}
\explain{对A,由题意得\(\left\{ \begin{array}{r}
a_{1}q^{2} = 1 \\
a_{1} + a_{1}q + a_{1}q^{2} = 7
\end{array} \right.\),结合\(q > 0\),解得\(\left\{ \begin{array}{r}
a_{1} = 4 \\
q = \frac{1}{2}
\end{array} \right.\)或\(\left\{ \begin{array}{r}
a_{1} = 9 \\
q = - \frac{1}{3}
\end{array} \right.\)(舍去),故A正确;\par
对B,则\(a_{5} = a_{1}q^{4} = 4 \times \left( \frac{1}{2} \right)^{4} = \frac{1}{4}\),故B错误;\par
对C,\(S_{5} = \frac{a_{1}\left( 1 - q^{5} \right)}{1 - q} = \frac{4 \times \left( 1 - \frac{1}{32} \right)}{1 - \frac{1}{2}} = \frac{31}{4}\),故C错误;\par
对D,\(a_{n} = 4 \times \left( \frac{1}{2} \right)^{n - 1} = 2^{3 - n}\),\(S_{n} = \frac{4 \times \left\lbrack 1 - \left( \frac{1}{2} \right)^{n} \right\rbrack}{1 - \frac{1}{2}} = 8 - 2^{- n + 3}\),\par
则\(a_{n} + S_{n} = 2^{3 - n} + 8 - 2^{3 - n} = 8\),故D正确}
\end{question}

\begin{question}
已知\(f(x)\)是定义在*\emph{R}*上的奇函数,
且当\(x > 0\)时,
\(f(x) = \left( x^{2} - 3 \right)e^{x} + 2\),
则(   )
\begin{choices}
  \item \(f(0) = 0\)
  \item 当\(x < 0\)时,\(f(x) = - \left( x^{2} - 3 \right)e^{- x} - 2\)
  \item \(f(x) \geq 2\)当且仅当\(x \geq \sqrt{3}\)
  \item \(x = - 1\)是\(f(x)\)的极大值点
\end{choices}
\topics{由奇偶性求函数解析式;函数奇偶性的应用;求已知函数的极值点}
\difficulty{0.65}
\answer{ABD}
\explain{对A,因为\(f(x)\)定义在\(R\)上奇函数,则\(f(0) = 0\),
故A正确;\par
对B,当\(x < 0\)时,\(- x > 0\),
则\(f(x) = - f( - x) = - \left\lbrack \left( ( - x)^{2} - 3 \right)e^{- x} + 2 \right\rbrack = - \left( x^{2} - 3 \right)e^{- x} - 2\),
故B正确;\par
对C,
\(f( - 1) = - (1 - 3)e - 2 = 2(e - 1) > 2\),
 故C错误;\par
对D,当\(x < 0\)时,
\(f(x) = \left( 3 - x^{2} \right)e^{- x} - 2\),
则\(f'(x) = - \left( 3 - x^{2} \right)e^{- x} - 2xe^{- x} = \left( x^{2} - 2x - 3 \right)e^{- x}\),\par
令\(f'(x) = 0\),解得\(x = - 1\)或\(3\)(舍去),\par
当\(x \in ( - \infty, - 1)\)时,
\(f'(x) > 0\),此时\(f(x)\)单调递增,\par
当\(x \in ( - 1,0)\)时,\(f'(x) < 0\),
此时\(f(x)\)单调递减,\par
则\(x = - 1\)是\(f(x)\)极大值点,故D正确}
\end{question}

\begin{question}
双曲线\(C:\frac{x^{2}}{a^{2}} - \frac{y^{2}}{b^{2}} = 1(a > 0,b > 0)\)的左、右焦点分别为\(F_{1},F_{2}\),
左、右顶点分别为\(A_{1},A_{2}\),
以\(F_{1}F_{2}\)为直径的圆与\emph{C}的一条渐近线交于\emph{M},
\emph{N}两点,
且\(\angle NA_{1}M = \frac{5\pi}{6}\),则(   )
\begin{choices}
  \item \(\angle A_{1}MA_{2} = \frac{\pi}{6}\)
  \item \(\left| MA_{1} \right| = 2\left| MA_{2} \right|\)
  \item \emph{C}的离心率为\(\sqrt{13}\)
  \item 当\(a = \sqrt{2}\)时,四边形\(NA_{1}MA_{2}\)的面积为\(8\sqrt{3}\)
\end{choices}

\begin{center}
% IMAGE_TODO_START id=auto_0c1b2d-Q11-img1 path=/Users/muryor/code/mynote/word\\_to\\_tex/output/figures/auto\\_0c1b2d/media/image3.png width=60% inline=false question_index=11 sub_index=1
% CONTEXT_AFTER: 方法二:因为$$\tan\angle MOA_{2} = {a}$$,因为双曲线
\begin{tikzpicture}[scale=1.05,>=Stealth,line cap=round,line join=round]
  % TODO: AI_AGENT_REPLACE_ME (id=auto_0c1b2d-Q11-img1)
\end{tikzpicture}
% IMAGE_TODO_END id=auto_0c1b2d-Q11-img
1
\end{center}


\begin{center}
% IMAGE_TODO_START id=auto_0c1b2d-Q11-img2 path=/Users/muryor/code/mynote/word\\_to\\_tex/output/figures/auto\\_0c1b2d/media/image4.png width=60% inline=false question_index=11 sub_index=1
% CONTEXT_AFTER: 方法三:在$$\bigtriangleup OMA_{2}$$利用余弦定理知,$$\left|
\begin{tikzpicture}[scale=1.05,>=Stealth,line cap=round,line join=round]
  % TODO: AI_AGENT_REPLACE_ME (id=auto_0c1b2d-Q11-img2)
\end{tikzpicture}
% IMAGE_TODO_END id=auto_0c1b2d-Q11-img
2
\end{center}

\topics{用定义求向量的数量积;已知数量积求模;双曲线的对称性;求双曲线的离心率或离心率的取值范围}
\difficulty{0.65}
\answer{ACD}
\explain{不妨设渐近线为\(y = \frac{b}{a}x\),\(M\)在第一象限,
\(N\)在第三象限,\par
对于A,由双曲线的对称性可得\(A_{1}MA_{2}N\)为平行四边形,
故\(\angle A_{1}MA_{2} = \pi - \frac{5\pi}{6} = \frac{\pi}{6}\),\par
故A正确;\par
对于B,方法一:因为\(M\)在以\(F_{1}F_{2}\)为直径的圆上,
故\(F_{1}M\bot F_{2}M\)且\(|MO| = c\),\par
设\(M\left( x_{0},y_{0} \right)\),则\(\left\{ \begin{array}{r}
x_{0}^{2} + y_{0}^{2} = c^{2} \\
\frac{y_{0}}{x_{0}} = \frac{b}{a}
\end{array} \right.\),故\(\left\{ \begin{array}{r}
x_{0} = a \\
y_{0} = b
\end{array} \right.\),故\(MA_{2}\bot A_{1}A_{2}\),\par
由A得\(\angle A_{1}MA_{2} = \frac{\pi}{6}\),故\(\left| MA_{2} \right| = \left| MA_{1} \right| \times \frac{\sqrt{3}}{2}\)即\(\left| MA_{1} \right| = \frac{2\sqrt{3}}{3}\left| MA_{2} \right|\),故B错误;\par
方法二:因为\(\tan\angle MOA_{2} = \frac{b}{a}\),因为双曲线中,\(c^{2} = a^{2} + b^{2}\),\par
则\(\cos\angle MOA_{2} = \frac{a}{c}\),又因为以\(F_{1}F_{2}\)为直径的圆与\(C\)的一条渐近线交于\(M\)、\(N\),则\(OM = c\),\par
则若过点\(M\)往\(x\)轴作垂线,垂足为\(H\),则\(|OH| = c \cdot \frac{a}{c} = a = \left| OA_{2} \right|\),则点\(H\)与\(A_{2}(H)\)重合,则\(MA_{2}\bot x\)轴,则\(\left| MA_{2} \right| = \sqrt{c^{2} - a^{2}} = b\),\par
方法三:在\(\bigtriangleup OMA_{2}\)利用余弦定理知,\(\left| MA_{2} \right|^{2} = |OM|^{2} + \left| OA_{2} \right|^{2} - 2|OM|\left| OA_{2} \right|\cos\angle MOA_{2}\),\par
即\(\left| MA_{2} \right|^{2} = c^{2} + a^{2} - 2ac \cdot \frac{a}{c} = b^{2}\),则\(\left| MA_{2} \right| = b\),\par
则\(\bigtriangleup A_{1}A_{2}M\)为直角三角形,且\(\angle A_{1}MA_{2} = \frac{\pi}{6}\),则\(2\left| MA_{2} \right| = \sqrt{3}\left| MA_{1} \right|\),故B错误;\par
对于C,方法一:因为\(\overrightarrow{MO} = \frac{1}{2}\left( \overrightarrow{MA_{1}} + \overrightarrow{MA_{2}} \right)\),故\(4{\overrightarrow{MO}}^{2} = {\overrightarrow{MA_{1}}}^{2} + 2\overrightarrow{MA_{1}} \cdot \overrightarrow{MA_{2}} + {\overrightarrow{MA_{2}}}^{2}\),\par
由B可知\(\left| MA_{2} \right| = b,\left| MA_{1} \right| = \frac{2\sqrt{3}}{3}b\),\par
故\(4c^{2} = b^{2} + \frac{4}{3}b^{2} + 2 \times b \times \frac{2\sqrt{3}}{3}b \times \frac{\sqrt{3}}{2} = \frac{13}{3}b^{2} = \frac{13}{3}\left( c^{2} - a^{2} \right)\)即\(c^{2} = 13a^{2}\),\par
故离心率\(e = \sqrt{13}\),故C正确;\par
方法二:因为\(\frac{\left| MA_{2} \right|}{\left| A_{1}A_{2} \right|} = \frac{b}{2a} = \sqrt{3}\),则\(\frac{b}{a} = 2\sqrt{3}\),则\(e = \frac{c}{a} = \sqrt{1 + \frac{b^{2}}{a^{2}}} = \sqrt{1 + {(2\sqrt{3})}^{2}} = \sqrt{13}\),故C正确;\par
对于D,当\(a = \sqrt{2}\)时,由C可知\(e = \sqrt{13}\),故\(c = \sqrt{26}\),\par
故\(b = 2\sqrt{6}\),故四边形\(NA_{1}MA_{2}\)为\(2S_{\bigtriangleup MA_{1}A_{2}} = 2 \times \frac{1}{2} \times 2\sqrt{6} \times 2\sqrt{2} = 8\sqrt{3}\),\par
故D正确}
\end{question}

\section{填空题}

\begin{question}
已知平面向量\(\overrightarrow{a} = (x,1),\overrightarrow{b} = (x - 1,2x)\),若\(\overrightarrow{a}\bot\left( \overrightarrow{a} - \overrightarrow{b} \right)\),
则\(|\overrightarrow{a}| =\)
\topics{平面向量线性运算的坐标表示;坐标计算向量的模;向量垂直的坐标表示}
\difficulty{0.94}
\answer{\(\sqrt{2}\)}
\explain{\(\overrightarrow{a} - \overrightarrow{b} = (1,1 - 2x)\),
因为\(\overrightarrow{a}\bot\left( \overrightarrow{a} - \overrightarrow{b} \right)\),
则\(\overrightarrow{a} \cdot \left( \overrightarrow{a} - \overrightarrow{b} \right) = 0\),\par
则\(x + 1 - 2x = 0\),解得\(x = 1\).\par
则\(\overrightarrow{a} = (1,1)\),
则\(|\overrightarrow{a}| = \sqrt{2}\).\(\sqrt{2}\).}
\end{question}

\begin{question}
若\(x = 2\)是函数\(f(x) = (x - 1)(x - 2)(x - a)\)的极值点,则\(f(0) =\)
\topics{求函数值;导数的运算法则;根据极值点求参数}
\difficulty{0.85}
\answer{\(- 4\)}
\explain{由题意有\(f(x) = (x - 1)(x - 2)(x - a)\),\par
所以\(f'(x) = (x - a)(x - 1) + (x - 1)(x - 2) + (x - a)(x - 2)\),\par
因为\(2\)是函数\(f(x)\)极值点,
所以\(f'(2) = 2 - a = 0\),得\(a = 2\),\par
当\(a = 2\)时,
\(f'(x) = 2(x - 2)(x - 1) + (x - 2)^{2} = (x - 2)(3x - 4)\),\par
当\(x \in \left( - \infty,\frac{4}{3} \right),f'(x) > 0,f(x)\)单调递增,
当\(x \in \left( \frac{4}{3},2 \right),f'(x) < 0,f(x)\)单调递减,\par
当\(x \in (2, + \infty),f'(x) > 0,f(x)\)单调递增,\par
所以\(x = 2\)是函数\(f(x) = (x - 1)(x - 2)(x - a)\)的极小值点,
符合题意;\par
所以\(f(0) = - 1 \times ( - 2) \times ( - a) = - 2a = - 4\).\(- 4\).}
\end{question}

\begin{question}
一个底面半径为\(\text{4cm}\),
高为\(\text{9cm}\)的封闭圆柱形容器(容器壁厚度忽略不计)内有两个半径相等的铁球,
则铁球半径的最大值为
\(\text{cm}\).

\begin{center}
% IMAGE_TODO_START id=auto_0c1b2d-Q14-img1 path=/Users/muryor/code/mynote/word\\_to\\_tex/output/figures/auto\\_0c1b2d/media/image5.png width=60% inline=false question_index=14 sub_index=1
% CONTEXT_BEFORE: 、球的截面的性质及计算 【分析】根据圆柱与球的性质以及球的体积公式可求出球的半径; 【详解】
% CONTEXT_AFTER: 圆柱的底面半径为$$4$$,设铁球的半径为*r*,且$$r < 4$$, 由
\begin{tikzpicture}[scale=1.05,>=Stealth,line cap=round,line join=round]
  % TODO: AI_AGENT_REPLACE_ME (id=auto_0c1b2d-Q14-img1)
\end{tikzpicture}
% IMAGE_TODO_END id=auto_0c1b2d-Q14-img
1
\end{center}

\topics{圆柱的结构特征辨析;球的截面的性质及计算}
\difficulty{0.4}
\answer{\(2.5\)}
\explain{圆柱的底面半径为\(4\text{cm}\),设铁球的半径为\emph{r},
且\(r < 4\),\par
由圆柱与球的性质知\(AB^{2} = {(2r)}^{2} = {(8 - 2r)}^{2} + {(9 - 2r)}^{2}\),\par
即\(4r^{2} - 68r + 145 = (2r - 5)(2r - 29) = 0\),
\(\because r < 4\),\par
\(\therefore r = 2.5\).\(2.5\).}
\end{question}

\section{解答题}

\begin{question}
已知函数\(f(x) = \cos(2x + \varphi)(0 \leq \varphi < \pi),f
\begin{enumerate}[label=(\arabic*)]
  \item = \frac{1}{2}\).
  \item 求\(\varphi\);
  \item 设函数\(g(x) = f(x) + f\left( x - \frac{\pi}{6} \right)\),
\item 求\(g(x)\)的值域和单调区间.
\end{enumerate}
\topics{求含cosx的函数的单调性;利用余弦函数的单调性求参数;求cosx(型)函数的值域}
\difficulty{0.65}
\answer{(1)\(\varphi = \frac{\pi}{3}\)
(2)答案见解析}
\explain{(1)由题意\(f(0) = \cos\varphi = \frac{1}{2},\left( 0 \leq \varphi < \pi \right)\),
所以\(\varphi = \frac{\pi}{3}\);\par
(2)由(1)可知\(f(x) = \cos\left( 2x + \frac{\pi}{3} \right)\),\par
所以\(g(x) = f(x) + f\left( x - \frac{\pi}{6} \right) = \cos\left( 2x + \frac{\pi}{3} \right) + \cos 2x= \frac{1}{2}\cos 2x - \frac{\sqrt{3}}{2}\sin 2x + \cos 2x = \frac{3}{2}\cos 2x - \frac{\sqrt{3}}{2}\sin 2x = \sqrt{3}\cos\left( 2x + \frac{\pi}{6} \right)\),\par
所以函数\(g(x)\)的值域为\(\left\lbrack - \sqrt{3},\sqrt{3} \right\rbrack\),\par
令\(2k\pi \leq 2x + \frac{\pi}{6} \leq \pi + 2k\pi,k \in \mathbb{Z}\),
解得\(- \frac{\pi}{12} + k\pi \leq x \leq \frac{5\pi}{12} + k\pi,k \in \mathbb{Z}\),\par
令\(\pi + 2k\pi \leq 2x + \frac{\pi}{6} \leq 2\pi + 2k\pi,k \in \mathbb{Z}\),
解得\(\frac{5\pi}{12} + k\pi \leq x \leq \frac{11\pi}{12} + k\pi,k \in \mathbb{Z}\),\par
所以函数\(g(x)\)的单调递减区间为\(\left\lbrack - \frac{\pi}{12} + k\text{\pi,}\frac{5\pi}{12} + k\pi \right\rbrack,k \in \mathbb{Z}\),\par
函数\(g(x)\)的单调递增区间为\(\left\lbrack \frac{5\pi}{12} + k\text{\pi,}\frac{11\pi}{12} + k\pi \right\rbrack,k \in \mathbb{Z}\).}
\end{question}

\begin{question}
已知椭圆\(C:\frac{x^{2}}{a^{2}} + \frac{y^{2}}{b^{2}} = 1(a > b > 0)\)的离心率为\(\frac{\sqrt{2}}{2}\),
长轴长为4.
\begin{enumerate}[label=(\arabic*)]
  \item 求\emph{C}的方程;
  \item 过点\((0, - 2)\)的直线\emph{l}交\emph{C}于\(A,B\)两点,
\item \(O\)为坐标原点.若\(\bigtriangleup OAB\)的面积为\(\sqrt{2}\),
\item 求\(|AB|\).
\end{enumerate}

\begin{center}
% IMAGE_TODO_START id=auto_0c1b2d-Q16-img1 path=/Users/muryor/code/mynote/word\\_to\\_tex/output/figures/auto\\_0c1b2d/media/image6.png width=60% inline=false question_index=16 sub_index=1
% CONTEXT_AFTER: 由题设直线$$AB$$的斜率不为0,故设直线$$l:x = t(y + 2)$$,$$A\lef
\begin{tikzpicture}[scale=1.05,>=Stealth,line cap=round,line join=round]
  % TODO: AI_AGENT_REPLACE_ME (id=auto_0c1b2d-Q16-img1)
\end{tikzpicture}
% IMAGE_TODO_END id=auto_0c1b2d-Q16-img
1
\end{center}

\topics{根据a;b;c求椭圆标准方程;根据离心率求椭圆的标准方程;椭圆中三角形(四边形)的面积}
\difficulty{0.85}
\answer{(1)\(\frac{x^{2}}{4} + \frac{y^{2}}{2} = 1\)
(2)\(\sqrt{5}\)}
\explain{(1)因为长轴长为4,故\(a = 2\),
而离心率为\(\frac{\sqrt{2}}{2}\),故\(c = \sqrt{2}\),\par
故\(b = \sqrt{2}\),
故椭圆方程为:\(\frac{x^{2}}{4} + \frac{y^{2}}{2} = 1\).\par
(2)\par
由题设直线\(AB\)的斜率不为0,故设直线\(l:x = t(y + 2)\),
\(A\left( x_{1},y_{1} \right),B\left( x_{2},y_{2} \right)\),\par
由\(\left\{ \begin{array}{r}
x = t(y + 2) \\
x^{2} + 2y^{2} = 4
\end{array} \right.\)可得\(\left( t^{2} + 2 \right)y^{2} + 4t^{2}y + 4t^{2} - 4 = 0\),\par
故\(\Delta = 16t^{4} - 4\left( t^{2} + 2 \right)\left( 4t^{2} - 4 \right) = 4\left( 8 - 4t^{2} \right) > 0\)即\(- \sqrt{2} < t < \sqrt{2}\),\par
且\(y_{1} + y_{2} = - \frac{4t^{2}}{t^{2} + 2},y_{1}y_{2} = \frac{4t^{2} - 4}{t^{2} + 2}\),\par
故\(S_{\bigtriangleup OAB} = \frac{1}{2} \times |2t| \times \left| y_{1} - y_{2} \right| = |t|\sqrt{\left( y_{1} + y_{2} \right)^{2} - 4y_{1}y_{2}} = \frac{|t|\sqrt{32 - 16t^{2}}}{t^{2} + 2} = \sqrt{2}\),\par
解得\(t = \pm \frac{\sqrt{6}}{3}\),\par
故\(|AB| = \sqrt{1 + t^{2}}\left| y_{1} - y_{2} \right| = \sqrt{1 + \frac{2}{3}} \times \sqrt{\left( y_{1} + y_{2} \right)^{2} - 4y_{1}y_{2}} = \sqrt{\frac{5}{3}} \times \frac{\sqrt{32 - 16 \times \frac{2}{3}}}{\frac{2}{3} + 2} = \sqrt{5}\).}
\end{question}

\begin{question}
如图,在四边形\(ABCD\)中,
\(AB//CD,\angle DAB = 90{^\circ}\),
\emph{F}为*CD*的中点,点\emph{E}在*AB*上,\(EF//AD\),
\(AB = 3AD,CD = 2AD\).将四边形\(EFDA\)沿\(EF\)翻折至四边形\(EFD'A'\),
使得面\(EFD'A'\)与面*EFCB*所成的二面角为\(60{^\circ}\).
\begin{enumerate}[label=(\arabic*)]
  \item 证明:\(A'B//\)平面\(CD'F\);
  \item 求面\(BCD'\)与面\(EFD'A'\)所成的二面角的正弦值.
\end{enumerate}

\begin{center}
% IMAGE_TODO_START id=auto_0c1b2d-Q17-img1 path=/Users/muryor/code/mynote/word\\_to\\_tex/output/figures/auto\\_0c1b2d/media/image7.png width=60% inline=false question_index=17 sub_index=1
% CONTEXT_BEFORE: A'$$,使得面$$EFD'A'$$与面*EFCB*所成的二面角为$$60{^\circ}$$.
% CONTEXT_AFTER: (1)证明:$$A'B//$$平面$$CD'F$$; (2)求面$$BCD'$$与面$$EFD
\begin{tikzpicture}[scale=1.05,>=Stealth,line cap=round,line join=round]
  % TODO: AI_AGENT_REPLACE_ME (id=auto_0c1b2d-Q17-img1)
\end{tikzpicture}
% IMAGE_TODO_END id=auto_0c1b2d-Q17-img
1
\end{center}


\begin{center}
% IMAGE_TODO_START id=auto_0c1b2d-Q17-img2 path=/Users/muryor/code/mynote/word\\_to\\_tex/output/figures/auto\\_0c1b2d/media/image8.png width=60% inline=false question_index=17 sub_index=1
% CONTEXT_BEFORE: \subset$$平面$$A'EB$$,所以$$A'B//$$平面$$CD'F$$. (2)
% CONTEXT_AFTER: 因为$$\angle DAB = 90{^\circ}$$,所以$$AD\bot AB$$,又因
\begin{tikzpicture}[scale=1.05,>=Stealth,line cap=round,line join=round]
  % TODO: AI_AGENT_REPLACE_ME (id=auto_0c1b2d-Q17-img2)
\end{tikzpicture}
% IMAGE_TODO_END id=auto_0c1b2d-Q17-img
2
\end{center}

\topics{证明线面平行;求二面角;面面角的向量求法}
\difficulty{0.65}
\answer{(1)证明见解析
(2)\(\frac{\sqrt{42}}{7}\)}
\explain{(1)设\(AD = 1\),所以\(AB = 3,CD = 2\),
因为\(F\)为\(CD\)中点,所以\(DF = 1\),
因为\(EF//AD\),\(AB//CD\),所以\(AEFD\)是平行四边形,\par
所以\(AE//DF\),所以\(A'E//D'F\),\par
因为\(D'F \subset\)平面\(CD'F,A'E ⊄\)平面\(CD'F\),
所以\(A'E//\)平面\(CD'F\),\par
因为\(FC//EB,FC \subset\)平面\(CD'F,EB ⊄\)平面\(CD'F\),
所以\(EB//\)平面\(CD'F\),\par
又\(EB \cap A'E = E\),
\(EB,A'E \subset\)平面\(A'EB\),
所以平面\(A'EB//\)平面\(CD'F\),\par
又\(A'B \subset\)平面\(A'EB\),
所以\(A'B//\)平面\(CD'F\).\par
(2)\par
因为\(\angle DAB = 90{^\circ}\),
所以\(AD\bot AB\),又因为\(AB//FC,EF//AD\),
所以\(EF\bot FC\),\par
以\(F\)为原点,
\(FE,FC\)以及垂直于平面\(BECF\)的直线分别为\(x,y,z\)轴,
建立空间直角坐标系.\par
因为\(D'F\bot EF,CF\bot EF\),
平面\(EFD'A'\)与平面\(EFCB\)所成二面角为60°
,\par
所以\(\angle D'FC = 60^{{^\circ}}\).\par
则\(B(1,2,0)\),\(C(0,1,0)\),
\(D'\left( 0,\frac{1}{2},\frac{\sqrt{3}}{2} \right)\),
\(E(1,0,0)\),\(F(0,0,0)\),.\par
所以\(\overrightarrow{BC} = ( - 1, - 1,0),\overrightarrow{CD'} = \left( 0, - \frac{1}{2},\frac{\sqrt{3}}{2} \right),\overrightarrow{FE} = (1,0,0),\overrightarrow{FD'} = \left( 0,\frac{1}{2},\frac{\sqrt{3}}{2} \right)\).\par
设平面\(BCD'\)的法向量为\(\overrightarrow{n} = (x,y,z)\),
则\par
\(\left\{ \begin{array}{r}
\overrightarrow{BC} \cdot \overrightarrow{n} = 0 \\
\overrightarrow{CD'} \cdot \overrightarrow{n} = 0
\end{array} \right.\),所以\(\left\{ \begin{array}{r}
 - \frac{1}{2}y + \frac{\sqrt{3}}{2}z = 0 \\
 - x - y = 0
\end{array} \right.\),令\(y = \sqrt{3}\),则\(z = 1,x = - \sqrt{3}\),则\(\overrightarrow{n} = \left( - \sqrt{3},\sqrt{3},1 \right)\).\par
设平面\(EFD'A'\)的法向量为\(\overrightarrow{m} = \left( x_{1},y_{1},z_{1} \right)\),\par
则\(\left\{ \begin{array}{r}
\overrightarrow{FE} \cdot \overrightarrow{m} = 0 \\
\overrightarrow{FD'} \cdot \overrightarrow{m} = 0
\end{array} \right.\),所以\(\left\{ \begin{array}{r}
\frac{1}{2}y + \frac{\sqrt{3}}{2}z = 0 \\
x = 0
\end{array} \right.\),\par
令\(y = \sqrt{3}\),则\(z = - 1,x = 0\),所以\(\overrightarrow{m} = \left( 0,\sqrt{3}, - 1 \right)\).\par
所以\(\cos\left\langle \overrightarrow{m},\overrightarrow{n} \right\rangle = \frac{\left| \overrightarrow{m} \cdot \overrightarrow{n} \right|}{\left| \overrightarrow{m} \right|\overrightarrow{|n|}} = \frac{0 + 3 - 1}{\sqrt{3 + 3 + 1} \times \sqrt{1 + 3}} = \frac{1}{\sqrt{7}}\).\par
所以平面\(BCD'\)与平面\(EFD'A'\)夹角的正弦值为\(\sqrt{1 - \left( \frac{1}{\sqrt{7}} \right)^{2}} = \frac{\sqrt{42}}{7}\).}
\end{question}

\begin{question}
已知函数\(f(x) = \ln(1 + x) - x + \frac{1}{2}x^{2} - kx^{3}\),
其中\(0 < k < \frac{1}{3}\).
\begin{enumerate}[label=(\arabic*)]
  \item 证明:\(f(x)\)在区间\((0, + \infty)\)存在唯一的极值点和唯一的零点;
  \item 设\(x_{1},x_{2}\)分别为\(f(x)\)在区间\((0, + \infty)\)的极值点和零点.

\item (i)设函数\(g(t) = f\left( x_{1} + t \right) - f\left( x_{1} - t \right)\).证明:\(g(t)\)在区间\(\left( 0,x_{1} \right)\)单调递减;


\item (ii)比较\(2x_{1}\)与\(x_{2}\)的大小,并证明你的结论.
\end{enumerate}
\topics{用导数判断或证明已知函数的单调性;利用导数证明不等式;利用导数研究函数的零点;求已知函数的极值点}
\difficulty{0.65}
\answer{(1)证明见解析;
(2)(i)证明见解析;(ii)\(2x_{1} > x_{2}\),证明见解析.}
\explain{(1)由题得\(f'(x) = \frac{1}{1 + x} - 1 + x - 3kx^{2} = \frac{x^{2}}{1 + x} - 3kx^{2} = x^{2}\left( \frac{1}{1 + x} - 3k \right)\),\par
因为\(x \in (0, + \infty)\),
所以\(x^{2} > 0\),
设\(g(x) = \frac{1}{1 + x} - 3k,x > 0\),\par
则\(g'(x) = - \frac{1}{(1 + x)^{2}} < 0\)在\((0, + \infty)\)上恒成立,
所以\(g(x)\)在\((0, + \infty)\)上单调递减,\par
\(g(0) = 1 - 3k > 0\),
令\(g\left( x_{0} \right) = 0 \Rightarrow x_{0} = \frac{1}{3k} - 1\),\par
所以当\(x \in \left( 0,x_{0} \right)\)时,
\(g(x) > 0\),则\(f'(x) > 0\);
当\(x \in \left( x_{0}, + \infty \right)\)时,
\(g(x) < 0\),则\(f'(x) < 0\),\par
所以\(f(x)\)在\(\left( 0,x_{0} \right)\)上单调递增,
在\(\left( x_{0}, + \infty \right)\)上单调递减,\par
所以\(f(x)\)在\((0, + \infty)\)上存在唯一极值点,\par
对函数\(y = \ln(1 + x) - x\)有\(y' = \frac{1}{1 + x} - 1 = - \frac{x}{1 + x} < 0\)在\((0, + \infty)\)上恒成立,\par
所以\(y = \ln(1 + x) - x\)在\((0, + \infty)\)上单调递减,\par
所以\(y = \ln(1 + x) - x < y|_{x = 0} = 0\)在\((0, + \infty)\)上恒成立,\par
又因为\(f(0) = 0\),
\(x \rightarrow + \infty\)时\(\frac{1}{2}x^{2} - kx^{3} = \frac{1}{2}x^{2}(1 - 2kx) < 0\),\par
所以\(x arrow + \infty\)时\(f(x) < 0\),\par
所以存在唯一\(x_{2} \in (0, + \infty)\)使得\(f\left( x_{2} \right) = 0\),
即\(f(x)\)在\((0, + \infty)\)上存在唯一零点.\par
(2)(i)由(1)知\(x_{1} = \frac{1}{3k} - 1\),
则\(x_{1} + 1 = \frac{1}{3k}\),
\(f'(x) = x^{2}\left( \frac{1}{1 + x} - \frac{1}{1 + x_{1}} \right)\),\par
\(\because g(t) = f\left( x_{1} + t \right) - f\left( x_{1} - t \right)\),\par
则\(g'(t) = \left( x_{1} + t \right)^{2}\left( \frac{1}{x_{1} + t + 1} - \frac{1}{1 + x_{1}} \right) + \left( x_{1} - t \right)^{2}\left( \frac{1}{x_{1} - t + 1} - \frac{1}{1 + x_{1}} \right)= \frac{- t\left( x_{1} + t \right)^{2}}{\left( x_{1} + t + 1 \right)\left( x_{1} + 1 \right)} + \frac{t\left( x_{1} - t \right)^{2}}{\left( x_{1} - t + 1 \right)\left( x_{1} + 1 \right)}= 3kt\left\lbrack - \frac{\left( x_{1} + t \right)^{2}}{x_{1} + t + 1} + \frac{\left( x_{1} - t \right)^{2}}{x_{1} - t + 1} \right\rbrack= \frac{6kt^{2}\left( t^{2} - x_{1}^{2} - 2x_{1} \right)}{\left( 1 + x_{1} \right)^{2} - t^{2}}\),\par
\(\because k > 0,t \in \left( 0,x_{1} \right),\therefore t^{2} - x_{1}^{2} - 2x_{1} < 0,\left( 1 + x_{1} \right)^{2} - t^{2} > 0\),\par
\(\therefore g'(t) = \frac{6kt^{2}\left( t^{2} - x_{1}^{2} - 2x_{1} \right)}{\left( 1 + x_{1} \right)^{2} - t^{2}} < 0\),\par
即\(g(t)\)在\(t \in \left( 0,x_{1} \right)\)上单调递减.\par
(ii)\(2x_{1} > x_{2}\),证明如下:\par
由(i)知:函数\(g(t)\)在区间\(\left( 0,x_{1} \right)\)上单调递减,\par
所以\(g(0) > g\left( x_{1} \right)\)即\(0 > f\left( 2x_{1} \right)\),
又\(f\left( x_{2} \right) = 0\),\par
由(1)可知\(f(x)\)在\(\left( x_{0}, + \infty \right)\)上单调递减,
\(x_{2} \in \left( x_{0}, + \infty \right)\),
且对任意\(x \in \left( 0,x_{2} \right)f(x) > 0\),\par
所以\(2x_{1} > x_{2}\).}
\end{question}

\begin{question}
甲、乙两人进行乒乓球练习,每个球胜者得1分,
负者得0分.设每个球甲胜的概率为\(p\left( \frac{1}{2} < p < 1 \right)\),
乙胜的概率为\emph{q},\(p + q = 1\),
且各球的胜负相互独立.对正整数\(k \geq 2\),
记\(p_{k}\)为打完\emph{k}个球后甲比乙至少多得2分的概率,
\(q_{k}\)为打完\emph{k}个球后乙比甲至少多得2分的概率.
\begin{enumerate}[label=(\arabic*)]
  \item 求\(p_{3},p_{4}\)(用\emph{p}表示).
  \item 若\(\frac{p_{4} - p_{3}}{q_{4} - q_{3}} = 4\),
\item 求\emph{p}.
  \item 证明:对任意正整数\emph{m},
\item \(p_{2m + 1} - q_{2m + 1} < p_{2m} - q_{2m} < p_{2m + 2} - q_{2m + 2}\).
\end{enumerate}
\topics{计算古典概型问题的概率;服从二项分布的随机变量概率最大问题;建立二项分布模型解决实际问题}
\difficulty{0.15}
\answer{(1)\(p_{3} = p^{3}\),\(p_{4} = p^{3}(4 - 3p)\)
(2)\(p = \frac{2}{3}\)
(3)证明过程见解析}
\explain{(1)\(p_{3}\)为打完3个球后甲比乙至少多得两分的概率,故只能甲胜三场,\par
故所求为\(p_{3} = \mathbb{C}_{3}^{3}(1 - p)^{0}p^{3} = p^{3}\),\par
\(p_{4}\)为打完4个球后甲比乙至少多得两分的概率,故甲胜三场或四场,\par
故所求为\(p_{4} = \mathbb{C}_{4}^{3}(1 - p)^{1}p^{3} + \mathbb{C}_{4}^{4}(1 - p)^{0}p^{4} = 4p^{3}(1 - p) + p^{4} = p^{3}(4 - 3p)\);\par
(2)由(1)得\(p_{3} = p^{3}\),
\(p_{4} = p^{3}(4 - 3p)\),
同理\(q_{3} = q^{3},q_{4} = q^{3}(4 - 3q)\),\par
若\(\frac{p_{4} - p_{3}}{q_{4} - q_{3}} = 4\),
\(p + q = 1\),\par
则\(\frac{p_{4} - p_{3}}{q_{4} - q_{3}} = \frac{p^{3}(4 - 3p) - p^{3}}{q^{3}(4 - 3q) - q^{3}} = \frac{3p^{3}(1 - p)}{3q^{3}(1 - q)} = \frac{p^{3}q}{q^{3}p} = \left( \frac{p}{q} \right)^{2} = 4\),\par
由于\(0 < p,q < 1\),
所以\(p = 2q = 2(1 - p) > 0\),
解得\(p = \frac{2}{3}\);\par
(3)设打完\(k\)个球,甲的得分为\(X_{k}\),
乙的得分为\(Y_{k}\),\(X_{k} + Y_{k} = k\),\par
所以\(p_{2m} = P(X_{2m} \geq m + 1)\),
\(p_{2m + 1} = P(X_{2m + 1} \geq m + 2)\),
\(p_{2m + 2} = P(X_{2m + 2} \geq m + 2)\),\par
\(q_{2m} = P(Y_{2m} \geq m + 1)\),
\(q_{2m + 1} = P(Y_{2m + 1} \geq m + 2)\),
\(q_{2m + 2} = P(Y_{2m + 2} \geq m + 2)\),\par
要证明\(p_{2m + 1} - q_{2m + 1} < p_{2m} - q_{2m} < p_{2m + 2} - q_{2m + 2}\),\par
即证明①\(p_{2m + 1} - p_{2m} < q_{2m + 1} - q_{2m}\),
②\(p_{2m + 2} - p_{2m} > q_{2m + 2} - q_{2m}\),\par
先证明①\(p_{2m + 1} - p_{2m} < q_{2m + 1} - q_{2m}\),\par
\(p_{2m + 1} - p_{2m} = P(X_{2m + 1} \geq m + 2) - P(X_{2m} \geq m + 1)= P(X_{2m} \geq m + 2) + P(X_{2m} = m + 1)p - P(X_{2m} \geq m + 1)= P(X_{2m} = m + 1)p - P(X_{2m} = m + 1)= (p - 1)C_{2m}^{m + 1}p^{m + 1}q^{m - 1}\),\par
同理可得\(q_{2m + 1} - q_{2m} = (q - 1)C_{2m}^{m + 1}q^{m + 1}p^{m - 1}\),\par
所以①\(\Leftrightarrow (p - 1)C_{2m}^{m + 1}p^{m + 1}q^{m - 1} < (q - 1)C_{2m}^{m + 1}q^{m + 1}p^{m - 1} \Leftrightarrow p^{2}(p - 1) < q^{2}(q - 1) \Leftrightarrow - p^{2}q < - q^{2}p \Leftrightarrow - p < - q \Leftrightarrow p > q\),
故成立;\par
证明②\(p_{2m + 2} - p_{2m} > q_{2m + 2} - q_{2m}\):\par
\(p_{2m + 2} - p_{2m} = P(X_{2m + 2} \geq m + 2) - P(X_{2m} \geq m + 1)= P(X_{2m} = m)p^{2} + P(X_{2m} = m + 1)\lbrack 1 - {(1 - p)}^{2}\rbrack + P(X_{2m} \geq m + 2) - P(X_{2m} \geq m + 1)= P(X_{2m} = m)p^{2} + P(X_{2m} = m + 1)\lbrack 1 - {(1 - p)}^{2}\rbrack + P(X_{2m} \geq m + 1) - P(X_{2m} = m + 1) - P(X_{2m} \geq m + 1)= P(X_{2m} = m)p^{2} + P(X_{2m} = m + 1)(1 - q^{2}) - P(X_{2m} = m + 1)= C_{2m}^{m}p^{m}q^{m}p^{2} - q^{2}C_{2m}^{m + 1}p^{m + 1}q^{m - 1}= C_{2m}^{m}p^{m + 2}q^{m} - C_{2m}^{m + 1}p^{m + 1}q^{m + 1}\),\par
同理可得\(q_{2m + 2} - q_{2m} = C_{2m}^{m}q^{m + 2}p^{m} - C_{2m}^{m + 1}q^{m + 1}p^{m + 1}\),\par
所以②\(\Leftrightarrow C_{2m}^{m}p^{m + 2}q^{m} - C_{2m}^{m + 1}p^{m + 1}q^{m + 1} > C_{2m}^{m}q^{m + 2}p^{m} - C_{2m}^{m + 1}q^{m + 1}p^{m + 1} \Leftrightarrow C_{2m}^{m}p^{m + 2}q^{m} > C_{2m}^{m}q^{m + 2}p^{m} \Leftrightarrow p^{2} > q^{2} \Leftrightarrow p > q\),
故成立;\par
综上,
不等式\(p_{2m + 1} - q_{2m + 1} < p_{2m} - q_{2m} < p_{2m + 2} - q_{2m + 2}\)成立.}
\end{question}
