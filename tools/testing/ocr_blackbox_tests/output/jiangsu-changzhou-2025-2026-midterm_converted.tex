\examxtitle{测试试卷 - jiangsu-changzhou-2025-2026-midterm}

\section{单选题}

\begin{question}
已知集合\(A = \left\{ x \mid x^{2} - 5x + 4 < 0 \right\}B = \left\{ n\left| \frac{2n}{n - 1} \right.\  \right\).是质数,\(\left. \ n \in \mathbb{N} \right\}\),
则\(A \cap B =\)(    )
\begin{choices}
  \item \(\varnothing\)
  \item {2}
  \item {3}
  \item \(\left\{ 2,3 \right\}\)
\end{choices}
\topics{列举法表示集合;交集的概念及运算;解不含参数的一元二次不等式}
\difficulty{0.65}
\answer{C}
\explain{由\(x^{2} - 5x + 4 = (x - 1)(x - 4) < 0\),
解得\(1 < x < 4\),
故\(A = \left\{ x\mid 1 < x < 4 \  \right\}\).\par
因为\(\frac{2n}{n - 1} = \frac{2(n - 1) + 2}{n - 1} = 2 + \frac{2}{n - 1}\left( n \in \mathbb{N},n \neq 1 \right)\),\par
要使\(\frac{2n}{n - 1}\)是质数,
\(\frac{2}{n - 1}\)必须是整数,而\(n - 1\)必须是2的正因数.\par
因为2的正因数有1和2,\par
所以当\(n - 1 = 1\)时,\(n = 2\),
此时\(\frac{2n}{n - 1} = \frac{2 \times 2}{2 - 1} = 4\),
4不是质数,不符合要求,舍去;\par
所以当\(n - 1 = 2\)时,\(n = 3\),
此时\(\frac{2n}{n - 1} = \frac{2 \times 3}{3 - 1} = 3\),
3是质数,符合要求,故\(B = \left\{ 3 \right\}\).\par
所以\(A \cap B = \left\{ 3 \right\}\)}
\end{question}

\begin{question}
某店日盈利\(y\)(单位:百元)与当天平均气温\(x\)(单位:\({}^{\circ}\mathbb{C}\))之间有如下数据:

\begin{center}
\begin{tabular}{cccccc}
\hline
\(x/{}^{\circ}\mathbb{C}\) & -2 & -1 & 0 & 1 & 2 \\
\hline
\(y/\)百元 & 5 & 4 & 2 & 2 & 1 \\
\hline
\end{tabular
\end{center}

小明对上述数据进行分析,发现\(y\)与\(x\)之间具有线性相关关系,则\(y\)关于\(x\)的经验回归方程为(    )
\begin{choices}
  \item \(\widehat{y} = x + 2.8\)
  \item \(\widehat{y} = - x + 2.8\)
  \item \(\widehat{x} = y + 2.8\)
  \item \(\widehat{x} = - y + 2.8\)
\end{choices}
\topics{计算样本的中心点}
\difficulty{0.85}
\answer{B}
\explain{由题意可知\(\overline{x} = 0\),
\(\overline{y} = \frac{1}{5}(5 + 4 + 2 + 2 + 1) = \frac{14}{5}\),
样本中心点为\(\left( 0,\frac{14}{5} \right)\),\par
由样本数据可知,\(y\)随着\(x\)的增大而减小,
所以\(\widehat{y} = - x + 2.8\)符合条件}
\end{question}

\begin{question}
下列四个命题中,是假命题的为(    )
\begin{choices}
  \item \(\exists x > 1,x + \frac{1}{x} \geq 2\)
  \item \(\exists x > - 1,\frac{4}{x + 1} > 2 - x\)
  \item \(\forall x > 2,\sqrt{x} + \frac{2}{\sqrt{x}} \geq 2\sqrt{2}\)
  \item \(\forall x < 0,x + \frac{1}{x} < - 2\)
\end{choices}
\topics{判断全称命题的真假;判断特称(存在性)命题的真假;基本不等式求和的最小值}
\difficulty{0.85}
\answer{D}
\explain{对于A,取\(x = 2\),
\(x + \frac{1}{x} = 2 + \frac{1}{2} = \frac{5}{2} > 2\),
A是真命题;\par
对于B,取\(x = 0\),
\(\frac{4}{x + 1} = 4 > 2 = 2 - x\),B是真命题;\par
对于C,
\(\sqrt{x} + \frac{2}{\sqrt{x}} \geq 2\sqrt{\sqrt{x} \cdot \frac{2}{\sqrt{x}}} = 2\sqrt{2}\),
当且仅当\(\sqrt{x} = \frac{2}{\sqrt{x}}\),
即\(x = 2\)时取等号,\par
因此当\(\forall x > 2\)时,
\(\sqrt{x} + \frac{2}{\sqrt{x}} > 2\sqrt{2}\),C是真命题;\par
对于D,当\(x = - 1\)时,
\(x + \frac{1}{x} = - 2\),D是假命题}
\end{question}

\begin{question}
已知随机变量\(X \sim N(\mu,9)\),
若\(P(X < 1 - a) = P(X > 7 + a)\left( a \in \mathbb{R} \right)\),
则(    )
\begin{choices}
  \item \(E(X) = 3,D(X) = 3\)
  \item \(E(X) = 4,D(X) = 3\)
  \item \(E(X) = 3,D(X) = 9\)
  \item \(E(X) = 4,D(X) = 9\)
\end{choices}
\topics{二项分布的均值;二项分布的方差;正态曲线的性质}
\difficulty{0.85}
\answer{D}
\explain{因为\(P(X < 1 - a) = P(X > 7 + a)\),
所以\(\mu = \frac{1 - a + 7 + a}{2} = 4\),
所以\(E(X) = 4\),\par
又\(D(X) = \sigma^{2} = 9\)}
\end{question}

\begin{question}
将函数\(f(x) = \cos(x + \frac{\pi}{3})\)的图象向左平移\(\frac{\pi}{4}\)个单位长度,
再将得到的曲线上每一个点的横坐标变为原来的2倍(纵坐标不变),
得到函数\(y = g(x)\)的图象,则\(g(x) =\)(    )
\begin{choices}
  \item \(\cos(\frac{x}{2} + \frac{7\pi}{12})\)
  \item \(\cos(\frac{x}{2} + \frac{\pi}{12})\)
  \item \(\cos(2x + \frac{7\pi}{12})\)
  \item \(\cos(2x + \frac{\pi}{12})\)
\end{choices}
\topics{求图象变化前(后)的解析式}
\difficulty{0.85}
\answer{A}
\explain{将函数\(f(x) = \cos(x + \frac{\pi}{3})\)的图象向左平移\(\frac{\pi}{4}\)个单位长度,
得到\(y = \cos(x + \frac{\pi}{4} + \frac{\pi}{3}) = \cos(x + \frac{7\pi}{12})\),\par
再将得到的曲线上每一个点的横坐标变为原来的2倍(纵坐标不变),
得\(g(x) = \cos(\frac{x}{2} + \frac{7\pi}{12})\)}
\end{question}

\begin{question}
已知圆柱和圆锥的底面半径相同,母线长也相同,则它们的表面积之比为(    )
\begin{choices}
  \item \(\sqrt{2}:1\)
  \item \(\sqrt{3}:1\)
  \item \(2:1\)
  \item 3:1
\end{choices}
\topics{圆柱表面积的有关计算;圆锥表面积的有关计算}
\difficulty{0.85}
\answer{C}
\explain{设它们底面圆半径为\(r\),母线长为\(l\),\par
记圆柱的表面积为\(S_{1}\),
则\(S_{1} = 2\pir^{2} + 2\pirl\),\par
记圆锥的表面积为\(S_{2}\),
则\(S_{2} = \pir^{2} + \pirl\),\par
所以圆柱与圆锥表面积之比\(S_{1}:S_{2} = 2:1\)}
\end{question}

\begin{question}
若实数\(x = 10\),\(y = 5\text{eln}2\),\(z = 2\text{eln}5\)则\(x,y,z\)的大小关系是(    )
\begin{choices}
  \item \(x > y > z\)
  \item \(x > z > y\)
  \item \(y > x > z\)
  \item \(y > z > x\)
\end{choices}
\topics{函数单调性;极值与最值的综合应用;比较对数式的大小}
\difficulty{0.65}
\answer{A}
\explain{因为\(y = 5\text{eln}2 = \text{eln}2^{5} = \text{eln32}\),
\(z = 2\text{eln}5 = \text{eln}5^{2} = \text{eln}25\),\par
由\(\ln 32 > \ln 25 > 0\),
所以\(\text{e}\ln 32 > \text{e}\ln 25\),即\(y > z\).\par
设函数\(f(x) = \frac{\ln x}{x}\),\(x > 0\),
则\(f'(x) = \frac{1 - \ln x}{x^{2}}\),
\(x > 0\).\par
由\(f'(x) > 0\) \(\Rightarrow\) \(0 < x < \text{e}\);
由\(f'(x) < 0\Rightarrow\) \(x > \text{e}\).\par
即\(f(x)\)在\(\left( 0,\text{e} \right)\)上单调递增,
在\(\left( \text{e}, + \infty \right)\)上单调递减.\par
所以\(f(x) \leq f\left( \text{e} \right) = \frac{1}{\text{e}}\),
所以\(\text{e}f(x) \leq 1\).\par
所以\(\text{e}f(2) < 1\) \(\Rightarrow\) \(\frac{\text{e}\ln 2}{2} < 1\Rightarrow\) \(\text{e}\ln 2 < 2\) \(\Rightarrow5\text{e}\ln 2 < 10\),
即\(y < x\).\par
综上,\(x > y > z\)}
\end{question}

\begin{question}
已知函数\(f(x) = x^{3} + 3ax^{2} + x\left( a \in \mathbb{R} \right),P\)是函数\(f(x)\)的图象上的定点,
过\(P\)的动直线与函数\(f(x)\)的图象有异于\(P\)的两个公共点\(M,N\),
且它们的纵坐标之和恒为2,则\(P\)的横坐标为(    )
\begin{choices}
  \item 1
  \item \(- 1\)
  \item 2
  \item \(- 2\)
\end{choices}
\topics{二次函数的图象分析与判断}
\difficulty{0.4}
\answer{B}
\explain{因为\(P\)在函数\(f(x)\)的图象上,
所以可设\(P\left( x_{0},x_{0}^{3} + 3ax_{0}^{2} + x_{0} \right)\),\par
设直线\(MN\)方程为:\(y - \left( x_{0}^{3} + 3ax_{0}^{2} + x_{0} \right) = k\left( x - x_{0} \right)\Rightarrowy = k\left( x - x_{0} \right) + \left( x_{0}^{3} + 3ax_{0}^{2} + x_{0} \right)\),\par
代入\(y = x^{3} + 3ax^{2} + x\)得:\(x^{3} + 3ax^{2} + x = k\left( x - x_{0} \right) + \left( x_{0}^{3} + 3ax_{0}^{2} + x_{0} \right)\),\par
化简得:\(x^{3} + 3ax^{2} + (1 - k)x - x_{0}^{3} - 3ax_{0}^{2} - (1 - k)x_{0} = 0\).\par
因为\(x = x_{0}\)为该方程的1个根,
所以方程可化成\(\left( x - x_{0} \right)\left( x^{2} + bx + c \right) = 0\),\par
即\(x^{3} + \left( b - x_{0} \right)x^{2} + \left( c - bx_{0} \right)x - cx_{0} = 0\).\par
所以\(b - x_{0} = 3a\) \(\Rightarrow\) \(b = x_{0} + 3a\).\par
设\(M\left( x_{1},y_{1} \right)\),
\(N\left( x_{2},y_{2} \right)\),\par
则\(x_{1}\),
\(x_{2}\)为方程\(x^{2} + bx + c = 0\)的两根,
所以\(x_{1} + x_{2} = - b = - x_{0} - 3a\),\par
由\(y_{1} + y_{2} = 2\) \(\Rightarrowk\left( x_{1} - x_{0} \right) + \left( x_{0}^{3} + 3ax_{0}^{2} + x_{0} \right) + k\left( x_{2} - x_{0} \right) + \left( x_{0}^{3} + 3ax_{0}^{2} + x_{0} \right) = 2\),\par
即\(k\left( x_{1} + x_{2} - 2x_{0} \right) + 2\left( x_{0}^{3} + 3ax_{0}^{2} + x_{0} \right) = 2\)恒成立.\par
所以\(\{\begin{array}{r}
x_{1} + x_{2} - 2x_{0} = 0 \\
2(x_{0}^{3} + 3ax_{0}^{2} + x_{0}) = 2
\end{array}\),\par
由\(x_{1} + x_{2} - 2x_{0} = 0\)及\(x_{1} + x_{2} = - x_{0} - 3a\)可得\(( - x_{0} - 3a) - 2x_{0} = 0\),解得\(x_{0} = - a\),\par
由\(2(x_{0}^{3} + 3ax_{0}^{2} + x_{0}) = 2\)可得\(x_{0}^{3} + 3ax_{0}^{2} + x_{0} = 1\),\par
将\(x_{0} = - a\)代入该式得\(( - a)^{3} + 3a( - a)^{2} + ( - a) = 1\),即\(2a^{3} - a - 1 = 0\),\par
\((a - 1)\left( 2a^{2} + 2a + 1 \right) = 0\),\par
所以\(a = 1\),所以\(x_{0} = - 1\),即\(P\)点的横坐标为:\(- 1\)}
\end{question}

\section{多选题}

\begin{question}
已知\(z\)是虚数,且\(|z| = 1\).下列四个选项中,\(\frac{1}{z} + z\)的可能取值有(    )
\begin{choices}
  \item 0
  \item i
  \item 1
  \item \(1 + \text{i}\)
\end{choices}
\topics{求复数的模;复数的除法运算}
\difficulty{0.85}
\answer{AC}
\explain{由\(z\)是虚数,\(|z| = 1\),
设\(z = \cos\theta + \text{i}\sin\theta,\theta \neq k\pi,k \in \mathbb{Z}\),\par
则\(\frac{1}{z} + z = \frac{1}{\cos\theta + \text{i}\sin\theta} + \cos\theta + \text{i}\sin\theta = 2\cos\theta \in ( - 2,2)\),\par
因此\(\frac{1}{z} + z\)的可能取值有0和1}
\end{question}

\begin{question}
立德中学某班\(5\)名同学参加"青春向党"知识竞赛答题活动,其成绩均为正整数,
中位数为\(70\),唯一众数为\(80\),极差为\(15\),
则下列说法正确的是(    )
\begin{choices}
  \item 该组数据的最小值可能为\(64\)
  \item 该组数据的平均数不超过\(73\)
  \item 该数据的第\(60\)百分位数为\(75\)
  \item 该组数据的方差超过\(36\)
\end{choices}
\topics{计算几个数的平均数;计算几个数据的极差;方差;标准差;总体百分位数的估计}
\difficulty{0.65}
\answer{BCD}
\explain{由题意设该组数据从小到大为\(a\)、\(b\)、\(c\)、\(d\)、\(e\),\par
由题意可得\(c = 70\),\(d = e = 80\),
\(80 - a = 15\),可得\(a = 65\),A错;\par
这组数据为\(65\)、\(b\)、\(70\)、\(80\)、\(80\),
则\(65 < b < 70\),\par
这组数据的平均数为\(\overline{x} = \frac{65 + b + 70 + 80 + 80}{5} = 59 + \frac{b}{5} \in (72,73)\),
B对;\par
对于C选项,因为\(5 \times 0.6 = 3\),
所以该数据的第\(60\)百分位数为\(\frac{70 + 80}{2} = 75\),
C对;\par
对于D选项,当\(b = 66\)时,
这组数据的平均数为\(\overline{x} = 59 + \frac{66}{5} = 72.2\),\par
这组数的方差为\(s^{2} = \frac{1}{5}\left\lbrack (65 - 72.2)^{2} + (66 - 72.2)^{2} + (70 - 72.2)^{2} + 2 \times (80 - 72.2)^{2} \right\rbrack = 43.36 > 36\),\par
当\(b = 67\)时,
这组数的平均数为\(\overline{x} = 59 + \frac{67}{5} = 72.4\),\par
这组数的方差为\(s^{2} = \frac{1}{5}\left\lbrack (65 - 72.4)^{2} + (67 - 72.4)^{2} + (70 - 72.4)^{2} + 2 \times (80 - 72.4)^{2} \right\rbrack = 41.04 > 36\),\par
当\(b = 68\)时,
这组数据的平均数为\(\overline{x} = 59 + \frac{68}{5} = 72.6\),\par
这组数的方差为\(s^{2} = \frac{1}{5}\left\lbrack (65 - 72.6)^{2} + (68 - 72.6)^{2} + (70 - 72.6)^{2} + 2 \times (80 - 72.6)^{2} \right\rbrack = 39.04 > 36\),\par
当\(b = 69\),
此时这组数据的平均数为\(\overline{x} = 59 + \frac{69}{5} = 72.8\),\par
这组数的方差为\(s^{2} = \frac{1}{5}\left\lbrack (65 - 72.8)^{2} + (69 - 72.8)^{2} + (70 - 72.8)^{2} + 2 \times (80 - 72.8)^{2} \right\rbrack = 37.36 > 36\),\par
因此,这组数据的方差大于\(36\),D对}
\end{question}

\begin{question}
已知在矩形\(ABCD\)中,\(AB = \sqrt{2},BC = 1\),
\(P\)为线段\(CD\)的中点,
将\(\bigtriangleup ADP, \bigtriangleup BCP\)分别沿\(AP,BP\)翻折,
使得\(C,D\)两点重合于点\(Q\),则(    )
\begin{choices}
  \item \(AQ\bot BQ\)
  \item 三棱锥\(P - ABQ\)的体积为\(\frac{\sqrt{2}}{4}\)
  \item 点\(Q\)到平面\(ABP\)的距离为\(\frac{1}{2}\)
  \item 存在半径为\(\frac{\sqrt{10}}{2}\)的球\(O\),使得\(A,B,P,Q\)四点均在球\(O\)的球面上
\end{choices}

\begin{center}
% IMAGE_TODO_START id=jiangsu-changzhou-2025-2026-midterm-Q11-img1 path=/Users/muryor/code/mynote/word\\_to\\_tex/output/figures/jiangsu-changzhou-2025-2026-midterm/media/image2.png width=60% inline=false question_index=11 sub_index=1
% CONTEXT_AFTER: 
\begin{tikzpicture}[scale=1.05,>=Stealth,line cap=round,line join=round]
  % TODO: AI_AGENT_REPLACE_ME (id=jiangsu-changzhou-2025-2026-midterm-Q11-img1)
\end{tikzpicture}
% IMAGE_TODO_END id=jiangsu-changzhou-2025-2026-midterm-Q11-img
1
\end{center}

\topics{锥体体积的有关计算;多面体与球体内切外接问题;求点面距离}
\difficulty{0.4}
\answer{AC}
\explain{对A:\(AQ = AD = 1\),\(BQ = BC = 1\),
\(AB = \sqrt{2}\),\par
有\(AQ^{2} + BQ^{2} = AB^{2}\),
故\(AQ\bot BQ\),故A正确;\par
对B:由\(\angle D = \angle C = \frac{\pi}{2}\),
故\(AQ\bot PQ\)、\(BQ\bot PQ\),\par
又\(AQ\)、\(BQ \subset\)平面\(ABQ\),
\(AQ \cap BQ = Q\),故\(PQ\bot\)平面\(ABQ\),\par
故\(V_{P - ABQ} = \frac{1}{3} \cdot PQ \cdot S_{\bigtriangleup ABQ} = \frac{1}{3} \times \frac{\sqrt{2}}{2} \cdot \frac{1}{2} \times 1 \times 1 = \frac{\sqrt{2}}{12}\),
故B错误;\par
对C:设点\(Q\)到平面\(ABP\)的距离为\(d\),
则由\(V_{P - ABQ} = V_{Q - ABP}\)可得:\par
\(\frac{\sqrt{2}}{12} = \frac{1}{3} \cdot d \cdot S_{{}_{\bigtriangleup ABP}} = \frac{1}{3} \cdot d \cdot \frac{1}{2} \times \sqrt{2} \times 1 = \frac{\sqrt{2}}{6}d\),
则\(d = \frac{1}{2}\),故C正确;\par
对D:设三棱锥\(P - ABQ\)外接球球心为\(O\),半径为\(r\),\par
由\(AQ\bot BQ\),\(PQ\bot\)平面\(ABQ\),
取\(AB\)中点\(E\),\par
则\(OE//PQ\),且\(OP = OQ = r\),
则\(OE = \frac{1}{2}PQ = \frac{\sqrt{2}}{4}\),\par
则有\(r^{2} = QE^{2} + OE^{2} = \left( \frac{\sqrt{2}}{2} \right)^{2} + \left( \frac{\sqrt{2}}{4} \right)^{2} = \frac{5}{8}\),\par
即\(r = \sqrt{\frac{5}{8}} = \frac{\sqrt{10}}{4}\),
故D错误}
\end{question}

\section{填空题}

\begin{question}
在\((x - 1)(x + 2)(x - 3)(x + 4)\)的展开式中,含\(x^{3}\)的项的系数为
.
\topics{分步乘法计数原理及简单应用}
\difficulty{0.85}
\answer{2}
\explain{在\((x - 1)(x + 2)(x - 3)(x + 4)\)的展开式中,\par
从\(4\)个因式中,\(3\)个因式选择\(x\),
\(1\)个因式选择常数相乘的积即可得含\(x^{3}\)的项,\par
所以含\(x^{3}\)的项的系数为\(- 1 + 2 - 3 + 4 = 2\).\(2\)}
\end{question}

\begin{question}
已知平面向量\(\overrightarrow{a} = (\cos\alpha,\sqrt{3}\sin(\pi + \alpha)),\overrightarrow{b} = (\sqrt{3}\cos\beta,\text{cos(}\frac{\pi}{2} - \beta))\),
其中\(\alpha,\beta\)是锐角.若\(\overrightarrow{a}\bot\overrightarrow{b}\),
则\(\alpha + \beta =\)
.
\topics{三角函数的化简;求值------诱导公式;用和;差角的余弦公式化简;求值;向量垂直的坐标表示}
\difficulty{0.65}
\answer{\(\frac{\pi}{2}\)/\(\frac{1}{2}\pi\)}
\explain{向量\(\overrightarrow{a} = (\cos\alpha,\sqrt{3}\sin(\pi + \alpha)),\overrightarrow{b} = (\sqrt{3}\cos\beta,\text{cos(}\frac{\pi}{2} - \beta))\),
由\(\overrightarrow{a}\bot\overrightarrow{b}\),\par
得\(\overrightarrow{a} \cdot \overrightarrow{b} = \sqrt{3}\cos\alpha\cos\beta + \sqrt{3}\sin(\pi + \alpha)\text{cos(}\frac{\pi}{2} - \beta) = \sqrt{3}\text{(cos}\alpha\cos\beta - \sin\alpha\sin\beta)= \sqrt{3}\cos(\alpha + \beta) = 0\),
由\(\alpha,\beta\)是锐角,
得\(0 < \alpha + \beta < \pi\),\par
所以\(\alpha + \beta = \frac{\pi}{2}\).\(\frac{\pi}{2}\)}
\end{question}

\begin{question}
某校学生在研究民间剪纸艺术时,发现剪纸时经常会把纸沿某直线折叠,
现有一张长方形纸\(ABCD)(AB > BC\)).若将长方形纸\(ABCD\)对折,
使得\(AD,BC\)重合,得到新的长方形,
发现长边与短边的长度比保持不变.若将长方形纸\(ABCD\)的顶点\(A\)折到边\(CD\)上,
设折痕所在直线与\(CD\)的夹角为\(\theta\),当折痕最短时,
\(\sin\theta =\)
.

\begin{center}
% IMAGE_TODO_START id=jiangsu-changzhou-2025-2026-midterm-Q14-img1 path=/Users/muryor/code/mynote/word\\_to\\_tex/output/figures/jiangsu-changzhou-2025-2026-midterm/media/image3.png width=60% inline=false question_index=14 sub_index=1
% CONTEXT_BEFORE: {2}b$$. 情形1: 如图1,折痕$$PQ \geq AB = b$$.
% CONTEXT_AFTER: 情形2: 如图2,$$AF = {\cos\angle AFE} = \fr
\begin{tikzpicture}[scale=1.05,>=Stealth,line cap=round,line join=round]
  % TODO: AI_AGENT_REPLACE_ME (id=jiangsu-changzhou-2025-2026-midterm-Q14-img1)
\end{tikzpicture}
% IMAGE_TODO_END id=jiangsu-changzhou-2025-2026-midterm-Q14-img
1
\end{center}


\begin{center}
% IMAGE_TODO_START id=jiangsu-changzhou-2025-2026-midterm-Q14-img2 path=/Users/muryor/code/mynote/word\\_to\\_tex/output/figures/jiangsu-changzhou-2025-2026-midterm/media/image4.png width=60% inline=false question_index=14 sub_index=1
% CONTEXT_AFTER: 情形3: 如图3.当$$F$$与$$O$$重合时,$$PQ$$取得最小值. 如图4,此时$$
\begin{tikzpicture}[scale=1.05,>=Stealth,line cap=round,line join=round]
  % TODO: AI_AGENT_REPLACE_ME (id=jiangsu-changzhou-2025-2026-midterm-Q14-img2)
\end{tikzpicture}
% IMAGE_TODO_END id=jiangsu-changzhou-2025-2026-midterm-Q14-img
2
\end{center}


\begin{center}
% IMAGE_TODO_START id=jiangsu-changzhou-2025-2026-midterm-Q14-img3 path=/Users/muryor/code/mynote/word\\_to\\_tex/output/figures/jiangsu-changzhou-2025-2026-midterm/media/image5.png width=60% inline=false question_index=14 sub_index=1
% CONTEXT_AFTER: [公式:7302f77e-58ec-42e3-8bdb-6643e6c8a9d9]
\end{tikzpicture}
% IMAGE_TODO_END id=jiangsu-changzhou-2025-2026-midterm-Q14-img
3
\end{center}


\begin{center}
% IMAGE_TODO_START id=jiangsu-changzhou-2025-2026-midterm-Q14-img4 path=/Users/muryor/code/mynote/word\\_to\\_tex/output/figures/jiangsu-changzhou-2025-2026-midterm/media/image6.png width=60% inline=false question_index=14 sub_index=1
% CONTEXT_BEFORE: su-changzhou-2025-2026-midterm/media/image5.png)
\begin{tikzpicture}[scale=1.05,>=Stealth,line cap=round,line join=round]
  % TODO: AI_AGENT_REPLACE_ME (id=jiangsu-changzhou-2025-2026-midterm-Q14-img4)
\end{tikzpicture}
% IMAGE_TODO_END id=jiangsu-changzhou-2025-2026-midterm-Q14-img
4
\end{center}

\topics{三角函数的化简;求值------同角三角函数基本关系;基本不等式的实际应用}
\difficulty{0.4}
\answer{\(\frac{\sqrt{6}}{3}\)/\(\frac{1}{3}\sqrt{6}\)}
\explain{设\(AD\)中点为\(E\),\(CD,AB\)中点依次记为\(M,N\),\(MN\)中点记为\(O\),连接\(OE\).\par
设\(AB = a,AD = b\),因为\(A'\)在\(DC\)上,
所以\(AA'\)的中点在线段\(OE\)上.\par
折痕为\(PQ\)过\(F\),且\(PQ\bot AA'\),情况分三种,
如图1,2,3.\par
由题意,
\(\frac{a}{b} = \frac{b}{\frac{a}{2}}\),得\(a = \sqrt{2}b\).\par
情形1:\par
如图1,折痕\(PQ \geq AB = \sqrt{2}b\).\par
情形2:\par
如图2,
\(AF = \frac{AE}{\cos\angle AFE} = \frac{b}{2\cos\theta},AP = \frac{AF}{\sin\angle APF} = \frac{AF}{\sin\theta} = \frac{b}{2\sin\theta\cos\theta}\),\par
\(PQ = \frac{AP}{\cos\angle APQ} = \frac{AP}{\cos\theta} = \frac{b}{2\sin\theta\cos^{2}\theta}\),\par
\(\left( \sqrt{2}\sin\theta\cos^{2}\theta \right)^{2} = \left( \sqrt{2}\sin\theta \right)^{2}\cos^{2}\theta\cos^{2}\theta \leq \frac{\left( \left( \sqrt{2}\sin\theta \right)^{2} + \cos^{2}\theta + \cos^{2}\theta \right)^{3}}{27} = \frac{8}{27}\),\par
所以\(PQ \geq \frac{b}{\sqrt{2} \times \sqrt{\frac{8}{27}}} = \frac{3\sqrt{3}b}{4}\).\par
情形3:\par
如图3.当\(F\)与\(O\)重合时,\(PQ\)取得最小值.\par
如图4,
此时\(PQ = \frac{2ON}{\sin\angle ONQ} = \frac{2ON}{\sin\angle ONQ} = \frac{b}{\frac{\sqrt{2}}{\sqrt{3}}} = \frac{\sqrt{6}b}{2}\).\par
由\(24 < 27 < 32\),得\(2\sqrt{6} < 3\sqrt{3} < 4\sqrt{2}\),所以\(\frac{\sqrt{6}}{2} < \frac{3\sqrt{3}}{4} < \sqrt{2}\).\par
故当折痕最短时,
\(\sin\theta = \frac{a}{\sqrt{a^{2} + b^{2}}} = \frac{\sqrt{6}}{3}\).\(\frac{\sqrt{6}}{3}\)}
\end{question}

\section{解答题}

\begin{question}
为调查某地区老人是否需要志愿者提供帮助,
用简单随机抽样方法从该地区调查了500位老年人,调查结果如下表:

\begin{center}
\begin{tabular}{ccc}
\hline
男性 & 女性 &  \\
\hline
需要 & 40 & 20 \\
不需要 & 160 & 280 \\
\hline
\end{tabular}
\end{center}
\begin{enumerate}[label=(\arabic*)]
  \item 在该地区男性老年人中,随机选择一位,
\item 他需要志愿者提供帮助的概率记为\(P\),求\(P\)的估计值;
  \item 能否有99%的把握认为该地区的老年人是否需要志愿者提供帮助与性别有关;
\item 并指出该调查中更优的抽样方法.

\item 参考公式:\(K^{2} = \frac{n(ad - bc)^{2}}{(a + b)(c + d)(a + c)(b + d)}\),
\item 其中\(n = a + b + c + d\).

\item 参考数据:

\item \begin{center}
\item \begin{tabular}{ccccc}
\item \hline
\item \(P\left( K^{2} \geq k_{0} \right)\) & 0.10 & 0.05 & 0.010 & 0.005 \\
\item \hline
\item \(k_{0}\) & 2.706 & 3.841 & 6.635 & 7.879 \\
\item \hline
\item \end{tabular
\item \end{center}
\end{enumerate}
\topics{独立性检验解决实际问题;计算古典概型问题的概率}
\difficulty{0.85}
\answer{(1)\(\frac{1}{5}\)
(2)能,详见解析}
\explain{(1)抽取的样本中,男性老年人共有200人,需要志愿者提供帮助的有40人,\par
频率为\(\frac{40}{200} = \frac{1}{5}\),
所以\(P\)的估计值是\(\frac{1}{5}\).\par
(2)列联表如下:\par
\begin{center}
\begin{tabular}{cccc}
\hline
男性 & 女性 & 合计 &  \\
\hline
需要 & 40 & 20 & 60 \\
不需要 & 160 & 280 & 440 \\
合计 & 200 & 300 & 500 \\
\hline
\end{tabular}
\end{center}\par
\(K^{2} = \frac{500(40 \times 280 - 20 \times 160)^{2}}{60 \times 440 \times 200 \times 300} \approx 20.202 > 6.635\),
所以有99%的把握认为该地区的老年人是否需要志愿者提供帮助与性别有关.\par
由于该地区老年人是否需要帮助与性别有关,
并且从样本数据能看出男性老年人需要帮助的需求较高,与女性老年人有明显差异,
因此调查时先确定男女老年人的比例,然后按照男、女两层进行分层抽样.}
\end{question}

\begin{question}
有10只不同的试验产品,其中有4只不合格品、6只合格品.现每次取1只测试,
直到4只不合格品全部测出为止.
\begin{enumerate}[label=(\arabic*)]
  \item 求最后1只不合格品正好在第5次测试时被发现的不同情形种数;
  \item 已知最后1只不合格品正好在第5次测试时被发现,求第2次测得合格品的概率.
\end{enumerate}
\topics{元素(位置)有限制的排列问题;代数中的组合计数问题;计算古典概型问题的概率;计算条件概率}
\difficulty{0.65}
\answer{(1)576;
(2)\(\frac{1}{4}\).}
\explain{(1)由最后1只不合格品正好在第5次测试时被发现,得第5次测得不合格品,\par
且前4次测得3只不合格品、1只合格品,\par
所以不同情形种数为\(\mathbb{C}_{4}^{1}\mathbb{C}_{6}^{1}\text{A}_{4}^{4} = 576\).\par
(2)记事件\(A\):最后1只不合格品正好在第5次测试时被发现,
事件\(B\):第2次测得合格品,\par
\(P\text{(}A\text{)} = \frac{576}{10 \times 9 \times 8 \times 7 \times 6} = \frac{2}{105}\),
\(P\text{(}AB\text{)} = \frac{\mathbb{C}_{6}^{1}\text{A}_{4}^{4}}{10 \times 9 \times 8 \times 7 \times 6} = \frac{1}{210}\),\par
因此\(P\text{(}B\text{|}A\text{)} = \frac{P(AB)}{P(A)} = \frac{1}{4}\),\par
所以最后1只不合格品正好在第5次测试时被发现,
第2次测得合格品的概率为\(\frac{1}{4}\).}
\end{question}

\begin{question}
在\(\bigtriangleup ABC\)中,
\(a,b,c\)分别是角\(A,B,C\)所对的边,
点\(D\)在边\(BC\)上,
已知\(\sin A = \cos\frac{B + C}{2}\),
\(c = 6, \bigtriangleup ABC\)的面积为\(30\sqrt{3}\).
\begin{enumerate}[label=(\arabic*)]
  \item 求\(A,a\);
  \item 若\(2CD = 3BD\),求\(\angle BAD\)的正切值.
\end{enumerate}
\topics{三角恒等变换的化简问题;正弦定理解三角形;三角形面积公式及其应用;余弦定理解三角形}
\difficulty{0.65}
\answer{(1)\(\frac{2\pi}{3},2\sqrt{139}\)
(2)\(- 10\sqrt{3}\)}
\explain{(1)因为\(\sinA = \cos\frac{B + C}{2}\),
所以\(2\sin\frac{A}{2}\cos\frac{A}{2} = \cos\frac{\pi - A}{2} = \sin\frac{A}{2}\),\par
因为\(0 < \frac{A}{2} < \frac{\pi}{2}\),
则\(\sin\frac{A}{2} \neq 0\),
故由\(\cos\frac{A}{2} = \frac{1}{2}\),
可得\(A = \frac{2\pi}{3}\).\par
因为\(S_{\bigtriangleup ABC} = \frac{1}{2}bc\sin A = 30\sqrt{3}\),
\(c = 6\),解得\(b = 20\),\par
由余弦定理得\(a^{2} = b^{2} + c^{2} - 2bc\cosA = 400 + 36 - 2 \times 20 \times 6 \times \left( - \frac{1}{2} \right) = 556\),\par
解得\(a = 2\sqrt{139}\).\par
(2)设\(\angle BAD = \theta\),
所以\(\angle CAD = \frac{2\pi}{3} - \theta,0 < \theta < \frac{2\pi}{3}\).\par
在\(\bigtriangleup ABD\)中,由正弦定理得,
\(\frac{AB}{\sin\angle ADB} = \frac{BD}{\sin\theta}\),
即\(BD\sin\angle ADB = 6\sin\theta\),\par
在\(\bigtriangleup ACD\)中,由正弦定理得,
\(\frac{AC}{\sin\angle ADC} = \frac{CD}{\sin(\frac{2\pi}{3} - \theta)}\),
即\(CD\sin\angle ADC = 20\sin(\frac{2\pi}{3} - \theta)\),\par
因\(2CD = 3BD,\sin\angle ADB = \sin\angle ADC\),
代入化简得\(20\sin(\frac{2\pi}{3} - \theta) = 9\sin\theta\),\par
即\(10\sqrt{3}\cos\theta + 10\sin\theta = 9\sin\theta\),
解得\(\tan\theta = - 10\sqrt{3}\),
即\(\tan\angle BAD = - 10\sqrt{3}\).}
\end{question}

\begin{question}
如图,在直三棱柱\(ABC - A_{1}B_{1}C_{1}\)中,
\(\angle BAC = 90^{\circ},AB = AC = AA_{1} = 2\),
两点\(M,N\)分别在直线\(BC,AC_{1}\)上,
\(MN\bot BC,MN\bot AC_{1}\).
\begin{enumerate}[label=(\arabic*)]
  \item 证明:\(MN\bot\)平面\(AB_{1}C_{1}\);
  \item 求线段\(MN\)的长度;
  \item 求二面角\(M - AB_{1} - N\)的余弦值.
\end{enumerate}

\begin{center}
% IMAGE_TODO_START id=jiangsu-changzhou-2025-2026-midterm-Q18-img1 path=/Users/muryor/code/mynote/word\\_to\\_tex/output/figures/jiangsu-changzhou-2025-2026-midterm/media/image7.png width=60% inline=false question_index=18 sub_index=1
% CONTEXT_BEFORE: 分别在直线$$BC,AC_{1}$$上,$$MN\bot BC,MN\bot AC_{1}$$.
% CONTEXT_AFTER: (1)证明:$$MN\bot$$平面$$AB_{1}C_{1}$$; (2)求线段$$MN$$
\begin{tikzpicture}[scale=1.05,>=Stealth,line cap=round,line join=round]
  % TODO: AI_AGENT_REPLACE_ME (id=jiangsu-changzhou-2025-2026-midterm-Q18-img1)
\end{tikzpicture}
% IMAGE_TODO_END id=jiangsu-changzhou-2025-2026-midterm-Q18-img
1
\end{center}


\begin{center}
% IMAGE_TODO_START id=jiangsu-changzhou-2025-2026-midterm-Q18-img2 path=/Users/muryor/code/mynote/word\\_to\\_tex/output/figures/jiangsu-changzhou-2025-2026-midterm/media/image8.png width=60% inline=false question_index=18 sub_index=1
\begin{tikzpicture}[scale=1.05,>=Stealth,line cap=round,line join=round]
  % TODO: AI_AGENT_REPLACE_ME (id=jiangsu-changzhou-2025-2026-midterm-Q18-img2)
\end{tikzpicture}
% IMAGE_TODO_END id=jiangsu-changzhou-2025-2026-midterm-Q18-img
2
\end{center}

\topics{由坐标解决线段平行和长度问题;证明线面垂直;面面角的向量求法}
\difficulty{0.65}
\answer{(1)证明见解析
(2)\(\frac{2}{3}\sqrt{3}\)
(3)\(\frac{\sqrt{3}}{3}\)}
\explain{(1)因\(ABC - A_{1}B_{1}C_{1}\)为直三棱柱,
则\(BC\text{//}B_{1}C_{1}\),\par
因\(MN\bot BC\),则\(MN\bot B_{1}C_{1}\),\par
又\(MN\bot AC_{1}\),
\(AC_{1} \cap B_{1}C_{1} = C_{1}\),
\(AC_{1},B_{1}C_{1} \subset\)平面\(AB_{1}C_{1}\),
则\(MN\bot\)平面\(AB_{1}C_{1}\);\par
(2)因\(ABC - A_{1}B_{1}C_{1}\)为直三棱柱,
且\(\angle BAC = 90^{\circ}\),\par
则以\(A\)为原点,
\(AB,AC,AA_{1}\)所在直线为\(x\)轴、\(y\)轴、\(z\)轴建立如图所示的空间直角坐标系,\par
则\(B(2,0,0),C(0,2,0),C_{1}(0,2,2)\),\par
则\(\overrightarrow{BC} = ( - 2,2,0),\overrightarrow{AC_{1}} = (0,2,2)\),\par
设\(\overrightarrow{BM} = \lambda\overrightarrow{BC} = ( - 2\lambda,2\lambda,0),\overrightarrow{AN} = \mu\overrightarrow{AC_{1}} = (0,2\mu,2\mu)\),\par
则\(M(2 - 2\lambda,2\lambda,0),N(0,2\mu,2\mu)\),
则\(\overrightarrow{MN} = (2\lambda - 2,2\mu - 2\lambda,2\mu)\),\par
则\(\left\{ \begin{array}{r}
\overrightarrow{MN} \cdot \overrightarrow{BC} = - 4\lambda + 4 + 4\mu - 4\lambda = 0 \\
\overrightarrow{MN} \cdot \overrightarrow{AC_{1}} = 4\mu - 4\lambda + 4\mu = 0
\end{array} \right.\),解得\(\mu = \frac{1}{3},\lambda = \frac{2}{3}\),\par
则\(M\left( \frac{2}{3},\frac{4}{3},0 \right),N\left( 0,\frac{2}{3},\frac{2}{3} \right)\),故\(|MN| = \sqrt{\left( \frac{2}{3} \right)^{2} + \left( \frac{2}{3} \right)^{2} + \left( \frac{2}{3} \right)^{2}} = \frac{2}{3}\sqrt{3}\);\par
(3)因\(B_{1}(2,0,2)\),则\(\overrightarrow{AB_{1}} = (2,0,2)\),\(\overrightarrow{AM} = \left( \frac{2}{3},\frac{4}{3},0 \right),\overrightarrow{AN} = \left( 0,\frac{2}{3},\frac{2}{3} \right)\),\par
设平面\(MAB_{1}\)和平面\(AB_{1}N\)的法向量分别为\(\overrightarrow{m} = \left( x_{1},y_{1},z_{1} \right),\overrightarrow{n} = \left( x_{2},y_{2},z_{2} \right)\),\par
则\(\left\{ \begin{array}{r}
\overrightarrow{AB_{1}} \cdot \overrightarrow{m} = 2x_{1} + 2z_{1} = 0 \\
\overrightarrow{AM} \cdot \overrightarrow{m} = \frac{2}{3}x_{1} + \frac{4}{3}y_{1} = 0
\end{array} \right.\),\(\left\{ \begin{array}{r}
\overrightarrow{AB_{1}} \cdot \overrightarrow{n} = 2x_{2} + 2z_{2} = 0 \\
\overrightarrow{AN} \cdot \overrightarrow{n} = \frac{2}{3}y_{2} + \frac{2}{3}z_{2} = 0
\end{array} \right.\),\par
不妨令\(y_{1} = 1,x_{2} = 1\),则\(\overrightarrow{m} = ( - 2,1,2),\overrightarrow{n} = (1,1, - 1)\),\par
则\(\cos\left\langle \overrightarrow{m},\overrightarrow{n} \right\rangle = \frac{\overrightarrow{m} \cdot \overrightarrow{n}}{\left| \overrightarrow{m} \right| \cdot \left| \overrightarrow{n} \right|} = \frac{- 2 + 1 - 2}{\sqrt{9} \times \sqrt{3}} = - \frac{\sqrt{3}}{3}\),\par
有图可知二面角\(M - AB_{1} - N\)为锐二面角,故二面角\(M - AB_{1} - N\)的余弦值为\(\frac{\sqrt{3}}{3}\)}
\end{question}

\begin{question}
已知函数\(f(x) = \ln(x + 1) + a\cos x,a \in R\).
\begin{enumerate}[label=(\arabic*)]
  \item 当\(a = 1\),
\item 求\(f(x)\)在点\(\left( 0,f
  \item \right)\)处的切线方程;
  \item 当\(a = 1\),求函数\(f(x)\)的零点个数;
  \item \(\forall x \geq 0,f(x) \geq 0\),
\item 求整数\(a\)的值.
\end{enumerate}
\topics{求在曲线上一点处的切线方程(斜率);利用导数研究不等式恒成立问题;利用导数研究函数的零点}
\difficulty{0.4}
\answer{(1)\(y = x + 1\)
(2)\(1\)个
(3)\(0\)或\(1\)}
\explain{(1)当\(a = 1\)时,
\(f(x) = \ln(x + 1) + \cos x\),
则\(f'(x) = \frac{1}{x + 1} - \sin x\),\par
有\(f'(0) = 1 - \sin 0 = 1\),
又\(f(0) = \ln(0 + 1) + \cos 0 = 1\),\par
故\(f(x)\)在点\(\left( 0,f(0) \right)\)处的切线方程为\(y = x + 1\);\par
(2)\(f(x) = \ln(x + 1) + \cos x\),
定义域为\(( - 1, + \infty)\),
\(f'(x) = \frac{1}{x + 1} - \sin x\),\par
①\(x \in ( - 1,0\rbrack\),
因为\(\frac{1}{x + 1} \geq 1,\sin x \in \lbrack - 1,1\rbrack,f'(x) \geq 0\),
所以\(f(x)\)在\(( - 1,0\rbrack\)上单调递增.\par
又因为\(f(0) = 1 > 0,f\left( \frac{1}{\text{e}} - 1 \right) = \ln\left( \frac{1}{\text{e}} \right) + \cos\left( \frac{1}{\text{e}} - 1 \right) < \ln\left( \frac{1}{\text{e}} \right) + 1 = 0\),\par
由零点存在性定理可得,
存在\(x_{1} \in \left( \frac{1}{\text{e}} - 1,0 \right)\)使得\(f\left( x_{1} \right) = 0\).\par
②当\(x \in \left( 0,\frac{\pi}{2} \right\rbrack,\cos x \geq 0\)且\(\ln(x + 1) > 0\),\par
因此\(f(x) > 0\)恒成立,即\(f(x)\)在该区间内无零点.\par
③当\(x \in \left\lbrack \frac{\pi}{2},\frac{3\pi}{4} \right\rbrack\)时,
\(f'(x) = \frac{1}{x + 1} - \sin x \leq \frac{1}{\frac{\pi}{2} + 1} - \sin x\).\par
因为\(\sin x \geq \frac{\sqrt{2}}{2},\frac{1}{\frac{\pi}{2} + 1} = \frac{2}{\pi + 2} < \frac{1}{2} < \frac{\sqrt{2}}{2},f'(x) \leq \frac{2}{\pi + 2} - \frac{\sqrt{2}}{2} < 0\),\par
所以\(f(x)\)在区间\(\left\lbrack \frac{\pi}{2},\frac{3\pi}{4} \right\rbrack\)上单调递减.\par
\(f{(x)}_{\min} = f\left( \frac{3\pi}{4} \right) = \ln\left( \frac{3\pi}{4} + 1 \right) - \frac{\sqrt{2}}{2}\).\par
因为\(\frac{3\pi + 4}{4} > 3 > e,\ln\left( \frac{3\pi + 4}{4} \right) > \ln e = 1 > \frac{\sqrt{2}}{2}\),
所以\(f\left( \frac{3\pi}{4} \right) > 0\),
在\(\left\lbrack \frac{\pi}{2},\frac{3\pi}{4} \right\rbrack\)上\(f(x) > 0\),
故该区间内无零点.\par
④当\(x \in \left\lbrack \frac{3\pi}{4}, + \infty \right)\)时,
\(f(x) = \ln(x + 1) + \cos x \geq \ln(x + 1) - 1 \geq \ln\left( \frac{3\pi}{4} + 1 \right) - 1 > 0\),\par
因此该区间内也无零点.\par
综上,
函数\(f(x)\)在\(( - 1, + \infty)\)上仅有一个零点.\par
(3)函数\(f(x) = \ln(x + 1) + a\cos x(a \in Z)\),
由题意\(f(0) = a \geq 0\).\par
当\(a = 0\)时,
\(f(x) = \ln(x + 1) \geq 0\)对于任意\(x \geq 0\)恒成立,\par
当\(a = 1\)时,
由(2)知\(f(x) \geq 0\)在\(\lbrack 0, + \infty)\)上恒成立.\par
当\(a \geq 2\)且\(a \in Z\)时,取\(x = \pi\),
有\(f(\pi) = \ln(\pi + 1) + a\cos\pi = \ln(\pi + 1) - a\),\par
由于\(\ln(\pi + 1) < 2\)且\(a \geq 2\),
故\(f(\pi) < 0\),与\(f(x) \geq 0\)矛盾.\par
综上所述,\(a\)的值为0或1.}
\end{question}
