\examxtitle{测试试卷 - js-suxichang-2025-q2}

\section{单选题}

\begin{question}
已知集合\(A = \left\{ 1,2,3,4,5 \right\}\),
\(B = \left\{ x\left| 0 < \log_{2}x < 2 \right.\  \right\}\),
则\(A \cap B =\)(    )
\begin{choices}
  \item \(\left\{ 2,3 \right\}\)
  \item \(\left\{ 2,3,4 \right\}\)
  \item \(\left\{ 1,2,3 \right\}\)
  \item \(\left\{ 1,2,3,4 \right\}\)
\end{choices}
\topics{交集的概念及运算;由对数函数的单调性解不等式}
\difficulty{0.85}
\answer{A}
\explain{\(\log_{2}x > 0 = \log_{2}1\),故\(x > 1\),
\(\log_{2}x < 2 = \log_{2}4\),故\(x < 4\),\par
所以\(B = \left\{ x\mid 1 < x < 4 \  \right\}\),\par
又\(A = \left\{ 1,2,3,4,5 \right\}\),
故\(A \cap B = \left\{ 2,3 \right\}\)}
\end{question}

\begin{question}
已知\(z + 2\text{i} = \frac{\text{i} - 1}{\text{i} + 1}\),则\(z =\)(    )
\begin{choices}
  \item \(- 2\text{i}\)
  \item \(- \text{i}\)
  \item \(\text{i}\)
  \item \(2\text{i}\)
\end{choices}
\topics{复数加减法的代数运算;复数的除法运算}
\difficulty{0.85}
\answer{B}
\explain{\(z = \frac{\text{i} - 1}{\text{i} + 1} - 2\text{i} = \frac{\left( \text{i} - 1 \right)\left( 1 - \text{i} \right)}{\left( 1 + \text{i} \right)\left( 1 - \text{i} \right)} - 2\text{i} = \frac{\text{i} - \text{i}^{2} - 1 + \text{i}}{2} - 2\text{i} = \text{i} - 2\text{i} = - \text{i}\)}
\end{question}

\begin{question}
诗歌朗诵比赛共有八位评委分别给出某选手的原始评分,评定该选手的成绩时,
从8个原始评分中去掉1个最高分和1个最低分,得到6个有效评分,
6个有效评分与8个原始评分相比,一定不变的数字特征是(    )
\begin{choices}
  \item 极差
  \item 平均数
  \item 中位数
  \item 标准差
\end{choices}
\topics{计算几个数的中位数;计算几个数的平均数;计算几个数据的极差;方差;标准差}
\difficulty{0.85}
\answer{C}
\explain{根据题意,将8个数据从小到大排列,从8个原始评分中去掉1个最高分、1个最低分,\par
得到6个有效评分,\par
6个有效评分与8个原始评分相比,最中间的两个分数不变,\par
而最高分、最低分、平均分、标准差都有可能发生变化,\par
因此一定不变的数字特征是中位数}
\end{question}

\begin{question}
已知圆\(C:x^{2} + (y - 2)^{2} = \frac{10}{3}\),
将直线\(l_{1}:\sqrt{3}x - y = 0\)绕原点按顺时针方向旋转\(30{^\circ}\)后得到直线\(l_{2}\),
则(    )
\begin{choices}
  \item 直线\(l_{2}\)过圆心\(C\)
  \item 直线\(l_{2}\)与圆\(C\)相交,但不过圆心
  \item 直线\(l_{2}\)与圆\(C\)相切
  \item 直线\(l_{2}\)与圆\(C\)无公共点
\end{choices}
\topics{直线的倾斜角;判断直线与圆的位置关系}
\difficulty{0.85}
\answer{B}
\explain{直线\(l_{1}:\sqrt{3}x - y = 0\)即\(y = \sqrt{3}x\),
斜率为\(\sqrt{3}\),倾斜角为\(60{^\circ}\),\par
将直线\(l_{1}\)绕原点顺时针方向旋转\(30{^\circ}\)得到直线\(l_{2}\),
则直线\(l_{2}\)的倾斜角为\(30{^\circ}\),\par
所以直线\(l_{2}\)的方程为\(y = \frac{\sqrt{3}}{3}x\),
即\(x - \sqrt{3}y = 0\),\par
圆\(C:x^{2} + (y - 2)^{2} = \frac{10}{3}\)的圆心坐标为\(C(0,2)\),
半径\(r = \frac{\sqrt{10}}{\sqrt{3}}\),\par
圆心到直线\(l_{2}\)的距离\(d = \frac{2\sqrt{3}}{2} = \sqrt{3} < \frac{\sqrt{10}}{\sqrt{3}}\),\par
\(\therefore\)直线\(l_{2}\)与圆\(C\)相交但不过圆心}
\end{question}

\begin{question}
已知\(\sin\alpha + \cos\alpha = \frac{1}{5}\left( 0 < \alpha < \pi \right)\),
则\(\tan2\alpha =\)(    )
\begin{choices}
  \item \(- \frac{24}{7}\)
  \item \(\frac{24}{7}\)
  \item \(- \frac{24}{25}\)
  \item \(\frac{24}{25}\)
\end{choices}
\topics{三角函数的化简;求值------同角三角函数基本关系;二倍角的正切公式}
\difficulty{0.85}
\answer{B}
\explain{由\(\sin\alpha + \cos\alpha = \frac{1}{5}\)与\(\sin^{2}\alpha + \cos^{2}\alpha = 1\)联立,
结合\(\alpha \in \left( 0,\pi \right)\)可解得:\par
\(\sin\alpha = \frac{4}{5}\),
\(\cos\alpha = - \frac{3}{5}\),
\(\tan\alpha = - \frac{4}{3}\),\par
再由二倍角公式可得\(\tan2\alpha = \frac{2\tan\alpha}{1 - \tan^{2}\alpha} = \frac{- \frac{8}{3}}{1 - \frac{16}{9}} = \frac{24}{7}\)}
\end{question}

\begin{question}
已知等比数列\(\left\{ a_{n} \right\}\)的公比\(q \neq - 1\),
前\(n\)项和为\(S_{n}\),
则对于\(\forall n \in N^{\ast}\),下列结论一定正确的是(    )
\begin{choices}
  \item \(S_{n} + S_{3n} = 2S_{2n}\)
  \item \(3S_{n} + S_{3n} = 2S_{2n}\)
  \item \(S_{2n}^{2} = S_{n}S_{3n}\)
  \item \(S_{2n}\left( S_{2n} - S_{n} \right) = S_{n}\left( S_{3n} - S_{n} \right)\)
\end{choices}
\topics{求等比数列前n项和}
\difficulty{0.85}
\answer{D}
\explain{令\(a_{n} = 2^{n}\),\(S_{1} = 2\),
\(S_{2} = 6\),\(S_{3} = 14\),
\(S_{1} + S_{3} \neq 2S_{2}\),A错;\par
\(3S_{1} + S_{3} \neq 2S_{2}\),B错;\par
\(S_{2}^{2} \neq S_{1}S_{3}\),C错;\par
一般情况,\(q = 1\)时,\(S_{n} = na_{1}\),
\(S_{2n} - S_{n} = na_{1}\),
\(S_{2n}\left( S_{2n} - S_{n} \right) = 2n^{2}a_{1}^{2}\),\par
\(S_{n}\left( S_{3n} - S_{n} \right) = na_{1} \cdot 2na_{1} = 2n^{2}a_{1}^{2}\),
此时\(S_{2n}\left( S_{2n} - S_{n} \right) = S_{n}\left( S_{3n} - S_{n} \right)\);\par
\(q \neq 1\)时,
\(S_{n} = \frac{a_{1}\left( 1 - q^{n} \right)}{1 - q}\),\par
左边\(= \frac{a_{1}\left( 1 - q^{2n} \right)}{1 - q}\left( \frac{a_{1}\left( 1 - q^{2n} \right)}{1 - q} - \frac{a_{1}\left( 1 - q^{n} \right)}{1 - q} \right) = \frac{a_{1}^{2}\left( 1 - q^{2n} \right)\left( q^{n} - q^{2n} \right)}{(1 - q)^{2}} = \frac{a_{1}^{2}\left( q^{n} - q^{2n} - q^{3n} + q^{4n} \right)}{(1 - q)^{2}}\),\par
右边\(= \frac{a_{1}\left( 1 - q^{n} \right)}{1 - q}\left( \frac{a_{1}\left( 1 - q^{3n} \right)}{1 - q} - \frac{a_{1}\left( 1 - q^{n} \right)}{1 - q} \right) = \frac{a_{1}^{2}\left( 1 - q^{n} \right)\left( q^{n} - q^{3n} \right)}{(1 - q)^{2}} = \frac{a_{1}^{2}\left( q^{n} - q^{2n} - q^{3n} + q^{4n} \right)}{(1 - q)^{2}} =\)左边,
D对;}
\end{question}

\begin{question}
已知函数\(f(x)\)和\(g(x)\)的定义域均为\(R\).若\(f(x + 1)\)是奇函数,
\(g(x)\)是偶函数,
且\(f(x) - g(x - 2) = 2 - x\),
则\(f\left( g( - 1) \right) =\)(    )
\begin{choices}
  \item \(- 1\)
  \item \(0\)
  \item \(1\)
  \item \(2\)
\end{choices}
\topics{求函数值;函数奇偶性的应用}
\difficulty{0.65}
\answer{D}
\explain{因为\(f(x + 1)\)是奇函数,
则\(f( - x + 1) = - f(x + 1)\),\par
令\(x = 0\),可得\(f(1) = - f(1)\),
可得\(f(1) = 0\),\par
在\(f(x) - g(x - 2) = 2 - x\)中令\(x = 1\)得\(f(1) - g( - 1) = 1\),
所以\(g( - 1) = - 1\),\par
在\(f(x) - g(x - 2) = 2 - x\)中令\(x = 3\)得\(f(3) - g(1) = - 1\),\par
所以\(f(3) = g(1) - 1 = g( - 1) - 1 = - 2\),\par
所以\(f\left( g( - 1) \right) = f( - 1) = f( - 2 + 1) = - f(2 + 1) = - f(3) = 2\)}
\end{question}

\begin{question}
一个底面边长和侧棱长均为4的正三棱柱密闭容器\(ABC - A_{1}B_{1}C_{1}\),
其中盛有一定体积的水,当底面\(ABC\)水平放置时,
水面高为\(\frac{15}{4}\).当侧面\(AA_{1}B_{1}B\)水平放置时(如图),
容器内的水形成新的几何体.若该几何体的所有顶点均在同一个球面上,
则该球的表面积为(    )
\begin{choices}
  \item \(\frac{100}{3}\pi\)
  \item \(\frac{200}{3}\pi\)
  \item \(100\pi\)
  \item \(\frac{400}{3}\pi\)
\end{choices}

\begin{center}
% IMAGE_TODO_START id=js-suxichang-2025-q2-Q8-img1 path=/Users/muryor/code/mynote/word\\_to\\_tex/output/figures/js-suxichang-2025-q2/media/image2.png width=60% inline=false question_index=8 sub_index=1
% CONTEXT_BEFORE: ),容器内的水形成新的几何体.若该几何体的所有顶点均在同一个球面上,则该球的表面积为( )
% CONTEXT_AFTER: > A.$${3}$$ B.$${3}\
\begin{tikzpicture}[scale=1.05,>=Stealth,line cap=round,line join=round]
  % TODO: AI_AGENT_REPLACE_ME (id=js-suxichang-2025-q2-Q8-img1)
\end{tikzpicture}
% IMAGE_TODO_END id=js-suxichang-2025-q2-Q8-img
1
\end{center}


\begin{center}
% IMAGE_TODO_START id=js-suxichang-2025-q2-Q8-img3 path=/Users/muryor/code/mynote/word\\_to\\_tex/output/figures/js-suxichang-2025-q2/media/image4.png width=60% inline=false question_index=8 sub_index=1
% CONTEXT_BEFORE: $${16}$$,则边长之比为$$1:4$$,即"小三角形"边长为1.然后如图:
% CONTEXT_AFTER: 设圆的半径为$$r$$,由余弦定理可得$$BD = + AB^{2}
\begin{tikzpicture}[scale=1.05,>=Stealth,line cap=round,line join=round]
  % TODO: AI_AGENT_REPLACE_ME (id=js-suxichang-2025-q2-Q8-img3)
\end{tikzpicture}
% IMAGE_TODO_END id=js-suxichang-2025-q2-Q8-img
3
\end{center}

\topics{柱体体积的有关计算;球的表面积的有关计算;多面体与球体内切外接问题}
\difficulty{0.65}
\answer{A}
\explain{方法一:\par
\(V_{水} = \frac{1}{2} \times 4 \times 4 \times \frac{\sqrt{3}}{2} \times \frac{15}{4} = 15\sqrt{3}\),
\(V_{ABC - A_{1}B_{1}C_{1}} = \frac{1}{2} \times 4 \times 4 \times \frac{\sqrt{3}}{2} \times 4 = 16\sqrt{3}\)\par
如图\(V_{C_{1}D_{1}E_{1} - CDE} = 16\sqrt{3} - 15\sqrt{3} = \sqrt{3} = S_{\bigtriangleup C_{1}D_{1}E_{1}} \times 4\),
\(\therefore S_{\bigtriangleup C_{1}D_{1}E_{1}} = \frac{\sqrt{3}}{4}\),\par
而\(S_{\bigtriangleup A_{1}B_{1}C_{1}} = 4\sqrt{3}\),\par
%
% IMAGE_TODO_START id=js-suxichang-2025-q2-Q8-img2 path=/Users/muryor/code/mynote/word\\_to\\_tex/output/figures/js-suxichang-2025-q2/media/image3.png width=60% inline=true question_index=8 sub_index=1
% CONTEXT_BEFORE: _{\bigtriangleup A_{1}B_{1}C_{1}} = 4$$,
\begin{tikzpicture}[scale=0.8,baseline=-0.5ex]
  % TODO: AI_AGENT_REPLACE_ME (id=js-suxichang-2025-q2-Q8-img2)
\end{tikzpicture}
% IMAGE_TODO_END id=js-suxichang-2025-q2-Q8-img
2
\(\therefore\frac{S_{\bigtriangleup C_{1}D_{1}E_{1}}}{S_{\bigtriangleup A_{1}B_{1}C_{1}}} = \frac{1}{16}\),
\(\therefore\frac{C_{1}D_{1}}{C_{1}B_{1}} = \frac{C_{1}E_{1}}{C_{1}A_{1}} = \frac{1}{4}\),
即\(C_{1}D_{1} = C_{1}E_{1} = D_{1}E_{1} = 1\),\par
由于\(C_{1}\)到\(A_{1}B_{1}\)距离\(2\sqrt{3}\),
则\(D_{1}\)到\(A_{1}B_{1}\)距离\(\frac{3}{4} \times 2\sqrt{3} = \frac{3\sqrt{3}}{2}\),\par
设正方形\(ABB_{1}A_{1}\)外接圆圆心\(O_{1}\),
则\(r_{1} = \frac{\sqrt{2}}{2}AB = 2\sqrt{2}\)\par
设矩形\(DEE_{1}D_{1}\)外接圆圆心\(O_{2}\),
则\(r_{2} = \frac{1}{2}D_{1}E = \frac{1}{2} \times \sqrt{4^{2} + 1^{2}} = \frac{\sqrt{17}}{2}\),
设外接球半径\(R\left\{ \begin{array}{r}
O_{1}O^{2} + 8 = R^{2} \\
\left( O_{1}O + \frac{3\sqrt{3}}{2} \right)^{2} + \frac{17}{4} = R^{2}
\end{array} \right.\),\(\therefore R^{2} = \frac{25}{3}\),故外接球表面积为\(4\piR^{2} = \frac{100\pi}{3}\),\par
故选;A.\par
方法二:由当底面\(ABC\)水平放置时,水面高为\(\frac{15}{4}\)可知容器内的空气占容器体积的\(\frac{1}{16}\),于是侧放时,图中的空气区域的"小三棱柱"的体积为容器的\(\frac{1}{16}\),因此"小三棱柱"的底面"小三角形"的面积为大三角形的\(\frac{1}{16}\),则边长之比为\(1:4\),即"小三角形"边长为1.然后如图:\par
设圆的半径为\(r\),由余弦定理可得\(BD = \sqrt{AD^{2} + AB^{2} - 2AD \cdot AB\cos 60^{\circ}} = \sqrt{9 + 16 - 2 \times 3 \times 4 \times \frac{1}{2}} = \sqrt{13}\),\par
故\(2r = \frac{BD}{\sin 60^{\circ}} = \frac{\sqrt{13}}{\frac{\sqrt{3}}{2}}\),故\(r = \sqrt{\frac{13}{3}}\),\par
所以外接球的半径为\(R = \sqrt{2^{2} + \frac{13}{3}} = \frac{5}{\sqrt{3}}\),所以球的表面积为\(S = 4\piR^{2} = \frac{100}{3}\pi\)}
\end{question}

\section{多选题}

\begin{question}
\((1 - 2x)^{5}\)的展开式中,则(    )
\begin{choices}
  \item \(x\)的系数为\(- 10\)
  \item 第3项与第4项的二项式系数相等
  \item 所有项的二项式系数和为32
  \item 所有项的系数和为32
\end{choices}
\topics{求指定项的二项式系数;二项式的系数和;求指定项的系数;二项展开式各项的系数和}
\difficulty{0.85}
\answer{ABC}
\explain{A选项,
\((1 - 2x)^{5}\)展开式第\(r + 1\)项\(T_{r + 1} = \mathbb{C}_{5}^{r}( - 2x)^{r} = \mathbb{C}_{5}^{r}( - 2)^{r}x^{r}\),\par
\(r = 1\)时,
\(\mathbb{C}_{5}^{1}( - 2)^{1} = - 10\),A对;\par
B选项,第3项二项式系数为\(\mathbb{C}_{5}^{2} = 10\),
第4项的二项式系数为\(\mathbb{C}_{5}^{3} = 10\),
两者相同,B对.\par
C选项,所有项的二项式系数和为\(2^{5} = 32\),C对.\par
D选项,\(x = 1\)时,\((1 - 2)^{5} = - 1\),
即所有项的系数和为\(- 1\),D错}
\end{question}

\begin{question}
已知函数\(f(x) = \sinx + \cosx + \left| \sinx - \cosx \right|\),则(    )
\begin{choices}
  \(f(x)\)的图象关于点\(\left( \pi,0 \right)\)对称

\begin{enumerate}[label=(\arabic*)]
  \item \(f(x)\)的最小正周期为\(2\pi\)
  \item \(f(x)\)的最小值为\(- 2\)
  \item \(f(x) = \sqrt{3}\)在\(\left\lbrack 0,2\pi \right\rbrack\)上有四个不同的实数解
\item \end{choices}

\item \begin{center}
% IMAGE_TODO_START id=js-suxichang-2025-q2-Q10-img1 path=/Users/muryor/code/mynote/word\\_to\\_tex/output/figures/js-suxichang-2025-q2/media/image5.png width=60% inline=false question_index=10 sub_index=1
% CONTEXT_BEFORE: left\lbrack 0,2 \rbrack$$有4个根,D正确.
% CONTEXT_AFTER: 方法二:由$$f(x) = x + x + |
\item \begin{tikzpicture}[scale=1.05,>=Stealth,line cap=round,line join=round]
  % TODO: AI_AGENT_REPLACE_ME (id=js-suxichang-2025-q2-Q10-img1)
\item \end{tikzpicture}
% IMAGE_TODO_END id=js-suxichang-2025-q2-Q10-img
\item 1
\item \end{center}


\item \begin{center}
% IMAGE_TODO_START id=js-suxichang-2025-q2-Q10-img2 path=/Users/muryor/code/mynote/word\\_to\\_tex/output/figures/js-suxichang-2025-q2/media/image6.png width=60% inline=false question_index=10 sub_index=1
% CONTEXT_BEFORE: t{sin}x$$和$$h(x) = 2x$$的图像,取位于上方的部分即可:
% CONTEXT_AFTER: 由图可知,AC错误,B正确, 对于D,计算知$$2x$$与$$2\text
\item \begin{tikzpicture}[scale=1.05,>=Stealth,line cap=round,line join=round]
  % TODO: AI_AGENT_REPLACE_ME (id=js-suxichang-2025-q2-Q10-img2)
\item \end{tikzpicture}
% IMAGE_TODO_END id=js-suxichang-2025-q2-Q10-img
\item 2
\item \end{center}

\item \topics{求正弦(型)函数的最小正周期;三角函数图象的综合应用;辅助角公式}
\item \difficulty{0.4}
\item \answer{BD}
\item \explain{方法一:由\(f(x) = \sinx + \cosx + \left| \sinx - \cosx \right|\),\par
\item 则\(f(0) = 2\),
\item \(f\left( 2\pi \right) = 2\),
\item 则\(f\left( 2\pi \right) + f(0) \neq 0\),\par
\item 所以\(f(x)\)不可能关于\(\left( \pi,0 \right)\)对称,
\item A错误;\par
\item 因为函数\(y_{1} = \sin x + \cos x = \sqrt{2}\sin\left( x + \frac{\pi}{4} \right)\)的最小正周期为\(2\pi\),\par
\item 函数\(y_{2} = \left| \sin x - \cos x \right| = \left| \sqrt{2}\sin\left( x - \frac{\pi}{4} \right) \right|\)的最小正周期为\(\pi\),\par
\item 则\(f(x) = y_{1} + y_{2}\)的最小正周期为\(2\pi\),
\item B正确;\par
\item 当\(0 \leq x \leq \frac{\pi}{4}\)时,
\item \(f(x) = 2\cosx\),
\item 当\(\frac{\pi}{4} < x \leq \frac{5\pi}{4}\)时,
\item \(f(x) = 2\sinx\);\par
\item 当\(\frac{5\pi}{4} < x \leq 2\pi\)时,
\item \(f(x) = 2\cosx\),作出函数\(f(x)\)大致图象,如图,\par
\item 则\(f(x)_{\min} = - \sqrt{2}\),C错误,\par
\item \(f(x) = \sqrt{3}\)在\(\left\lbrack 0,2\pi \right\rbrack\)有4个根,
\item D正确.\par
\item 方法二:由\(f(x) = \sinx + \cosx + \left| \sinx - \cosx \right| = \left\{ \begin{array}{r}
\item 2\sin x,\sin x \geq \cos x \\
\item 2\cos x,\sin x < \cos x
\item \end{array} \right.\  = 2\max\left\{ \sinx,\cosx \right\}\),\par
\item 作出\(g(x) = 2\sinx\)和\(h(x) = 2\cosx\)的图像,取位于上方的部分即可:\par
\item 由图可知,AC错误,B正确,\par
\item 对于D,计算知\(2\sinx\)与\(2\cosx\)在\(\left( 0,\pi \right)\)内的交点坐标为\(\left( \frac{\pi}{4},\sqrt{2} \right)\),\par
\item 而\(\sqrt{2} < \sqrt{3} < 2\),结合函数\(f(x)\)的图象特征可知函数\(f(x)\)与\(y = \sqrt{3}\)图象在\(\left\lbrack 0,2\pi \right\rbrack\)内有四个交点,\par
\item 所以\(f(x) = \sqrt{3}\)在\(\left\lbrack 0,2\pi \right\rbrack\)上有四个不同的实数解,故D正确}
\end{enumerate}
\end{question}

\begin{question}
已知\(P\)为曲线\(E:y^{4} = 4x\)上一个动点(异于原点),
\(E\)在\(P(x,y)(y \neq 0)\)处的切线是指曲线\(y = \pm \sqrt[4]{4x}\)在\(P\)处的切线.直线\(m\)为\(E\)在\(P\)处的切线,
过\(P\)作\(m\)的垂线\(n\),若\(m\),
\(n\)分别与\(x\)轴交于\(A\),\(B\)两点,则(    )
\begin{choices}
  \item \(E\)关于\(x\)轴对称
  \item \(P\)到点\(F(1,0)\)的距离不小于\(P\)到直线\(x = - 1\)的距离
  \item 存在\(P\),使得\(2|PA| = |PB|\)
  \item 当\(|AB|\)取得最小值时,直线\(OP\)的斜率为\(\pm 4\sqrt{2}\)
\end{choices}
\topics{求在曲线上一点处的切线方程(斜率);基本不等式求和的最小值;求平面两点间的距离}
\difficulty{0.65}
\answer{ACD}
\explain{A,
若点\(\left( x_{0},y_{0} \right)\)满足方程\(y^{4} = 4x\),
则点\(\left( x_{0}, - y_{0} \right)\)也满足方程\(y^{4} = 4x\),\par
则\(E\)关于\(x\)轴对称,故A正确;\par
B,设\(P\left( x_{0},y_{0} \right)\),
则\(y_{0}^{4} = 4x_{0}\),\par
则\(P\)到点\(F(1,0)\)的距离\(d_{1} = \sqrt{\left( x_{0} - 1 \right)^{2} + y_{0}^{2}}\),
\(P\)到直线\(x = - 1\)的距离\(d_{2} = x_{0} + 1\),\par
则\(d_{1}^{2} - d_{2}^{2} = \left( x_{0} - 1 \right)^{2} + y_{0}^{2} - \left( x_{0} + 1 \right)^{2} = y_{0}^{2} - 4x_{0} = y_{0}^{2} - y_{0}^{4} = y_{0}^{2}\left( 1 - y_{0}^{2} \right)\),\par
当\(y_{0}^{2} > 1\)时,
\(d_{1}^{2} - d_{2}^{2} < 0\),
即\(d_{1} < d_{2}\),所以B错误;\par
C,设\(P\left( x_{0},y_{0} \right)\),
则\(y_{0}^{4} = 4x_{0}\),\par
因\(y = \pm \sqrt[4]{4x}\),
则\(y' = \pm \frac{1}{4} \times 4 \times (4x)^{- \frac{3}{4}} = \pm (4x)^{- \frac{3}{4}} = \pm \left( \frac{1}{\sqrt[4]{4x}} \right)^{3} = \left( \pm \frac{1}{\sqrt[4]{4x}} \right)^{3} = \frac{1}{y^{3}}\),\par
则曲线\(E\)在点\(P\)处切线斜率为\(\frac{1}{y_{0}^{3}}\),\par
所以直线\(m\)为\(y - y_{0} = \frac{1}{y_{0}^{3}}\left( x - \frac{y_{0}^{4}}{4} \right)\),
直线\(n\)为\(y - y_{0} = - y_{0}^{3}\left( x - \frac{y_{0}^{4}}{4} \right)\),\par
所以\(A\left( - \frac{3}{4}y_{0}^{4},0 \right)\),
\(B\left( \frac{1}{y_{0}^{2}} + \frac{y_{0}^{4}}{4},0 \right)\),\par
可得\(PA^{2} = \left( x_{0} + \frac{3}{4}y_{0}^{4} \right)^{2} + y_{0}^{2} = \left( \frac{y_{0}^{4}}{4} + \frac{3}{4}y_{0}^{4} \right)^{2} + y_{0}^{2} = y_{0}^{8} + y_{0}^{2}\),\par
\(PB^{2} = \left( x_{0} - \frac{1}{y_{0}^{2}} - \frac{y_{0}^{4}}{4} \right)^{2} + y_{0}^{2} = \left( \frac{y_{0}^{4}}{4} - \frac{1}{y_{0}^{2}} - \frac{y_{0}^{4}}{4} \right)^{2} + y_{0}^{2} = \frac{1}{y_{0}^{4}} + y_{0}^{2}\),\par
则\(\frac{|PA|^{2}}{|PB|^{2}} = \frac{y_{0}^{8} + y_{0}^{2}}{\frac{1}{y_{0}^{4}} + y_{0}^{2}} = \frac{\left( y_{0}^{6} + 1 \right)y_{0}^{6}}{y_{0}^{6} + 1} = y_{0}^{6}\)\par
因\(y_{0} \in \mathbb{R}\),故存在\(P\),
使得\(2|PA| = |PB|\)时,故C正确;\par
D,由C选项可知,
\(|AB| = \frac{1}{y_{0}^{2}} + \frac{y_{0}^{4}}{4} - \left( - \frac{3}{4}y_{0}^{4} \right) = \frac{1}{y_{0}^{2}} + y_{0}^{4} = \frac{1}{2y_{0}^{2}} + \frac{1}{2y_{0}^{2}} + y_{0}^{4} \geq 3 \times \sqrt[3]{\frac{1}{2y_{0}^{2}} \cdot \frac{1}{2y_{0}^{2}} \cdot y_{0}^{4}} = \frac{3}{\sqrt[3]{4}}\),\par
等号成立时,
\(\frac{1}{2y_{0}^{2}} = y_{0}^{4}\),
即\(y_{0} = \pm \frac{1}{\sqrt[6]{2}}\),\par
此时\(OP\)的斜率为\(\frac{y_{0}}{x_{0}} = \frac{4}{y_{0}^{3}} = \pm 4\sqrt{2}\),
故D正确}
\end{question}

\section{填空题}

\begin{question}
已知平面向量\(\overrightarrow{a} = (1,2)\),
\(\overrightarrow{b} = (x,3)\),
若\(\overrightarrow{a} \parallel \left( \overrightarrow{a} + 2\overrightarrow{b} \right)\),
则实数\(x\)的值为
.
\topics{平面向量线性运算的坐标表示;由向量共线(平行)求参数}
\difficulty{0.85}
\answer{\(\frac{3}{2}\)}
\explain{\(\overrightarrow{a} + 2\overrightarrow{b} = (1 + 2x,8)\),
\(\overrightarrow{a} \parallel \left( \overrightarrow{a} + 2\overrightarrow{b} \right)\),\par
\(\therefore 8 = 2(1 + 2x)\),
\(\therefore x = \frac{3}{2}\).\(\frac{3}{2}\)}
\end{question}

\begin{question}
在平面直角坐标系\(xOy\)中,双曲线\(C\)的中心在原点,
焦点在\(x\)轴上,
焦距长为\(4\sqrt{3}\).若\(C\)和抛物线\(y^{2} = x\)交于\(A\),
\(B\)两点,且\(\bigtriangleup OAB\)为正三角形,
则\(C\)的离心率为
.

\begin{center}
% IMAGE_TODO_START id=js-suxichang-2025-q2-Q13-img1 path=/Users/muryor/code/mynote/word\\_to\\_tex/output/figures/js-suxichang-2025-q2/media/image7.png width=60% inline=false question_index=13 sub_index=1
\begin{tikzpicture}[scale=1.05,>=Stealth,line cap=round,line join=round]
  % TODO: AI_AGENT_REPLACE_ME (id=js-suxichang-2025-q2-Q13-img1)
\end{tikzpicture}
% IMAGE_TODO_END id=js-suxichang-2025-q2-Q13-img
1
\end{center}

\topics{求双曲线的离心率或离心率的取值范围;求直线与抛物线的交点坐标}
\difficulty{0.85}
\answer{\(\sqrt{2}\)}
\explain{由对称性知\(A\)、\(B\)关于\(x\)轴对称,
\(\bigtriangleup AOB\)为正三角形,\par
则由正三角形对称性可知\(A\)、\(B\)为\(y = \pm \tan30{^\circ}x = \pm \frac{\sqrt{3}}{3}x\)与抛物线的交点,\par
联立\(y = \frac{\sqrt{3}}{3}x\)与\(y^{2} = x\)得\(x = 3\)或0(舍去),
当\(x = 3\)时,\(y = \sqrt{3}\),\par
故其中一个交点为\(\left( 3,\sqrt{3} \right)\),\par
设双曲线方程为\(\frac{x^{2}}{a^{2}} - \frac{y^{2}}{b^{2}} = 1\),
故\(2c = 4\sqrt{3}\),解得\(c = 2\sqrt{3}\),\par
\(\left( 3, \pm \sqrt{3} \right)\)在双曲线上,
\(\left\{ \begin{array}{r}
\frac{9}{a^{2}} - \frac{3}{b^{2}} = 1 \\
a^{2} = 12 - b^{2}
\end{array} \right.\),\(\therefore\left\{ \begin{array}{r}
a^{2} = 6 \\
b^{2} = 6
\end{array} \right.\),\par
故离心率为\(e = \frac{c}{a} = \frac{2\sqrt{3}}{\sqrt{6}} = \sqrt{2}\);\(\sqrt{2}\)}
\end{question}

\begin{question}
已知随机变量\(X\),\(Y\)相互独立,且\(X \sim N(4,4)\),
\(Y \sim B\left( 8,\frac{1}{2} \right)\),
则\(P(X \leq 4,Y \leq 4) =\)
;若\(Z = X + Y\),
则\(\sum_{t = 1}^{15}{P(Z \leq t)} =\) .
\topics{独立事件的乘法公式;独立重复试验的概率问题;指定区间的概率}
\difficulty{0.4}
\answer{\(\frac{163}{512}\) \(\frac{15}{2}\)}
\explain{\(X \sim N(4,4)\),
\(P(X \leq 4) = \frac{1}{2}\),
\(Y \sim B\left( 8,\frac{1}{2} \right)P(Y \leq 4) = P(Y = 0) + P(Y = 1) + P(Y = 2) + P(Y = 3) + P(Y = 4)= \left( \frac{1}{2} \right)^{8} + 8\left( \frac{1}{2} \right)^{8} + 28 \times \left( \frac{1}{2} \right)^{8} + 56\left( \frac{1}{2} \right)^{8} + 70\left( \frac{1}{2} \right)^{8} = 163 \times \left( \frac{1}{2} \right)^{8}P(X \leq 4,Y \leq 4) = \frac{1}{2} \times 163 \times \left( \frac{1}{2} \right)^{8} = \frac{163}{512}\).\par
\(\sum_{t = 1}^{15}{P(Z \leq t)} = \sum_{\text{t} = 1}^{15}{P(X + Y \leq t) = P(Z \leq 1) + P(Z \leq 2) + \cdot \cdot \cdot + P(Z \leq 15)}= P(X \leq 1,Y = 0) + P(X \leq 0,Y = 1) + \cdot \cdot \cdot + P(X \leq - 7,Y = 8) + P(X \leq 2,Y = 0)+ P(X \leq 1,Y = 1) + \cdot \cdot \cdot + P(X \leq - 6,Y = 8) + \cdots + P(X \leq 7,Y = 0) + P(X \leq 6,Y = 1) + \cdot \cdot \cdot+ P(X \leq - 1,Y = 8) + \cdot \cdot \cdot + P(X \leq 15,Y = 0) + P(X \leq 14,Y = 1) + \cdot \cdot \cdot + P(X \leq 7,Y = 8)\)\par
并利用\(P(Y = k) = P(Y = 8 - k)\),
\(P(X \leq k) + P(X \leq 8 - k) = 1\)\par
记原式\(= S\),\par
倒序相加\(\Rightarrow 15\left\lbrack P(Y = 0) + P(Y = 1) + \cdot \cdot \cdot + P(Y = 8) \right\rbrack = 2S \Rightarrow S = \frac{15}{2}\).\(\frac{163}{512};\frac{15}{2}\).}
\end{question}

\section{解答题}

\begin{question}
某种产品可以采用甲、乙两种工艺来生产,
为了研究产品的质量与所采用的生产工艺的关联性,现对该种产品进行随机抽查,
得到的结果如下表所示.

\begin{center}
\begin{tabular}{cccc}
\hline
工艺甲 & 工艺乙 & 合计 &  \\
\hline
合格 & 60 & 40 & 100 \\
不合格 & 20 & 30 & 50 \\
合计 & 80 & 70 & 150 \\
\hline
\end{tabular}
\end{center}
\begin{enumerate}[label=(\arabic*)]
  \item 依据小概率值\(\alpha = 0.05\)的独立性检验,
\item 分析产品的质量是否与采用的工艺有关;
  \item 在不合格的50件样本产品中任选3件,
\item 求在这3件样本产品中至少有1件是采用工艺甲生产的条件下,
\item 这3件样本产品中恰有一件是采用工艺乙生产的概率.
\end{enumerate}
\topics{卡方的计算;独立性检验解决实际问题;计算条件概率}
\difficulty{0.65}
\answer{(1)产品的质量与采用的工艺有关
(2)\(\frac{95}{259}\)}
\explain{(1)零假设:产品的质量与采用的工艺无关,\par
\(\chi^{2} = \frac{150 \times (60 \times 30 - 40 \times 20)^{2}}{100 \times 50 \times 80 \times 70} \approx 5.357 > 3.841\therefore\)根据小概率值\(\alpha = 0.05\)的独立性检验,
产品的质量与采用的工艺有关.\par
(2)记事件\(A\)为3件样本产品中至少有1件是采用工艺甲,
事件\(B\)为这3件样本产品中恰有一件是采用工艺乙.\par
\(\therefore P(B \mid A) = \frac{n(AB)}{n(A)} = \frac{\mathbb{C}_{30}^{1}\mathbb{C}_{20}^{2}}{\mathbb{C}_{50}^{3} - \mathbb{C}_{30}^{3}} = \frac{95}{259}\).}

\vspace{1em}
\textbf{附:}

附:\(\chi^{2} = \frac{n(ad - bc)^{2}}{(a + b)(c + d)(a + c)(b + d)}\)

\begin{center}
\begin{tabular}{cccccc}
\hline
\(\alpha\) & 0.1 & 0.05 & 0.01 & 0.005 & 0.001 \\
\hline
\(x_{\alpha}\) & 2.706 & 3.841 & 6.635 & 7.879 & 10.828 \\
\hline
\end{tabular
\end{center}

\end{question}

\begin{question}
如图,在三棱柱\(ABC - A_{1}B_{1}C_{1}\)中,
\(AB = 2\),\(BC = 2\sqrt{2}\),
\(\angle ABC = 45{^\circ}\),
\(BC_{1}\bot AC\).
\begin{enumerate}[label=(\arabic*)]
  \item 证明:\(AC\bot\)平面\(ABC_{1}\);
  \item 若\(CC_{1} = 2\sqrt{2}\),
\item 二面角\(C_{1} - AC - B\)的大小为\(60{^\circ}\),
\item 求直线\(BC_{1}\)与平面\(AA_{1}B_{1}B\)所成角的正弦值.
\end{enumerate}

\begin{center}
% IMAGE_TODO_START id=js-suxichang-2025-q2-Q16-img1 path=/Users/muryor/code/mynote/word\\_to\\_tex/output/figures/js-suxichang-2025-q2/media/image8.png width=60% inline=false question_index=16 sub_index=1
% CONTEXT_BEFORE: $,$$\angle ABC = 45{^\circ}$$,$$BC_{1}\bot AC$$.
% CONTEXT_AFTER: (1)证明:$$AC\bot$$平面$$ABC_{1}$$; (2)若$$CC_{1} = 2
\begin{tikzpicture}[scale=1.05,>=Stealth,line cap=round,line join=round]
  % TODO: AI_AGENT_REPLACE_ME (id=js-suxichang-2025-q2-Q16-img1)
\end{tikzpicture}
% IMAGE_TODO_END id=js-suxichang-2025-q2-Q16-img
1
\end{center}

\topics{余弦定理解三角形;证明线面垂直;线面角的向量求法;由二面角大小求线段长度或距离}
\difficulty{0.65}
\answer{(1)证明见解析
(2)\(\frac{\sqrt{21}}{7}\)}
\explain{(1)\(\because AB = 2\),
\(BC = 2\sqrt{2}\),\(\angle ABC = 45{^\circ}\),\par
由余弦定理得\(AC^{2} = AB^{2} + BC^{2} - 2AB \cdot BC\cos 45{^\circ} = 4 + 8 - 2 \times 2 \times 2\sqrt{2} \times \frac{\sqrt{2}}{2} = 4\),\par
\(\therefore AC = 2\),\par
\(\therefore AB^{2} + AC^{2} = BC^{2}\),
\(\therefore AC\bot AB\),\par
又\(\because AC\bot BC_{1}\),
\(AB \cap BC_{1} = B\),
\(AB,BC_{1} \subset\)平面\(ABC_{1}\),\par
\(\therefore AC\bot\)平面\(ABC_{1}\);\par
(2)方法一:\(\because AC\bot\)平面\(ABC_{1}\),
\(C_{1}A,AB \subset\)平面\(ABC_{1}\),\par
\(\therefore AC\bot C_{1}A\)且\(AC\bot AB\),\par
\(\therefore\)二面角\(C_{1} - AC - B\)的平面角为\(\angle BAC_{1} = 60{^\circ}\),
而\(CC_{1} = 2\sqrt{2}\),\par
\(\therefore AC_{1} = \sqrt{CC_{1}^{2} - AC^{2}} = \sqrt{8 - 4} = 2 = AB\),
\(\therefore \bigtriangleup ABC_{1}\)为等边三角形,\par
以\(A\)为坐标原点,\(AB,AC\)所在直线分别为\(x,y\)轴,
\(\bigtriangleup ABC_{1}\)所在平面为\(xAz\)平面,
建立如图所示的空间直角坐标系,\par
%
% IMAGE_TODO_START id=js-suxichang-2025-q2-Q16-img2 path=/Users/muryor/code/mynote/word\\_to\\_tex/output/figures/js-suxichang-2025-q2/media/image9.png width=60% inline=true question_index=16 sub_index=1
% CONTEXT_BEFORE: iangleup ABC_{1}$$所在平面为$$xAz$$平面,建立如图所示的空间直角坐标系,
% CONTEXT_AFTER: $$\therefore A(0,0,0)$$,$$B(2,0,0)$$,$$C(0,2,0)$$,
\begin{tikzpicture}[scale=0.8,baseline=-0.5ex]
  % TODO: AI_AGENT_REPLACE_ME (id=js-suxichang-2025-q2-Q16-img2)
\end{tikzpicture}
% IMAGE_TODO_END id=js-suxichang-2025-q2-Q16-img
2
\(\therefore A(0,0,0)\),
\(B(2,0,0)\),\(C(0,2,0)\),
\(C_{1}\left( 1,0,\sqrt{3} \right)\),\par
由\(\overrightarrow{A_{1}C_{1}} = \overrightarrow{AC} \Rightarrow A_{1}\left( 1, - 2,\sqrt{3} \right)\),
\(\therefore\overrightarrow{BC_{1}} = \left( - 1,0,\sqrt{3} \right)\),
\(\overrightarrow{AB} = (2,0,0)\),
\(\overrightarrow{AA_{1}} = \left( 1, - 2,\sqrt{3} \right)\),\par
设平面\(AA_{1}B_{1}B\)的一个法向量\(\overrightarrow{n} = (x,y,z)\),\par
\(\therefore\left\{ \begin{array}{r}
\overrightarrow{AB} \cdot \overrightarrow{n} = (2,0,0) \cdot (x,y,z) = 2x = 0 \\
\overrightarrow{AA_{1}} \cdot \overrightarrow{n} = \left( 1, - 2,\sqrt{3} \right) \cdot (x,y,z) = x - 2y + \sqrt{3}z = 0
\end{array} \right.\),\par
解得\(x = 0\),令\(y = \sqrt{3}\),则\(z = 2\),故\(\overrightarrow{n} = \left( 0,\sqrt{3},2 \right)\),\par
设\(BC_{1}\)与平面\(AA_{1}B_{1}B\)所成角为\(\theta\),\par
\(\therefore\sin\theta = \frac{\left| \overrightarrow{BC_{1}} \cdot \overrightarrow{n} \right|}{\left| \overrightarrow{BC_{1}} \right|\left| \overrightarrow{n} \right|} = \frac{2\sqrt{3}}{2 \times \sqrt{7}} = \frac{\sqrt{21}}{7}\).\par
方法二:因为\(AC\bot\)平面\(ABC_{1}\),又\(AC_{1} \subset\)平面\(ABC_{1}\),所以\(AC\bot AC_{1}\).\par
又\(AC\bot AB\),所以\(\angle C_{1}AB\)为二面角\(C_{1} - AC - B\)的平面角,即\(\angle C_{1}AB = 60^{{^\circ}}\),\par
在\(Rt \bigtriangleup ACC_{1}\)中,因为\(CC_{1} = 2\sqrt{2}\),\(AC = 2\),所以\(AC_{1} = 2\).故\(\bigtriangleup ABC_{1}\)是等边三角形\par
所以\(V_{A - ABC_{1}} = V_{C - ABC_{1}} = \frac{1}{3} \cdot S_{\bigtriangleup ABC_{1}} \cdot AC = \frac{1}{3} \times \frac{\sqrt{3}}{4} \times 2^{2} \times 2 = \frac{2\sqrt{3}}{3}\).\par
在三棱柱\(ABC - A_{1}B_{1}C_{1}\)中,\(AC//A_{1}C_{1}\),又\(AC\bot\)平面\(ABC_{1}\),所以\(A_{1}C_{1}\bot\)平面\(ABC_{1}\),\par
又\(BC_{1} \subset\)平面\(ABC_{1}\),所以\(A_{1}C_{1}\bot BC_{1}\).故\(\bigtriangleup A_{1}BC_{1}\)为直角三角形.\par
在直角\(\bigtriangleup A_{1}BC_{1}\)中,因为\(BC_{1} = 2\),\(A_{1}C_{1} = 2\),所以\(A_{1}B = 2\sqrt{2}\),故\(S_{\bigtriangleup ABA_{1}} = \sqrt{7}\).\par
设点\(C_{1}\)到平面\(ABA_{1}\)的距离为\(d\),由\(V_{C_{1} - ABA_{1}} = V_{A - ABC_{1}}\),\par
得\(\frac{1}{3} \times \sqrt{7} \times d = \frac{2\sqrt{3}}{3}\),解得\(d = \frac{2\sqrt{21}}{7}\).\par
设直线\(BC_{1}\)与平面\(AA_{1}B_{1}B\)所成角为\(\theta\),则\(\sin\theta = \frac{d}{BC_{1}} = \frac{\sqrt{21}}{7}\),\par
即直线\(BC_{1}\)与平面\(AA_{1}B_{1}B\)所成角的正弦值为\(\frac{\sqrt{21}}{7}\).}
\end{question}

\begin{question}
已知椭圆\(E:\frac{x^{2}}{a^{2}} + \frac{y^{2}}{b^{2}} = 1(a > b > 0)\)的离心率为\(\frac{3}{5}\),
且经过点\((3,\frac{16}{5})\).\(F_{1}\),
\(F_{2}\)是\(E\)的左、右焦点.
\begin{enumerate}[label=(\arabic*)]
  \item 求\(E\)的标准方程;
  \item 过\(F_{2}\)的直线与\(E\)交于\(P\),
\item \(Q\)两点.若\(\bigtriangleup F_{1}PQ\)的内切圆半径为\(r\),
\item \(r = \frac{\sqrt{3}}{10}|PQ|\),
\item 求\(\left| F_{2}P \right| \cdot \left| F_{2}Q \right|\)的值.
\end{enumerate}

\begin{center}
% IMAGE_TODO_START id=js-suxichang-2025-q2-Q17-img1 path=/Users/muryor/code/mynote/word\\_to\\_tex/output/figures/js-suxichang-2025-q2/media/image10.png width=60% inline=false question_index=17 sub_index=1
\begin{tikzpicture}[scale=1.05,>=Stealth,line cap=round,line join=round]
  % TODO: AI_AGENT_REPLACE_ME (id=js-suxichang-2025-q2-Q17-img1)
\end{tikzpicture}
% IMAGE_TODO_END id=js-suxichang-2025-q2-Q17-img
1
\end{center}

\topics{根据椭圆过的点求标准方程;根据离心率求椭圆的标准方程;求椭圆中的弦长;根据韦达定理求参数}
\difficulty{0.65}
\answer{(1)\(\frac{x^{2}}{25} + \frac{y^{2}}{16} = 1\);
(2)\(\frac{256}{19}\).}
\explain{(1)设椭圆\(E\)的半焦距为\(c\),
由离心率为\(\frac{3}{5}\),
得\(\frac{c}{a} = \frac{3}{5}\),令\(a = 5t,t > 0\),
\(c = 3t,b = 4t\),\par
椭圆\(E:\frac{x^{2}}{25t^{2}} + \frac{y^{2}}{16t^{2}} = 1\)过点\((3,\frac{16}{5})\),
则\(\frac{9}{25t^{2}} + \frac{16}{25t^{2}} = 1\),
解得\(t^{2} = 1\),\par
所以椭圆\(E\)的标准方程为\(\frac{x^{2}}{25} + \frac{y^{2}}{16} = 1\).\par
(2)由(1)知\(F_{1}( - 3,0),F_{2}(3,0)\),
设直线\(PQ\)的方程为\(x = my + 3\),
\(P(x_{1},y_{1})\),\(Q(x_{2},y_{2})\),\par
由\(\left\{ \begin{array}{r}
x = my + 3 \\
16x^{2} + 25y^{2} = 400
\end{array} \right.\)消去\(x\)得\((16m^{2} + 25)y^{2} + 96my - 256 = 0\),\par
\(\text{Δ} = {(96m)}^{2} + 4 \times 256(16m^{2} + 25) = 100 \times 256(m^{2} + 1) > 0\),\(y_{1}y_{2} = \frac{- 256}{16m^{2} + 25}\),\par
\(S_{\bigtriangleup F_{1}PQ} = \frac{1}{2}|F_{1}F_{2}||y_{1} - y_{2}| = 3|y_{1} - y_{2}|\),\(r = \frac{2S_{\bigtriangleup F_{1}PQ}}{4a} = \frac{6|y_{1} - y_{2}|}{20} = \frac{3}{10}|y_{1} - y_{2}|\),\par
而\(|PQ| = \sqrt{1 + m^{2}} \cdot |y_{1} - y_{2}|\),\(r = \frac{\sqrt{3}}{10}|PQ|\),则\(\frac{3}{10}|y_{1} - y_{2}| = \frac{\sqrt{3}}{10}\sqrt{1 + m^{2}} \cdot |y_{1} - y_{2}|\),解得\(m^{2} = 2\),\par
所以\(|F_{2}P| \cdot |F_{2}Q| = \sqrt{1 + m^{2}}|y_{1}| \cdot \sqrt{1 + m^{2}} \cdot |y_{2}| = \frac{256(1 + m^{2})}{16m^{2} + 25} = \frac{3 \times 256}{57} = \frac{256}{19}\).}
\end{question}

\begin{question}
已知函数\(f(x) = \sinx\),
\(g(x) = \mathrm{e}^{ax}\),\(a \in R\).
\begin{enumerate}[label=(\arabic*)]
  \item 若曲线\(y = f(x)\)在点\(O(0,0)\)的切线也是曲线\(y = g(x)\)的切线,
\item 求\(a\)的值;
  \item 讨论函数\(h(x) = \frac{x - 1}{g(x)}\)在区间\((0, + \infty)\)上的单调性;
  \item 若\(f(x)g(x) < x\)对任意\(x \in (0, + \infty)\)恒成立,
\item 求\(a\)的取值范围.
\end{enumerate}
\topics{求在曲线上一点处的切线方程(斜率);利用导数证明不等式;利用导数研究不等式恒成立问题;利用导数求函数(含参)的单调区间}
\difficulty{0.4}
\answer{(1)\(\frac{1}{\text{e}}\)
(2)答案见解析
(3)\(( - \infty,0\rbrack\)}
\explain{(1)由已知得\(f'(x) = \cosx\),
\(k = f'(0) = 1\),
\(\therefore f(x)\)在点\(O(0,0)\)处的切线方程为\(y = x\).\par
设与\(y = g(x)\)切于\(P\left( x_{0},\mathrm{e}^{ax_{0}} \right)\),
\(g'(x) = a\mathrm{e}^{ax}\),
\(k = a\mathrm{e}^{ax_{0}}\),\par
则\(y = g(x)\)过该点的切线方程为:\(y - \mathrm{e}^{ax_{0}} = a\mathrm{e}^{ax_{0}}\left( x - x_{0} \right)\),\par
整理得\(y = a\mathrm{e}^{ax_{0}}x - a\mathrm{e}^{ax_{0}}x_{0} + \mathrm{e}^{ax_{0}}\),
由于该切线与\(y = x\)重合,\par
则\(\therefore\left\{ \begin{array}{r}
a\mathrm{e}^{ax_{0}} = 1 \\
 - a\mathrm{e}^{ax_{0}}x_{0} + \mathrm{e}^{ax_{0}} = 0
\end{array} \Rightarrow \left\{ \begin{array}{r} a = \frac{1}{\text{e}} \\
x_{0} = \text{e}
\end{array} \right.\  \right..\par
(2)由\(h(x) = \frac{x - 1}{\mathrm{e}^{ax}}\),求导得\(h'(x) = \frac{\mathrm{e}^{ax} - a\mathrm{e}^{ax}(x - 1)}{\left( \mathrm{e}^{ax} \right)^{2}} = \frac{a + 1 - ax}{\mathrm{e}^{ax}}\),\par
①当\(- 1 \leq a \leq 0\)时,\(\because x > 0\),\(\therefore h'(x) > 0\),\(h(x)\)在\((0, + \infty)\)上单调递增;\par
②当\(a < - 1\)时,令\(h'(x) = 0 \Rightarrow x = \frac{a + 1}{a}\)\par
当\(0 < x < \frac{a + 1}{a}\)时,\(h'(x) < 0\),\(h(x)\)在区间\(\left( 0,\frac{a + 1}{a} \right)\)上单调递减,\par
当\(x > \frac{a + 1}{a}\)时,\(h'(x) > 0\),\(h(x)\)在区间\(\left( \frac{a + 1}{a}, + \infty \right)\)上单调递增\par
③当\(a > 0\)时,令\(h'(x) = 0 \Rightarrow x = \frac{a + 1}{a}\)\par
当\(0 < x < \frac{a + 1}{a}\)时,\(h'(x) > 0\),\(h(x)\)在区间\(\left( 0,\frac{a + 1}{a} \right)\)上单调递增;\par
当\(x > \frac{a + 1}{a}\)时,\(h'(x) < 0\),\(h(x)\)在区间\(\left( \frac{a + 1}{a}, + \infty \right)\)上单调递减\par
(3)由题意得\(\mathrm{e}^{ax}\sinx < x\),即\(\mathrm{e}^{ax}\sinx - x < 0\)对\(\forall x \in (0, + \infty)\)恒成立.\par
令\(F(x) = \mathrm{e}^{ax}\sinx - x\),\(F'(x) = a\mathrm{e}^{ax}\sinx + \mathrm{e}^{ax}\cosx - 1 = \mathrm{e}^{ax}\left( a\sinx + \cosx - \mathrm{e}^{- ax} \right)\),\par
令\(\varphi(x) = a\sinx + \cosx - \mathrm{e}^{- ax}\),\(\varphi'(x) = a\cosx - \sinx + a\mathrm{e}^{- ax}\),\par
因为\(F(0) = \mathrm{e}^{0}\sin0 - 0 = 0\),\(F'(0) = 0\),\par
若\(\varphi'(0) = 2a > 0\),则\(\varphi(x) = a\sinx + \cosx - \mathrm{e}^{- ax}\)在\(x = 0\)处的切线必然是上升的,\par
又因为\(\varphi(0) = a\sin0 + \cos0 - \mathrm{e}^{0} = 0\),所以当\(x > 0\)且靠近\(0\)的函数值满足\(\varphi(x) > 0\),\par
此时就有\(F'(x) = \mathrm{e}^{ax}\left( a\sinx + \cosx - \mathrm{e}^{- ax} \right) > 0\),\par
从而可推导\(F(x) = \mathrm{e}^{ax}\sinx - x\)在\(x > 0\)且靠近\(0\)的附近是递增的,\par
又因为\(F(0) = \mathrm{e}^{0}\sin0 - 0 = 0\),\par
所以在\(x > 0\)且靠近\(0\)的附近必有\(F(x) = \mathrm{e}^{ax}\sinx - x > 0\)\par
则必然不满足对\(\forall x \in (0, + \infty)\)恒有\(F(x) < 0\),\par
所以要满足对\(\forall x \in (0, + \infty)\)恒有\(F(x) < 0\),\par
首先必需满足在\(x > 0\)且靠近\(0\)的附近\(F'(x) \leq 0\),\par
所以满足\(\varphi'(0) = 2a \leq 0 \Rightarrow a \leq 0\),\par
从而可得参数\(a\)满足的必要条件是\(a \leq 0\);\par
下面再证充分性,当\(a \leq 0\),\(x > 0\)时,则\(\mathrm{e}^{ax} < \mathrm{e}^{0} = 1\),即有\(\frac{x}{\mathrm{e}^{ax}} > x\),\par
又构造\(\varphi(x) = x - \sin x\),\(x > 0\),可得\(\varphi'(x) = 1 - \cos x \geq 0\),\par
所以\(\varphi(x) = x - \sin x\)在区间\((0, + \infty)\)上单调递增,即\(\varphi(x) = x - \sin x > \varphi(0) = 0\),\par
则可知\(\sinx - \frac{x}{\mathrm{e}^{ax}} \leq \sinx - x < 0\),则\(F(x) = \mathrm{e}^{ax}\left( \sinx - \frac{x}{\mathrm{e}^{ax}} \right) < 0\),\par
\(\therefore F(x) < 0\)恒成立,符合题意,\par
综上:\(a\)的取值范围为\(( - \infty,0\rbrack\).}
\end{question}

\begin{question}
若无穷数列\(\left\{ a_{n} \right\}\)满足:,
\(\forall n \in N^{\ast}\),
\(a_{1} > \frac{a_{1} + a_{2}}{2} > \cdot \cdot \cdot > \frac{a_{1} + a_{2} + \cdot \cdot \cdot + a_{n}}{n} > \cdot \cdot \cdot\),
则称\(\left\{ a_{n} \right\}\)为"均值递减数列".
\begin{enumerate}[label=(\arabic*)]
  \item 已知无穷数列\(\left\{ a_{n} \right\}\)的前\(n\)项和为\(S_{n}\),
\item 若\(\left\{ a_{n} \right\}\)为"均值递减数列",
\item 求证:\(\forall n \in N^{\ast}\),
\item \(S_{n} > na_{n + 1}\);
  \item 若数列\(\left\{ b_{n} \right\}\)的通项公式\(b_{n} = - 6(n - 4)^{3} + n^{2}\),
\item 判断\(\left\{ b_{n} \right\}\)是否为"均值递减数列",
\item 并说明理由;
  \item 若两个正项数列\(\left\{ c_{n} \right\}\)和\(\left\{ d_{n} \right\}\)均为"均值递减数列",
\item 证明:数列\(\left\{ c_{n}d_{n} \right\}\)也为"均值递减数列".
\end{enumerate}
\topics{判断数列的增减性;数学归纳法;数列新定义}
\difficulty{0.4}
\answer{(1)证明见解析
(2)是,理由见解析
(3)证明见解析}
\explain{(1)法一:\par
\(\frac{S_{n}}{n} > \frac{S_{n + 1}}{n + 1} \Leftrightarrow \frac{S_{n}}{n} > \frac{a_{n + 1} + S_{n}}{n + 1} \Leftrightarrow (n + 1)S_{n} > na_{n + 1} + nS_{n} \Leftrightarrow S_{n} > na_{n + 1}\);\par
法二:\par
\(\because\left\{ a_{n} \right\}\)为"均值递减数列",
\(\therefore\left\{ \frac{a_{1} + a_{2} + \cdot \cdot \cdot + a_{n}}{n} \right\}\)关于\(n\)单调递减,\par
即\(\left\{ \frac{S_{n}}{n} \right\}\)关于\(n\)单调递减,
\(\therefore\frac{S_{n + 1}}{n + 1} < \frac{S_{n}}{n}\),
\(\therefore nS_{n + 1} < (n + 1)S_{n} = nS_{n} + S_{n}\Rightarrow S_{n} > n\left( S_{n + 1} - S_{n} \right) = na_{n + 1}\);\par
(2)法一:\par
设\(b_{n}\)的前\(n\)项和为\(T_{n}\),\par
\(T_{n} = 216 - \frac{3}{2}(n - 4)^{2}(n - 3)^{2} + \frac{1}{6}n(n + 1)(2n + 1)= n\left( \frac{1}{6}(n + 1)(2n + 1) - \frac{3}{2}(n - 7)\left( n^{2} - 7n + 24 \right) \right)\),\par
令\(G_{n} = \frac{T_{n}}{n} = \frac{1}{6}(n + 1)(2n + 1) - \frac{3}{2}(n - 7)\left( n^{2} - 7n + 24 \right)\),
则\(G_{n + 1} - G_{n} = \frac{- 27n^{2} + 229n - 535}{6}\),
判别式小于零,所以\(G_{n}\)递减,\par
因此\(\left\{ b_{n} \right\}\)是"均值递减数列";\par
法二:\par
易知\(1 \leq n \leq 3\)时,
\(\left\{ b_{n} \right\}\)单调递减;\(3 \leq n \leq 5\)时,\par
\(\left\{ b_{n} \right\}\)单调递增且\(1 \leq n \leq 5\)时,
\(b_{n} > 0\)当\(n \geq 6\)时,
\(\left\{ b_{n} \right\}\)单调递减且\(b_{n} < 0\),\par
且计算易得\(b_{1} > \frac{b_{1} + b_{2}}{2} > \frac{b_{1} + b_{2} + b_{3}}{3} > \cdot \cdot \cdot > \frac{b_{1} + b_{2} + b_{3} + \cdot \cdot \cdot + b_{6}}{6}\),\par
设\(\left\{ b_{n} \right\}\)前\(n\)项和为\(S_{n}\),
归纳假设\(n = k\),\(k \geq 6\),
\(k \in N^{\ast}\)时,\par
\(\frac{S_{n}}{n} < \frac{S_{n - 1}}{n - 1}\),
即\(\frac{S_{k}}{k} < \frac{S_{k - 1}}{k - 1}\),
即\(kb_{k} < S_{k}\),
\(\therefore\frac{S_{k + 1}}{k + 1} - \frac{S_{k}}{k} = \frac{kS_{k + 1} - (k + 1)S_{k}}{k(k + 1)} = \frac{kb_{k + 1} - S_{k}}{k(k + 1)} < \frac{k\left( b_{k + 1} - b_{k} \right)}{k(k + 1)} < 0\),\par
\(\therefore\frac{S_{k + 1}}{k + 1} < \frac{S_{k}}{k}\),
即\(n = k + 1\),\(k \geq 6\),
\(k \in N^{\ast}\)时,
\(\frac{S_{n}}{n} < \frac{S_{n - 1}}{n - 1}\)成立.\par
而\(\frac{S_{6}}{6} < \frac{S_{5}}{5}\)成立,
\(\therefore\frac{S_{n}}{n} < \frac{S_{n - 1}}{n - 1}\)对\(\forall n \geq 6\)且\(n \in N^{\ast}\)恒成立,\par
\(\therefore\)也有\(\frac{S_{5}}{5} > \frac{S_{6}}{6} > \cdot \cdot \cdot > \frac{S_{n}}{n} < \cdot \cdot \cdot \Rightarrow S_{1} > \frac{S_{2}}{2} > \frac{S_{3}}{3} > \cdot \cdot \cdot > \frac{S_{6}}{6} > \cdot \cdot \cdot > \frac{S_{n}}{n} > \cdot \cdot \cdot\),\par
即\(\left\{ b_{n} \right\}\)为"均值递减数列";\par
(3)法一:\par
设\(\left\{ \begin{array}{r}
\frac{c_{1} + c_{2} + \cdot \cdot \cdot + c_{n}}{n} = x_{n} \\
\frac{d_{1} + d_{2} + \cdot \cdot \cdot + d_{n}}{n} = y_{n}
\end{array} \right.\)依题意\(\left\{ x_{n} \right\}\),\(\left\{ y_{n} \right\}\)均为递减数列,\par
而\(\left\{ \begin{array}{r}
c_{n} = nx_{n} - (n - 1)x_{n - 1} \\
d_{n} = nx_{n} - (n - 1)x_{n - 1}
\end{array} \right.\),\par
相乘展开得\(c_{n}d_{n} = n^{2}x_{n}y_{n} + (n - 1)^{2}x_{n - 1}y_{n - 1} - n(n - 1)\left( x_{n}y_{n - 1} + y_{n}x_{n - 1} \right)\),\par
由于\(x_{n} < x_{n - 1}\),\(y_{n} < y_{n - 1}\),则由补充不等式有\(x_{n}y_{n - 1} + y_{n}x_{n - 1} < x_{n}y_{n} + y_{n - 1}x_{n - 1}\),\par
所以\(c_{n}d_{n} > n^{2}x_{n}y_{n} + (n - 1)^{2}x_{n - 1}y_{n - 1} - n(n - 1)\left( x_{n}y_{n} + y_{n - 1}x_{n - 1} \right)= nx_{n}y_{n} - (n - 1)x_{n - 1}y_{n - 1}\),\par
求和得\(\sum_{i = 1}^{n}{}c_{i}d_{i} > nx_{n}y_{n}\),由(1)的结论知\(x_{n} > c_{n + 1}\),\(y_{n} > d_{n + 1}\),\par
所以\(\sum_{i = 1}^{n}{}c_{i}d_{i} > nc_{n + 1}d_{n + 1}\),\par
于是再由(1)的结论即可知\(\left\{ c_{n}d_{n} \right\}\)是"均值递减数列";\par
法二:\par
设\(\left\{ c_{n} \right\}\)的前\(n\)项和为\(S_{n}\),\(\left\{ d_{n} \right\}\)的前\(n\)项和为\(T_{n}\),\par
\(\because\left\{ c_{n} \right\}\)和\(\left\{ d_{n} \right\}\)均为均值递减数列,\par
由(1)知\(\left\{ \begin{array}{r}
S_{n} > nc_{n + 1},① \\
T_{n} > nd_{n + 1},②
\end{array} \right.\)对\(\forall n \in N^{\ast}\)恒成立,\par
由①②知\(\left\{ \begin{array}{r}
c_{1} > c_{2} \\
d_{1} > d_{2}
\end{array} \right.\),\(\therefore c_{1}d_{1} > c_{2}d_{2}\),记\(\left\{ c_{n}d_{n} \right\}\)的前\(n\)项和为\(R_{n}\),\par
\(\Leftrightarrow\)证对\(\forall n \in N^{\ast}\),\(R_{n} > nc_{n + 1}d_{n + 1}\),\(n = 1\)时不等式显然成立,\par
设当\(n = k\),\(k \in N^{\ast}\)时,\(R_{n} > nc_{n + 1}d_{n + 1}\)成立,\par
即\(R_{k} > kc_{k + 1}d_{k + 1}\),\(k \in N^{\ast}\),\par
\(\therefore R_{k + 1} - (k + 1)c_{k + 2}d_{k + 2} = R_{k} + c_{k + 1}d_{k + 1} - (k + 1)c_{k + 2}d_{k + 2}\),\par
\(> kc_{k + 1}d_{k + 1} + c_{k + 1}d_{k + 1} - (k + 1)c_{k + 2}d_{k + 2} = (k + 1)\left( c_{k + 1}d_{k + 1} - c_{k + 2}d_{k + 2} \right) > 0\),\par
\(\therefore R_{k + 1} > (k + 1)c_{k + 2}d_{k + 2}\),即\(n = k + 1\)时,不等式也成立,\par
\(\therefore R_{n} > nc_{n + 1}d_{n + 1}\)对\(\forall n \in N^{\ast}\)恒成立,\par
\(\therefore\left\{ c_{n}d_{n} \right\}\)也为"均值递减数列".}
\end{question}
