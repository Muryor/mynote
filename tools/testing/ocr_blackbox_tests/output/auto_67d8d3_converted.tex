\examxtitle{测试试卷 - auto_67d8d3}

\section{单选题}

\begin{question}
已知集合\(A = \left\{ x \mid x^{2} - 3x + 2 < 0 \right\},B = \{ x \mid x < a\}\),
若\(A \subseteq B\),则实数\(a\)的取值范围是(    )
\begin{choices}
  \item \(a > 2\)
  \item \(a < 2\)
  \item \(a \leq 2\)
  \item \(a \geq 2\)
\end{choices}
\topics{根据集合的包含关系求参数;解不含参数的一元二次不等式}
\difficulty{0.85}
\answer{D}
\explain{\(A = \left\{ x \mid x^{2} - 3x + 2 < 0 \right\} = \{ x|1 < x < 2\}\),
又\(A \subseteq B\),\(B = \{ x|x < a\}\),\par
所以\(a \geq 2\)}
\end{question}

\begin{question}
已知\(x > 0,y > 0,x,a,b,y\)依次成等差数列,
\(x,c,d,y\)依次成等比数列,
则\(\frac{(a + b)^{2}}{2cd}\)的最小值是(  )
\begin{choices}
  \item 2
  \item \(2\sqrt{2}\)
  \item 4
  \item 8
\end{choices}
\topics{等差中项的应用;等比中项的应用;基本不等式求和的最小值}
\difficulty{0.85}
\answer{A}
\explain{\(\because x,a,b,y\)成等差数列,
\(x,c,d,y\)成等比数列,\par
所以\(a + b = x + y,cd = xy\),
且\(x > 0,y > 0\),
则\(\frac{(a + b)^{2}}{2cd} = \frac{(x + y)^{2}}{2xy} \geq \frac{\left( 2\sqrt{xy} \right)^{2}}{2xy} = 2\),\par
当且仅当\(x = y\)时取等号}
\end{question}

\begin{question}
已知向量\(\overrightarrow{a} = (1,1),\overrightarrow{b} = (\sqrt{3},1)\),
则向量\(\overrightarrow{a}\)在\(\overrightarrow{b}\)上的投影向量为(  )
\begin{choices}
  \item \((\frac{3 + \sqrt{3}}{2},\frac{\sqrt{3} + 1}{2})\)
  \item \(\frac{\sqrt{3} + 1}{2}\overrightarrow{b}\)
  \item \(\frac{\sqrt{3} + 1}{4}\)
  \item \(\frac{\sqrt{3} + 1}{4}\overrightarrow{b}\)
\end{choices}
\topics{数量积的坐标表示;坐标计算向量的模;求投影向量}
\difficulty{0.85}
\answer{D}
\explain{向量\(\overrightarrow{a} = (1,1),\overrightarrow{b} = (\sqrt{3},1)\),
则\(\overrightarrow{a} \cdot \overrightarrow{b} = \sqrt{3} + 1,|\overrightarrow{b}| = 2\),\par
所以向量\(\overrightarrow{a}\)在\(\overrightarrow{b}\)上的投影向量为\(\frac{\overrightarrow{a} \cdot \overrightarrow{b}}{|\overrightarrow{b}|^{2}}\overrightarrow{b} = \frac{\sqrt{3} + 1}{4}\overrightarrow{b} = (\frac{3 + \sqrt{3}}{4},\frac{\sqrt{3} + 1}{4})\)}
\end{question}

\begin{question}
已知点\(P(x,y)\)满足\(\sqrt{(x - 1)^{2} + y^{2}} = |x + 1|,Q(4,0)\),
则\(|PQ|\)的最小值为(  )
\begin{choices}
  \item 2
  \item \(2\sqrt{2}\)
  \item \(2\sqrt{3}\)
  \item 4
\end{choices}
\topics{求平面两点间的距离;利用抛物线定义求动点轨迹}
\difficulty{0.65}
\answer{C}
\explain{因为\(\sqrt{(x - 1)^{2} + y^{2}}\)表示点\(P(x,y)\)到点\((1,0)\)的距离;
\(|x + 1|\)表示点\(P(x,y)\)到直线\(x = - 1\)的距离,\par
又\(\sqrt{(x - 1)^{2} + y^{2}} = |x + 1|\),
所以点\(P(x,y)\)到点\((1,0)\)的距离等于点\(P(x,y)\)到直线\(x = - 1\)的距离,\par
由抛物线的定义知,点\(P\)的轨迹为抛物线,
抛物线方程为\(y^{2} = 4x\),\par
设\(P\left( \frac{t^{2}}{4},t \right)\),
则\(|PQ| = \sqrt{\left( \frac{t^{2}}{4} - 4 \right)^{2} + t^{2}} = \sqrt{\frac{t^{4}}{16} - t^{2} + 16} = \sqrt{\left( \frac{t^{2}}{4} - 2 \right)^{2} + 12} \geq 2\sqrt{3}\),\par
当且仅当\(t = \pm 2\sqrt{2}\)时,等号成立}
\end{question}

\begin{question}
下列函数中,\(f(x)\)为周期函数,
且在区间\(\left( \frac{\pi}{6},\frac{\pi}{4} \right)\)上单调递减的是(  )
\begin{choices}
  \item \(f(x) = \sin|x| + 1\)
  \item \(f(x) = \left| \sin2x \right| + 2\)
  \item \(f(x) = \cos|3x| + 3\)
  \item \(f(x) = \left| \cos4x \right| + 4\)
\end{choices}
\topics{求cosx型三角函数的单调性;求含cosx的函数的最小正周期;求sinx型三角函数的单调性}
\difficulty{0.85}
\answer{C}
\explain{对于A:当\(x \in \left( \frac{\pi}{6},\frac{\pi}{4} \right)\)时,
\(f(x) = \sin x + 1\),
函数在\(\left( \frac{\pi}{6},\frac{\pi}{4} \right)\)上显然单调递增,
故A错误;\par
对于B:当\(x \in \left( \frac{\pi}{6},\frac{\pi}{4} \right)\)时,
\(\sin 2x > 0\),
则\(f(x) = \sin 2x + 2\)在\(\left( \frac{\pi}{6},\frac{\pi}{4} \right)\)上显然单调递增,
故B错误\par
;\par
对于D:\(x \in \left( \frac{\pi}{6},\frac{\pi}{4} \right)\)时,
\(\frac{2\pi}{3} < 4x < \pi\),
则\(\cos 4x < 0\),
\(f(x) = - \cos 4x + 4\).该函数在\(\left( \frac{\pi}{6},\frac{\pi}{4} \right)\)单调递增,
故D错误;\par
对于C:\(x \in \left( \frac{\pi}{6},\frac{\pi}{4} \right)\)时,
\(\frac{\pi}{2} < 3x < \frac{3\pi}{4}\),
则\(f(x) = \cos 3x + 3\)在\(\left( \frac{\pi}{6},\frac{\pi}{4} \right)\)上单调递减,
且为最小正周期是\(\frac{2\pi}{3}\)的周期函数,故C正确}
\end{question}

\begin{question}
已知\(f(x)\)是定义在\(R\)上的奇函数,
\(f(x + 1)\)为偶函数,
且当\(x \in \lbrack 0,1\rbrack\)时,
\(f(x) = \frac{1}{3}\sin\frac{\pi}{2}x\),
则函数\(g(x) = f(x) - \frac{1}{x - 4}\)在\(\left\lbrack - 5,4) \cup (4,13 \right\rbrack\)上所有零点的和为(  )
\begin{choices}
  \item 16
  \item 24
  \item 32
  \item 48
\end{choices}

\begin{center}
% IMAGE_TODO_START id=auto_67d8d3-Q6-img1 path=/Users/muryor/code/mynote/word\\_to\\_tex/output/figures/auto\\_67d8d3/media/image2.png width=60% inline=false question_index=6 sub_index=1
% CONTEXT_BEFORE: $对称, 所以$$g(x)$$所有零点和为$$8 \times 3 = 24$$. 
\begin{tikzpicture}[scale=1.05,>=Stealth,line cap=round,line join=round]
  % TODO: AI_AGENT_REPLACE_ME (id=auto_67d8d3-Q6-img1)
\end{tikzpicture}
% IMAGE_TODO_END id=auto_67d8d3-Q6-img
1
\end{center}

\topics{函数周期性的应用;函数对称性的应用;函数图象的应用}
\difficulty{0.65}
\answer{B}
\explain{依题意,\(f(x)\)是定义在\(R\)上的奇函数,图象关于原点对称,
则\(f(x) = - f( - x)\).\par
由于\(f(x + 1)\)为偶函数,
则\(f(x + 1) = f(1 - x) \Rightarrow f(x) = f(2 - x)\).\par
从而\(f( - x) = - f(2 - x) \Rightarrow f(x + 2) = - f(x) \Rightarrow f(x + 4) = - f(x + 2) = f(x)\).\par
所以\(f(x)\)是一个周期为4的周期函数.\par
令\(g(x) = f(x) - \frac{1}{x - 4} = 0\),
得\(f(x) = \frac{1}{x - 4}\),\par
函数\(y = \frac{1}{x - 4}\)的图象关于点\((4,0)\)对称,
\(y = f(x)\)的图象也关于点\((4,0)\)对称,\par
画出函数\(y = f(x)\)和\(y = \frac{1}{x - 4}\)的图象如图所示,\par
由图可知,
两个函数图象在\(\lbrack - 5,4) \cup (4,13\rbrack\)上有6个交点,
且交点关于\((4,0)\)对称,\par
所以\(g(x)\)所有零点和为\(8 \times 3 = 24\).}
\end{question}

\begin{question}
已知点\(P\)为直线\(l:2x + y + 2 = 0\)上的一个动点,
\(A,B\)为圆\(M:x^{2} + y^{2} - 2x - 2y - 2 = 0\)上任意两个不重合的点,
记\(\cos\angle APB\)的最小值为\(m,\sin\angle APB\)的最大值为\(n\),
则\(m + n =\)(  )
\begin{choices}
  \item \(\frac{2}{5}\)
  \item \(\frac{1}{5}\)
  \item \(\frac{5 - 2\sqrt{5}}{5}\)
  \item \(\frac{5 - \sqrt{5}}{5}\)
\end{choices}

\begin{center}
% IMAGE_TODO_START id=auto_67d8d3-Q7-img1 path=/Users/muryor/code/mynote/word\\_to\\_tex/output/figures/auto\\_67d8d3/media/image3.png width=60% inline=false question_index=7 sub_index=1
\begin{tikzpicture}[scale=1.05,>=Stealth,line cap=round,line join=round]
  % TODO: AI_AGENT_REPLACE_ME (id=auto_67d8d3-Q7-img1)
\end{tikzpicture}
% IMAGE_TODO_END id=auto_67d8d3-Q7-img
1
\end{center}

\topics{二倍角的余弦公式;求点到直线的距离;判断直线与圆的位置关系}
\difficulty{0.65}
\answer{A}
\explain{由题意得\(\odot M\)的标准方程为\((x - 1)^{2} + (y - 1)^{2} = 4\),
所以圆心\(M(1,1)\),半径为2,\par
如图:\par
所以圆心\(M\)到直线\(l\)的距离为\(\frac{|2 + 1 + 2|}{\sqrt{4 + 1}} = \sqrt{5} > 2\),
所以直线\(l\)与\(\odot M\)相离,\par
所以当\(PA,PB\)分别为圆的切线,且\(|MP|\)最小时,\par
\(\sin\angle APM = \frac{|AM|}{|PM|} = \frac{2}{|PM|} = \frac{2\sqrt{5}}{5}\)最大,
又\(0 < \angle APM < \frac{\pi}{2}\),
则\(\angle APM\)最大,\par
所以\(\angle APB = 2\angle APM\)最大,
此时\(\cos\angle APB\)最小,\par
此时\(\cos\angle APB = \cos2\angle APM = 1 - 2\sin^{2}\angle APM = 1 - 2 \times \left( \frac{2}{\sqrt{5}} \right)^{2} = - \frac{3}{5}\).\par
显然\(\sin\angle APB\)的最大值为1,
故\(m + n = - \frac{3}{5} + 1 = \frac{2}{5}\).}
\end{question}

\begin{question}
已知正实数\(a,b\)满足\(a\mathrm{e}^{a - 2} = \mathrm{e}^{2025}\)和\(b\left( \lnb - 2 \right) = \mathrm{e}^{2029}\).则\(ab\)的值为(  )
\begin{choices}
  \item \(\mathrm{e}^{2029}\)
  \item \(\mathrm{e}^{2028}\)
  \item \(\mathrm{e}^{2027}\)
  \item \(\mathrm{e}^{2026}\)
\end{choices}
\topics{指数式与对数式的互化;对数的运算性质的应用;对数函数单调性的应用}
\difficulty{0.15}
\answer{A}
\explain{\(\because\) \(a\mathrm{e}^{a - 2} = \mathrm{e}^{2025}\),\par
\(\therefore\ln\left( a\mathrm{e}^{a - 2} \right) = \ln\mathrm{e}^{2025}\),
即\(\ln a + \ln\mathrm{e}^{a - 2} = \ln\mathrm{e}^{2025} \Rightarrow \ln a + a = 2027\),\par
\(\because\) \(b\left( \lnb - 2 \right) = \mathrm{e}^{2029}\),\par
\(\therefore\ln\left\lbrack b\left( \ln b - 2 \right) \right\rbrack = \ln\mathrm{e}^{2029}\),
即\(\ln b + \ln\left( \ln b - 2 \right) = 2029 \Rightarrow \left( \ln b - 2 \right) + \ln\left( \ln b - 2 \right) = 2027\),\par
令\(f(x) = \lnx + x,x > 0\),
则\(f(x)\)在\((0, + \infty)\)上单调递增,\par
\(\therefore\)方程\(f(x) = 2027\)有唯一解,\par
\(\therefore\) \(a = \lnb - 2\),\par
\(\thereforeab = \left( \lnb - 2 \right)b = \mathrm{e}^{2029}\).}
\end{question}

\section{多选题}

\begin{question}
在平行六面体\(ABCD - A_{1}B_{1}C_{1}D_{1}\)中,侧棱\(BB_{1}\)垂直于底面\(ABCD\),下列结论正确的是(  )
\begin{choices}
  \item 若\(AB = AD\),则\(AC\bot BD_{1}\)
  \item 若\(AC = BD\),则\(AC\bot BD_{1}\)
  \item 若\(A_{1}D = A_{1}B\),则\(BD\bot\)平面\(ACC_{1}A_{1}\)
  \item 若\(AD = AA_{1}\),则\(AD_{1}\bot\)平面\(DA_{1}B_{1}C\)
\end{choices}
\topics{证明线面垂直;线面垂直证明线线垂直}
\difficulty{0.4}
\answer{AC}
\explain{对于选项A,\(BB_{1}\bot\)平面\(ABCD\),
\(AC \subset\)平面\(ABCD\),则\(BB_{1}\bot AC\),\par
又\(AB = AD\),底面为菱形,则\(AC\bot BD\),\par
\(BD \cap BB_{1} = B\),
则\(AC\bot\)平面\(DBB_{1}D_{1}\),\par
因为\(BD_{1} \subset\)平面\(DBB_{1}D_{1}\),
所以\(AC\bot BD_{1}\),A正确;\par
对于选项B,\(AC = BD\),底面为矩形,
无法得到\(AC\bot\)平面\(DBB_{1}D_{1}\),B错误;\par
对于选项C,设\(AC\)与\(BD\)交于点\(O\),
由\(A_{1}D = A_{1}B\),\(O\)为\(BD\)中点,
得\(A_{1}O\bot BD\),\par
因为\(AA_{1}//BB_{1}\),
\(BB_{1}\bot\)平面\(ABCD\),
所以\(AA_{1}\bot\)平面\(ABCD\),\par
因为\(BD \subset\)平面\(ABCD\),
所以\(A_{1}A\bot BD\),\par
因为\(AA_{1} \cap A_{1}O = A_{1}\),
所以\(BD\bot\)平面\(A_{1}ACC_{1}\),C正确;\par
对于选项D,因为\(AD = AA_{1}\),
所以平面\(ADD_{1}A_{1}\)为正方形,
所以\(AD_{1}\bot A_{1}D\),
而\(AD_{1}\)与\(A_{1}B_{1}\)不一定垂直,D错误.}
\end{question}

\begin{question}
如果两个椭圆的离心率相等,
我们称这两个椭圆为"相似椭圆".已知椭圆\(C_{1}:\frac{x^{2}}{4} + \frac{y^{2}}{2} = 1\)和椭圆\(C_{2}:\frac{x^{2}}{2} + y^{2} = 1,P\)为\(C_{1}\)上一点,
过\(P\)点作\(C_{2}\)的两条切线交\(C_{1}\)于\(A,B\),
切点分别为\(M\),\(N\),则(  )
\begin{choices}
  \item \(C_{1}\)与\(C_{2}\)是相似椭圆
  \item \(M\)为\(PA\)中点
  \item \(\overrightarrow{PA} \cdot \overrightarrow{PN} = \overrightarrow{PB} \cdot \overrightarrow{PM}\)
  \item \(|PA| \cdot |PB|\)为定值
\end{choices}
\topics{求椭圆的离心率或离心率的取值范围;椭圆中的定值问题}
\difficulty{0.65}
\answer{ABC}
\explain{对于A,
\(e_{1} = \frac{c_{1}}{a_{1}} = \sqrt{\frac{a_{1}^{2} - b_{1}^{2}}{a_{1}^{2}}} = \sqrt{\frac{4 - 2}{4}} = \frac{\sqrt{2}}{2},e_{2} = = \sqrt{\frac{a_{2}^{2} - b_{2}^{2}}{a_{2}^{2}}} = \sqrt{\frac{2 - 1}{2}} = \frac{\sqrt{2}}{2}\),
故\(C_{1}\)与\(C_{2}\)是相似椭圆.故A正确;\par
对于B,设\(M\left( x_{0},y_{0} \right)\),
则\(l_{PA}:\frac{x_{0}x}{2} + y_{0}y = 1\),
与\(C_{1}:\frac{x^{2}}{4} + \frac{y^{2}}{2} = 1\)联立,\par
消去\(y\)得\(\left( \frac{x_{0}^{2}}{8y_{0}^{2}} + \frac{1}{4} \right)x^{2} - \frac{x_{0}}{2y_{0}^{2}}x + \frac{1 - 2y_{0}^{2}}{2y_{0}^{2}} = 0\),\par
所以\(x_{A} + x_{P} = \frac{4x_{0}}{x_{0}^{2} + 2y_{0}^{2}}\),
由于\(y_{0}^{2} = 1 - \frac{x_{0}^{2}}{2}\),
代入化简得\(x_{A} + x_{P} = 2x_{0}\),\par
又点\(M\)为直线\(PA\)与\(C_{2}\)的切点,
故\(M\)为\(PA\)的中点.故B正确;\par
对于C,由\(\text{B}\)可知\(M\)为\(PA\)中点,
同理\(N\)为\(PB\)中点,故\(MN//AB\),\par
且\(\bigtriangleup PMN\)和\(\bigtriangleup PAB\)相似,
所以\(|PM| \cdot |PB| = |PN| \cdot |PA|\),\par
因为\(\overrightarrow{PA} \cdot \overrightarrow{PN} = |PA| \cdot |PN| \cdot \cos\angle APN,\overrightarrow{PB} \cdot \overrightarrow{PM} = |PB| \cdot |PM| \cdot \cos\angle BPM\),\par
而\(\angle APN = \angle BPM\),
故\(\overrightarrow{PA} \cdot \overrightarrow{PN} = \overrightarrow{PB} \cdot \overrightarrow{PM}\).故C正确;\par
对于D,如下左图,若取\(P_{1}(2,0)\),
则易得\(A,B\)为上、下顶点,
不妨取\(A(0,\sqrt{2}),B(0, - \sqrt{2})\),
此时\(|P_{1}A| \cdot |P_{1}B| = \sqrt{2^{2} + {(\sqrt{2})}^{2}} \times \sqrt{2^{2} + {(\sqrt{2})}^{2}} = 6\);\par
若取\(P_{2}\left( 0,\sqrt{2} \right)\),
则易得\(A,B\)为左、右顶点,不妨取\(A(2,0),B( - 2,0)\),
此时\(|P_{2}A| \cdot |P_{2}B| = \sqrt{2^{2} + {(\sqrt{2})}^{2}} \times \sqrt{2^{2} + {(\sqrt{2})}^{2}} = 6\),
即\(|P_{1}A| \cdot \left| P_{1}B \right| = \left| P_{2}A \right| \cdot \left| P_{2}B \right| = 6\);\par
如下右图,
再取\(P_{3}\left( \sqrt{2},1 \right),P_{4}\left( - \sqrt{2}, - 1 \right)\),
此时\(P_{3}AP_{4}B\)围成一个矩形,
其中\(A( - \sqrt{2},1),B(\sqrt{2}, - 1)\),\par
有\(\left| P_{3}A \right| \cdot \left| P_{3}B \right| = \left| P_{4}A \right| \cdot \left| P_{4}B \right| = 2\sqrt{2} \times 2 = 4\sqrt{2}\),
即\(|PA| \cdot |PB|\)不是定值.故D错误.\par
%
% IMAGE_TODO_START id=auto_67d8d3-Q10-img1 path=/Users/muryor/code/mynote/word\\_to\\_tex/output/figures/auto\\_67d8d3/media/image4.png width=60% inline=true question_index=10 sub_index=1
% CONTEXT_BEFORE: s 2 = 4$$,即$$|PA| \cdot |PB|$$不是定值.故D错误.
% CONTEXT_AFTER: [公式:754a54e6-5cf3-4024-a82c-a21ad3e05d5e]
\end{tikzpicture}
% IMAGE_TODO_END id=auto_67d8d3-Q10-img
1
[公式:754a54e6-5cf3-4024-a82c-a21ad3e05d5e]}
\end{question}

\begin{question}
在\(\bigtriangleup ABC\)中,
\(C > B > A,\sin^{2}A + \sin^{2}B + \sin^{2}C = 2,D\)是\(AB\)的中点,
则下列结论正确的是(  )
\begin{choices}
  \item \(\bigtriangleup ABC\)可以是钝角三角形
  \item \(\sin^{2}A + \sinB < \frac{\sqrt{2} + 1}{2}\)
  \item 若\(AB = 4\),则\(2AC + BC \in \left( 8,4\sqrt{5} \right\rbrack\)
  \item \(\tanB \cdot \tan\angle BDC > 1\)
\end{choices}
\topics{求含sinx(型)函数的值域和最值;解正切不等式;用和;差角的正切公式化简;求值}
\difficulty{0.4}
\answer{BCD}
\explain{对于A,
\(\sin^{2}A + \sin^{2}B + \sin^{2}C = \sin^{2}A + \sin^{2}B + \sin^{2}(A + B)= \sin^{2}A + \sin^{2}B + \sin^{2}A\cos^{2}B + 2\sinA\cosB\sinB\cosA + \sin^{2}B\cos^{2}A = 2\),\par
即\(2\sin A\cos B\sin B\cos A = 1 - \sin^{2}A + 1 - \sin^{2}B - \sin^{2}A\cos^{2}B - \sin^{2}B\cos^{2}A\),\par
所以\(2\sin A\cos B\sin B\cos A = \cos^{2}A + \cos^{2}B - \sin^{2}A\cos^{2}B - \sin^{2}B\cos^{2}A\)\par
所以\(2\sin A\cos B\sin B\cos A = \cos^{2}A\left( 1 - \sin^{2}B \right) + \cos^{2}B\left( 1 - \sin^{2}A \right)\),\par
即得\(2\sinA\cosB\sinB\cosA = 2\cos^{2}A\cos^{2}B\),
即\(2\cosA\cosB\cos(A + B) = 0\),\par
因为\(C > B > A\),
所以\(A \in \left( 0,\frac{\pi}{2} \right),B \in \left( 0,\frac{\pi}{2} \right)\),\par
故\(\cosA > 0,\cosB > 0\),
则\(\cos(A + B) = 0\),\par
所以\(A + B = \frac{\pi}{2},C =\frac{\pi}{2},B \in \left( \frac{\pi}{4},\frac{\pi}{2} \right),A \in \left( 0,\frac{\pi}{4} \right)\),\par
故\(\bigtriangleup ABC\)为直角三角形,故A错误;\par
对于B,由\(A + B = \frac{\pi}{2}\),
则\(\sin B = \cos A\),
所以\(\sin^{2}A + \sinB = 1 - \cos^{2}A + \cosA\),\par
由于\(\cosA \in \left( \frac{\sqrt{2}}{2},1 \right)\),
故\(1 - \cos^{2}A + \cosA = - \left( \cos A - \frac{1}{2} \right)^{2} + \frac{5}{4} < \frac{\sqrt{2} + 1}{2}\),
故B正确;\par
对于C,由\(C = \frac{\pi}{2}\),
则\(AC^{2} + BC^{2} = AB^{2} = 16\),\par
所以\(2AC + BC = 8\sinB + 4\sinA = 8\cosA + 4\sinA = 4\sqrt{5}\sin(A + \alpha)\),\par
其中\(\sin\alpha = \frac{2\sqrt{5}}{5},\cos\alpha = \frac{\sqrt{5}}{5},A + \alpha \in \left( \alpha,\frac{\pi}{4} + \alpha \right)\),\par
又\(\alpha > \frac{\pi}{4},\sin\left( \frac{\pi}{4} + \alpha \right) = \frac{3\sqrt{10}}{10} > \frac{2\sqrt{5}}{5} = \sin\alpha\),\par
故\(2AC + BC \in \left( 8,4\sqrt{5} \right\rbrack\),
故C正确;\par
对于D,
\(\angle B = \angle BCD > \frac{\pi}{4},\angle BDC < \frac{\pi}{2}\),
所以\(\bigtriangleup BCD\)必是锐角三角形,\par
即证\(\tan\angle BDC > \frac{1}{\tanB} = \tan\left( \frac{\pi}{2} - B \right)\),\par
又\(\frac{\pi}{2} - B \in \left( 0,\frac{\pi}{2} \right)\),
即证\(\angle BDC > \frac{\pi}{2} - B\),
即\(\angle BDC + B > \frac{\pi}{2}\),成立,故D正确}
\end{question}

\section{填空题}

\begin{question}
已知\(\text{i}^{2025} = a - b\text{i})(\text{i}\)为虚数单位,
\(a,b \in R\)),则\(a + b\)的值为
.
\topics{复数的相等;复数的乘方}
\difficulty{0.85}
\answer{\(- 1\)}
\explain{因为\(\text{i}^{2025} = \text{i}^{4 \times 506 + 1} = \text{i}\),
由复数相等的充要条件得\(a = 0,b = - 1\),\par
所以\(a + b = - 1\).\(- 1\).}
\end{question}

\begin{question}
若数列\(\left\{ a_{n} \right\}\)满足\(a_{1} = a_{2} = 1,a_{n} + a_{n + 1} + a_{n + 2} = n^{2}\left( n \in N^{\text{*}} \right)\),
则\(a_{20} =\)
.
\topics{根据数列递推公式写出数列的项;由递推数列研究数列的有关性质;求等差数列前n项和}
\difficulty{0.65}
\answer{121}
\explain{由题意可得\(\left\{ \begin{array}{r}
a_{n} + a_{n + 1} + a_{n + 2} = n^{2} \\
a_{n + 1} + a_{n + 2} + a_{n + 3} = (n + 1)^{2}
\end{array} \right.\),作差得\(a_{n + 3} - a_{n} = 2n + 1\),\par
故\(a_{20} = \left( a_{20} - a_{17} \right) + \left( a_{17} - a_{14} \right) + \cdots + \left( a_{5} - a_{2} \right) + a_{2} = (2 \times 17 + 1) + (2 \times 14 + 1) + \cdots + (2 \times 2 + 1) + 1= \frac{(35 + 5)}{2} \times 6 + 1 = 121\).121.}
\end{question}

\begin{question}
某同学每次投篮命中的概率为0.8,且各次投篮是否投中相互独立,该同学若出现连续投中两次的情况,则停止投掷,那么投篮总次数的数学期望为
.
\topics{独立事件的乘法公式;求离散型随机变量的均值}
\difficulty{0.65}
\answer{\(\frac{45}{16}\)/\(2.8125\)}
\explain{设投篮总次数的数学期望为\(E(X)\),\par
若第一次没有投中,则后续需重新投篮,
且后续重新投篮的总次数的数学期望仍为\(E(X)\),\par
此情况下发生的概率为0.2,投篮总次数为\(1 + E(X)\),\par
若第一次投中,且第二次没有投中,则后续需重新投篮,
且后续重新投篮的总次数的数学期望仍为\(E(X)\),\par
此情况发生的概率为\(0.8 \times 0.2\),
投篮总次数为\(2 + E(X)\),\par
若第一次投中,第二次投中,
则此情况发生的概率为\(0.8 \times 0.8\),投篮总次数为2,\par
则投篮总次数的数学期望为\(0.2 \times \left( 1 + E(X) \right) + 0.8 \times 0.2 \times \left( 2 + E(X) \right) + 0.8 \times 0.8 \times 2 = E(X)\),\par
解得\(E(X) = \frac{45}{16}\)\(\frac{45}{16}\)}
\end{question}

\section{解答题}

\begin{question}
在锐角\(\bigtriangleup ABC\)中,
内角\(A,B,C\)的对边分别是\(a,b,c\),
且满足\(a^{2} + c^{2} - ac = b^{2}\).
\begin{enumerate}[label=(\arabic*)]
  \item 求角\(B\)的大小;
  \item 若\(b = 1\),求\(a\)的取值范围.
\end{enumerate}
\topics{余弦定理解三角形;求三角形中的边长或周长的最值或范围}
\difficulty{0.65}
\answer{(1)\(B = \frac{\pi}{3}\)
(2)\(\left( \frac{\sqrt{3}}{3},\frac{2\sqrt{3}}{3} \right)\)}
\explain{(1)因为\(a^{2} + c^{2} - ac = b^{2}\),
则\(ac = a^{2} + c^{2} - b^{2}\),\par
则\(\cosB = \frac{a^{2} + c^{2} - b^{2}}{2ac} = \frac{ac}{2ac} = \frac{1}{2}\),\par
因为\(B \in \left( 0,\frac{\pi}{2} \right)\),
所以\(B = \frac{\pi}{3}\);\par
(2)因为\(B = \frac{\pi}{3}\),
所以\(A + C = \pi - B = \frac{2\pi}{3}\),
则\(C = \frac{2\pi}{3} - A\),\par
由正弦定理,
得\(\frac{a}{\sinA} = \frac{b}{\sinB}\),\par
所以\(a = \frac{b}{\sinB} \cdot \sinA = \frac{2\sqrt{3}}{3}\sinA\),\par
因为\(\bigtriangleup ABC\)为锐角三角形,\par
所以\(\left\{ \begin{array}{r}
0 < A < \frac{\pi}{2} \\
0 < \frac{2\pi}{3} - A < \frac{\pi}{2}
\end{array} \right.\),解得\(\frac{\pi}{6} < A < \frac{\pi}{2}\),\par
所以\(\frac{1}{2} < \sinA < 1\),所以\(\frac{\sqrt{3}}{3} < \frac{2\sqrt{3}}{3}\sinA < \frac{2\sqrt{3}}{3}\),\par
所以\(\frac{\sqrt{3}}{3} < a < \frac{2\sqrt{3}}{3}\),即\(a\)的取值范围为\(\left( \frac{\sqrt{3}}{3},\frac{2\sqrt{3}}{3} \right)\).}
\end{question}

\begin{question}
在多面体\(ABCDE\)中,
平面\(ABE\bot\)平面\(BCE,BC\bot CE,\angle EBC = 30^{\circ},BE = 2, \bigtriangleup ABE\)是以\(BE\)为斜边的等腰直角三角形,
四边形\(ABCD\)为平行四边形,
\(M,N\)分别为\(AD,BE\)的中点.
\begin{enumerate}[label=(\arabic*)]
  \item 求证:\(MN\bot CE\);
  \item 求平面\(ABC\)与平面\(ACE\)的夹角的余弦值.
\end{enumerate}

\begin{center}
% IMAGE_TODO_START id=auto_67d8d3-Q16-img1 path=/Users/muryor/code/mynote/word\\_to\\_tex/output/figures/auto\\_67d8d3/media/image6.png width=60% inline=false question_index=16 sub_index=1
% CONTEXT_BEFORE: 腰直角三角形,四边形$$ABCD$$为平行四边形,$$M,N$$分别为$$AD,BE$$的中点.
% CONTEXT_AFTER: (1)求证:$$MN\bot CE$$; (2)求平面$$ABC$$与平面$$ACE$$的夹角
\begin{tikzpicture}[scale=1.05,>=Stealth,line cap=round,line join=round]
  % TODO: AI_AGENT_REPLACE_ME (id=auto_67d8d3-Q16-img1)
\end{tikzpicture}
% IMAGE_TODO_END id=auto_67d8d3-Q16-img
1
\end{center}

\topics{线面垂直证明线线垂直;面面垂直证线面垂直;面面角的向量求法}
\difficulty{0.4}
\answer{(1)证明见解析;
(2)\(\frac{\sqrt{105}}{35}\).}
\explain{(1)连接\(AN\),\par
%
% IMAGE_TODO_START id=auto_67d8d3-Q16-img2 path=/Users/muryor/code/mynote/word\\_to\\_tex/output/figures/auto\\_67d8d3/media/image7.png width=60% inline=true question_index=16 sub_index=1
% CONTEXT_BEFORE: 线垂直; (2)利用空间向量法来求二面角的夹角余弦值即可. 【详解】(1)连接$$AN$$,
% CONTEXT_AFTER: $$\because \bigtriangleup ABE$$是以$$BE$$为斜边的等腰直角三角形
\begin{tikzpicture}[scale=0.8,baseline=-0.5ex]
  % TODO: AI_AGENT_REPLACE_ME (id=auto_67d8d3-Q16-img2)
\end{tikzpicture}
% IMAGE_TODO_END id=auto_67d8d3-Q16-img
2
\(\because \bigtriangleup ABE\)是以\(BE\)为斜边的等腰直角三角形,
\(N\)为\(BE\)的中点,\par
\(\therefore AN\bot BE\),\par
又平面\(ABE\bot\)平面\(BCE\),
平面\(ABE \cap\)平面\(BCE = BE,AN \subset\)平面\(ABE\),\par
\(\therefore AN\bot\)平面\(BCE\),
又\(\because CE \subset\)平面\(BCE\),\par
\(\therefore AN\bot CE\),\par
\(\because\)四边形\(ABCD\)为平行四边形,
\(M\)为\(AD\)的中点,\par
\(\therefore BC//AD,BC//AM\),\par
又\(\because CE\bot BC\),
\(\therefore CE\bot AM\),\par
又\(AN \cap AM = A\),
\(AN,AM \subset\)平面\(AMN\),\par
\(\therefore CE\bot\)平面\(AMN\),
又\(\because MN \subset\)平面\(AMN\),\par
\(\therefore CE\bot MN\).\par
(2)在平面\(BCE\)中,
过\(N\)作\(NF\bot BE\)交\(BC\)于点\(F\),
结合(1)易知\(NF,NE,NA\)两两垂直.所以以\(NF,NE,NA\)分别为\(x\)轴、\(y\)轴、\(z\)轴建立空间直角坐标系如图所示.\par
%
% IMAGE_TODO_START id=auto_67d8d3-Q16-img3 path=/Users/muryor/code/mynote/word\\_to\\_tex/output/figures/auto\\_67d8d3/media/image8.png width=60% inline=true question_index=16 sub_index=1
% CONTEXT_BEFORE: $NF,NE,NA$$分别为$$x$$轴、$$y$$轴、$$z$$轴建立空间直角坐标系如图所示.
% CONTEXT_AFTER: $$\because\angle EBC = 30^{\circ},BE = 2$$,$$\ther
\begin{tikzpicture}[scale=0.8,baseline=-0.5ex]
  % TODO: AI_AGENT_REPLACE_ME (id=auto_67d8d3-Q16-img3)
\end{tikzpicture}
% IMAGE_TODO_END id=auto_67d8d3-Q16-img
3
\(\because\angle EBC = 30^{\circ},BE = 2\),
\(\therefore AN = BN = NE = \frac{1}{2}BE = 1,CE = \frac{1}{2}BE = 1,\angle BEC = 60^{\circ}\),\par
则\(N(0,0,0),A(0,0,1),B(0, - 1,0),E(0,1,0),C\left( \frac{\sqrt{3}}{2},\frac{1}{2},0 \right)\),\par
\(\therefore\overrightarrow{AE} = (0,1, - 1),\overrightarrow{CE} = \left( - \frac{\sqrt{3}}{2},\frac{1}{2},0 \right),\overrightarrow{BA} = (0,1,1),\overrightarrow{BC} = \left( \frac{\sqrt{3}}{2},\frac{3}{2},0 \right)\).\par
设平面\(ACE\)的法向量为\(\overrightarrow{n} = (x,y,z)\),\par
\(\therefore\left\{ \begin{array}{r}
\overrightarrow{n} \cdot \overrightarrow{CE} = 0 \\
\overrightarrow{n} \cdot \overrightarrow{AE} = 0
\end{array} \right.\)\  \Rightarrow \left\{ \begin{array}{r}
 - \frac{\sqrt{3}}{2}x + \frac{1}{2}y = 0, \\
y - z = 0,
\end{array} \right.取\(x = 1\),则\(y = \sqrt{3},z = \sqrt{3}\),\par
故\(\overrightarrow{n} = \left( 1,\sqrt{3},\sqrt{3} \right)\)是平面\(ACE\)的一个法向量,\par
设平面\(ABC\)的法向量为\(\overrightarrow{m} = \left( x_{1},y_{1},z_{1} \right)\),\par
\(\therefore\left\{ \begin{array}{r}
\overrightarrow{m} \cdot \overrightarrow{BA} = 0 \\
\overrightarrow{m} \cdot \overrightarrow{BC} = 0
\end{array} \right.\)\  \Rightarrow \left\{ \begin{array}{r}
y_{1} + z_{1} = 0, \\
\frac{\sqrt{3}}{2}x_{1} + \frac{3}{2}y_{1} = 0,
\end{array} \right.取\(y_{1} = - 1\),则\(x_{1} = \sqrt{3},z_{1} = 1\),\par
故平面\(ABC\)的法向量为\(\overrightarrow{m} = \left( \sqrt{3}, - 1,1 \right)\),\par
设平面\(ABC\)与平面\(ACE\)的夹角为\(\alpha\),\par
则\(\cos\alpha = \frac{\left| \overrightarrow{m} \cdot \overrightarrow{n} \right|}{\left| \overrightarrow{m} \right|\left| \overrightarrow{n} \right|} = \frac{\left| 1 \times \sqrt{3} + \sqrt{3} \times ( - 1) + \sqrt{3} \times 1 \right|}{\sqrt{1^{2} + \left( \sqrt{3} \right)^{2} + \left( \sqrt{3} \right)^{2}} \times \sqrt{\left( \sqrt{3} \right)^{2} + ( - 1)^{2} + 1^{2}}} = \frac{\sqrt{105}}{35}\),\par
故平面\(ABC\)与平面\(ACE\)的夹角的余弦值为\(\frac{\sqrt{105}}{35}\).}
\end{question}

\begin{question}
已知函数\(f(x) = \lnx + ax^{2},g(x) = \mathrm{e}^{x} - ax^{2},a \in R\).
\begin{enumerate}[label=(\arabic*)]
  \item 讨论\(f(x)\)的单调性;
  \item 若\(g(2x) \geq 4x^{2}\left\lbrack f(x) + \frac{1}{x} - ax^{2} \right\rbrack\)恒成立,
\item 求实数\(a\)的取值范围.
\end{enumerate}
\topics{由导数求函数的最值(不含参);利用导数研究不等式恒成立问题;利用导数求函数(含参)的单调区间}
\difficulty{0.65}
\answer{(1)答案见解析
(2)\(\left( - \infty,\frac{\mathrm{e}^{2}}{4} - 1 \right\rbrack\)}
\explain{(1)\(f(x)\)定义域为\((0, + \infty)\),
\(f'(x) = \frac{1}{x} + 2ax = \frac{2ax^{2} + 1}{x}\),\par
①当\(a \geq 0\)时,\(f'(x) > 0\)恒成立,
故\(f(x)\)在\((0, + \infty)\)上单调递增;\par
②当\(a < 0\)时,
令\(f'(x) > 0\)有\(2ax^{2} + 1 > 0\),
解得\(- \sqrt{- \frac{1}{2a}} < x < \sqrt{- \frac{1}{2a}}\),
又\(\because x > 0,\therefore 0 < x < \sqrt{- \frac{1}{2a}}\),\par
令\(f'(x) < 0\),
解得\(x > \sqrt{- \frac{1}{2a}}\),\par
则\(f(x)\)在\(\left( 0,\sqrt{- \frac{1}{2a}} \right)\)上单调递增,
在\(\left( \sqrt{- \frac{1}{2a}}, + \infty \right)\)上单调递减,\par
综上,当\(a \geq 0\)时,
\(f(x)\)在\((0, + \infty)\)上单调递增;\par
当\(a < 0\)时,
\(f(x)\)在\(\left( 0,\sqrt{- \frac{1}{2a}} \right)\)上单调递增,
在\(\left( \sqrt{- \frac{1}{2a}}, + \infty \right)\)上单调递减.\par
(2)由题\(g(2x) = \mathrm{e}^{2x} - 4ax^{2},4x^{2}\left\lbrack f(x) + \frac{1}{x} - ax^{2} \right\rbrack = 4x^{2}\lnx + 4x\),\par
所以\(g(2x) \geq 4x^{2}\left\lbrack f(x) + \frac{1}{x} - ax^{2} \right\rbrack\)恒成立等价于\(a \leq \frac{\mathrm{e}^{2x}}{4x^{2}} - \lnx - \frac{1}{x}\)对任意\(x > 0\)恒成立,\par
令\(h(x) = \frac{\mathrm{e}^{2x}}{4x^{2}} - \lnx - \frac{1}{x}(x > 0)\),\par
则\(h'(x) = \frac{(x - 1)\mathrm{e}^{2x}}{2x^{3}} - \frac{1}{x} + \frac{1}{x^{2}} = \frac{(x - 1)\left( \mathrm{e}^{2x} - 2x \right)}{2x^{3}}\),\par
令\(t(x) = \mathrm{e}^{2x} - 2x(x > 0)\),
则\(t'(x) = 2\left( \mathrm{e}^{2x} - 1 \right) > 0\),\par
即\(t(x)\)在\((0, + \infty)\)上单调递增,
故\(t(x) > t(0) = 1 > 0\),\par
令\(h'(x) = 0\)有\(x = 1\),\par
当\(0 < x < 1\)时,\(h'(x) < 0\),
此时\(h(x)\)单调递减;当\(x > 1\)时,\(h'(x) > 0\),
此时\(h(x)\)单调递增,\par
则\(x = 1\)为\(h(x)\)唯一的极小值点,也是最小值点,\par
故\(h(x)_{\min} = h(1) = \frac{\mathrm{e}^{2}}{4} - 1\),
从而\(a \leq \frac{\mathrm{e}^{2}}{4} - 1\),\par
因此实数\(a\)的取值范围为\(\left( - \infty,\frac{\mathrm{e}^{2}}{4} - 1 \right\rbrack\).}
\end{question}

\begin{question}
张明在暑假为了锻炼身体,
制定了一项坚持晨跑的计划:30天晨跑训练.规则如下:张明从第1天开始晨跑,
若第\(i\)天晨跑,
则他第\((i + 1)\)天晨跑的概率为\(\frac{1}{4}\),
且他不能连续两天没有晨跑.设他第\(n\)天晨跑的概率为\(P_{n}(1 \leq n \leq 30,n \in N)\).
\begin{enumerate}[label=(\arabic*)]
  \item 求\(P_{1},P_{2},P_{3}\)的值;
  \item 求数列\(\left\{ P_{n} \right\}\)的通项公式;
  \item 若\(X,Y\)都是离散型随机变量,
\item 则\(E(X + Y) = E(X) + E(Y)\),
\item 记张明前\(n\)天晨跑的天数为\(X\),求\(E(X)\).
\end{enumerate}
\topics{写出等比数列的通项公式;独立事件的乘法公式;求离散型随机变量的均值}
\difficulty{0.4}
\answer{(1)\(P_{1} = 1,P_{2} = \frac{1}{4},P_{3} = \frac{13}{16}\);
(2)\(P_{n} = \frac{4}{7} + \frac{3}{7} \times \left( - \frac{3}{4} \right)^{n - 1}(1 \leq n \leq 30)\)(或\(P_{n} = \frac{4}{7} - \frac{4}{7} \times \left( - \frac{3}{4} \right)^{n}\));
(3)\(\frac{4n}{7} + \frac{12}{49}\left\lbrack 1 - \left( - \frac{3}{4} \right)^{n} \right\rbrack(1 \leq n \leq 30).)\)}
\explain{(1)已知第1天一定晨跑,故\(P_{1} = 1\),\par
第2天晨跑的概率由第1天晨跑决定,
故\(P_{2} = \frac{1}{4}\),\par
第3天晨跑的情况分两种:\par
第1天晨跑,第2天不晨跑,第3天晨跑,
概率为\(1 \times \left( 1 - \frac{1}{4} \right) \times 1 = \frac{3}{4}\),\par
第1天晨跑,第2天晨跑,第3天晨跑,
概率为\(1 \times \frac{1}{4} \times \frac{1}{4} = \frac{1}{16}\),\par
故\(P_{3} = 1 \times \left( 1 - \frac{1}{4} \right) \times 1 + 1 \times \frac{1}{4} \times \frac{1}{4} = \frac{13}{16}\).\par
(2)由题意得,张明第\((n - 2)\)天晨跑后,
下一次晨跑在第\(n\)天的概率为\(\frac{3}{4}P_{n - 2}\),\par
张明第\((n - 1)\)天晨跑后,
再在第\(n\)天晨跑的概率为\(\frac{1}{4}P_{n - 1}\),\par
所以\(P_{n} = \frac{1}{4}P_{n - 1} + \frac{3}{4}P_{n - 2}(n \geq 3)\),\par
即\(4P_{n} = P_{n - 1} + 3P_{n - 2}(n \geq 3)\),
则\(4P_{n} = 4P_{n - 1} - 3P_{n - 1} + 3P_{n - 2}\),\par
所以\(4\left( P_{n} - P_{n - 1} \right) = - 3\left( P_{n - 1} - P_{n - 2} \right)\),
即\(\frac{P_{n} - P_{n - 1}}{P_{n - 1} - P_{n - 2}} = - \frac{3}{4}(n \geq 3)\),\par
所以\(\left\{ P_{n + 1} - P_{n} \right\}\)是以\(P_{2} - P_{1}\)为首项,
\(- \frac{3}{4}\)为公比的等比数列.\par
由(1)得,\(P_{1} = 1\),
\(P_{2} = \frac{1}{4}\),
所以\(P_{2} - P_{1} = - \frac{3}{4}\),\par
所以\(P_{n} - P_{n - 1} = \left( - \frac{3}{4} \right)^{n - 1}\),\par
则\(P_{2} - P_{1} = \left( - \frac{3}{4} \right)^{1},P_{3} - P_{2} = \left( - \frac{3}{4} \right)^{2},\cdots ,P_{n} - P_{n - 1} = \left( - \frac{3}{4} \right)^{n - 1}\),\par
所以\(P_{n} = P_{1} + \left( - \frac{3}{4} \right) + \left( - \frac{3}{4} \right)^{2} + \cdots + \left( - \frac{3}{4} \right)^{n - 1} = 1 + \frac{\left( - \frac{3}{4} \right) \times \left\lbrack 1 - \left( - \frac{3}{4} \right)^{n - 1} \right\rbrack}{1 - \left( - \frac{3}{4} \right)}\),\par
所以\(P_{n} = \frac{4}{7} + \frac{3}{7} \times \left( - \frac{3}{4} \right)^{n - 1}(1 \leq n \leq 30)\).(或\(P_{n} = \frac{4}{7} - \frac{4}{7} \times \left( - \frac{3}{4} \right)^{n}\))\par
(3)记他前\(n\)天中,第\(i\)天晨跑的次数为\(X_{i}\).\par
由题意得,\(X_{i}\)服从两点分布,
且\(P\left( X_{i} = 1 \right) = P_{i}\),\par
因为\(X = X_{1} + X_{2} + \cdots + X_{n}\),
且对于离散型随机变量\(X,Y\),
都有\(E(X + Y) = E(X) + E(Y)\),\par
所以\(E(X) = E\left\lbrack X_{1} + \left( X_{2} + \cdots + X_{n} \right) \right\rbrack = E\left( X_{1} \right) + E\left( X_{2} + \cdots + X_{n} \right)= E\left( X_{1} \right) + E\left\lbrack X_{2} + \left( X_{3} + \cdots + X_{n} \right) \right\rbrack = E\left( X_{1} \right) + E\left( X_{2} \right) + E\left( X_{3} + \cdots + X_{n} \right) = \cdots= E\left( X_{1} \right) + E\left( X_{2} \right) + \cdots + E\left( X_{n} \right)\),\par
所以\(E(X) = P_{1} + P_{2} + \cdots + P_{n}\),\par
所以\(E(X) = 1 + \left\lbrack \frac{4}{7} + \frac{3}{7} \times \left( - \frac{3}{4} \right)^{1} \right\rbrack + \left\lbrack \frac{4}{7} + \frac{3}{7} \times \left( - \frac{3}{4} \right)^{2} \right\rbrack + \cdots + \left\lbrack \frac{4}{7} + \frac{3}{7} \times \left( - \frac{3}{4} \right)^{n - 1} \right\rbrack= 1 + \frac{4(n - 1)}{7} + \frac{3}{7}\lbrack\left( - \frac{3}{4} \right)^{1} + \left( - \frac{3}{4} \right)^{2} + \cdots + \left( - \frac{3}{4} \right)^{n - 1}\rbrack\)\par
所以\(E(X) = 1 + \frac{4(n - 1)}{7} - \frac{3}{7} \times \frac{3}{7}\left\lbrack 1 - \left( - \frac{3}{4} \right)^{n - 1} \right\rbrack = \frac{4n}{7} + \frac{9}{49} \times \left( - \frac{3}{4} \right)^{n - 1} + \frac{12}{49}(1 \leq n \leq 30)\).\par
(或\(E(X) = \left\lbrack \frac{4}{7} - \frac{4}{7} \times \left( - \frac{3}{4} \right)^{1} \right\rbrack + \left\lbrack \frac{4}{7} - \frac{4}{7} \times \left( - \frac{3}{4} \right)^{2} \right\rbrack + \cdots + \left\lbrack \frac{4}{7} - \frac{4}{7} \times \left( - \frac{3}{4} \right)^{n} \right\rbrack= \frac{4n}{7} - \frac{4}{7} \times \frac{\left( - \frac{3}{4} \right)\left\lbrack 1 - \left( - \frac{3}{4} \right)^{n} \right\rbrack}{1 - \left( - \frac{3}{4} \right)} = \frac{4n}{7} + \frac{12}{49}\left\lbrack 1 - \left( - \frac{3}{4} \right)^{n} \right\rbrack(1 \leq n \leq 30).)\)}
\end{question}

\begin{question}
已知双曲线\(\text{Γ}:\frac{x^{2}}{a^{2}} - \frac{y^{2}}{b^{2}} = 1(a > 0,b > 0)\)的焦距为\(2\sqrt{3}\),
其中一条渐近线方程为\(\sqrt{2}x - y= 0,P,Q\)为双曲线\(\text{Γ}\)的左、右顶点.
\begin{enumerate}[label=(\arabic*)]
  \item 求双曲线\(\text{Γ}\)的方程.
  \item 过点\(P\)作动圆\(D\)(以\((0,a)\)为圆心)的两条切线分别交双曲线\(\text{Γ}\)于异于点\(P\)的\(B\),
\item \(C\)两点,试判断直线\(BC\)是否过定点?若是,请求出此定点的坐标;
\item 若不是,请说明理由.
  \item 已知动点\(H\)满足直线\(HP,HQ\)的斜率的乘积的绝对值为2,
\item 记动点\(H\)的轨迹为曲线\(G\).过点\(Q\)作直线\(l_{1},l_{2}\)交曲线\(G\)分别于\(M,N\)和\(E,F\)(其中\(M,E\)的横坐标的绝对值均大于1),
\item 求证:直线\(MF\)与\(NE\)的交点在定直线上.
\end{enumerate}

\begin{center}
% IMAGE_TODO_START id=auto_67d8d3-Q19-img1 path=/Users/muryor/code/mynote/word\\_to\\_tex/output/figures/auto\\_67d8d3/media/image9.png width=60% inline=false question_index=19 sub_index=1
% CONTEXT_BEFORE: . 故直线$$BC$$过定点$$\left( {3},0 \right)$$.
% CONTEXT_AFTER: (3)由(1)知$$P( - 1,0),Q(1,0)$$,设$$H(x,y)$$,依题意,$$\
\begin{tikzpicture}[scale=1.05,>=Stealth,line cap=round,line join=round]
  % TODO: AI_AGENT_REPLACE_ME (id=auto_67d8d3-Q19-img1)
\end{tikzpicture}
% IMAGE_TODO_END id=auto_67d8d3-Q19-img
1
\end{center}


\begin{center}
% IMAGE_TODO_START id=auto_67d8d3-Q19-img2 path=/Users/muryor/code/mynote/word\\_to\\_tex/output/figures/auto\\_67d8d3/media/image10.png width=60% inline=false question_index=19 sub_index=1
% CONTEXT_BEFORE: {- 2(m + n)}{m^{2} + n^{2}})$$,在定直线$$x = - 1$$上.
\begin{tikzpicture}[scale=1.05,>=Stealth,line cap=round,line join=round]
  % TODO: AI_AGENT_REPLACE_ME (id=auto_67d8d3-Q19-img2)
\end{tikzpicture}
% IMAGE_TODO_END id=auto_67d8d3-Q19-img
2
\end{center}

\topics{根据a;b;c求双曲线的标准方程;双曲线中的直线过定点问题;双曲线中的动点在定直线上问题}
\difficulty{0.15}
\answer{(1)\(x^{2} - \frac{y^{2}}{2} = 1\)
(2)是,\(\left( \frac{1}{3},0 \right)\)
(3)证明见解析}
\explain{(1)由双曲线\(\text{Γ}:\frac{x^{2}}{a^{2}} - \frac{y^{2}}{b^{2}} = 1(a > 0,b > 0)\)的焦距为\(2\sqrt{3}\)可得\(c = \sqrt{a^{2} + b^{2}} = \sqrt{3}\),\par
又其中一条渐近线方程为\(\sqrt{2}x - y= 0\),
则\(\frac{b}{a} = \sqrt{2}\),\par
解得\(a = 1,b = \sqrt{2}\),\par
所以双曲线\(\text{Γ}\)的方程为\(x^{2} - \frac{y^{2}}{2} = 1\).\par
(2)由题意,切线\(PB,PC\)的斜率都存在,
设过\(P\)点的切线\(l\)的方程为\(y = k(x + 1)\),
动圆\(D\)的半径为\(r(r \neq 1)\),
所以圆心\(D(0,1)\)到切线\(l\)的距离为\(\frac{|k - 1|}{\sqrt{k^{2} + 1}} = r\),\par
化简得\(\left( 1 - r^{2} \right)k^{2} - 2k + 1 - r^{2} = 0\),
则\(PB,PC\)的斜率\(k_{1},k_{2}\)是该方程的两个根,
可得\(k_{1} \cdot k_{2} = 1\).\par
设直线\(PB:y = k_{1}(x + 1),B\left( x_{1},y_{1} \right),C\left( x_{2},y_{2} \right)\),\par
联立方程\(\left\{ \begin{array}{r}
y = k_{1}(x + 1), \\
x^{2} - \frac{y^{2}}{2} = 1,
\end{array} \right.\)得\(\left( 2 - k_{1}^{2} \right)x^{2} - 2k_{1}^{2}x - k_{1}^{2} - 2 = 0\).\par
由韦达定理,\(( - 1) \cdot x_{1} = \frac{k_{1}^{2} + 2}{k_{1}^{2} - 2}\),则\(x_{1} = \frac{2 + k_{1}^{2}}{2 - k_{1}^{2}}\),将其代入\(y = k_{1}(x + 1)\)可得\(y_{1} = \frac{4k_{1}}{2 - k_{1}^{2}}\),\par
即得\(B\left( \frac{2 + k_{1}^{2}}{2 - k_{1}^{2}},\frac{4k_{1}}{2 - k_{1}^{2}} \right)\),同理可得\(C(\frac{2 + k_{2}^{2}}{2 - k_{2}^{2}},\frac{4k_{2}}{2 - k_{2}^{2}})\),因\(k_{2} = \frac{1}{k_{1}}\),则得\(C(\frac{2k_{1}^{2} + 1}{2k_{1}^{2} - 1},\frac{4k_{1}}{2k_{1}^{2} - 1})\)\par
又因为\(k_{BC} = \frac{y_{2} - y_{1}}{x_{2} - x_{1}} = \frac{\frac{4k_{1}}{2k_{1}^{2} - 1} - \frac{4k_{1}}{2 - k_{1}^{2}}}{\frac{2k_{1}^{2} + 1}{2k_{1}^{2} - 1} - \frac{2 + k_{1}^{2}}{2 - k_{1}^{2}}} = \frac{- 12k_{1}^{3} + 12k_{1}}{- 4k_{1}^{4} + 4} = \frac{3k_{1}}{k_{1}^{2} + 1}\),\par
所以直线\(BC\)的方程为\(y - \frac{4k_{1}}{2 - k_{1}^{2}} = \frac{3k_{1}}{k_{1}^{2} + 1}(x - \frac{2 + k_{1}^{2}}{2 - k_{1}^{2}})\),\par
法一:直线\(BC\)的方程可化为\par
\(y = \frac{3k_{1}}{k_{1}^{2} + 1}x + \frac{4k_{1}}{2 - k_{1}^{2}} - \frac{3k_{1}}{k_{1}^{2} + 1} \cdot \frac{2 + k_{1}^{2}}{2 - k_{1}^{2}}= \frac{3k_{1}}{k_{1}^{2} + 1}x + \frac{4k_{1}\left( k_{1}^{2} + 1 \right) - 3k_{1}\left( 2 + k_{1}^{2} \right)}{\left( 2 - k_{1}^{2} \right)\left( k_{1}^{2} + 1 \right)}= \frac{3k_{1}}{k_{1}^{2} + 1}x + \frac{k_{1}^{3} - 2k_{1}}{\left( 2 - k_{1}^{2} \right)\left( k_{1}^{2} + 1 \right)}= \frac{k_{1}}{k_{1}^{2} + 1}(3x - 1)\),\par
故直线\(BC\)过定点\(\left( \frac{1}{3},0 \right)\).\par
法二:根据双曲线的对称性,若定点存在,则一定在\(x\)轴上,不妨设为\(\left( x_{0},0 \right)\),\par
将\(\left( x_{0},0 \right)\)代入\(BC\)方程,得\(0 - \frac{4k_{1}}{2 - k_{1}^{2}} = \frac{3k_{1}}{k_{1}^{2} + 1}(x_{0} - \frac{2 + k_{1}^{2}}{2 - k_{1}^{2}})\),\par
化简整理,得\(\left( 1 - 3x_{0} \right)k_{1}^{2} + 6x_{0} - 2 = 0\),\par
因\(k_{1} \in \mathbb{R}\),故由\(\left\{ \begin{array}{r}
1 - 3x_{0} = 0 \\
6x_{0} - 2 = 0
\end{array} \right.\),解得\(x_{0} = \frac{1}{3}\).\par
故直线\(BC\)过定点\(\left( \frac{1}{3},0 \right)\).\par
(3)由(1)知\(P( - 1,0),Q(1,0)\),设\(H(x,y)\),依题意,\(\left| \frac{y}{x + 1} \cdot \frac{y}{x - 1} \right| = 2\),\par
化简得:\(y^{2} = 2\left| x^{2} - 1 \right|\),两边取平方,整理即得动点\(H\)的轨迹方程为\(G:\left( x^{2} - \frac{y^{2}}{2} - 1 \right)\left( x^{2} + \frac{y^{2}}{2} - 1 \right) = 0\),其中\(y \neq 0\).\par
由题意可设直线\(l_{2},l_{1}\)的方程分别为\(l_{2}:x = my + 1\)和\(l_{1}:x = ny + 1\),其中\(mn \neq 0\),\par
联立方程\(\left\{ \begin{array}{r}
x = my + 1, \\
x^{2} + \frac{y^{2}}{2} = 1
\end{array} \right.\)得\(\left( 2m^{2} + 1 \right)y^{2} + 4my = 0\),所以\(y_{F} = - \frac{4m}{2m^{2} + 1}\),\par
将\(y_{F}\)代入到直线\(x = my + 1\)得到\(F\left( \frac{1 - 2m^{2}}{1 + 2m^{2}}, - \frac{4m}{2m^{2} + 1} \right)\);\par
联立方程\(\left\{ \begin{array}{r}
x = my + 1, \\
x^{2} - \frac{y^{2}}{2} = 1
\end{array} \right.\)得\(\left( 2m^{2} - 1 \right)y^{2} + 4my = 0\),所以\(y_{E} = - \frac{4m}{2m^{2} - 1}\),\par
将\(y_{E}\)代入到直线\(x = my + 1\)得到\(E\left( \frac{1 + 2m^{2}}{1 - 2m^{2}},\frac{4m}{1 - 2m^{2}} \right)\),\par
同理可得\(N\left( \frac{1 - 2n^{2}}{1 + 2n^{2}}, - \frac{4n}{2n^{2} + 1} \right),M\left( \frac{1 + 2n^{2}}{1 - 2n^{2}},\frac{4n}{1 - 2n^{2}} \right)\).\par
将点\(F\),\(M\)同时向右平移一个单位长度,分别得到\(F'\left( \frac{2}{1 + 2m^{2}}, - \frac{4m}{2m^{2} + 1} \right)\),\(M'\left( \frac{2}{1 - 2n^{2}},\frac{4n}{1 - 2n^{2}} \right)\),直线\(M'F'\)与\(y\)轴交点\(A\)的纵坐标为\par
\(y_{A} = \frac{x_{M'}y_{F'} - x_{F'}y_{M'}}{x_{M'} - x_{F'}} = \frac{\frac{2}{1 - 2n^{2}} \cdot \left( \frac{- 4m}{2m^{2} + 1} \right) - \frac{2}{1 + 2m^{2}} \cdot \frac{4n}{1 - 2n^{2}}}{\frac{2}{1 - 2n^{2}} - \frac{2}{1 + 2m^{2}}} = \frac{- 2(m + n)}{m^{2} + n^{2}}\),\par
因此直线\(MF\)经过点\(( - 1,\frac{- 2(m + n)}{m^{2} + n^{2}})\),\par
同理可得(将\(m,n\)互换)直线\(NE\)也经过点\(( - 1,\frac{- 2(m + n)}{m^{2} + n^{2}})\),\par
所以直线\(MF\)与\(NE\)的交点为\(( - 1,\frac{- 2(m + n)}{m^{2} + n^{2}})\),在定直线\(x = - 1\)上.}
\end{question}
