\examxtitle{测试试卷 - auto_5c82df}

\section{单选题}

\begin{question}
若\(\frac{z + \text{i}}{z} = 2\text{i}\),则\(|z| =\)(   )
\begin{choices}
  \item \(\frac{3\sqrt{5}}{5}\)
  \item \(\frac{\sqrt{5}}{5}\)
  \item \(\sqrt{5}\)
  \item \(\frac{2\sqrt{5}}{5}\)
\end{choices}
\topics{求复数的模;复数的除法运算}
\difficulty{0.65}
\answer{B}
\explain{\(\because\frac{z + \text{i}}{z} = 2\text{i}\),
\(\therefore z = \frac{\text{i}}{2\text{i} - 1} = \frac{\left( - 2\text{i} - 1 \right)\text{i}}{\left( 2\text{i} - 1 \right)\left( - 2\text{i} - 1 \right)} = \frac{2 - \text{i}}{5} = \frac{2}{5} - \frac{1}{5}\text{i}\),
\(|z| = \sqrt{\left( \frac{2}{5} \right)^{2} + \left( - \frac{1}{5} \right)^{2}} = \frac{\sqrt{5}}{5}\)}
\end{question}

\begin{question}
设\(\tan\alpha = - \frac{1}{2}\),
则\(\frac{\sin^{2}\alpha + 1}{\sin\alpha\cos\alpha - 2\cos^{2}\alpha} =\)(   )
\begin{choices}
  \item \(- \frac{3}{5}\)
  \item \(\frac{3}{5}\)
  \item \(- 1\)
  \item 1
\end{choices}
\topics{正;余弦齐次式的计算;三角函数的化简;求值------同角三角函数基本关系}
\difficulty{0.85}
\answer{A}
\explain{因为\(\tan\alpha = - \frac{1}{2}\),\par
所以\(\frac{\sin^{2}\alpha + 1}{\sin\alpha\cos\alpha - 2\cos^{2}\alpha} = \frac{\text{2sin}^{2}\alpha + \cos^{2}\alpha}{\sin\alpha\cos\alpha - 2\cos^{2}\alpha}= \frac{\text{2tan}^{2}\alpha + 1}{\tan\alpha - 2}= \frac{2 \times \left( - \frac{1}{2} \right)^{2} + 1}{- \frac{1}{2} - 2} = \frac{\frac{3}{2}}{- \frac{5}{2}} = - \frac{3}{5}\)}
\end{question}

\begin{question}
已知\(0 < a < 1,0 < b < 1\),则"\(a < b\)"是"\(a\log_{a}b < b\)"成立的(   )
\begin{choices}
  \item 充要条件
  \item 充分不必要条件
  \item 必要不充分条件
  \item 既不充分也不必要条件
\end{choices}
\topics{充要条件的证明;用导数判断或证明已知函数的单调性;根据函数的单调性解不等式}
\difficulty{0.4}
\answer{A}
\explain{因为\(b > a\log_{a}b = \frac{a\lnb}{\lna}\),
因为\(0 < a < 1\),所以\(\ln a < 0\),
上式等价于\(\frac{\lna}{a} < \frac{\lnb}{b}\),\par
设\(f(x) = \frac{\lnx}{x}\),
则\(f'(x) = \frac{1 - \lnx}{x^{2}}\),\par
当\(x \in (0,1)\)时,有\(f'(x) > 0\),\par
所以\(f(x)\)在区间\((0,1)\)上单调递增,
所以\(f(a) < f(b)\)等价于\(a < b\),\par
故"\(a < b\)"是"\(a\log_{a}b < b\)"成立的充要条件}
\end{question}

\begin{question}
设全集为自然数集\(N\),
\(E = \left\{ x|x = 2n,n \in N \right\},F = \left\{ x|x = 4n,n \in N \right\}\).那么集合\(N\)可以表示成(    )
\begin{choices}
  \item \(E \cup F\)
  \item \(\overline{E} \cup F\)
  \item \(E \cup \overline{F}\)
  \item \(\overline{E} \cap \overline{F}\)
\end{choices}
\topics{交集的概念及运算;并集的概念及运算;交并补混合运算}
\difficulty{0.85}
\answer{C}
\explain{集合\(E = \{ x|x = 2n,n \in N\}\)是所有偶数的集合,
\(F = \{ x|x = 4n,n \in N\}\)是所有4的倍数的集合.\par
A选项,由于\(E \cup F = E\),所以A选项错误.\par
B选项,
由于\(2 \notin \left( \overline{E} \cup F \right)\),所以B选项错误.\par
C选项,由于\(F\)\(E\),
所以\(\left( E \cup \overline{F} \right) = \mathbb{N}\),
所以C选项正确.\par
D选项,
由于\(\overline{E} \cap \overline{F} = \overline{E \cup F} = \overline{E}\),
所以D选项错误}
\end{question}

\begin{question}
已知向量\(\overrightarrow{a} = (1,2),\overrightarrow{b} = (x,y)\),
若\(\overrightarrow{b}\)在\(\overrightarrow{a}\)上的投影向量是\(\frac{1}{5}\overrightarrow{a}\),
则\(x^{2} + y^{2}\)的最小值为(   )
\begin{choices}
  \item \(\frac{2}{9}\)
  \item \(\frac{1}{2}\)
  \item \(\frac{1}{4}\)
  \item \(\frac{1}{5}\)
\end{choices}
\topics{求二次函数的值域或最值;数量积的坐标表示;坐标计算向量的模;求投影向量}
\difficulty{0.65}
\answer{D}
\explain{\(\because\) 向量\(\overrightarrow{a} = (1,2),\overrightarrow{b} = (x,y)\),
\(\therefore\) \(\overrightarrow{a} \cdot \overrightarrow{b} = x + 2y,\left| \overrightarrow{a} \right| = \sqrt{1^{2} + 2^{2}} = \sqrt{5}\),\par
\(\therefore\) \(\overrightarrow{b}\)在\(\overrightarrow{a}\)上的投影向量是\(\left( \frac{\overrightarrow{a} \cdot \overrightarrow{b}}{\left| \overrightarrow{a} \right|} \right)\frac{\overrightarrow{a}}{\left| \overrightarrow{a} \right|} = \frac{x + 2y}{5}\overrightarrow{a} = \frac{1}{5}\overrightarrow{a}\),
\(\therefore\) \(x + 2y = 1\),\par
\(\therefore\) \(x^{2} + y^{2} = (1 - 2y)^{2} + y^{2} = 5y^{2} - 4y + 1 = 5\left( y - \frac{2}{5} \right)^{2} + \frac{1}{5}\),\par
\(\therefore\) 当\(y = \frac{2}{5},x = \frac{1}{5}\)时,
\(x^{2} + y^{2}\)取最小值\(\frac{1}{5}\)}
\end{question}

\begin{question}
若函数\(f(x) = \left\{ \begin{array}{r}
(1 - 4a)x + 1,x < 2, \\
a^{x} - 8a + \frac{11}{4},x \geq 2.
\end{array}(a > 0 \right.\)且\(a \neq 1)\)在\(\mathbb{R}\)上为减函数,则\(a\)的取值范围是(    )
\begin{choices}
  \item \(\left( \frac{1}{4},1 \right)\)
  \item \(\left( 0,\frac{1}{4} \right)\)
  \item \(\left( \frac{1}{4},\frac{1}{2} \right\rbrack\)
  \item \(\left\lbrack \frac{1}{2},1 \right)\)
\end{choices}
\topics{根据分段函数的单调性求参数;由指数(型)的单调性求参数}
\difficulty{0.4}
\answer{C}
\explain{当\(x < 2\)时,
\(f(x) = (1 - 4a)x + 1\)单调递减须满足\(1 - 4a < 0\),
解得\(a > \frac{1}{4}\),\par
当\(x \geq 2\)时,
\(f(x) = a^{x} - 8a + \frac{11}{4}\)单调递减须满足\(0 < a < 1\),\par
且\(f{(x)}_{\max} = f(2) = a^{2} - 8a + \frac{11}{4}\);\par
所以要使函数\(f(x)\)在\(\mathbb{R}\)上为减函数,须满足\par
\(\left\{ \begin{array}{r}
a > \frac{1}{4} \\
0 < a < 1 \\
3 - 8a \geq a^{2} - 8a + \frac{11}{4}
\end{array} \right.\),即\(\left\{ \begin{array}{r}
a > \frac{1}{4} \\
0 < a < 1 \\
 - \frac{1}{2} \leq a \leq \frac{1}{2}
\end{array} \right.\),解得\(\frac{1}{4} < a \leq \frac{1}{2}\),\par
所以\(a\)的取值范围是\(\left( \frac{1}{4},\frac{1}{2} \right\rbrack\)}
\end{question}

\begin{question}
一个锐角三角形的三边长成等差数列,则该三角形的最小内角余弦值的取值范围是(   )
\begin{choices}
  \item \(\left( 0,\frac{1}{2} \right\rbrack\)
  \item \(\left\lbrack \frac{1}{2},\frac{3}{5} \right)\)
  \item \(\left( \frac{1}{2},\frac{4}{5} \right)\)
  \item \(\left\lbrack \frac{1}{2},\frac{4}{5} \right)\)
\end{choices}
\topics{余弦定理解三角形;等差中项的应用}
\difficulty{0.4}
\answer{D}
\explain{由题意可设三角形的三边长为\(a,b,c\),
不妨设\(a \leq b \leq c\),\par
由于三边长成等差数列,故\(2b = a + c\),\par
由于三角形中,需满足\(a + b > c\),
(\(a + c > b,b + c > a\)恒成立),\par
结合\(c = 2b - a\),则\(a + b > 2b - a\),
得\(2a > b\);\par
又三角形为锐角三角形,需满足\(\cos C > 0\),\par
即\(\cos C = \frac{a^{2} + b^{2} - c^{2}}{2ab} = \frac{a^{2} + b^{2} - (2b - a)^{2}}{2ab} = \frac{4a - 3b}{2a} > 0\),
即\(4a - 3b > 0\),\par
即\(b < \frac{4}{3}a\),结合\(2a > b\),
可得\(a \leq b < \frac{4}{3}a\);\par
又\(\cos A = \frac{b^{2} + c^{2} - a^{2}}{2bc} = \frac{b^{2} + (2b - a)^{2} - a^{2}}{2b(2b - a)} = \frac{5b^{2} - 4ab}{4b^{2} - 2ab} = \frac{5b - 4a}{4b - 2a}\)\par
令\(t = \frac{b}{a}\),
则\(t \in \lbrack 1,\frac{4}{3})\),
故\(\frac{5b - 4a}{4b - 2a} = \frac{5 \times \frac{b}{a} - 4}{2\left( \frac{2b}{a} - 1 \right)} = \frac{5t - 4}{2(2t - 1)} = \frac{5}{4} - \frac{3}{4(2t - 1)}\),\par
由于\(y = 2t - 1\)在\(t \in \lbrack 1,\frac{4}{3})\)时单调递增,
故\(y = \frac{5}{4} - \frac{3}{4(2t - 1)}\)在\(\lbrack 1,\frac{4}{3})\)上单调递增,\par
故当\(t = 1\)时,
\(y = \frac{5}{4} - \frac{3}{4(2t - 1)}\)取最小值\(\frac{5}{4} - \frac{3}{4(2 \times 1 - 1)} = \frac{1}{2}\),\par
当\(t < \frac{4}{3}\)时,
\(\frac{5}{4} - \frac{3}{4(2t - 1)} < \frac{5}{4} - \frac{3}{4\left( 2 \times \frac{4}{3} - 1 \right)} = \frac{4}{5}\),\par
故该三角形的最小内角余弦值的取值范围是\(\left\lbrack \frac{1}{2},\frac{4}{5} \right)\)}
\end{question}

\begin{question}
已知函数\(f(x) = \mathrm{e}^{- x}\),
若曲线\(y = f(x)\)在\(x = 0\)处的切线交\(x\)轴于点\(\left( a_{1},0 \right)\),
在\(x = a_{1}\)处的切线交\(x\)轴于点\(\left( a_{2},0 \right)\),
依此类推,
曲线\(y = f(x)\)在\(x = a_{n - 1}\)处的切线交\(x\)轴于点\(\left( a_{n},0 \right)\),
则\(\frac{1}{a_{1}a_{2}} + \frac{1}{a_{2}a_{3}} + \frac{1}{a_{3}a_{4}} + \cdots + \frac{1}{a_{2024}a_{2025}}\)的值是(   )
\begin{choices}
  \item \(\frac{2026}{2025}\)
  \item \(\frac{2024}{2023}\)
  \item \(\frac{2023}{2024}\)
  \item \(\frac{2024}{2025}\)
\end{choices}
\topics{求在曲线上一点处的切线方程(斜率);等差数列通项公式的基本量计算;裂项相消法求和}
\difficulty{0.65}
\answer{D}
\explain{由\(f(x) = \mathrm{e}^{- x}\),
得\(f'(x) = - \mathrm{e}^{- x}\),\par
所以\(f(0) = 1\),\(f'(0) = - 1\),\par
则函数在\(x = 0\)处的切线方程为\(y = - x + 1\),\par
令\(y = 0\),解得\(x = 1\),即\(a_{1} = 1\),\par
同理可得曲线\(y = f(x)\)在\(x = a_{n - 1}(n \geq 2)\)处的切线方程为\(y - \mathrm{e}^{- a_{n - 1}} = - \mathrm{e}^{- a_{n - 1}}\left( x - a_{n - 1} \right)\),\par
令\(y =\)`<!-- -->`{=html}0,
解得\(x = a_{n - 1} + 1\),
即\(a_{n} = a_{n - 1} + 1(n \geq 2)\),\par
所以\(a_{n} - a_{n - 1} = 1(n \geq 2)\),\par
即数列\(\left\{ a_{n} \right\}\)是以1为首项,
1为公差的等差数列,\par
所以\(a_{n} = n\),
则\(\frac{1}{a_{n}a_{n + 1}} = \frac{1}{n(n + 1)} = \frac{1}{n} - \frac{1}{n + 1}\),\par
所以\(\frac{1}{a_{1}a_{2}} + \frac{1}{a_{2}a_{3}} + \frac{1}{a_{3}a_{4}} + \cdots + \frac{1}{a_{2024}a_{2025}} = 1 - \frac{1}{2} + \frac{1}{2} - \frac{1}{3} + \frac{1}{3} - \frac{1}{4} + \cdots + \frac{1}{2024} - \frac{1}{2025}= 1 - \frac{1}{2025} = \frac{2024}{2025}\).}
\end{question}

\section{多选题}

\begin{question}
已知i为虚数单位,则下列结论正确的是(    )
\begin{choices}
  若复数\(z_{1},z_{2}\)满足\(\left| z_{1} \right| = \left| z_{2} \right|\),则\(z_{1}^{2} = z_{2}^{2}\)

\begin{enumerate}[label=(\arabic*)]
  \item 若\(z - (2 + \text{i}) > 0\),则\(z > 2 + \text{i}\)
  \item 若复数\(z_{1},z_{2}\),满足\(z_{1}z_{2} = 0\),则\(z_{1} = 0\)或\(z_{2} = 0\)
  \item 若复数\emph{z}满足,\(|z| = 1\),则\(|z - 2\text{i}|\)最大值为3
\item \end{choices}
\item \topics{与复数模相关的轨迹(图形)问题;复数代数形式的乘法运算;复数的乘方}
\item \difficulty{0.65}
\item \answer{CD}
\item \explain{对于A,显然若\(z_{1} = 1 + \text{i}\),
\item \(z_{2} = - 1 + \text{i}\),
\item 则\(\left| z_{1} \right| = \left| z_{2} \right| = \sqrt{2}\),\par
\item 但\(z_{1}^{2} = 2\text{i},z_{2}^{2} = - 2\text{i}\),
\item 故A错误;\par
\item 对于B,举例\(z = 3 + \text{i}\),
\item 则满足\(z - (2 + \text{i}) > 0\),
\item 但是复数不能直接比较大小,即\(z > 2 + \text{i}\)不成立,
\item 故B错误;\par
\item 对于C,由\(z_{1}z_{2} = 0\),
\item 得\(|z_{1}z_{2}| = 0\),
\item 即\(|z_{1}||z_{2}| = 0\),
\item 因此\(|z_{1}| = 0\)或\(|z_{2}| = 0\),
\item \(z_{1} = 0\)或\(z_{2} = 0\),C正确;\par
\item 对于D,设\(z = x + y\text{i}\),\par
\item 则\(x^{2} + y^{2} = 1,\therefore y \in \lbrack - 1,1\rbrack,\left| z - 2\text{i} \right| = \sqrt{x^{2} + (y - 2)^{2}} = \sqrt{1 - y^{2} + (y - 2)^{2}} = \sqrt{5 - 4y} \in \lbrack 1,3\rbrack\),
\item 则\(|z - 2\text{i}|\)最大值为3,故D正确}
\end{enumerate}
\end{question}

\begin{question}
(多选题)已知函数\(f(x) = \mathrm{e}^{- x} - \mathrm{e}^{x} - \sin x + x + 3\),则(   )
\begin{choices}
  \item \(f(x)\)的图象为中心对称图形
  \item \(f(x)\)有且仅有1个零点
  \item 若\(f\left( |2t - 1| - 5 \right) > 3\),则实数\(t\)的取值范围为\(( - 2,3)\)
  \item 若\(f\left( \log_{\frac{1}{2}}t \right) + f(2) < 6\),则实数\(t\)的取值范围为\((4, + \infty)\)
\end{choices}
\topics{函数奇偶性的应用;利用导数求函数的单调区间(不含参);根据函数的单调性解不等式;由对数函数的单调性解不等式}
\difficulty{0.4}
\answer{ABC}
\explain{令\(g(x) = \mathrm{e}^{- x} - \mathrm{e}^{x} - \sin x + x\),
则\(f(x) = g(x) + 3\),\par
因为\(x \in \mathbb{R}\),
所以\(g( - x) = \mathrm{e}^{x} - \mathrm{e}^{- x} - \sin( - x) - x = \mathrm{e}^{x} - \mathrm{e}^{- x} + \sin x - x = - g(x)\),\par
所以\(g(x)\)为奇函数,图象关于点\((0,0)\)对称,\par
\(g'(x) = - \mathrm{e}^{- x} - \mathrm{e}^{x} - \cos x + 1 = - \left( \mathrm{e}^{- x} + \mathrm{e}^{x} \right) - \cos x + 1\),\par
因为\(\mathrm{e}^{- x} + \mathrm{e}^{x} \geq 2\),
当且仅当\(x = 0\)时等号成立,
所以\(- \left( \mathrm{e}^{- x} + \mathrm{e}^{x} \right) \leq - 2\),\par
因为\(- \cos x + 1 \in \lbrack 0,2\rbrack\),
所以\(g'(x) = - \left( \mathrm{e}^{- x} + \mathrm{e}^{x} \right) - \cos x + 1 \leq 0\)\par
所以函数\(g(x)\)在\(\mathbb{R}\)上单调递减,\par
对于A,
因为\(f(x) = g(x) + 3\)图象为函数\(g(x)\)图象向上平移3个单位得到,\par
所以\(f(x)\)的图象关于点\((0,3)\)对称,
即\(f(x)\)的图象为中心对称图形,故A正确;\par
对于B,
因为\(f(x) = g(x) + 3\)图象可由函数\(g(x)\)的图象向上平移3个单位得到,\par
则函数\(f(x)\)在\(\mathbb{R}\)上单调递减,\par
又\(f(0) = 3 > 0\),
\(f(3) = \mathrm{e}^{- 3} - \mathrm{e}^{3} - \sin 3 + 6 < 0\),
\(f(x)\)有且仅有1个零点,故正确;\par
对于C,因为\(f(0) = 3\),
所以\(f\left( |2t - 1| - 5 \right) > 3 = f(0)\),\par
因函数\(f(x)\)在\(\mathbb{R}\)上单调递减,
则得\(|2t - 1| - 5 < 0\),
即\(- 5 < 2t - 1 < 5\),解得:\(- 2 < t < 3\),\par
所以,当\(f\left( |2t - 1| - 5 \right) > 3\),
实数\(t\)的取值范围为\(( - 2,3)\),故C正确;\par
对于D,
由\(f\left( \log_{\frac{1}{2}}t \right) + f(2) < 6\)得\(g\left( \log_{\frac{1}{2}}t \right) + g(2) < 0\),\par
因为\(g(x)\)为奇函数,
所以\(g\left( \log_{\frac{1}{2}}t \right) + g(2) < 0 \Leftrightarrow g\left( \log_{\frac{1}{2}}t \right) < - g(2) = g( - 2)\),\par
因函数\(g(x)\)在\(\mathbb{R}\)上单调递减,
则\(\log_{\frac{1}{2}}t > - 2 = \log_{\frac{1}{2}}4\),
解得\(0 < t < 4\),故D错误}
\end{question}

\begin{question}
以下命题正确的有(    )
\begin{choices}
  若\(P\)是\(\bigtriangleup ABC\)的重心,则有\(\overrightarrow{PA} + \overrightarrow{PB} + \overrightarrow{PC} = \overrightarrow{0}\)

\begin{enumerate}[label=(\arabic*)]
  \item 若\(a\overrightarrow{PA} + b\overrightarrow{PB} + c\overrightarrow{PC} = \overrightarrow{0}\)成立,则\(P\)是\(\bigtriangleup ABC\)的内心
  \item 若\(\overrightarrow{AP} = \frac{2}{5}\overrightarrow{AB} + \frac{1}{5}\overrightarrow{AC}\),则\(S_{\bigtriangleup ABP}:S_{\bigtriangleup ABC} = 2:5\)
  \item 若\(P\)是\(\bigtriangleup ABC\)的外心,\(A = \frac{\pi}{4}\),\(\overrightarrow{PA} = m\overrightarrow{PB} + n\overrightarrow{PC}\),则\(m + n \in \left\lbrack - \sqrt{2},1 \right)\)
\item \end{choices}

\item \begin{center}
% IMAGE_TODO_START id=auto_5c82df-Q11-img2 path=/Users/muryor/code/mynote/word\\_to\\_tex/output/figures/auto\\_5c82df/media/image2.png width=60% inline=false question_index=11 sub_index=1
% CONTEXT_BEFORE: \bigtriangleup ABC$$内一点,延长$$AP$$,交$$BC$$于点$$F$$,
% CONTEXT_AFTER: $${FC} = }{S_
\item \begin{tikzpicture}[scale=1.05,>=Stealth,line cap=round,line join=round]
  % TODO: AI_AGENT_REPLACE_ME (id=auto_5c82df-Q11-img2)
\item \end{tikzpicture}
% IMAGE_TODO_END id=auto_5c82df-Q11-img
\item 2
\item \end{center}

\item \topics{根据向量关系判断三角形的心}
\item \difficulty{0.4}
\item \answer{ABD}
\item \explain{如图所示,\(P\)是\(\bigtriangleup ABC\)内一点,
\item 延长\(AP\),交\(BC\)于点\(F\),\par
% IMAGE_TODO_START id=auto_5c82df-Q11-img1 path=/Users/muryor/code/mynote/word\\_to\\_tex/output/figures/auto\\_5c82df/media/image2.png width=60% inline=true question_index=11 sub_index=1
% CONTEXT_BEFORE: \bigtriangleup ABC$$内一点,延长$$AP$$,交$$BC$$于点$$F$$,
% CONTEXT_AFTER: $${FC} =}{S_
\item \begin{tikzpicture}[scale=0.8,baseline=-0.5ex]
  % TODO: AI_AGENT_REPLACE_ME (id=auto_5c82df-Q11-img1)
\item \end{tikzpicture}
% IMAGE_TODO_END id=auto_5c82df-Q11-img
\item 1
\item \(\frac{FB}{FC} = \frac{S_{\bigtriangleup PBF}}{S_{\bigtriangleup PCF}} = \frac{S_{\bigtriangleup ABF}}{S_{\bigtriangleup ACF}} = \frac{S_{\bigtriangleup ABF} - S_{\bigtriangleup PBF}}{S_{\bigtriangleup ACF} - S_{\bigtriangleup PCF}} = \frac{S_{\bigtriangleup PAB}}{S_{\bigtriangleup PAC}}\)\par
\item 因为\(\overrightarrow{PF} = \frac{FC}{BC} \cdot \overrightarrow{PB} + \frac{FB}{BC} \cdot \overrightarrow{PC} = \frac{S_{\bigtriangleup PAC}}{S_{\bigtriangleup PAC} + S_{\bigtriangleup PAB}} \cdot \overrightarrow{PB} + \frac{S_{\bigtriangleup PAB}}{S_{\bigtriangleup PAC} + S_{\bigtriangleup PAB}} \cdot \overrightarrow{PC}\),\par
\item \(\frac{PF}{PA} = \frac{S_{\bigtriangleup PBF}}{S_{\bigtriangleup PAB}} = \frac{S_{\bigtriangleup PCF}}{S_{\bigtriangleup PAC}} = \frac{S_{\bigtriangleup PBF} + S_{\bigtriangleup PCF}}{S_{\bigtriangleup PAB} + S_{\bigtriangleup PAC}} = \frac{S_{\bigtriangleup PBC}}{S_{PAB} + S_{\bigtriangleup PAC}}\),\par
\item 所以\(\overrightarrow{PF} = \frac{S_{\bigtriangleup PBC}}{S_{\bigtriangleup PAC} + S_{\bigtriangleup PAB}} \cdot \overrightarrow{AP} = \frac{S_{\bigtriangleup PAC}}{S_{\bigtriangleup PAC} + S_{\bigtriangleup PAB}} \cdot \overrightarrow{PB} + \frac{S_{\bigtriangleup PAB}}{S_{\bigtriangleup PAC} + S_{\bigtriangleup PAB}} \cdot \overrightarrow{PC}\),\par
\item 所以\(S_{\bigtriangleup PBC} \cdot \overrightarrow{AP} = S_{\bigtriangleup PAC} \cdot \overrightarrow{PB} + S_{\bigtriangleup PAB} \cdot \overrightarrow{PC}\),\par
\item 所以\(S_{\bigtriangleup PBC} \cdot \overrightarrow{PA} + S_{\bigtriangleup PAC} \cdot \overrightarrow{PB} + S_{\bigtriangleup PAB} \cdot \overrightarrow{PC} = \overrightarrow{0}\),\par
\item 对于A,如图所示,\(D,E,F\)分别为\(CA,AB,BC\)的中点,
\item 连接\(AF,BD,CE\),\par
\item 因为\(P\)是\(\bigtriangleup ABC\)的重心,
\item 所以\(CP = 2PE\),\par
\item 所以\(S_{\bigtriangleup AEC} = \frac{1}{2}S_{\bigtriangleup ABC},S_{\bigtriangleup PAC} = \frac{2}{3}S_{\bigtriangleup AEC} = \frac{1}{3}S_{\bigtriangleup ABC}\),\par
\item 同理可得\(S_{\bigtriangleup PAB} = \frac{1}{3}S_{\bigtriangleup ABC},S_{\bigtriangleup PBC} = \frac{1}{3}S_{\bigtriangleup ABC}\),
\item 所以\(S_{\bigtriangleup PBC} = S_{\bigtriangleup PAC} = S_{\bigtriangleup PAB}\),\par
\item 所以\(\overrightarrow{PA} + \overrightarrow{PB} + \overrightarrow{PC} = \overrightarrow{0}\),
\item 故A正确;\par
\item 对于B,
\item 设点\(P\)到\(AB,BC,CA\)的距离分别为\(h_{1},h_{2},h_{3}\),\par
\item 则\(S_{\bigtriangleup PBC} = \frac{1}{2}a \cdot h_{2},S_{\bigtriangleup PAC} = \frac{1}{2}b \cdot h_{3},S_{\bigtriangleup PAB} = \frac{1}{2}c \cdot h_{1}\),\par
\item 因为\(S_{\bigtriangleup PBC} \cdot \overrightarrow{PA} + S_{\bigtriangleup PAC} \cdot \overrightarrow{PB} + S_{\bigtriangleup PAB} \cdot \overrightarrow{PC} = \overrightarrow{0}\),\par
\item 则\(\frac{1}{2}a \cdot h_{2} \cdot \overrightarrow{PA} + \frac{1}{2}b \cdot h_{3} \cdot \overrightarrow{PB} + \frac{1}{2}c \cdot h_{1} \cdot \overrightarrow{PC} = \overrightarrow{0}\),
\item 即\(a \cdot h_{2} \cdot \overrightarrow{PA} + b \cdot h_{3} \cdot \overrightarrow{PB} + c \cdot h_{1} \cdot \overrightarrow{PC} = \overrightarrow{0}\),\par
\item 又\(a\overrightarrow{PA} + b\overrightarrow{PB} + c\overrightarrow{PC} = \overrightarrow{0}\),
\item 所以\(h_{1} = h_{2} = h_{3}\),
\item 所以点\(P\)是\(\bigtriangleup ABC\)的内心,故B正确;\par
\item 对于C,
\item 因为\(\overrightarrow{AP} = \frac{2}{5}\overrightarrow{AB} + \frac{1}{5}\overrightarrow{AC}\),\par
\item 所以\(\overrightarrow{PA} = - \frac{2}{5}\overrightarrow{AB} - \frac{1}{5}\overrightarrow{AC},\overrightarrow{PB} = \overrightarrow{PA} + \overrightarrow{AB} = \frac{3}{5}\overrightarrow{AB} - \frac{1}{5}\overrightarrow{AC}\),
\item \(\overrightarrow{PC} = \overrightarrow{PA} + \overrightarrow{AC} = - \frac{2}{5}\overrightarrow{AB} + \frac{4}{5}\overrightarrow{AC}\),\par
\item 所以\(S_{\bigtriangleup PBC} \cdot \left( - \frac{2}{5}\overrightarrow{AB} - \frac{1}{5}\overrightarrow{AC} \right) + S_{\bigtriangleup PAC} \cdot \left( \frac{3}{5}\overrightarrow{AB} - \right.\ \left. \ \frac{1}{5}\overrightarrow{AC} \right) + S_{\bigtriangleup PAB} \cdot \left( - \frac{2}{5}\overrightarrow{AB} + \frac{4}{5}\overrightarrow{AC} \right) = \overrightarrow{0}\),\par
\item 化简得\(\left( - \frac{2}{5}S_{\bigtriangleup PBC} + \frac{3}{5}S_{\bigtriangleup PAC} - \frac{2}{5}S_{\bigtriangleup PAB} \right) \cdot \overrightarrow{AB} + \left( - \frac{1}{5}S_{\bigtriangleup PBC} - \frac{1}{5}S_{\bigtriangleup PAC} + \frac{4}{5}S_{\bigtriangleup PAB} \right) \cdot \overrightarrow{AC} = \overrightarrow{0}\),\par
\item 又\(\overrightarrow{AB},\overrightarrow{AC}\)不共线,\par
\item 所以\(\left\{ \begin{array}{r}
 \item - \frac{2}{5}S_{\bigtriangleup PBC} + \frac{3}{5}S_{\bigtriangleup PAC} - \frac{2}{5}S_{\bigtriangleup PAB} = 0 \\
 \item - \frac{1}{5}S_{\bigtriangleup PBC} - \frac{1}{5}S_{\bigtriangleup PAC} + \frac{4}{5}S_{\bigtriangleup PAB} = 0
\item \end{array} \right.\),即\(\left\{ \begin{array}{r}
\item S_{\bigtriangleup PBC} = 2S_{\bigtriangleup PAB} \\
\item S_{\bigtriangleup PAC} = 2S_{\bigtriangleup PAB}
\item \end{array} \right.\),\par
\item 所以\(\frac{S_{\bigtriangleup ABP}}{S_{\bigtriangleup ABC}} = \frac{S_{\bigtriangleup PAB}}{S_{\bigtriangleup PBC} + S_{\bigtriangleup PAC} + S_{\bigtriangleup PAB}} = \frac{1}{5}\),故C错误;\par
\item 对于D,因为\(P\)是\(\bigtriangleup ABC\)的外心,\(A = \frac{\pi}{4}\),\par
\item 所以\(\angle BPC = \frac{\pi}{2}\),\(\left| \overrightarrow{PA} \right| = \left| \overrightarrow{PB} \right| = \left| \overrightarrow{PC} \right|\),\(\overrightarrow{PB} \cdot \overrightarrow{PC} = \left| \overrightarrow{PB} \right| \times \left| \overrightarrow{PC} \right| \times \cos\angle BPC = 0\).\par
\item 因为\(\overrightarrow{PA} = m\overrightarrow{PB} + n\overrightarrow{PC}\),则\(|\overrightarrow{PA}|^{2} = m^{2}|\overrightarrow{PB}|^{2} + 2mn\overrightarrow{PB} \cdot \overrightarrow{PC} + n^{2}|\overrightarrow{PC}|^{2}\),化简得\(m^{2} + n^{2} = 1\).\par
\item 由题意知\(m,n\)不同时为正,记\(\left\{ \begin{array}{r}
\item m = \cos\alpha \\
\item n = \sin\alpha
\item \end{array} \right.\),\(\frac{\pi}{2} < \alpha < 2\pi\),\par
\item 则\(m + n = \cos\alpha + \sin\alpha = \sqrt{2}\sin\left( \alpha + \frac{\pi}{4} \right)\),\par
\item 因为\(\frac{3\pi}{4} < \alpha + \frac{\pi}{4} < \frac{9\pi}{4}\),所以\(- 1 \leq \sin\left( \alpha + \frac{\pi}{4} \right) < \frac{\sqrt{2}}{2}\),即\(- \sqrt{2} \leq \sqrt{2}\sin\left( \alpha + \frac{\pi}{4} \right) < 1\),\par
\item 所以\(m + n \in \left\lbrack - \sqrt{2},1 \right)\),故D正确.}
\end{enumerate}
\end{question}

\section{填空题}

\begin{question}
若\(\overrightarrow{a} = (1,2)\),
\(\overrightarrow{b} = (m, - 1)\),
\(\overrightarrow{a}\)与\(\overrightarrow{b}\)的夹角是钝角,
那么实数\emph{m}的取值范围是
.
\topics{利用数量积求参数}
\difficulty{0.65}
\answer{\(m < 2\)且\(m \neq - \frac{1}{2}\)}
\explain{由于\(\overrightarrow{a}\)与\(\overrightarrow{b}\)的夹角是钝角,
则\(\overrightarrow{a} \cdot \overrightarrow{b} < 0\)且\(\overrightarrow{a}\)与\(\overrightarrow{b}\)不共线\par
由\(\overrightarrow{a} \cdot \overrightarrow{b} = m - 2 < 0\),
可得\(m < 2\),\par
由\(\overrightarrow{a}\)与\(\overrightarrow{b}\)共线,
可得\(2m = - 1\),即\(m = - \frac{1}{2}\).\par
故实数\emph{m}的取值范围是\(m < 2\)且\(m \neq - \frac{1}{2}\).\(m < 2\)且\(m \neq - \frac{1}{2}\).}
\end{question}

\begin{question}
已知\(S_{n}\)为等比数列\(\left\{ a_{n} \right\}\)前\(n\)项和,
若\(a_{4} = 6a_{3} - 9a_{2}\),
则\(\frac{S_{6}}{a_{1} + a_{2}} =\)
.
\topics{等比数列通项公式的基本量计算}
\difficulty{0.85}
\answer{91}
\explain{设该数列公比为\(q\),
由\(a_{4} = 6a_{3} - 9a_{2}\)可知\(a_{2}q^{2} = 6a_{2}q - 9a_{2}\),\par
所以\(q^{2} - 6q + 9 = 0\),可得\(q = 3\),\par
而\(\frac{S_{6}}{a_{1} + a_{2}} = \frac{a_{1}\left( 1 + q + q^{2} + q^{3} + q^{4} + q^{5} \right)}{a_{1}(1 + q)} = 1 + q^{2} + q^{4} = 1 + 9 + 81 = 91\).91}
\end{question}

\begin{question}
已知函数\(f(x) = x^{3} - x^{2}\text{sin\pi}x + \frac{x}{4}\)的零点分别为\(x_{1},x_{2}, \cdot \cdot \cdot ,x_{n}\left( n \in N^{\text{*}} \right)\),
则\(x_{1}^{2} + x_{2}^{2} + \cdot \cdot \cdot + x_{n}^{2} =\)
.
\topics{求函数的零点;正弦函数图象的应用}
\difficulty{0.65}
\answer{\(\frac{1}{2}\)/0.5}
\explain{令\(f(x) = 0\),
则有\(x^{3} - x^{2}\text{sin\pi}x + \frac{x}{4} = 0\),
即\(x\left( x^{2} - \right.\ x\text{sin\pi}x + \frac{1}{4}) = 0\),
所以有\(f(0) = 0\).\par
令\(g(x) = x^{2} - x\text{sin\pi}x + \frac{1}{4}\),
则\(g(0) \neq 0\),\par
令\(g(x) = 0\),
则有\(x^{2} + \frac{1}{4} = x\text{sin\pi}x\),
即有\(\frac{x^{2} + \frac{1}{4}}{x} = \text{sin\pi}x\),
\((x \neq 0)\),\par
因为\(- 1 \leq \text{sin\pi}x \leq 1\),
所以\(\left| \frac{x^{2} + \frac{1}{4}}{x} \right| \leq 1\),
则\(x^{2} + \frac{1}{4} \leq |x|\),
即有\(\left( |x| - \frac{1}{2} \right)^{2} \leq 0\),\par
则有\(|x| = \frac{1}{2}\),
即当\(x = \pm \frac{1}{2}\)时,\(g(x) = 0\),\par
所以\(f(x)\)共有3个零点,
分别为\(0, - \frac{1}{2},\frac{1}{2}\),\par
所以\(x_{1}^{2} + x_{2}^{2} + \cdot \cdot \cdot + x_{n}^{2} = 0^{2} + \left( - \frac{1}{2} \right)^{2} + \left( \frac{1}{2} \right)^{2} = \frac{1}{2}\).\(\frac{1}{2}\).}
\end{question}

\section{解答题}

\begin{question}
已知公差不为0的等差数列\(\left\{ a_{n} \right\}\)的前\(n\)项和为\(S_{n},S_{4} = 16\),
且\(a_{2},a_{5},a_{14}\)依次成等比数列.
\begin{enumerate}[label=(\arabic*)]
  \item 求\(\left\{ a_{n} \right\}\)的通项公式;
  \item 对于任意\(n \in \mathbb{N}^{*},\lambda \cdot 2^{n} \geq S_{n}\),
\item 求实数\(\lambda\)的取值范围.
\end{enumerate}
\topics{确定数列中的最大(小)项;等差数列通项公式的基本量计算;等比中项的应用;数列不等式恒成立问题}
\difficulty{0.65}
\answer{(1)\(a_{n} = 2n - 1\)
(2)\(\left\lbrack \frac{9}{8}, + \infty \right)\)}
\explain{(1)设等差数列\(\left\{ a_{n} \right\}\)的公差为\(d\),\par
由已知可得\(\left( a_{1} + 4d \right)^{2} = \left( a_{1} + d \right)\left( a_{1} + 13d \right)\),\par
因为\(d \neq 0\),解得\(d = 2a_{1}\),\par
又\(S_{4} = 4a_{1} + 6d = 16a_{1} = 16\),\par
得\(a_{1} = 1,d = 2\),\par
所以\(a_{n} = 2n - 1\).\par
(2)由(1)可知\(a_{n} = 2n - 1\),
则\(S_{n} = \frac{\left( a_{1} + a_{n} \right)n}{2} = \frac{(1 + 2n - 1)n}{2} = n^{2}\),\par
由\(\lambda \cdot 2^{n} \geq S_{n}\)可得\(\lambda \geq \frac{n^{2}}{2^{n}}\),\par
令\(b_{n} = \frac{n^{2}}{2^{n}}\),\par
\(b_{n + 1} - b_{n} = \frac{{(n + 1)}^{2}}{2^{n + 1}} - \frac{n^{2}}{2^{n}} = \frac{- n^{2} + 2n + 1}{2^{n + 1}} = \frac{- {(n - 1)}^{2} + 2}{2^{n + 1}}\),\par
当\(1 \leq n \leq 2\)时,
\(b_{n + 1} - b_{n} > 0\),\par
当\(n \geq 3\)时,\(b_{n + 1} - b_{n} < 0\),\par
则数列\(\left\{ b_{n} \right\}\)的最大项为\(b_{3} = \frac{9}{8}\),\par
故\(\lambda \geq \frac{9}{8}\),\par
即实数\(\lambda\)的取值范围为\(\left\lbrack \frac{9}{8}, + \infty \right)\).}
\end{question}

\begin{question}
如图,平面凸四边形\(ABCD\)中,\(AB\bot AC\),
且\(\bigtriangleup ACD\)是边长为2的等边三角形.
\begin{enumerate}[label=(\arabic*)]
  \item 若\(\angle ABC = 60{^\circ}\),
\item 求\(\sin\angle ABD\);
  \item 若线段\(AC\)(不含端点)上存在动点\(P\),
\item 满足\(BP = DP\),求\(AB\)长度的取值范围.
\end{enumerate}

\begin{center}
% IMAGE_TODO_START id=auto_5c82df-Q16-img1 path=/Users/muryor/code/mynote/word\\_to\\_tex/output/figures/auto\\_5c82df/media/image3.png width=60% inline=false question_index=16 sub_index=1
% CONTEXT_BEFORE: $AB\bot AC$$,且$$\bigtriangleup ACD$$是边长为2的等边三角形.
% CONTEXT_AFTER: (1)若$$\angle ABC = 60{^\circ}$$,求$$\sin\angle AB
\begin{tikzpicture}[scale=1.05,>=Stealth,line cap=round,line join=round]
  % TODO: AI_AGENT_REPLACE_ME (id=auto_5c82df-Q16-img1)
\end{tikzpicture}
% IMAGE_TODO_END id=auto_5c82df-Q16-img
1
\end{center}


\begin{center}
% IMAGE_TODO_START id=auto_5c82df-Q16-img2 path=/Users/muryor/code/mynote/word\\_to\\_tex/output/figures/auto\\_5c82df/media/image4.png width=60% inline=false question_index=16 sub_index=1
% CONTEXT_AFTER: 设$$AP = x,x \in \left( 0,2 \righ
\begin{tikzpicture}[scale=1.05,>=Stealth,line cap=round,line join=round]
  % TODO: AI_AGENT_REPLACE_ME (id=auto_5c82df-Q16-img2)
\end{tikzpicture}
% IMAGE_TODO_END id=auto_5c82df-Q16-img
2
\end{center}

\topics{正弦定理解三角形;余弦定理解三角形}
\difficulty{0.65}
\answer{(1)\(\sin\angle ABD = \frac{\sqrt{21}}{14}\)
(2)\((0,2)\)}
\explain{(1)由题意知\(AB = \frac{2\sqrt{3}}{3},\text{ }\angle BAD = 150{^\circ}\).\par
在\(\bigtriangleup ABD\)中,
由余弦定理可得\(BD = \sqrt{AB^{2} + AD^{2} - 2 \cdot AB \cdot AD \cdot \cos\angle BAD} = \frac{2\sqrt{21}}{3}\).\par
由正弦定理得\(\frac{BD}{\sin\angle BAD} = \frac{AD}{\sin\angle ABD}\),
解得\(\sin\angle ABD = \frac{\sqrt{21}}{14}\).\par
(2)如图:\par
设\(AP = x,\text{ }x \in \left( 0,\text{ }2 \right),\text{ }AB = y\),\par
在\(\bigtriangleup ADP\)中,由余弦定理得\par
\(DP = \sqrt{AP^{2} + AD^{2} - 2 \cdot AP \cdot AD \cdot \cos\angle DAP} = \sqrt{x^{2} - 2x + 4}\),\par
由\(AB\bot AC\)知\(BP = \sqrt{AP^{2} + AB^{2}} = \sqrt{x^{2} + y^{2}}\).\par
由题意得关于\(x\)的方程\(\sqrt{x^{2} - 2x + 4} = \sqrt{x^{2} + y^{2}}\)在\(x \in \left( 0,\text{ }2 \right)\)时有解,\par
等价于关于\(x\)的方程\(y^{2} = - 2x + 4\)在\(x \in \left( 0,\text{ }2 \right)\)时有解,\par
由\(x \in \left( 0,\text{ }2 \right),\text{ }y > 0\)知,
\(y \in \left( 0,\text{ }2 \right)\),
即\(AB\)长度的取值范围为\((0,2)\).}
\end{question}

\begin{question}
设计一个帐篷,
它下部的形状是正四棱柱\(A_{1}B_{1}C_{1}D_{1} - ABCD\),
上部的形状是正四棱锥\(P - A_{1}B_{1}C_{1}D_{1}\),
且该帐篷外接于球\(O\)(如图所示).
\begin{enumerate}[label=(\arabic*)]
  \item 若正四棱柱\(A_{1}B_{1}C_{1}D_{1} - ABCD\)是棱长为\(2\text{m}\)的正方体,
\item 求该帐篷的顶点\(P\)到底面\(ABCD\)中心\(O_{2}\)的距离;
  \item 若该帐篷外接球\(O\)的半径\(3\text{m}\),
\item 设\(\angle POC_{1} = \theta,\theta \in (0,\frac{\pi}{2})\),
\item 该帐篷的体积为\(V\),则当\(\cos\theta\)为何值时,
\item 体积\(V\)取得最大值.
\end{enumerate}

\begin{center}
% IMAGE_TODO_START id=auto_5c82df-Q17-img1 path=/Users/muryor/code/mynote/word\\_to\\_tex/output/figures/auto\\_5c82df/media/image5.png width=60% inline=false question_index=17 sub_index=1
% CONTEXT_BEFORE: $P - A_{1}B_{1}C_{1}D_{1}$$,且该帐篷外接于球$$O$$(如图所示).
% CONTEXT_AFTER: (1)若正四棱柱$$A_{1}B_{1}C_{1}D_{1} - ABCD$$是棱长为$$2\t
\begin{tikzpicture}[scale=1.05,>=Stealth,line cap=round,line join=round]
  % TODO: AI_AGENT_REPLACE_ME (id=auto_5c82df-Q17-img1)
\end{tikzpicture}
% IMAGE_TODO_END id=auto_5c82df-Q17-img
1
\end{center}


\begin{center}
% IMAGE_TODO_START id=auto_5c82df-Q17-img2 path=/Users/muryor/code/mynote/word\\_to\\_tex/output/figures/auto\\_5c82df/media/image5.png width=60% inline=false question_index=17 sub_index=1
\begin{tikzpicture}[scale=1.05,>=Stealth,line cap=round,line join=round]
  % TODO: AI_AGENT_REPLACE_ME (id=auto_5c82df-Q17-img2)
\end{tikzpicture}
% IMAGE_TODO_END id=auto_5c82df-Q17-img
2
\end{center}

\topics{由导数求函数的最值(不含参);面积;体积最大问题;锥体体积的有关计算;多面体与球体内切外接问题}
\difficulty{0.65}
\answer{(1)\(\sqrt{3} + 1\)
(2)\(\frac{2\sqrt{19} - 1}{15}\)}
\explain{(1)设外接球的半径为\(R\),
因为正四棱柱\(A_{1}B_{1}C_{1}D_{1} - ABCD\)是棱长为\(2\text{m}\)的正方体,\par
易知外接球的球心为正方体的中心,
所以\(R = OA_{1} = OP = \frac{\sqrt{2^{2} + 2^{2} + 2^{2}}}{2} = \sqrt{3}\),
而\(OO_{2} = 1\),\par
得到\(O_{2}P = OP + OO_{2} = \sqrt{3} + 1\).\par
(2)\(\because\angle POC_{1} = \theta ,\theta \in (0,\frac{\pi}{2})\),\par
\(\therefore OO_{1} = 3\cos\theta,O_{1}C_{1} = 3\sin\theta\),\par
\(\therefore O_{1}P = 3 - 3\cos\theta,AB = 3\sqrt{2}\sin\theta\).\par
\(\therefore V = AB^{2} \times AA_{1} + \frac{1}{3}AB^{2} \times O_{1}P = 18\sin^{2}\theta(1 + 5\cos\theta) = 18(1 - \cos^{2}\theta)(1 + 5\cos\theta)\),\par
令\(t = \cos\theta \in (0,1) \Rightarrow V = 18(1 - t^{2})(1 + 5t)\),\par
由\(V' = 18( - 15t^{2} - 2t + 5) = 0\),
得到\(t = \frac{2\sqrt{19} - 1}{15}\),\par
\(\therefore V\)在\((0,\frac{2\sqrt{19} - 1}{15})\)上递增,
在\((\frac{2\sqrt{19} - 1}{15},1)\)递减.\par
\(\therefore\cos\theta = \frac{2\sqrt{19} - 1}{15}\)时,
体积\(V\)取得最大值.}
\end{question}

\begin{question}
已知函数\(f(x) = ax^{a}\lnx(a > 0)\),
\(g(x) = x\mathrm{e}^{x}\).
\begin{enumerate}[label=(\arabic*)]
  \item 当\(a = 1\)时,
\item 求曲线\(y = f(x)\)在\(x = 1\)处的切线方程;
  \item 设\(h(x) = g(x) - 2\mathrm{e}^{x} - \frac{1}{2}bx^{2} + bx\),
\item \((b > 0)\),求\(h(x)\)的单调区间;
  \item 若\(f(x) \leq g(x)\)对于任意的\(x > 1\)都成立,
\item 求\(a\)的最大值.
\end{enumerate}
\topics{求在曲线上一点处的切线方程(斜率);利用导数研究不等式恒成立问题;利用导数求函数(含参)的单调区间}
\difficulty{0.4}
\answer{(1)\(y = x - 1\)
(2)答案见解析
(3)\(\text{e}\)}
\explain{(1)当\(a = 1\)时,\(f(x) = x\lnx\),
得\(f'(x) = \lnx + 1\),\par
则\(f(1) = 0\),\(f'(1) = 1\),\par
所以\(y = f(x)\)在\(x = 1\)处的切线方程为\(y = x - 1\).\par
(2)由\(h(x) = g(x) - 2\mathrm{e}^{x} - \frac{1}{2}bx^{2} + bx = (x - 2)\mathrm{e}^{x} - \frac{1}{2}bx^{2} + bx\),\par
则\(h'(x) = (x - 1)\mathrm{e}^{x} - bx + b = (x - 1)\left( \mathrm{e}^{x} - b \right)\),\par
令\(h'(x) = 0\),得\(x = 1\)或\(x = \ln b\),\par
当\(\ln b = 1\),即\(b = \text{e}\)时,
\(h'(x) \geq 0\),\par
则函数\(h(x)\)的单调递增区间为\(( - \infty, + \infty)\),
无单调递减区间;\par
当\(\ln b < 1\),即\(0 < b < \text{e}\)时,
令\(h'(x) < 0\),得\(\ln b < x < 1\),
令\(h'(x) > 0\),得\(x < \ln b\)或\(x > 1\),\par
则函数\(h(x)\)的单调递增区间为\(\left( - \infty,\ln b \right)\)和\((1, + \infty)\),
单调递减区间为\(\left( \ln b,1 \right)\);\par
当\(\ln b > 1\),即\(b > \text{e}\)时,
令\(h'(x) < 0\),得\(1 < x < \ln b\),
令\(h'(x) > 0\),得\(x < 1\)或\(x > \ln b\),\par
则函数\(h(x)\)的单调递增区间为\(( - \infty,1)\)和\(\left( \ln b, + \infty \right)\),
单调递减区间为\(\left( 1,\ln b \right)\).\par
综上所述,当\(0 < b < \text{e}\)时,
函数\(h(x)\)的单调递增区间为\(\left( - \infty,\ln b \right)\)和\((1, + \infty)\),
单调递减区间为\(\left( \ln b,1 \right)\);\par
当\(b = \text{e}\)时,
函数\(h(x)\)的单调递增区间为\(( - \infty, + \infty)\),
无单调递减区间;\par
当\(b > \text{e}\)时,
函数\(h(x)\)的单调递增区间为\(( - \infty,1)\)和\(\left( \ln b, + \infty \right)\),
单调递减区间为\(\left( 1,\ln b \right)\).\par
(3)由\(f(x) \leq g(x)\),\(x > 1\),
\(a > 0\),\par
则\(ax^{a}\lnx \leq xe^{x}\),
即\(x^{a}\lnx^{a} \leq e^{x} \cdot \lne^{x}\),
而\(x^{a} > 1\),\(\mathrm{e}^{x} > 1\),\par
设\(F(x) = x\lnx\),\(x > 1\),
则\(F\left( x^{a} \right) \leq F\left( e^{x} \right)\),\par
而\(F'(x) = \lnx + 1 > 0\),\par
所以函数\(F(x) = x\lnx\)在\((1, + \infty)\)上单调递增,\par
则\(x^{a} \leq e^{x}\)对于任意的\(x > 1\)都成立,
即\(a \leq \frac{x}{\lnx}\)对于任意的\(x > 1\)都成立,\par
令\(p(x) = \frac{x}{\lnx}\),\(x > 1\),
则\(p'(x) = \frac{\lnx - 1}{{(\lnx)}^{2}}\),\par
令\(p'(x) > 0\),得\(x > e\),令\(p'(x) < 0\),
得\(1 < x < e\),\par
则\(p(x)\)在\((1,e)\)上单调递减,
在\((e, + \infty)\)上单调递增,\par
故\(p(x)_{\min} = p(e) = e\),
则\(a \leq e\),\par
所以\(a\)的最大值为\(e\).}
\end{question}

\begin{question}
已知数列\(\left\{ a_{n} \right\}\)的前\emph{n}项和为\(S_{n}\),
且\(\frac{S_{n}}{a_{n}} = \frac{1}{2}a_{n + 1}\left( n \in \mathbf{N}^{*} \right)\),
其中\(a_{1} = 1,a_{n} \neq 0\).
\begin{enumerate}[label=(\arabic*)]
  \item 求数列\(\left\{ a_{n} \right\}\)的通项公式.
  \item 设数列\(\left\{ b_{n} \right\}\)满足\(\left( 2a_{n} - 1 \right)\left( 2^{b_{n}} - 1 \right) = 1,T_{n}\)为\(\left\{ b_{n} \right\}\)的前\emph{n}项和,
\item 求证:\(2T_{n} > \log_{2}\left( 2a_{n} + 1 \right),n \in \mathbf{N}^{*}\).
  \item 已知正整数\(m,d\)且\(m \geq d\),
\item 对\(\forall n \in \mathbf{N}^{\text{*}}\)有\(\left( \frac{1}{3} \right)^{m} + \left( \frac{1}{3} \right)^{m + d} + \left( \frac{1}{3} \right)^{m + 2d} + \cdots + \left( \frac{1}{3} \right)^{m + (n - 1)d} \leq \frac{n + 1}{a_{k}\sqrt{2S_{n}}}\)恒成立,
\item 正整数\(k\)的最大值为8,求\emph{m}和\emph{d}的值.
\end{enumerate}
\topics{由递推关系证明等比数列;利用an与sn关系求通项或项;数列不等式恒成立问题}
\difficulty{0.15}
\answer{(1)\(a_{n} = n\left( n \in N^{*} \right)\)
(2)见详解
(3)\(m = d = 2\)}
\explain{(1)已知\(S_{n} = \frac{1}{2}a_{n}a_{n + 1}\),
\(a_{n + 1} = S_{n + 1} - S_{n} = \frac{1}{2}a_{n + 1}a_{n + 2} - \frac{1}{2}a_{n}a_{n + 1}\),\par
因为\(a_{n} \neq 0\),
当然\(a_{n + 1} \neq 0\),
所以\(a_{n + 2} - a_{n} = 2\left( n \in \mathbf{N}^{*} \right)\).\par
由于\(a_{1} = S_{1} = \frac{1}{2}a_{1}a_{2}\),
且\(a_{1} = 1\),故\(a_{2} = 2\).\par
于是\(a_{2m - 1} = 1 + 2(m - 1) = 2m - 1,a_{2m} = 2 + 2(m - 1) = 2m\),\par
所以\(a_{n} = n\left( n \in \mathbf{N}^{*} \right)\).\par
(2)\(\left( 2a_{n} - 1 \right)\left( 2^{b_{n}} - 1 \right) = 1\),
得\((2n - 1)\left( 2^{b_{n}} - 1 \right) = 1,2^{b_{n}} = \frac{2n}{2n - 1},b_{n} = \log_{2}\frac{2n}{2n - 1}\).\par
\(T_{n} = b_{1} + b_{2} + \cdots + b_{n} = \log_{2}\left( \frac{2}{1} \cdot \frac{4}{3} \cdot \frac{6}{5}\cdots\frac{2n}{2n - 1} \right)2T_{n} = 2\log_{2}\left( \frac{2}{1} \cdot \frac{4}{3} \cdot \frac{6}{5}\cdots\frac{2n}{2n - 1} \right) =\log_{2}\left( \frac{2}{1} \cdot \frac{4}{3} \cdot \frac{6}{5}\cdots\frac{2n}{2n - 1} \right)^{2}2T_{n} - \log_{2}\left( 2a_{n} + 1 \right) =\log_{2}\left( \frac{2}{1} \cdot \frac{4}{3} \cdot \frac{6}{5}\cdots\frac{2n}{2n - 1} \right)^{2} - \log_{2}(2n + 1)\),\par
设\(f(n) = \left( \frac{2}{1} \cdot \frac{4}{3} \cdot \frac{6}{5}\cdots\frac{2n}{2n - 1} \right)^{2} \cdot \frac{1}{2n + 1}\),\par
则\(f(n + 1) = \left( \frac{2}{1} \cdot \frac{4}{3} \cdot \frac{6}{5} \cdot \frac{2n}{2n - 1} \cdot \frac{2n + 2}{2n + 1} \right)^{2} \cdot \frac{1}{2n + 3}\)\par
故\(\frac{f(n + 1)}{f(n)} = \frac{2n + 1}{2n + 3} \cdot \left( \frac{2n + 2}{2n + 1} \right)^{2} = \frac{{(2n + 2)}^{2}}{(2n + 3)(2n + 1)} = \frac{4n^{2} + 8n + 4}{4n^{2} + 8n + 3} > 1\),\par
注意到\(f(n) > 0\),所以\(f(n + 1) > f(n)\),\par
特别地\(f(n) \geq f(1) = \frac{4}{3} > 1\),
从而\(2T_{n} - \log_{2}\left( 2a_{n} + \right\).`<!-- -->`{=html}1)\(= \log_{2}f(n) > 0\),\par
所以\(2T_{n} > \log_{2}\left( 2a_{n} + 1 \right),n \in N^{*}\).\par
(3)由(1)得\(a_{n} = n,S_{n} = \frac{1}{2}n(n + 1)\),\par
则\(\frac{n + 1}{a_{k}\sqrt{2S_{n}}} = \frac{n + 1}{k\sqrt{2 \times \frac{1}{2}n(n + 1)}} = \frac{n + 1}{k\sqrt{n(n + 1)}} = \frac{\sqrt{n + 1}}{k\sqrt{n}}\),
\(\frac{\sqrt{n + 1}}{\sqrt{n}} \geq 1 \Rightarrow \frac{\sqrt{n + 1}}{k\sqrt{n}} \geq \frac{1}{k}(n \rightarrow \infty)\)取等,\par
\(\left( \frac{1}{3} \right)^{m} + \left( \frac{1}{3} \right)^{m + d} + \left( \frac{1}{3} \right)^{m + 2d} + \cdots + \left( \frac{1}{3} \right)^{m + (n - 1)d} = \left( \frac{1}{3} \right)^{m} \cdot \frac{1 - \left\lbrack \left( \frac{1}{3} \right)^{d} \right\rbrack^{n}}{1 - \left( \frac{1}{3} \right)^{d}} < \frac{\left( \frac{1}{3} \right)^{m}}{1 - \left( \frac{1}{3} \right)^{d}}\),\par
对\(\forall n \in \mathbf{N}^{\text{*}}\)恒成立可得:\(\frac{\left( \frac{1}{3} \right)^{m}}{1 - \left( \frac{1}{3} \right)^{d}} \leq \frac{1}{k}\),\par
正整数的最大值为8,因此给定\(k = 8\)时取等号,\par
即\(\left( \frac{1}{3} \right)^{m} = \frac{1}{8}\left\lbrack 1 - \left( \frac{1}{3} \right)^{d} \right\rbrack\),
整理得\(3^{m} - 3^{m - d} = 8\),\par
当\(m \geq d\)时,
得\(3^{m - d}\left( 3^{d} - 1 \right) = 8\),
因为\(m,d \in N^{*}\),
枚举\(m,d\)可得\(m = d = 2\).\par
综上所述,存在正整数\(m = d = 2\),\par
使得\(\lim_{n arrow \infty}\left\lbrack \left( \frac{1}{3} \right)^{m} + \left( \frac{1}{3} \right)^{m + d} + \left( \frac{1}{3} \right)^{m + 2d} + ... + \left( \frac{1}{3} \right)^{m + (n - 1)d} \right\rbrack= \frac{1}{a_{8}}\)成立.}
\end{question}
