\examxtitle{测试试卷 - jiangsu-yangzhou-2025-2026-midterm}

\section{单选题}

\begin{question}
已知\(A = \left\{ x\left| \lnx < 1,x \in Z \right.\  \right\},B = \left\{ x\mid |x| \leq 3 \  \right\}\),
则\(A \cap B =\)(    )
\begin{choices}
  \item \(\left\{ 1,2,3 \right\}\)
  \item \(\left\{ 1,2 \right\}\)
  \item \(\left( 0,\text{e} \right)\)
  \item \((0,3\rbrack\)
\end{choices}
\topics{交集的概念及运算;由对数函数的单调性解不等式}
\difficulty{0.65}
\answer{B}
\explain{\(\lnx < 1\)得\(0 < x < \text{e}\),
则\(A = \left\{ 1,2 \right\}\),\par
又\(B = \left\{ x\mid |x| \leq 3 \  \right\} = \left\{ x\mid - 3 \leq x \leq 3 \  \right\}\),
则\(A \cap B = \left\{ 1,2 \right\}\)}
\end{question}

\begin{question}
函数\(y = \sin3x\)图象的一条对称轴方程为(    )
\begin{choices}
  \item \(x = 0\)
  \item \(x = \frac{\pi}{12}\)
  \item \(x = \frac{\pi}{6}\)
  \item \(x = \frac{\pi}{3}\)
\end{choices}
\topics{求正弦(型)函数的对称轴及对称中心}
\difficulty{0.94}
\answer{C}
\explain{令\(3x = \frac{\pi}{2} + k\pi,\text{\quad}k \in \mathbb{Z}\),\par
得:\(x = \frac{\pi}{6} + \frac{k\pi}{3},\text{\quad}k \in \mathbb{Z}\),\par
所以函数\(y = \sin3x\)图象的对称轴方程为:\(x = \frac{\pi}{6} + \frac{k\pi}{3},k \in \mathbb{Z}\).\par
令\(k = - 1\)得:\(x = - \frac{\pi}{6}\),\par
令\(k = 0\)得:\(x = \frac{\pi}{6}\),\par
令\(k = 1\)得:\(x = \frac{\pi}{2}\),\par
故只有选项C正确}
\end{question}

\begin{question}
"\(\forall x_{1},x_{2} \in \lbrack 1,2\rbrack\),
当\(x_{1} < x_{2}\)时,
都有\(f\left( x_{1} \right) < f\left( x_{2} \right)\)"是"\(\forall x \in \lbrack 1,2\rbrack\),
都有\(f(x) \leq f(2)\)"的(    )
\begin{choices}
  \item 充分且不必要条件
  \item 必要且不充分条件
  \item 充要条件
  \item 既不充分又不必要条件
\end{choices}
\topics{判断命题的充分不必要条件;定义法判断或证明函数的单调性}
\difficulty{0.85}
\answer{A}
\explain{\(\forall x_{1},x_{2} \in \lbrack 1,2\rbrack\),
当\(x_{1} < x_{2}\)时,
都有\(f\left( x_{1} \right) < f\left( x_{2} \right)\),
则\(f(x)\)在\(\lbrack 1,2\rbrack\)单调递增,
所以\(f(x) \leq f(2)\);\par
反之\(\forall x \in \lbrack 1,2\rbrack\),
都有\(f(x) \leq f(2)\),
不能得到\(f(x)\)在\(\lbrack 1,2\rbrack\)单调递增,\par
例如\(f(x) = (x - 1.1)^{2}\),
该函数在\(\lbrack 1,1.1\rbrack\)上单调递减,
在\(\lbrack 1.1,2\rbrack\)上单调递增,\(f(1) < f(2)\),\par
显然满足\(f(x) \leq f(2)\),
但在\(\lbrack 1,2\rbrack\)上不单调递增;\par
所以"\(\forall x_{1},x_{2} \in \lbrack 1,2\rbrack\),
当\(x_{1} < x_{2}\)时,
都有\(f\left( x_{1} \right) < f\left( x_{2} \right)\)"是\par
"\(\forall x \in \lbrack 1,2\rbrack\),
都有\(f(x) \leq f(2)\)"的充分且不必要条件}
\end{question}

\begin{question}
函数\(y = f(x)\)的部分图象如图所示,则\(y = f(x)\)的解析式可以是(    )
\begin{choices}
  \item \(f(x) = x + \sinx\)
  \item \(f(x) = x\sinx\)
  \item \(f(x) = x + \cosx\)
  \item \(f(x) = x\cosx\)
\end{choices}

\begin{center}
% IMAGE_TODO_START id=jiangsu-yangzhou-2025-2026-midterm-Q4-img1 path=/Users/muryor/code/mynote/word\\_to\\_tex/output/figures/jiangsu-yangzhou-2025-2026-midterm/media/image2.png width=60% inline=false question_index=4 sub_index=1
% CONTEXT_BEFORE: $$y = f(x)$$的部分图象如图所示,则$$y = f(x)$$的解析式可以是( )
% CONTEXT_AFTER: > A.$$f(x) = x + x$$ B.$$f(x) = x\text
\begin{tikzpicture}[scale=1.05,>=Stealth,line cap=round,line join=round]
  % TODO: AI_AGENT_REPLACE_ME (id=jiangsu-yangzhou-2025-2026-midterm-Q4-img1)
\end{tikzpicture}
% IMAGE_TODO_END id=jiangsu-yangzhou-2025-2026-midterm-Q4-img
1
\end{center}

\topics{函数图像的识别}
\difficulty{0.85}
\answer{D}
\explain{由函数解析式及奇函数定义知,AD为奇函数,B为偶函数,C为非奇非偶函数,\par
由\(y = f(x)\)图象可知,函数为奇函数,排除BC,\par
又\(f(x) = x + \sinx = 0\)只有一解\(x = 0\),
即函数只有一个零点,故排除A}
\end{question}

\begin{question}
\(\forall x \in R,\lbrack x\rbrack\)表示不超过\(x\)的最大整数,
十八世纪,函数\(y = \lbrack x\rbrack\)被高斯采用,
因此得名高斯函数,
例如:\(\left\lbrack - 2.2\left\rbrack = - 3, \right\lbrack 2.1 \right\rbrack = 2\).若\({\lbrack x\rbrack}^{2} - 2\lbrack x\rbrack \leq 0\),
则实数\(x\)的取值范围是(    )
\begin{choices}
  \item \(\lbrack 0,2\rbrack\)
  \item \((0,2)\)
  \item \(\lbrack 0,3)\)
  \item \((0,3)\)
\end{choices}
\topics{解不含参数的一元二次不等式;函数新定义}
\difficulty{0.65}
\answer{C}
\explain{因为\({\lbrack x\rbrack}^{2} - 2\lbrack x\rbrack \leq 0\),
则\(0 \leq \lbrack x\rbrack \leq 2\),\par
根据高斯函数定义可得实数\(x \in \lbrack 0,3)\)}
\end{question}

\begin{question}
若\(2\sin\theta - \cos\theta = 2,\theta \in \left( \frac{\pi}{2},\pi \right)\),
则\(\cos2\theta =\)(    )
\begin{choices}
  \item \(- \frac{3}{5}\)
  \item \(\frac{3}{5}\)
  \item \(- \frac{7}{25}\)
  \item \(\frac{7}{25}\)
\end{choices}
\topics{三角函数的化简;求值------同角三角函数基本关系;二倍角的余弦公式}
\difficulty{0.85}
\answer{D}
\explain{联立\(\left\{ \begin{array}{r}
2\sin\theta - \cos\theta = 2 \\
\sin^{2}\theta + \cos^{2}\theta = 1
\end{array} \right.\),解得:\par
\(\left\{ \begin{array}{r}
\cos\theta = 0 \\
\sin\theta = 1
\end{array} \right.\)或\(\left\{ \begin{array}{r}
\cos\theta = - \frac{4}{5} \\
\sin\theta = \frac{3}{5}
\end{array} \right.\),\par
又因为\(\theta \in \left( \frac{\pi}{2},\pi \right)\),则\(\cos\theta < 0\),所以\(\left\{ \begin{array}{r}
\cos\theta = - \frac{4}{5} \\
\sin\theta = \frac{3}{5}
\end{array} \right.\),\par
所以\(\cos 2\theta = 1 - 2\sin^{2}\theta = 1 - 2 \times \left( \frac{3}{5} \right)^{2} = 1 - 2 \times \frac{9}{25} = 1 - \frac{18}{25} = \frac{7}{25}\)}
\end{question}

\begin{question}
在\(\bigtriangleup ABC\)中,
角\(A,B,C\)的对边分别为\(a,b,c\),
若\(a + 2b\cosA = 2c,2\sin\left( C + \frac{\pi}{3} \right) = \sqrt{3}a\),
则边\(b\)的值为(    )
\begin{choices}
  \item \(\frac{\sqrt{3}}{3}\)
  \item 1
  \item \(\sqrt{3}\)
  \item 2
\end{choices}
\topics{诱导公式二;三;四;正弦定理解三角形;余弦定理解三角形}
\difficulty{0.65}
\answer{B}
\explain{由\(a + 2b\cosA = 2c\)结合余弦定理可得\(a + 2b \cdot \frac{b^{2} + c^{2} - a^{2}}{2bc} = 2c\),\par
化简得\(a^{2} + c^{2} - b^{2} = ac\),
则\(\cos B = \frac{a^{2} + c^{2} - b^{2}}{2ac} = \frac{1}{2}\),\par
又\(B \in \left( 0,\pi \right)\),
则\(B = \frac{\pi}{3}\),\par
则\(2\sin\left( C + \frac{\pi}{3} \right) = 2\sin(C + B) = 2\sin\left( \pi - A \right) = 2\sin A = \sqrt{3}a\),
即\(\frac{2}{\sqrt{3}} = \frac{a}{\sin A}\),\par
则由正弦定理\(\frac{2}{\sqrt{3}} = \frac{a}{\sin A} = \frac{b}{\sin B}\),
得\(b = \frac{2}{\sqrt{3}}\sin B = 1\)}
\end{question}

\begin{question}
已知\(a = \sin\frac{\pi}{9},b = \ln\frac{9}{8},c = \frac{2}{9}\),
则它们的大小关系正确的是(    )
\begin{choices}
  \item \(b < a < c\)
  \item \(b < c < a\)
  \item \(a < c < b\)
  \item \(c < b < a\)
\end{choices}
\topics{用导数判断或证明已知函数的单调性;比较函数值的大小关系}
\difficulty{0.65}
\answer{B}
\explain{令\(f(x) = \ln(x + 1) - x,(x \geq 0)\),
求导得:\(f'(x) = \frac{1}{x + 1} - 1 = \frac{- x}{x + 1} \leq 0\),\par
故\(f(x)\)在\((0, + \infty)\)上单调递减,
所以\(f\left( \frac{1}{8} \right) < f(0) = 0\),
故\(\ln\frac{9}{8} - \frac{1}{8} < 0 \Rightarrow \ln\frac{9}{8} < \frac{1}{8}\),\par
又因为\(\frac{1}{8} < \frac{2}{9}\),
故\(\ln\frac{9}{8} < \frac{1}{8} < \frac{2}{9}\),
故\(b < c\);\par
令\(g(x) = \sin\pix - 2x,\left( 0 \leq x \leq \frac{1}{6} \right)\),
求导得:\(g'(x) = \pi\cos\pix - 2\),\par
由\(0 \leq x \leq \frac{1}{6}\)得:\(\frac{\sqrt{3}}{2} \leq \cos\pix \leq 1\),
所以\(\frac{\sqrt{3}\pi}{2} \leq \pi\cos\pix \leq \pi\),\par
而\(\frac{3\pi^{2}}{4} > \frac{16}{4} = 2^{2}\),
故\(g'(x) = \pi\cos\pix - 2 > 0\)对\(x \in \left\lbrack 0,\frac{1}{6} \right\rbrack\)上恒成立,\par
故\(g(x) = \sin\pix - 2x\)在\(x \in \left\lbrack 0,\frac{1}{6} \right\rbrack\)上单调递增,\par
故\(g\left( \frac{1}{9} \right) > g(0) = 0\),
故\(\sin\frac{\pi}{9} > \frac{2}{9}\),
故\(a > c\);\par
综上:\(b < c < a\)}
\end{question}

\section{多选题}

\begin{question}
在正\(\bigtriangleup ABC\)中,边长为3,\(D\)为边\(BC\)的中点,则下列结论正确的有(    )
\begin{choices}
  \item \(\overrightarrow{AB} + \overrightarrow{BC} + \overrightarrow{CA} = \overrightarrow{0}\)
  \item \(\overrightarrow{AB}\)在\(\overrightarrow{BC}\)上的投影向量为\(\overrightarrow{BD}\)
  \item \(\left( \overrightarrow{AB} + \overrightarrow{AC} \right) \cdot \overrightarrow{BC} = 0\)
  \item \(\overrightarrow{AB} \cdot \overrightarrow{AD} = \frac{27}{4}\)
\end{choices}

\begin{center}
% IMAGE_TODO_START id=jiangsu-yangzhou-2025-2026-midterm-Q9-img1 path=/Users/muryor/code/mynote/word\\_to\\_tex/output/figures/jiangsu-yangzhou-2025-2026-midterm/media/image3.png width=60% inline=false question_index=9 sub_index=1
% CONTEXT_BEFORE: s {4} = {4}$$. 故选项D正确. 
\begin{tikzpicture}[scale=1.05,>=Stealth,line cap=round,line join=round]
  % TODO: AI_AGENT_REPLACE_ME (id=jiangsu-yangzhou-2025-2026-midterm-Q9-img1)
\end{tikzpicture}
% IMAGE_TODO_END id=jiangsu-yangzhou-2025-2026-midterm-Q9-img
1
\end{center}

\topics{向量加法的法则;用定义求向量的数量积;数量积的运算律;求投影向量}
\difficulty{0.85}
\answer{ACD}
\explain{\(\overrightarrow{AB} + \overrightarrow{BC} + \overrightarrow{CA} = \overrightarrow{AC} + \overrightarrow{CA} = \overrightarrow{AC} + ( - \overrightarrow{AC}) = \overrightarrow{0}\),\par
故选项A正确;\par
\(\overrightarrow{AB}\)在\(\overrightarrow{BC}\)上的投影向量为:\par
\(\frac{\overrightarrow{AB} \cdot \overrightarrow{BC}}{\left| \overrightarrow{BC} \right|^{2}} \cdot \overrightarrow{BC} = \frac{( - \overrightarrow{BA}) \cdot \overrightarrow{BC}}{\left| \overrightarrow{BC} \right|^{2}} \cdot \overrightarrow{BC} = \frac{- \mid \overrightarrow{BA} \mid \cdot \mid \overrightarrow{BC} \mid \cdot \cos 60^{\circ}}{\left| \overrightarrow{BC} \right|^{2}} \cdot \overrightarrow{BC}\),\par
\(= \frac{- 3 \times 3 \times \frac{1}{2}}{9} \cdot \overrightarrow{BC} = - \frac{1}{2}\overrightarrow{BC} = - \overrightarrow{BD}\),\par
故选项B错误;\par
因为\(D\)为边\(BC\)的中点,\par
所以\(\overrightarrow{AB} + \overrightarrow{AC} = 2\overrightarrow{AD}\),\par
又因为\(AB = AC\),所以\(AD\bot BC\),\par
所以
\(\left( \overrightarrow{AB} + \overrightarrow{AC} \right) \cdot \overrightarrow{BC} = (2\overrightarrow{AD}) \cdot \overrightarrow{BC} = 2(\overrightarrow{AD} \cdot \overrightarrow{BC}) = 2 \times 0 = 0\).\par
故选项C正确;\par
因为\(\bigtriangleup ABC\)为等边三角形,
且\(D\)为边\(BC\)的中点,\par
所以\(\angle BAD = 30^{\circ}\),
\(\left| \overrightarrow{AB} \right| = 3\),
\(\left| \overrightarrow{AD} \right| = \frac{\sqrt{3}}{2} \times 3 = \frac{3\sqrt{3}}{2}\),
\(\cos 30^{\circ} = \frac{\sqrt{3}}{2}\),\par
所以:\(\overrightarrow{AB} \cdot \overrightarrow{AD} = 3 \times \frac{3\sqrt{3}}{2} \times \frac{\sqrt{3}}{2} = 3 \times \frac{3 \times 3}{4} = 3 \times \frac{9}{4} = \frac{27}{4}\).\par
故选项D正确}
\end{question}

\begin{question}
下列各式结果为1的有(    )
\begin{choices}
  \item \(2\left( 1 - 2\sin^{2}15^{\circ} \right)\)
  \item \(\frac{\sqrt{3} - \tan15^{\circ}}{1 + \sqrt{3}\tan15^{\circ}}\)
  \item \(\sin15^{\circ}\cos15^{\circ}\)
  \item \(\cos40^{\circ}\left( 1 + \sqrt{3}\tan10^{\circ} \right)\)
\end{choices}
\topics{用和;差角的正切公式化简;求值;二倍角的正弦公式;二倍角的余弦公式;辅助角公式}
\difficulty{0.65}
\answer{BD}
\explain{\(2\left( 1 - 2\sin^{2}15{^\circ} \right) = 2 \times \cos30{^\circ} = \sqrt{3}\),
A选项错;\par
\(\frac{\sqrt{3} - \tan15{^\circ}}{1 + \sqrt{3}\tan15^{{^\circ}}} = \frac{\tan60{^\circ} - \tan15{^\circ}}{1 + \tan60{^\circ}\tan15^{{^\circ}}} = \tan(60{^\circ} - 15{^\circ})\text{=tan}45{^\circ} = 1\),
B选项正确;\par
\(\sin15{^\circ}\cos15{^\circ} = \frac{1}{2} \times \left( 2\sin15{^\circ}\cos15{^\circ} \right) = \frac{1}{2}\sin30{^\circ} = \frac{1}{4}\),
C选项错误;\par
\(\cos40{^\circ}\left( 1 + \sqrt{3}\tan10{^\circ} \right) = \cos40{^\circ}\left( 1 + \frac{\sqrt{3}\sin10^{{^\circ}}}{\cos10{^\circ}} \right) = \cos40{^\circ}\left( \frac{\sqrt{3}\sin10^{{^\circ}} + \cos10{^\circ}}{\cos10{^\circ}} \right) = \cos40{^\circ}\left( \frac{2\text{sin4}0{^\circ}}{\cos10{^\circ}} \right) = \frac{\sin80{^\circ}}{\cos10{^\circ}} = \frac{\sin80{^\circ}}{\sin80{^\circ}} = 1\),
D选项正确}
\end{question}

\begin{question}
已知\(x > 0,y > 0\)且\(x + y = 1\),则下列结论正确的有(    )
\begin{choices}
  \item \(\frac{1}{x} + \frac{2}{y}\)的最小值为\(3 + 2\sqrt{2}\)
  \item \(\sqrt{x} + \sqrt{y}\)的最大值为\(\sqrt{2}\)
  \item \(\frac{1}{4x^{2} + 1} + \frac{1}{4y^{2} + 1}\)的最大值为\(1\)
  \item \(\frac{x + 2y + \sqrt{x^{2} + 4y^{2}}}{xy}\)的最小值为\(10\)
\end{choices}
\topics{基本(均值)不等式的应用}
\difficulty{0.4}
\answer{AB}
\explain{A选项,因为\(x > 0,y > 0\)且\(x + y = 1\),
所以\(\frac{1}{x} + \frac{2}{y} = (\frac{1}{x} + \frac{2}{y})(x + y) = 1 + \frac{y}{x} + \frac{2x}{y} + 2 = 3 + \frac{y}{x} + \frac{2x}{y}\),\par
根据基本不等式可得\(\frac{y}{x} + \frac{2x}{y} \geq 2\sqrt{\frac{y}{x} \cdot \frac{2x}{y}} = 2\sqrt{2}\),
当且仅当\(\frac{y}{x} = \frac{2x}{y}\)且\(x + y = 1\)时取等号,
所以\(\frac{1}{x} + \frac{2}{y} \geq 3 + 2\sqrt{2}\),\par
A 选项正确;\par
B选项,
\({(\sqrt{x} + \sqrt{y})}^{2} = x + y + 2\sqrt{xy} = 1 + 2\sqrt{xy}\),\par
因为\(xy \leq {(\frac{x + y}{2})}^{2} = \frac{1}{4}\),
当且仅当\(x = y = \frac{1}{2}\)时取等号,\par
所以\({(\sqrt{x} + \sqrt{y})}^{2} \leq 1 + 2 \times \frac{1}{2} = 2\),
则\(\sqrt{x} + \sqrt{y} \leq \sqrt{2}\),\par
当且仅当\(x = y = \frac{1}{2}\)时取等号,
即最大值为\(\sqrt{2}\),B选项正确;\par
C选项,因为\(x + y = 1\),
所以\(4x^{2} + 1 = 4x^{2} + {(x + y)}^{2} = 5x^{2} + 2xy + y^{2}\),
\(4y^{2} + 1 = 4y^{2} + {(x + y)}^{2} = 5y^{2} + 2xy + x^{2}\),
\(\frac{1}{4x^{2} + 1} + \frac{1}{4y^{2} + 1} = \frac{4y^{2} + 1 + 4x^{2} + 1}{(4x^{2} + 1)(4y^{2} + 1)} = \frac{4(x^{2} + y^{2}) + 2}{16x^{2}y^{2} + 4(x^{2} + y^{2}) + 1}\),
又\(x^{2} + y^{2} = 1 - 2xy\),
代入得:\(\frac{4(1 - 2xy) + 2}{16x^{2}y^{2} + 4(1 - 2xy) + 1} = \frac{6 - 8xy}{16x^{2}y^{2} - 8xy + 5}\),\par
令\(t = xy)(0 < t \leq \frac{1}{4}\)),
表达式变为\(\frac{6 - 8t}{16t^{2} - 8t + 5}\),\par
取特殊值\(x = \frac{1}{3}\text{,}y = \frac{2}{3}\)(此时\(t = \frac{2}{9}\)),
计算得:\par
\(\frac{1}{4\left( \frac{1}{3} \right)^{2} + 1} + \frac{1}{4\left( \frac{2}{3} \right)^{2} + 1} = \frac{9}{13} + \frac{9}{25} = \frac{342}{325} > 1\),\par
而当\(x = y = \frac{1}{2}\)时,值为\(1\),
故1不是最大值,选项C错误;\par
选项D,化简表达式:\par
\(\frac{x + 2y}{xy} + \frac{\sqrt{x^{2} + 4y^{2}}}{xy} = \frac{1}{y} + \frac{2}{x} + \frac{\sqrt{x^{2} + 4y^{2}}}{xy}\),\par
取特殊值\(x = \frac{2}{3}\text{,}y = \frac{1}{3}\)(此时\(x = 2y\)),
代入计算:\par
\(\frac{1}{\frac{1}{3}} + \frac{2}{\frac{2}{3}} + \frac{\sqrt{\left( \frac{2}{3} \right)^{2} + 4\left( \frac{1}{3} \right)^{2}}}{\frac{2}{3} \cdot \frac{1}{3}} = 3 + 3 + 3\sqrt{2} \approx 10.24 > 10\),说明最小值不是10,
选项D错误}
\end{question}

\section{填空题}

\begin{question}
已知向量\(\overrightarrow{a},\overrightarrow{b},\overrightarrow{c}\)均为单位向量,
且\(\overrightarrow{a} + \overrightarrow{b} = \overrightarrow{c}\),
则\(\overrightarrow{a}\)和\(\overrightarrow{b}\)的夹角大小为
.
\topics{向量夹角的计算}
\difficulty{0.85}
\answer{\(120^{\circ}\)/\(\frac{2\pi}{3}\)}
\explain{由于 \(\overrightarrow{a}\),
 \(\overrightarrow{b}\),
\(\overrightarrow{c}\) 均为单位向量,故
\(\left| \overset{\rightarrow}{a} | = 1\),\par
\(\left| \overset{\rightarrow}{b} | = 1\),\par
\(\left| \overset{\rightarrow}{c} | = 1\).\par
给定
\(\overrightarrow{a} + \overrightarrow{b} = \overrightarrow{c}\),
对两边取模的平方得:\par
\(\left| \overset{\rightarrow}{a} + {\overset{arrow}{b}}^{2} | = \left| \overset{\rightarrow}{c} |^{2}\)代入
\(\left| \overset{\rightarrow}{c} | = 1\),
得:\par
\(\left| \overset{\rightarrow}{a} + \overset{arrow}{b} |^{2} = \overset{arrow}{a} \cdot \overset{arrow}{a} + 2\overset{arrow}{a} \cdot \overset{arrow}{b} + \overset{arrow}{b} \cdot \overset{arrow}{b} = 1 + 2\overset{arrow}{a} \cdot \overset{arrow}{b} + 1 = 2 + 2 \times 1 \times 1 \times \cos\left. <\overset{\rightarrow}{a},\overset{arrow}{b} > = 1\),\par
解得:\(\cos\left\langle \overrightarrow{a},\overrightarrow{b} \right\rangle = - \frac{1}{2}\)
,
\(\left. <\overrightarrow{a},\overrightarrow{b} \right.> \in \lbrack 0,\pi\rbrack,\left. <\overrightarrow{a},\overrightarrow{b} \right.> = 120^{\circ}\).\(120^{\circ}\)}
\end{question}

\begin{question}
写出一个同时具有下列性质①②③的函数\(f(x) =\)
.①\(f\left( x_{1}x_{2} \right) = f\left( x_{1} \right)f\left( x_{2} \right)\);
②当\(x \in (0, + \infty)\)时,\(f'(x) < 0\);
③\(f(x)\)是奇函数.
\topics{函数奇偶性的定义与判断;用导数判断或证明已知函数的单调性}
\difficulty{0.65}
\answer{\(f(x) = \frac{1}{x}\)(答案不唯一)}
\explain{取\(f(x) = \frac{1}{x}\),
则\(f\left( x_{1}x_{2} \right) = \frac{1}{x_{1}x_{2}},f\left( x_{1} \right)f\left( x_{2} \right) = \frac{1}{x_{1}} \times \frac{1}{x_{2}} = \frac{1}{x_{1}x_{2}}\),\par
所以\(f\left( x_{1}x_{2} \right) = f\left( x_{1} \right)f\left( x_{2} \right)\),
满足①,\par
\(f'(x) = - \frac{1}{x^{2}}\),
当\(x \in (0, + \infty)\)时,有\(f'(x) < 0\),
满足②,\par
\(f(x)\)的定义域为\(( - \infty,0) \cup (0, + \infty)\),
又\(f( - x) = \frac{1}{- x} = - \frac{1}{x} = - f(x)\),\par
故\(f(x)\)是奇函数,
满足③.\(f(x) = \frac{1}{x}\)(答案不唯一).}
\end{question}

\begin{question}
在平面四边形\(ABCD\)中,
\(\angle A = \frac{\pi}{3},\angle B = \frac{2\pi}{3},AB = 2,AD = 3\),
若满足上述条件的平面四边形\(ABCD\)有且只有1个,
则边\(CD\)的取值范围是
.

\begin{center}
% IMAGE_TODO_START id=jiangsu-yangzhou-2025-2026-midterm-Q14-img1 path=/Users/muryor/code/mynote/word\\_to\\_tex/output/figures/jiangsu-yangzhou-2025-2026-midterm/media/image4.png width=60% inline=false question_index=14 sub_index=1
\begin{tikzpicture}[scale=1.05,>=Stealth,line cap=round,line join=round]
  % TODO: AI_AGENT_REPLACE_ME (id=jiangsu-yangzhou-2025-2026-midterm-Q14-img1)
\end{tikzpicture}
% IMAGE_TODO_END id=jiangsu-yangzhou-2025-2026-midterm-Q14-img
1
\end{center}

\topics{正弦定理解三角形;余弦定理解三角形;求三角形中的边长或周长的最值或范围}
\difficulty{0.4}
\answer{\(\left\{ \sqrt{3} \right\} \cup \left\lbrack \sqrt{7}, + \infty \right)\)}
\explain{在\(\bigtriangleup ABD\)中,
由余弦定理可得\(BD^{2} = AB^{2} + AD^{2} - 2AB \cdot AD\cos A = 4 + 9 - 2 \times 2 \times 3 \times \frac{1}{2} = 7\),\par
所以\(BD = \sqrt{7}\),\par
由正弦定理可得\(\frac{AB}{\sin\angle ADB} = \frac{BD}{\sin A}\),
即\(\sin\angle ADB = \frac{AB\sin A}{BD} = \frac{2 \times \frac{\sqrt{3}}{2}}{\sqrt{7}} = \frac{\sqrt{21}}{7}\),\par
因为\(\angle A = \frac{\pi}{3},\angle B = \frac{2\pi}{3}\),
所以\(AD//BC\),\par
所以\(\sin\angle DBC = \frac{\sqrt{21}}{7}\),\par
在\(\bigtriangleup DBC\)中,
由正弦定理可得\(\frac{CD}{\sin\angle DBC} = \frac{BD}{\sin C}\),\par
即\(CD = \frac{BD\sin\angle DBC}{\sin C} = \frac{\sqrt{7} \times \frac{\sqrt{21}}{7}}{\sin C} = \frac{\sqrt{3}}{\sin C}\),\par
因为满足上述条件的平面四边形\(ABCD\)有且只有1个,\par
所以\(0 < \sin C \leq \sin\angle DBC\)或\(\sin C = 1\),
得\(CD \geq \sqrt{7}\)或\(CD = \sqrt{3}\),\par
所以边\(CD\)的取值范围是\(\left\{ \sqrt{3} \right\} \cup \left\lbrack \sqrt{7}, + \infty \right)\).\(\left\{ \sqrt{3} \right\} \cup \left\lbrack \sqrt{7}, + \infty \right)\)}
\end{question}

\section{解答题}

\begin{question}
函数\(f(x) = A\sin(\omega x + \varphi)\left( A > 0,\omega > 0, - \frac{\pi}{2} < \varphi < \frac{\pi}{2} \right)\)的部分图象如图所示.
\begin{enumerate}[label=(\arabic*)]
  \item 求\(f(x)\)的解析式及其单调增区间;
  \item 若\(x \in \left\lbrack 0,\pi \right\rbrack\),
\item 求不等式\(f(x) \geq 1\)的解集.
\end{enumerate}

\begin{center}
% IMAGE_TODO_START id=jiangsu-yangzhou-2025-2026-midterm-Q15-img1 path=/Users/muryor/code/mynote/word\\_to\\_tex/output/figures/jiangsu-yangzhou-2025-2026-midterm/media/image5.png width=60% inline=false question_index=15 sub_index=1
% CONTEXT_AFTER: (1)求$$f(x)$$的解析式及其单调增区间; (2)若$$x \in \lbra
\begin{tikzpicture}[scale=1.05,>=Stealth,line cap=round,line join=round]
  % TODO: AI_AGENT_REPLACE_ME (id=jiangsu-yangzhou-2025-2026-midterm-Q15-img1)
\end{tikzpicture}
% IMAGE_TODO_END id=jiangsu-yangzhou-2025-2026-midterm-Q15-img
1
\end{center}

\topics{解正弦不等式;由图象确定正(余)弦型函数解析式;求sinx型三角函数的单调性}
\difficulty{0.65}
\answer{(1)\(f(x) = 2\sin\left( 2x - \frac{\pi}{3} \right)\),单调递增区间为\(\left\lbrack k\pi - \frac{\pi}{12},k\pi + \frac{5\pi}{12} \right\rbrack(k \in Z)\)
(2)\(\left\lbrack \frac{\pi}{4},\frac{7\pi}{12} \right\rbrack\)}
\explain{(1)由图象可知\(A = \frac{f(x)_{\max} - f(x)_{\min}}{2} = \frac{2 - ( - 2)}{2} = 2\),\par
函数\(f(x)\)的最小正周期\(T\)满足\(\frac{3}{4}T = \frac{5\pi}{12} - \left( - \frac{\pi}{3} \right) = \frac{3\pi}{4}\),
故\(T = \pi\),
所以\(\omega = \frac{2\pi}{T} = \frac{2\pi}{\pi} = 2\),\par
所以\(f(x) = 2\sin(2x + \varphi)\),\par
因为\(f\left( \frac{5\pi}{12} \right) = 2\sin\left( \frac{5\pi}{6} + \varphi \right) = 2\),
可得\(\sin\left( \frac{5\pi}{6} + \varphi \right) = 1\),\par
因为\(- \frac{\pi}{2} < \varphi < \frac{\pi}{2}\),
故\(\frac{\pi}{3} < \varphi + \frac{5\pi}{6} < \frac{4\pi}{3}\),
所以\(\varphi + \frac{5\pi}{6} = \frac{\pi}{2}\),
解得\(\varphi = - \frac{\pi}{3}\),\par
因此\(f(x) = 2\sin\left( 2x - \frac{\pi}{3} \right)\),\par
令\(2k\pi - \frac{\pi}{2} \leq 2x - \frac{\pi}{3} \leq 2k\pi + \frac{\pi}{2}(k \in Z)\),
解得\(k\pi - \frac{\pi}{12} \leq x \leq k\pi + \frac{5\pi}{12}(k \in Z)\),\par
所以\(f(x)\)的单调递增区间为\(\left\lbrack k\pi - \frac{\pi}{12},k\pi + \frac{5\pi}{12} \right\rbrack(k \in Z)\).\par
(2)由\(f(x) \geq 1\)得\(2\sin\left( 2x - \frac{\pi}{3} \right) \geq 1\),\par
即\(\sin\left( 2x - \frac{\pi}{3} \right) \geq \frac{1}{2}\),
则有\(\frac{\pi}{6} + 2k\pi \leq 2x - \frac{\pi}{3} \leq \frac{5\pi}{6} + 2k\text{\pi,}\left( k \in \mathbb{Z} \right)\),\par
解得\(\frac{\pi}{4} + k\pi \leq x \leq \frac{7\pi}{12} + k\text{\pi,}\left( k \in \mathbb{Z} \right)\),
又\(x \in \left\lbrack 0,\pi \right\rbrack\),
所以\(x \in \left\lbrack \frac{\pi}{4},\frac{7\pi}{12} \right\rbrack\),\par
综上,
不等式的解集为\(\left\lbrack \frac{\pi}{4},\frac{7\pi}{12} \right\rbrack\).}
\end{question}

\begin{question}
一个盒子中有\(6\)个大小重量相同的小球,其中\(2\)个白球,
\(4\)个黑球,甲同学从盒子中分\(3\)次随机抽取,每次抽取\(1\)个球.
\begin{enumerate}[label=(\arabic*)]
  \item 若每次抽出的球放回,求恰有\(2\)次抽取到黑球的概率;
  \item 若每次抽出的球不放回.

\item ①记抽取到的黑球个数为随机变量\(X\),求\(X\)的分布列和数学期望;

\item ②在抽取到\(1\)个黑球与\(2\)个白球的前提下,
\item 求第\(2\)次抽到黑球的概率.
\end{enumerate}
\topics{写出简单离散型随机变量分布列;计算条件概率;独立重复试验的概率问题;求离散型随机变量的均值}
\difficulty{0.65}
\answer{(1)\(\frac{4}{9}\)
(2)①分布列见解析,数学期望\(E(X) = 2\);②\(\frac{1}{3}\)}
\explain{(1)若每次抽出的球放回,
则每次抽取到黑球的概率为\(\frac{4}{6} = \frac{2}{3}\),\par
则随机抽取\(3\)次,
恰有\(2\)次抽取到黑球的概率\(p = \mathbb{C}_{3}^{2} \times \left( \frac{2}{3} \right)^{2} \times \frac{1}{3} = \frac{4}{9}\).\par
(2)①由题意知:\(X\)所有可能的取值为\(1,2,3\),\par
\(\because P(X = 1) = \frac{4}{6} \times \frac{2}{5} \times \frac{1}{4} + \frac{2}{6} \times \frac{4}{5} \times \frac{1}{4} + \frac{2}{6} \times \frac{1}{5} \times 1 = \frac{24}{120} = \frac{1}{5}\);\par
\(P(X = 2) = \frac{4}{6} \times \frac{3}{5} \times \frac{2}{4} + \frac{4}{6} \times \frac{2}{5} \times \frac{3}{4} + \frac{2}{6} \times \frac{4}{5} \times \frac{3}{4} = \frac{72}{120} = \frac{3}{5}\);\par
\(P(X = 3) = \frac{4}{6} \times \frac{3}{5} \times \frac{2}{4} = \frac{24}{120} = \frac{1}{5}\);\par
\(\therefore X\)的分布列为:\par
  ----------------- ----------------- ----------------- -----------------
        \(X\)             \(1\)             \(2\)             \(3P\)        \(\frac{1}{5}\)   \(\frac{3}{5}\)   \(\frac{1}{5}\)
  ----------------- ----------------- ----------------- -----------------\par
\(\therefore\)数学期望\(E(X) = 1 \times \frac{1}{5} + 2 \times \frac{3}{5} + 3 \times \frac{1}{5} = 2\).\par
②记事件\(A\)为"抽取到\(1\)个黑球与\(2\)个白球",
事件\(B\)为"第\(2\)次抽到黑球",\par
则事件\(AB\)为"第\(1\)次和第\(3\)次抽到白球,
第\(2\)次抽到黑球";\par
\(\because P(AB) = \frac{2}{6} \times \frac{4}{5} \times \frac{1}{4} = \frac{1}{15}\),
\(P(A) = P(X = 1) = \frac{1}{5}\),
\(\therefore P\left( \left. \ B \right|A \right) = \frac{P(AB)}{P(A)} = \frac{\frac{1}{15}}{\frac{1}{5}} = \frac{1}{3}\),\par
即在抽取到\(1\)个黑球与\(2\)个白球的前提下,
第\(2\)次抽到黑球的概率为\(\frac{1}{3}\).}
\end{question}

\begin{question}
如图,四棱锥\(P - ABCD\)和四棱锥\(Q - ABCD\)中,
底面\(ABCD\)为边长为6的正方形,
\(PA\bot\)平面\(ABCD,QD\bot\)平面\(ABCD\),
且\(PA = QD = 8\).
\begin{enumerate}[label=(\arabic*)]
  \item 求证:\(PA//\)平面\(QCD\);
  \item 求直线\(QB\)与平面\(PCD\)所成角的正弦值;
  \item 求四棱锥\(P - ABCD\)和四棱锥\(Q - ABCD\)重合部分的体积.
\end{enumerate}

\begin{center}
% IMAGE_TODO_START id=jiangsu-yangzhou-2025-2026-midterm-Q17-img1 path=/Users/muryor/code/mynote/word\\_to\\_tex/output/figures/jiangsu-yangzhou-2025-2026-midterm/media/image6.png width=60% inline=false question_index=17 sub_index=1
% CONTEXT_BEFORE: t$$平面$$ABCD,QD\bot$$平面$$ABCD$$,且$$PA = QD = 8$$.
% CONTEXT_AFTER: (1)求证:$$PA//$$平面$$QCD$$; (2)求直线$$QB$$与平面$$PCD$$
\begin{tikzpicture}[scale=1.05,>=Stealth,line cap=round,line join=round]
  % TODO: AI_AGENT_REPLACE_ME (id=jiangsu-yangzhou-2025-2026-midterm-Q17-img1)
\end{tikzpicture}
% IMAGE_TODO_END id=jiangsu-yangzhou-2025-2026-midterm-Q17-img
1
\end{center}


\begin{center}
% IMAGE_TODO_START id=jiangsu-yangzhou-2025-2026-midterm-Q17-img2 path=/Users/muryor/code/mynote/word\\_to\\_tex/output/figures/jiangsu-yangzhou-2025-2026-midterm/media/image7.png width=60% inline=false question_index=17 sub_index=1
% CONTEXT_AFTER: (3)连接$$PQ$$,由$$PA//QD$$且$$PA = QD$$,可得:四边形$$APQD
\begin{tikzpicture}[scale=1.05,>=Stealth,line cap=round,line join=round]
  % TODO: AI_AGENT_REPLACE_ME (id=jiangsu-yangzhou-2025-2026-midterm-Q17-img2)
\end{tikzpicture}
% IMAGE_TODO_END id=jiangsu-yangzhou-2025-2026-midterm-Q17-img
2
\end{center}


\begin{center}
% IMAGE_TODO_START id=jiangsu-yangzhou-2025-2026-midterm-Q17-img3 path=/Users/muryor/code/mynote/word\\_to\\_tex/output/figures/jiangsu-yangzhou-2025-2026-midterm/media/image8.png width=60% inline=false question_index=17 sub_index=1
% CONTEXT_BEFORE: + 6 \times 4 \times {2} \times 3 = 60$$.
\begin{tikzpicture}[scale=1.05,>=Stealth,line cap=round,line join=round]
  % TODO: AI_AGENT_REPLACE_ME (id=jiangsu-yangzhou-2025-2026-midterm-Q17-img3)
\end{tikzpicture}
% IMAGE_TODO_END id=jiangsu-yangzhou-2025-2026-midterm-Q17-img
3
\end{center}

\topics{锥体体积的有关计算;证明线面平行;线面角的向量求法}
\difficulty{0.4}
\answer{(1)证明见解析
(2)\(\frac{12\sqrt{34}}{85}\)
(3)60}
\explain{(1)证明:因为\(PA\bot\)平面\(ABCD,QD\bot\)平面\(ABCD\),\par
所以\(PA//QD\),\par
又因为\(PA ⊄\)平面\(QCD\),
\(QD \subset\)平面\(QCD\),\par
所以\(PA//\)平面\(QCD\).\par
(2)分别以\(AB,AD,AP\)为\(x,y,z\)轴建立空间直角坐标系,
由题意得:\par
\(B(6,0,0),P(0,0,8),D(0,6,0),C(6,6,0),Q(0,6,8)\),\par
所以\(\overrightarrow{QB} = (6, - 6, - 8),\overrightarrow{CD} = ( - 6,0,0),\overrightarrow{PD} = (0,6, - 8)\),\par
设平面\(PCD\)的法向量为\(\overrightarrow{n} = (x,y,z)\),\par
故\(\left\{ \begin{array}{r}
\overrightarrow{n} \cdot \overrightarrow{CD} = (x,y,z) \cdot ( - 6,0,0) = - 6x = 0 \\
\overrightarrow{n} \cdot \overrightarrow{PD} = (x,y,z) \cdot (0,6, - 8) = 6y - 8z = 0
\end{array} \right.\),令\(y = 4\)得\(\left\{ \begin{array}{r}
x = 0 \\
y = 4 \\
z = 3
\end{array} \right.\),\par
故平面\(PCD\)的一个法向量为\(\overrightarrow{n} = (0,4,3)\),\par
直线\(QB\)与平面\(PCD\)所成角为\(\alpha\),\par
\(\sin\alpha = \left| \cos\left\langle \overrightarrow{QB},\overrightarrow{n} \right\rangle \right| = \left| \frac{\overrightarrow{QB} \cdot \overrightarrow{n}}{\left| \overrightarrow{QB} \right| \cdot \left| \overrightarrow{n} \right|} \right| = \left| \frac{- 6 \times 4 - 8 \times 3}{\sqrt{6^{2} + 6^{2} + 8^{2}} \cdot \sqrt{4^{2} + 3^{2}}} \right| = \frac{12\sqrt{34}}{85}\),\par
所以直线\(QB\)与平面\(PCD\)所成角的正弦值\(\frac{12\sqrt{34}}{85}\).\par
(3)连接\(PQ\),由\(PA//QD\)且\(PA = QD\),可得:四边形\(APQD\)为平行四边形,故\(PD,QA\)相交,设交点为\(H\);\par
易得四边形\(PBCQ\)为平行四边形,故\(PC,QB\)相交,设交点为\(G\),\par
故四棱锥\(P - ABCD\)和四棱锥\(Q - ABCD\)重合部分为几何体\(GH - ABCD\),\par
分别取\(AB,CD\)的中点\(M,N\),连接\(GM,GN,MN\),容易得到几何体\(GMN - HAD\)为三棱柱.\par
几何体\(GH - ABCD\)由四棱锥\(G - BMNC\)与三棱柱\(GMN - HAD\)组合而成.\par
所以几何体\(GH - ABCD\)的体积\(V = V_{G - BMNC} + V_{GMN - HAD} = \frac{1}{3} \times 6 \times 3 \times 4 + 6 \times 4 \times \frac{1}{2} \times 3 = 60\).}
\end{question}

\begin{question}
已知函数\(f(x) = a\mathrm{e}^{2x} + (a - 2)\mathrm{e}^{x} - x\).
\begin{enumerate}[label=(\arabic*)]
  \item 若函数\(f(x)\)在\(x = 0\)处的切线方程为\(y = 3x + b\),
\item 求\(b\)的值;
  \item 讨论\(f(x)\)的单调性;
  \item 当\(a = 1\)时,
\item 记函数\(g(x) = f(x) - x\left( \mathrm{e}^{x} - 1 \right)\),
\item 求证:函数\(g(x)\)有唯一的极大值点\(x_{0}\),
\item 且\(g\left( x_{0} \right) < \frac{1}{2\text{e}}\).
\end{enumerate}
\topics{已知切线(斜率)求参数;利用导数证明不等式;利用导数求函数(含参)的单调区间}
\difficulty{0.4}
\answer{(1)\(b = 2\);
(2)答案见解析;
(3)证明见解析.}
\explain{(1)\(f'(x) = 2a\mathrm{e}^{2x} + (a - 2)\mathrm{e}^{x} - 1\),\par
因函数\(f(x)\)在\(x = 0\)处的切线方程为\(y = 3x + b\),\par
则\(f'(0) = 3a - 3 = 3\),
\(f(0) = 2a - 2 = b\),得\(a = 2,b = 2\);\par
(2)\(f'(x) = 2a\mathrm{e}^{2x} + (a - 2)\mathrm{e}^{x} - 1 = \left( 2\mathrm{e}^{x} + 1 \right)\left( a\mathrm{e}^{x} - 1 \right)\),
\(2\mathrm{e}^{x} + 1 > 0\),\par
当\(a \leq 0\)时,\(f'(x) < 0\),
则\(f(x)\)在\(\mathbb{R}\)上单调递减;\par
当\(a > 0\)时,
\(f'(x) < 0\)得\(x < - \ln a\);
\(f'(x) > 0\)得\(x > - \ln a\);\par
则\(f(x)\)在\(\left( - \infty, - \ln a \right)\)上单调递减,
在\(\left( - \ln a, + \infty \right)\)上单调递增,\par
综上,\(a \leq 0\)时,
\(f(x)\)在\(\mathbb{R}\)上单调递减;\par
\(a > 0\)时,
\(f(x)\)在\(\left( - \infty, - \ln a \right)\)上单调递减,
在\(\left( - \ln a, + \infty \right)\)上单调递增.\par
(3)当\(a = 1\)时,
\(g(x) = f(x) - x\left( \mathrm{e}^{x} - 1 \right) = \mathrm{e}^{2x} - \mathrm{e}^{x} - x - x\left( \mathrm{e}^{x} - 1 \right) = \mathrm{e}^{2x} - (x + 1)\mathrm{e}^{x}\),\par
则\(g'(x) = 2\mathrm{e}^{2x} - (x + 2)\mathrm{e}^{x} = \mathrm{e}^{x}\left( 2\mathrm{e}^{x} - x - 2 \right)\),\par
令\(h(x) = 2\mathrm{e}^{x} - x - 2\),
则\(h'(x) = 2\mathrm{e}^{x} - 1\),\par
则\(h'(x) > 0\)得\(x > \ln\frac{1}{2}\);
\(h'(x) < 0\)得\(x < \ln\frac{1}{2}\),\par
则\(h(x)\)在\(\left( - \infty,\ln\frac{1}{2} \right)\)上单调递减,
在\(\left( \ln\frac{1}{2}, + \infty \right)\)上单调递增,
故\(h(x)_{\min} = h\left( \ln\frac{1}{2} \right) = \ln 2 - 1 < 0\),\par
又\(h(0) = 0\),
\(h( - 1) = 2\mathrm{e}^{- 1} - 1 < 0\),
\(h( - 2) = 2\mathrm{e}^{- 2} > 0\),\par
则由零点存在性定理可知,
\(\exists x_{0} \in ( - 2, - 1)\)使得\(h\left( x_{0} \right) = 2\mathrm{e}^{x_{0}} - x_{0} - 2 = 0\),\par
则\(x < x_{0}\)或\(x > 0\)时,\(h(x) > 0\),
\(g'(x) > 0\);\(x_{0} < x < 0\)时,
\(h(x) < 0\),\(g'(x) < 0\),\par
故\(g(x)\)在\(\left( - \infty,x_{0} \right)\)和\((0, + \infty)\)上单调递增,
在\(\left( x_{0},0 \right)\)上单调递减,
则\(g(x)\)有唯一的极大值点\(x_{0}\),\par
且\(g\left( x_{0} \right) = \mathrm{e}^{2x_{0}} - \left( x_{0} + 1 \right)\mathrm{e}^{x_{0}} = \mathrm{e}^{x_{0}}\left( \mathrm{e}^{x_{0}} - x_{0} - 1 \right) = \frac{x_{0} + 2}{2}\left( \frac{x_{0} + 2}{2} - x_{0} - 1 \right)= - \frac{x_{0}\left( x_{0} + 2 \right)}{4} = - \frac{x_{0}\mathrm{e}^{x_{0}}}{2}\),\par
令\(\varphi(x) = - x\mathrm{e}^{x},x \in ( - 2, - 1)\),
则\(\varphi'(x) = - (x + 1)\mathrm{e}^{x} > 0\),\par
则\(\varphi(x)\)在\(( - 2, - 1)\)上单调递增,
则\(\varphi(x) < \varphi( - 1) = \frac{1}{\text{e}}\),
故\(g\left( x_{0} \right) < \frac{1}{2\text{e}}\).}
\end{question}

\begin{question}
在\(\bigtriangleup ABC\)中,
角\(A,B,C\)的对边分别为\(a,b,c, \bigtriangleup ABC\)的面积为\(S\).已知\(\sinB = \sqrt{3}\sinC\).
\begin{enumerate}[label=(\arabic*)]
  \item 若\(a^{2} - c^{2} = \frac{2S}{\sinA}\),
\item 求\(\cosA\)的值;
  \item 若\(S = \frac{3}{2}\),求\(a^{2}\)的最小值;
  \item 若\(A = \frac{\pi}{3},c = 1,P,Q,R\).分别在边\(AB,BC,CA\)上,
\item 且\(PQ = QR = RP\),
\item 求\(\bigtriangleup PQR\)面积的最小值.
\end{enumerate}

\begin{center}
% IMAGE_TODO_START id=jiangsu-yangzhou-2025-2026-midterm-Q19-img1 path=/Users/muryor/code/mynote/word\\_to\\_tex/output/figures/jiangsu-yangzhou-2025-2026-midterm/media/image9.png width=60% inline=false question_index=19 sub_index=1
% CONTEXT_BEFORE: = c$$,可得$$b = $$, 如图,作出符合题意的图形,
% CONTEXT_AFTER: 设$$PQ = QR = RP = m$$,则$$\bigtriangleup PQR$$是等边
\begin{tikzpicture}[scale=1.05,>=Stealth,line cap=round,line join=round]
  % TODO: AI_AGENT_REPLACE_ME (id=jiangsu-yangzhou-2025-2026-midterm-Q19-img1)
\end{tikzpicture}
% IMAGE_TODO_END id=jiangsu-yangzhou-2025-2026-midterm-Q19-img
1
\end{center}


\begin{center}
% IMAGE_TODO_START id=jiangsu-yangzhou-2025-2026-midterm-Q19-img2 path=/Users/muryor/code/mynote/word\\_to\\_tex/output/figures/jiangsu-yangzhou-2025-2026-midterm/media/image10.png width=60% inline=false question_index=19 sub_index=1
% CONTEXT_AFTER: 由锐角三角函数的定义得$${\tan B} = {CD} =
\begin{tikzpicture}[scale=1.05,>=Stealth,line cap=round,line join=round]
  % TODO: AI_AGENT_REPLACE_ME (id=jiangsu-yangzhou-2025-2026-midterm-Q19-img2)
\end{tikzpicture}
% IMAGE_TODO_END id=jiangsu-yangzhou-2025-2026-midterm-Q19-img
2
\end{center}

\topics{正弦定理边角互化的应用;三角形面积公式及其应用;余弦定理解三角形;基本不等式求和的最小值}
\difficulty{0.4}
\answer{(1)\(\frac{\sqrt{3} - 1}{2}\)
(2)\(2\sqrt{3}\)
(3)\(\frac{36\sqrt{3} - 27}{208}\)}
\explain{(1)因为\(\sinB = \sqrt{3}\sinC\),
所以由正弦定理可得\(b = \sqrt{3}c\),\par
因为\(a^{2} - c^{2} = \frac{2S}{\sinA} = \frac{bc\sin A}{\sin A} = bc\),
所以\(a^{2} = \left( \sqrt{3} + 1 \right)c^{2}\),\par
所以由余弦定理得\(\cos A = \frac{b^{2} + c^{2} - a^{2}}{2bc} = \frac{3c^{2} + c^{2} - \left( \sqrt{3} + 1 \right)c^{2}}{2\sqrt{3}c^{2}} = \frac{\sqrt{3} - 1}{2}\);\par
(2)由三角形面积公式得\(S = \frac{1}{2}bc\sin A = \frac{3}{2}\),
且\(b = \sqrt{3}c\),
解得\(c^{2}\sin A = \sqrt{3}\),\par
由余弦定理得\(a^{2} = b^{2} + c^{2} - 2bc\cos A = 4c^{2} - 2\sqrt{3}c^{2}\cos A = \left( 4 - 2\sqrt{3}\cos A \right)c^{2}= \left( 4 - 2\sqrt{3}\cos A \right)\frac{\sqrt{3}}{\sin A} = 2\sqrt{3} \cdot \frac{2 - \sqrt{3}\cos A}{\sin A}\),\par
令\(\frac{2 - \sqrt{3}\cos A}{\sin A} = \lambda\),
则\(\lambda\sin A = 2 - \sqrt{3}\cos A\),
\(a^{2} = 2\sqrt{3}\lambda\),\par
即\(\sqrt{\lambda^{2} + 3}\sin(A + \varphi) = 2\),
其中\(\tan\varphi = \frac{\sqrt{3}}{\lambda}\),\par
可得\(\sqrt{\lambda^{2} + 3} \geq 2\),
解得\(\lambda \geq 1\),\par
则\(a^{2} \geq 2\sqrt{3}\),
当且仅当\(\lambda = 1\),即\(A = \frac{\pi}{6}\)时,
等号成立,\par
所以\(a^{2}\)的最小值为\(2\sqrt{3}\).\par
(3)由题意得\(c = 1\),而\(b = \sqrt{3}c\),
可得\(b = \sqrt{3}\),\par
如图,作出符合题意的图形,\par
设\(PQ = QR = RP = m\),
则\(\bigtriangleup PQR\)是等边三角形,
得到\(\angle QPR = \frac{\pi}{3}\),\par
因为\(A = \frac{\pi}{3}\),
所以\(\frac{\pi}{3} + \angle ARP + \angle APR = \pi\),
而\(\frac{\pi}{3} + \angle APR + \angle BPQ = \pi\),\par
故\(\angle ARP = \angle BPQ\),
设\(\angle ARP = \angle BPQ = \theta\),\par
在\(\bigtriangleup BPQ\)中,
由正弦定理得\(\frac{m}{\sin B} = \frac{BP}{\sin\angle BQP} = \frac{BP}{\sin(B + \theta)}\),\par
解得\(BP = \frac{m\sin(B + \theta)}{\sin B} = \frac{m(\sin B\cos\theta + \cos B\sin\theta)}{\sin B} = m(\cos\theta + \frac{\sin\theta}{\tan B})\),\par
在\(\bigtriangleup APR\)中,
由正弦定理得\(\frac{m}{\sin\frac{\pi}{3}} = \frac{AP}{\sin\theta}\),
解得\(AP = \frac{2\sqrt{3}}{3}m\sin\theta\),\par
如图,过\(C\)作\(CD\bot AB\),
可得\(AD = \frac{\sqrt{3}}{2}\),\(CD = \frac{3}{2}\),
\(BD = 1 - \frac{\sqrt{3}}{2}\),\par
由锐角三角函数的定义得\(\frac{1}{\tan B} = \frac{BD}{CD} = \frac{2 - \sqrt{3}}{3}\),
则\(BP = m(\cos\theta + \frac{2 - \sqrt{3}}{3}\sin\theta)\),\par
因为\(PA + PB = 1\),
所以\(\frac{2\sqrt{3}}{3}m\sin\theta + m(\cos\theta + \frac{2 - \sqrt{3}}{3}\sin\theta) = 1\),\par
化简得\(m(\cos\theta + \frac{2 + \sqrt{3}}{3}\sin\theta) = 1\),
则\(m = \frac{1}{\cos\theta + \frac{2 + \sqrt{3}}{3}\sin\theta}\),\par
由辅助角公式可得\(m = \frac{1}{\sqrt{{(\frac{2 + \sqrt{3}}{3})}^{2} + 1}\sin(\theta + \varphi)}\),
\(\tan\varphi = 3(2 - \sqrt{3})\),\par
由正弦函数有界性可得\(\frac{1}{\sqrt{{(\frac{2 + \sqrt{3}}{3})}^{2} + 1}\sin(\theta + \varphi)} \geq \frac{1}{\sqrt{{(\frac{2 + \sqrt{3}}{3})}^{2} + 1}}\),\par
即\(m \geq \frac{1}{\sqrt{{(\frac{2 + \sqrt{3}}{3})}^{2} + 1}}\),
故\(m^{2} \geq \frac{1}{{(\frac{2 + \sqrt{3}}{3})}^{2} + 1} = \frac{9(4 - \sqrt{3})}{52}\),\par
由三角形面积公式得\(S_{\bigtriangleup PQR} = \frac{1}{2} \times \frac{\sqrt{3}}{2} \times m^{2} = \frac{\sqrt{3}}{4}m^{2} \geq \frac{\sqrt{3}}{4} \times \frac{9(4 - \sqrt{3})}{52} = \frac{36\sqrt{3} - 27}{208}\).}
\end{question}
