\examxtitle{测试试卷 - jiangsu-huaian-2025-2026-mock1}

\section{单选题}

\begin{question}
已知集合\(A = \left\{ - 3, - 1,0,2,3 \right\},B = \{ x||x| \leq 2\}\),
则\(A \cap B =\)(    )
\begin{choices}
  \item \(\left\{ 0,2 \right\}\)
  \item \(\left\{ - 1,0,2 \right\}\)
  \item \(\left\{ - 1,2 \right\}\)
  \item \(\left\{ - 1,0 \right\}\)
\end{choices}
\topics{交集的概念及运算}
\difficulty{0.85}
\answer{B}
\explain{集合\(B = \left\{ x||x| \leq 2 \right\} = \left\{ x| - 2 \leq x \leq 2 \right\}\),\par
所以\(A \cap B = \left\{ - 1,0,2 \right\}\)}
\end{question}

\begin{question}
已知复数\(z = \frac{2 + \text{i}}{1 - \text{i}}\),则\(|z| =\)(    )
\begin{choices}
  \item \(\frac{\sqrt{6}}{2}\)
  \item \(\frac{3}{2}\)
  \item \(2\)
  \item \(\frac{\sqrt{10}}{2}\)
\end{choices}
\topics{求复数的模;复数的除法运算}
\difficulty{0.65}
\answer{D}
\explain{\(z = \frac{2 + \text{i}}{1 - \text{i}} = \frac{\left( 2 + \text{i} \right)\left( 1 + \text{i} \right)}{\left( 1 - \text{i} \right)\left( 1 + \text{i} \right)} = \frac{1 + 3\text{i}}{1 + 1} = \frac{1}{2} + \frac{3}{2}\text{i}\),
所以\(|z| = \left| \frac{1}{2} + \frac{3}{2}\text{i} \right| = \sqrt{\left( \frac{1}{2} \right)^{2} + \left( \frac{3}{2} \right)^{2}} = \sqrt{\frac{10}{4}} = \frac{\sqrt{10}}{2}\)}
\end{question}

\begin{question}
已知\(k\)为实数,
\(\overrightarrow{a} = (k,2),\overrightarrow{b} = (2,k - 3)\),
则"\(k = 4\)"是"向量\(\overrightarrow{a},\overrightarrow{b}\)共线"的(    )
\begin{choices}
  \item 充分不必要条件
  \item 充要条件
  \item 必要不充分条件
  \item 既不充分也不必要条件
\end{choices}
\topics{判断命题的充分不必要条件;由向量共线(平行)求参数}
\difficulty{0.85}
\answer{A}
\explain{若\(k = 4\),
则\(\overset{arrow}{a} = (4,2),\overset{arrow}{b} = (2,1)\),
\(\overset{arrow}{a} = 2\overset{arrow}{b}\),
即向量\(\overset{arrow}{a},\overset{arrow}{b}\)共线,\par
所以"\(k = 4\)"是"向量\(\overset{arrow}{a},\overset{arrow}{b}\)共线"的充分条件;\par
若"\(\overset{arrow}{a},\overset{arrow}{b}\)共线",
则\(k(k - 3) = 4\),
解得\(k = 4\)或\(k = - 1\),\par
所以"\(k = 4\)"不是"向量\(\overset{arrow}{a},\overset{arrow}{b}\)共线"的必要条件.\par
所以"\(k = 4\)"是"向量\(\overset{arrow}{a},\overset{arrow}{b}\)共线"的充分不必要条件}
\end{question}

\begin{question}
已知\(\left\{ a_{n} \right\}\)为等差数列,
\(\left\{ a_{n} \right\}\)前\(n\)项和为\(S_{n}\),
\(a_{1} + a_{4} + a_{7} = 1\),
\(S_{5} = 5\),则\(a_{6} =\)(    )
\begin{choices}
  \item \(- 1\)
  \item \(- \frac{1}{3}\)
  \item \(\frac{1}{2}\)
  \item \(3\)
\end{choices}
\topics{等差数列通项公式的基本量计算;利用等差数列的性质计算;等差数列前n项和的基本量计算}
\difficulty{0.85}
\answer{A}
\explain{因为\(S_{5} = 5\),\par
所以\(\frac{5(a_{1} + a_{5})}{2} = 5a_{3} = 5\),\par
所以\(a_{3} = 1\);\par
又因为\(a_{1} + a_{4} + a_{7} = 1\),\par
即\(3a_{4} = 1\),
解得\(a_{4} = \frac{1}{3}\),\par
所以等差数列的公差\(d = a_{4} - a_{3} = - \frac{2}{3}\),\par
所以\(a_{6} = a_{3} + 3d = 1 - 2 = - 1\).\par
故选;A.}
\end{question}

\begin{question}
函数\(f(x) = A\sin(\omega x + \varphi)\left( A > 0,\omega > 0,\varphi \in \left\lbrack 0,2\pi \right) \right)\)的图像如图所示,
则\(f(x) =\)(    )
\begin{choices}
  \item \(3\sin\left( \frac{1}{4}x + \frac{\pi}{4} \right)\)
  \item \(3\sin\left( \frac{1}{4}x - \frac{\pi}{4} \right)\)
  \item \(3\sin\left( \frac{\pi}{4}x + \frac{\pi}{4} \right)\)
  \item \(3\sin\left( \frac{\pi}{4}x - \frac{\pi}{4} \right)\)
\end{choices}

\begin{center}
% IMAGE_TODO_START id=jiangsu-huaian-2025-2026-mock1-Q5-img1 path=/Users/muryor/code/mynote/word\\_to\\_tex/output/figures/jiangsu-huaian-2025-2026-mock1/media/image2.png width=60% inline=false question_index=5 sub_index=1
% CONTEXT_BEFORE: xt{π} ) )$$的图像如图所示,则$$f(x) =$$( )
\begin{tikzpicture}[scale=1.05,>=Stealth,line cap=round,line join=round]
  % TODO: AI_AGENT_REPLACE_ME (id=jiangsu-huaian-2025-2026-mock1-Q5-img1)
\end{tikzpicture}
% IMAGE_TODO_END id=jiangsu-huaian-2025-2026-mock1-Q5-img
1
\end{center}

\topics{由图象确定正(余)弦型函数解析式}
\difficulty{0.85}
\answer{C}
\explain{观察图象可得函数\(f(x)\)的最大值为\(3\),最小值为\(- 3\),
又\(A > 0\),\par
所以\(A = 3\),\par
又\(\because\) \(3 - ( - 1) = \frac{T}{2}\),
\(\therefore\) \(T = 8\),
\(\therefore\) \(\omega = \frac{2\pi}{T} = \frac{\pi}{4}\),\par
因为\(x = \frac{3 + ( - 1)}{2} = 1\)时函数\(f(x)\)取最大值\(3\),\par
所以\(\frac{\pi}{4} \times 1 + \varphi = 2k\pi + \frac{\pi}{2}\),
\(k \in \mathbb{Z}\),
又\(\varphi \in \lbrack 0,2\pi)\),\par
\(\therefore\) \(\varphi = \frac{\pi}{4}\),\par
\(\therefore\) \(f(x) = 3\sin\left( \frac{\pi}{4}x + \frac{\pi}{4} \right)\)}
\end{question}

\begin{question}
已知函数\(f(x)\)定义域为\(R\),
且满足:\(f(x) = f(2 - x)\),
\(\forall x_{1},x_{2} \in \lbrack 1, + \infty)\),
\(x_{1} \neq x_{2}\),
\(\frac{f\left( x_{1} \right) - f\left( x_{2} \right)}{x_{1} - x_{2}} < 0\),
若\(f(2) < f\left( \lgx \right)\),
则实数\(x\)的取值范围为(    )
\begin{choices}
  \item \(1 < x < 100\)
  \item \(\frac{1}{10} < x < 100\)
  \item \(10 < x < 100\)
  \item \(1 < x < 10\)
\end{choices}
\topics{函数对称性的应用;根据函数的单调性解不等式}
\difficulty{0.65}
\answer{A}
\explain{因为\(f(x) = f(2 - x)\),
所以\(f(x)\)关于\(x = 1\)对称,\par
又\(\forall x_{1},x_{2} \in \lbrack 1, + \infty),x_{1} \neq x_{2},\frac{f\left( x_{1} \right) - f\left( x_{2} \right)}{x_{1} - x_{2}} < 0\),
所以\(f(x)\)在\(\lbrack 1, + \infty)\)单调递减,\par
所以\(f(x)\)在\(( - \infty,1\rbrack\)单调递增.\par
又\(f(2) = f(0)\),所以\(0 < \lgx < 2\),
所以\(1 < x < 100\),\par
所以实数\(x\)的取值范围为\(1 < x < 100\)}
\end{question}

\begin{question}
在无穷数列\(\left\{ a_{n} \right\}\)中,
若\(a_{n} \in N^{\text{*}}\),
且\(a_{n + 1} = \left\{ \begin{array}{r}
\frac{a_{n}}{2},当a_{n}为偶数 \\
3a_{n} + 1,当a_{n}为奇数
\end{array}(n = 1,2,3,\cdots) \right.\),记\(\left\{ a_{n} \right\}\)的前\(n\)项和为\(S_{n}\).若\(S_{3} = 29\),则\(a_{1} =\)(    )
\begin{choices}
  \item 16
  \item 10
  \item 6
  \item 5
\end{choices}
\topics{根据数列递推公式写出数列的项}
\difficulty{0.65}
\answer{D}
\explain{当\(a_{1}\)为偶数,\(a_{2}\)为偶数时,
\(S_{3} = a_{1} + \frac{a_{1}}{2} + \frac{a_{1}}{4} = 29\),
无整数解;\par
当\(a_{1}\)为偶数,\(a_{2}\)为奇数时,
\(S_{3} = a_{1} + \frac{a_{1}}{2} + \frac{3a_{1}}{2} + 1 = 29\),
无整数解;\par
当\(a_{1}\)为奇数,\(a_{2}\)为偶数时,
\(S_{3} = a_{1} + 3a_{1} + 1 + \frac{3a_{1} + 1}{2} = 29\),
解得\(a_{1} = 5\),验证成立;\par
当\(a_{1}\)为奇数,\(a_{2}\)为奇数时,
\(S_{3} = a_{1} + 3a_{1} + 1 + 9a_{1} + 4 = 29\),
无整数解;\par
综上所述:\(a_{1} = 5\)}
\end{question}

\begin{question}
在\(\bigtriangleup ABC\)中,
向量\(\overrightarrow{AC} - \overrightarrow{BC}\)与向量\(\overrightarrow{AC} + 3\overrightarrow{BC}\)垂直,
则\(\cosC + 4\sinA\cosB\)的最大值为(    )
\begin{choices}
  \item \(\sqrt{2}\)
  \item \(\sqrt{3}\)
  \item \(2\sqrt{3}\)
  \item \(2\sqrt{7}\)
\end{choices}
\topics{求含sinx(型)函数的值域和最值;正弦定理边角互化的应用;余弦定理解三角形;垂直关系的向量表示}
\difficulty{0.65}
\answer{A}
\explain{设\(\bigtriangleup ABC\)的三边\(a,b,c\)对的三角分别为\(A,B,C\),\par
因为向量\(\overrightarrow{AC} - \overrightarrow{BC}\)与向量\(\overrightarrow{AC} + 3\overrightarrow{BC}\)垂直,
可得\(\left( \overrightarrow{AC} - \overrightarrow{BC} \right) \cdot \left( \overrightarrow{AC} + 3\overrightarrow{BC} \right) = 0\),\par
即\(\overrightarrow{AB} \cdot \left( \overrightarrow{AC} + 3\overrightarrow{BC} \right) = 0\),可得\(\overrightarrow{AB} \cdot \overrightarrow{AC} + 3\overrightarrow{AB} \cdot \overrightarrow{BC} = 0\),
所以\(bc\cos A - 3ac\cos B = 0\),\par
又因为\(c \neq 0\),
可得\(b\cos A - 3a\cos B = 0\),
即\(\sin B\cos A = 3\sin A\cos B\),\par
所以\(\sin B\cos A + \sin A\cos B = 4\sin A\cos B\),
可得\(\sin(A + B) = 4\sin A\cos B\),\par
因为\(A + B + C = \pi\),
所以\(\sin C = 4\sin A\cos B\),\par
所以\(\cos C + 4\sin A\cos B = \cos C + \sin C = \sqrt{2}\sin(C + \frac{\pi}{4})\),\par
又因为\(C \in (0,\pi)\),
所以\(C + \frac{\pi}{4} \in (\frac{\pi}{4},\frac{5\pi}{4})\),\par
所以当\(C + \frac{\pi}{4} = \frac{\pi}{2}\),
即\(C = \frac{\pi}{4}\)时,
\(\cos C + 4\sin A\cos B\)取得最大值\(\sqrt{2}\)}
\end{question}

\section{多选题}

\begin{question}
下列选项正确的有(    )
\begin{choices}
  \item 若\(a > b\),且\(\frac{1}{a} > \frac{1}{b}\),则\(ab < 0\)
  \item 若\(ab = 1\),则\(a + b \geq 2\)
  \item 若\(a + b > 0\),则\(a^{2} + b^{2} > 0\)
  \item \(13\left( a^{2} + b^{2} \right) \geq {(2a + 3b)}^{2}\)
\end{choices}
\topics{由已知条件判断所给不等式是否正确;作差法比较代数式的大小}
\difficulty{0.85}
\answer{ACD}
\explain{对于A,因\(a > b\),
由\(\frac{1}{a} - \frac{1}{b} = \frac{b - a}{ab} > 0\),
可得\(ab < 0\),故A正确;\par
对于B,若取\(a = - 2,b = - \frac{1}{2}\),
满足\(ab = 1\),
但\(a + b = - \frac{5}{2}\)显然不满足\(a + b \geq 2\),
故B错误;\par
对于C,由\(a + b > 0\)可知\(a,b\)可以同为正数,一正一负,
或者一个为正数一个为0,
易得以上情况都能使\(a^{2} + b^{2} > 0\)成立,故C正确;\par
对于D,
因\(13\left( a^{2} + b^{2} \right) - {(2a + 3b)}^{2} = 9a^{2} - 12ab + 4b^{2} = {(3a - 2b)}^{2} \geq 0\),
故\(13\left( a^{2} + b^{2} \right) \geq {(2a + 3b)}^{2}\),
即D正确}
\end{question}

\begin{question}
已知函数\(f(x) = x + \log_{2}\frac{1}{2^{x} + 1}\),下列选项正确的有(    )
\begin{choices}
  \item \(f(x) = f( - x) + x\)
  \item \(f(2025) > f(2026)\)
  \item 函数\(f(x)\)有唯一零点
  \item 不等式\(f(x) < - 3\)的解集为\(\left( - \infty, - \log_{2}7 \right)\)
\end{choices}
\topics{对数的运算;对数的运算性质的应用;求函数零点或方程根的个数}
\difficulty{0.4}
\answer{AD}
\explain{对于A,
因为\(f(x) = x + \log_{2}\frac{1}{2^{x} + 1}\),\par
所以\(f( - x) = - x + \log_{2}\frac{1}{2^{- x} + 1} = - x + \log_{2}\frac{2^{x}}{2^{x} + 1} = - x + \log_{2}2^{x} + \log_{2}\frac{1}{2^{x} + 1} = \log_{2}\frac{1}{2^{x} + 1}\),\par
所以\(f( - x) + x = \log_{2}\frac{1}{2^{x} + 1} + x = f(x)\),
故A选项正确;\par
对于B,根据解析式,
得\(f(2025) - f(2026) = 2025 + \log_{2}\frac{1}{2^{2025} + 1} - 2026 - \log_{2}\frac{1}{2^{2026} + 1}= - 1 + \log_{2}\frac{2^{2026} + 1}{2^{2025} + 1} = - 1 + \log_{2}\frac{2^{2026} + 2 - 1}{2^{2025} + 1} = - 1 + \log_{2}\left( 2 - \frac{1}{2^{2025} + 1} \right) < 0\),\par
所以\(f(2025) - f(2026) < 0\),
所以\(f(2025) < f(2026)\),故B选项错误;\par
对于C,
令\(f(x) = x + \log_{2}\frac{1}{2^{x} + 1} = 0\),
即\(x + \log_{2}1 - \log_{2}\left( 2^{x} + 1 \right) = 0\),\par
则\(x = \log_{2}\left( 2^{x} + 1 \right)\),
即\(\log_{2}2^{x} = \log_{2}\left( 2^{x} + 1 \right)\),
所以\(2^{x} = 2^{x} + 1\),方程无解,故C选项错误;\par
对于D,不等式\(f(x) < - 3\),
即\(x + \log_{2}\frac{1}{2^{x} + 1} < - 3\),
化简得\(x - \log_{2}\left( 2^{x} + 1 \right) < - 3\),\par
即\(x + 3 < \log_{2}\left( 2^{x} + 1 \right)\),
所以\(\log_{2}2^{x + 3} < \log_{2}\left( 2^{x} + 1 \right)\),
所以\(2^{x + 3} < 2^{x} + 1\),\par
所以\(8 \cdot 2^{x} < 2^{x} + 1\),
即\(7 \cdot 2^{x} < 1\),
所以\(2^{x} < \frac{1}{7}\),
解不等式得\(x < \log_{2}\frac{1}{7}\),
即\(x < - \log_{2}7\),故D选项正确}
\end{question}

\begin{question}
如图,已知正方形\(ABCD\)和正方形\(BDEF\)所在的平面相互垂直,\(AC \cap BD = O\),\(AB = 2\).(    )
\begin{choices}
  \item \(BD\) \(\parallel\)平面\(CEF\)
  \item 二面角\(A - EF - C\)的正切值为\(\frac{4}{3}\)
  \item 三棱锥\(F - BCE\)外接球体积为\(16\pi\)
  \item 侧面\(AEF\)内的动点\(P\)满足\(OP\) \(\parallel\)平面\(CEF\),则\(P\)点轨迹长度为\(\sqrt{2}\)
\end{choices}

\begin{center}
% IMAGE_TODO_START id=jiangsu-huaian-2025-2026-mock1-Q11-img1 path=/Users/muryor/code/mynote/word\\_to\\_tex/output/figures/jiangsu-huaian-2025-2026-mock1/media/image3.png width=60% inline=false question_index=11 sub_index=1
% CONTEXT_BEFORE: $$所在的平面相互垂直,$$AC \cap BD = O$$,$$AB = 2$$.( )
% CONTEXT_AFTER: > A.$$BD$$ $$\parallel$$平面$$CEF$$ > > B.二面角$$A -
\begin{tikzpicture}[scale=1.05,>=Stealth,line cap=round,line join=round]
  % TODO: AI_AGENT_REPLACE_ME (id=jiangsu-huaian-2025-2026-mock1-Q11-img1)
\end{tikzpicture}
% IMAGE_TODO_END id=jiangsu-huaian-2025-2026-mock1-Q11-img
1
\end{center}


\begin{center}
% IMAGE_TODO_START id=jiangsu-huaian-2025-2026-mock1-Q11-img2 path=/Users/muryor/code/mynote/word\\_to\\_tex/output/figures/jiangsu-huaian-2025-2026-mock1/media/image4.png width=60% inline=false question_index=11 sub_index=1
% CONTEXT_BEFORE: $D$$为原点,$$DA,DC,DE$$分别为$$x,y,z$$轴建立空间直角坐标系,如图所示:
% CONTEXT_AFTER: 因为$$AB = 2$$,所以$$BD = 2$$, 则$$A(2,0,0)$
\begin{tikzpicture}[scale=1.05,>=Stealth,line cap=round,line join=round]
  % TODO: AI_AGENT_REPLACE_ME (id=jiangsu-huaian-2025-2026-mock1-Q11-img2)
\end{tikzpicture}
% IMAGE_TODO_END id=jiangsu-huaian-2025-2026-mock1-Q11-img
2
\end{center}


\begin{center}
% IMAGE_TODO_START id=jiangsu-huaian-2025-2026-mock1-Q11-img3 path=/Users/muryor/code/mynote/word\\_to\\_tex/output/figures/jiangsu-huaian-2025-2026-mock1/media/image5.png width=60% inline=false question_index=11 sub_index=1
% CONTEXT_BEFORE: $$P_{2}$$,连接$$P_{1}P_{2},OP_{1},OP_{2}$$,$$AC$$,
% CONTEXT_AFTER: 所以$$P_{1}P_{2}//EF$$,$$EF \subset$$平面$$EFC$$,$$P
\begin{tikzpicture}[scale=1.05,>=Stealth,line cap=round,line join=round]
  % TODO: AI_AGENT_REPLACE_ME (id=jiangsu-huaian-2025-2026-mock1-Q11-img3)
\end{tikzpicture}
% IMAGE_TODO_END id=jiangsu-huaian-2025-2026-mock1-Q11-img
3
\end{center}

\topics{球的体积的有关计算;求二面角;面面垂直证线面垂直;面面角的向量求法}
\difficulty{0.4}
\answer{ABD}
\explain{对于A:正方形\(BDEF\)中,\(BD//EF\),\par
又\(BD ⊄\)平面\(CEF\),
\(EF \subset\)平面\(CEF\),\par
所以\(BD\) \(//\)平面\(CEF\),故A正确;\par
对于B:以\(D\)为原点,
\(DA,DC,DE\)分别为\(x,y,z\)轴建立空间直角坐标系,如图所示:\par
因为\(AB = 2\),所以\(BD = 2\sqrt{2}\),\par
则\(A(2,0,0)\),\(E(0,0,2\sqrt{2})\),
\(F(2,2,2\sqrt{2})\),\(C(0,2,0)\),
\(B(2,2,0)\),\par
则\(\overrightarrow{AE} = ( - 2,0,2\sqrt{2})\),
\(\overrightarrow{EF} = (2,2,0)\),
\(\overrightarrow{EC} = (0,2, - 2\sqrt{2})\),\par
设平面\(AEF\)的法向量为\(\overrightarrow{m} = (x_{1},y_{1},z_{1})\),
平面\(EFC\)的法向量为\(\overrightarrow{n} = (x_{2},y_{2},z_{2})\),\par
则\(\left\{ \begin{array}{r}
 - 2x_{1}^{} + 2\sqrt{2}z_{1} = 0 \\
2x_{1} + 2y_{1} = 0
\end{array} \right.\),\(\left\{ \begin{array}{r}
2y_{2}^{} - 2\sqrt{2}z_{2} = 0 \\
2x_{2} + 2y_{2} = 0
\end{array} \right.\),\par
令\(x_{1} = \sqrt{2},x_{2} = \sqrt{2}\),则\(y_{1} = - \sqrt{2},z_{1} = 1,y_{2} = - \sqrt{2},z_{2} = - 1\),\par
所以\(\overrightarrow{m} = (\sqrt{2}, - \sqrt{2},1)\),\(\overrightarrow{n} = (\sqrt{2}, - \sqrt{2}, - 1)\),\par
所以\(\cos\left\langle \overrightarrow{m},\overrightarrow{n} \right\rangle = \frac{\overrightarrow{m} \cdot \overrightarrow{n}}{\left| \overrightarrow{m} \right|\left| \overrightarrow{n} \right|} = \frac{2 + 2 - 1}{\sqrt{5} \cdot \sqrt{5}} = \frac{3}{5}\),\par
设二面角\(A - EF - C\)的平面角为\(\theta\),\(\theta \in \lbrack 0,\pi\rbrack\),\par
所以\(\sin\theta = \sqrt{1 - \left( \frac{3}{5} \right)^{2}} = \frac{4}{5}\),\par
则\(\tan\theta = \frac{\frac{4}{5}}{\frac{3}{5}} = \frac{4}{3}\),故B正确;\par
对于C:设球心为\(G(x,y,z)\),\par
则\(\left\{ \begin{array}{r}
{(x - 2)}^{2} + {(y - 2)}^{2} + {(z - 2\sqrt{2})}^{2} = {(x - 2)}^{2} + {(y - 2)}^{2} + z^{2} \\
x^{2} + {(y - 2)}^{2} + z^{2} = {(x - 2)}^{2} + {(y - 2)}^{2} + z^{2} \\
x^{2} + y^{2} + {(z - 2\sqrt{2})}^{2} = x^{2} + {(y - 2)}^{2} + z^{2}
\end{array} \right.\),\par
解得\(\left\{ \begin{array}{r}
x = 1 \\
y = 1 \\
z = \sqrt{2}
\end{array} \right.\),\par
所以球的半径\(R = \sqrt{1^{2} + ( - 1)^{2} + \left( \sqrt{2} \right)^{2}} = 2\),\par
所以球的体积为\(V = \frac{4}{3}\pi R^{3} = \frac{32}{3}\pi\),故C错误;\par
对于D:取\(AE\)的中点\(P_{1}\),\(AF\)的中点\(P_{2}\),连接\(P_{1}P_{2},OP_{1},OP_{2}\),\(AC\),\par
所以\(P_{1}P_{2}//EF\),\(EF \subset\)平面\(EFC\),\(P_{1}P_{2} ⊄\)平面\(EFC\),\(P_{1}P_{2}//\)平面\(EFC\),\par
又因为\(O\)为\(AC\)中点,\par
所以\(OP_{2}//CF\),\(CF \subset\)平面\(EFC\),\(OP_{2} ⊄\)平面\(EFC\),\(OP_{2}//\)平面\(EFCOP_{2} \cap P_{1}P_{2} = P_{2},OP_{2},P_{1}P_{2} \subset\)平面\(OP_{1}P_{2}\),\par
所以平面\(EFC//\)平面\(OP_{1}P_{2}\),\par
所以当\(P\)在\(P_{1}P_{2}\)上运动时,\(OP//\)平面\(EFC\),\par
所以\(P\)点轨迹为\(P_{1}P_{2}\),长度为\(\frac{1}{2}EF = \sqrt{2}\),故D正确}
\end{question}

\section{填空题}

\begin{question}
已知向量\(\overrightarrow{a},\overrightarrow{b}\)满足\(\left| \overrightarrow{a} + 2\overrightarrow{b} \right| = 4,\overrightarrow{a} \cdot \overrightarrow{b} = - 1\),
则\(\left| \overrightarrow{a} - 2\overrightarrow{b} \right| =\)
.
\topics{已知数量积求模}
\difficulty{0.65}
\answer{\(2\sqrt{6}\)}
\explain{根据向量模的平方可得:\(|\overset{arrow}{a} + 2\overset{arrow}{b}|^{2} = (\overset{arrow}{a} + 2\overset{arrow}{b}) \cdot (\overset{arrow}{a} + 2\overset{arrow}{b}) = |\overset{arrow}{a}|^{2} + 4\overset{arrow}{a} \cdot \overset{arrow}{b} + 4|\overset{arrow}{b}|^{2} = 16\),\par
代入\(\overrightarrow{a} \cdot \overrightarrow{b} = - 1\)得:
\(|\overrightarrow{a}|^{2} + 4|\overrightarrow{b}|^{2} = 20\);\par
同理,
\(|\overrightarrow{a} - 2\overrightarrow{b}|^{2} = (\overrightarrow{a} - 2\overrightarrow{b}) \cdot (\overrightarrow{a} - 2\overrightarrow{b}) = |\overrightarrow{a}|^{2} - 4\overrightarrow{a} \cdot \overrightarrow{b} + 4|\overrightarrow{b}|^{2}\)\par
将\(|\overrightarrow{a}|^{2} + 4|\overrightarrow{b}|^{2} = 20\)和\(\overrightarrow{a} \cdot \overrightarrow{b} = - 1\)代入:\(|\overrightarrow{a}|^{2} - 4 \times ( - 1) + 4|\overrightarrow{b}|^{2} = (|\overrightarrow{a}|^{2} + 4|\overrightarrow{b}|^{2}) + 4 = 20 + 4 = 24\);\par
因为\(|\overrightarrow{a} - 2\overrightarrow{b}|^{2} = 24\),
所以\(|\overrightarrow{a} - 2\overrightarrow{b}| = \sqrt{24} = 2\sqrt{6}\).\(2\sqrt{6}\).}
\end{question}

\begin{question}
已知递增等比数列\(\left\{ a_{n} \right\}\)前\(n\)项和为\(S_{n}\),
且\(a_{2} = 8,S_{3} = 42\),
则数列\(\left\{ \frac{1}{\log_{2}a_{n} \cdot \log_{2}a_{n + 1}} \right\}\)的前\(10\)项和为
.
\topics{等比数列前n项和的基本量计算;裂项相消法求和}
\difficulty{0.65}
\answer{\(\frac{10}{21}\)}
\explain{由于\(a_{2} = 8,S_{3} = 42\),
则\(S_{3} = a_{1} + a_{2} + a_{3} = \frac{8}{q} + 8 + 8q = 42\),\par
解得\(q = 4\)或\(\frac{1}{4}\),
因为等比数列\(\left\{ a_{n} \right\}\)为递增数列,\(a_{2} = 8 > 0\),\par
所以\(q = 4\)\par
所以\(a_{1} = 2\),
故\(a_{n} = 2 \times 4^{n - 1} = 2^{2n - 1}\).\par
因为\(\frac{1}{\log_{2}a_{n} \cdot \log_{2}a_{n + 1}} = \frac{1}{(2n - 1)(2n + 1)} = \frac{1}{2}\left( \frac{1}{2n - 1} - \frac{1}{2n + 1} \right)\),\par
所以\(T_{10} = \frac{1}{2}\left( 1 - \frac{1}{3} + \frac{1}{3} - \frac{1}{5} + \cdots + \frac{1}{19} - \frac{1}{21} \right) = \frac{1}{2}(1 - \frac{1}{21}) = \frac{10}{21}\).\(\frac{10}{21}\)}
\end{question}

\begin{question}
法国著名数学家傅里叶用一个纯粹的数学定理表述了周期性声音的规则特征:任何周期性声音的公式是形如\(A\sinwt\)的简单正弦函数之和.某种叠加音波的函数模型为\(f(t) = \sint + \frac{1}{2}\sin2t + \frac{1}{3}\sin3t(t \geq 0)\),
则函数\(f(t)\)的最大值为
.
\topics{由导数求函数的最值(不含参);三角恒等变换的化简问题}
\difficulty{0.15}
\answer{\(\frac{1}{2} + \frac{2\sqrt{2}}{3}\)}
\explain{\(f\left( t + 2\pi \right) = \sin\left( t + 2\pi \right) + \frac{1}{2}\sin\left( 2t + 4\pi \right) + \frac{1}{3}\sin\left( 3t + 6\pi \right)= \sint + \frac{1}{2}\sin2t + \frac{1}{3}\sin3t = f(t)\),\par
故\(f(t) = \sint + \frac{1}{2}\sin2t + \frac{1}{3}\sin3t(t \geq 0)\)的一个周期为\(2\pi\),\par
考虑\(t \in \left\lbrack 0,2\pi \right\rbrack\),
\(f'(t) = \cos t + \cos 2t + \cos 3t\),\par
其中\(\cos 3t = \cos(2t + t) = \cos 2t\cos t - \sin 2t\sin t = \left( 1 - 2\sin^{2}t \right)\cos t - 2\sin^{2}t\cos t= \left( 1 - 4\sin^{2}t \right)\cos t\),\par
\(f'(t) = \cos t + \cos 2t + \left( 1 - 4\sin^{2}t \right)\cos t = \left( 2 - 4\sin^{2}t \right)\cos t + \cos 2t= 2\cos 2t\cos t + \cos 2t = \left( 2\cos t + 1 \right)\cos 2t = \left( 2\cos t + 1 \right)\left( 2\cos^{2}t - 1 \right)= \left( 2\cos t + 1 \right)\left( \sqrt{2}\cos t + 1 \right)\left( \sqrt{2}\cos t - 1 \right)\),\par
令\(f'(t) = 0\)得\(t = \frac{\pi}{4},\frac{2\pi}{3},\frac{3\pi}{4},\frac{5\pi}{4},\frac{4\pi}{3},\frac{7\pi}{4}\),\par
当\(t \in \left( 0,\frac{\pi}{4} \right) \cup \left( \frac{2\pi}{3},\frac{3\pi}{4} \right) \cup \left( \frac{5\pi}{4},\frac{4\pi}{3} \right) \cup \left( \frac{7\pi}{4},2\pi \right)\)时,
\(f'(t) > 0\),\par
当\(t \in \left( \frac{\pi}{4},\frac{2\pi}{3} \right) \cup \left( \frac{3\pi}{4},\frac{5\pi}{4} \right) \cup \left( \frac{4\pi}{3},\frac{7\pi}{4} \right)\)时,
\(f'(t) < 0\),\par
所以\(f(t)\)在\(\left( 0,\frac{\pi}{4} \right),\left( \frac{2\pi}{3},\frac{3\pi}{4} \right),\left( \frac{5\pi}{4},\frac{4\pi}{3} \right),\left( \frac{7\pi}{4},2\pi \right)\)上单调递增,\par
在\(\left( \frac{\pi}{4},\frac{2\pi}{3} \right),\left( \frac{3\pi}{4},\frac{5\pi}{4} \right),\left( \frac{4\pi}{3},\frac{7\pi}{4} \right)\)上单调递减,\par
故\(f(t)\)在\(t = \frac{\pi}{4},\frac{3\pi}{4},\frac{5\pi}{4}\)处取得极大值,\par
且\(f\left( \frac{\pi}{4} \right) = \sin\frac{\pi}{4} + \frac{1}{2}\sin\frac{\pi}{2} + \frac{1}{3}\sin\frac{3\pi}{4} = \frac{\sqrt{2}}{2} + \frac{1}{2} + \frac{1}{3} \times \frac{\sqrt{2}}{2} = \frac{1}{2} + \frac{2\sqrt{2}}{3}\),\par
\(f\left( \frac{3\pi}{4} \right) = \sin\frac{3\pi}{4} + \frac{1}{2}\sin\frac{3\pi}{2} + \frac{1}{3}\sin\frac{9\pi}{4} = \frac{\sqrt{2}}{2} - \frac{1}{2} + \frac{1}{3} \times \frac{\sqrt{2}}{2} = - \frac{1}{2} + \frac{2\sqrt{2}}{3}\),\par
\(f\left( \frac{5\pi}{4} \right) = \sin\frac{4\pi}{3} + \frac{1}{2}\sin\frac{8\pi}{3} + \frac{1}{3}\text{sin4}\pi = - \frac{\sqrt{3}}{2} + \frac{\sqrt{3}}{2} - \frac{1}{3} \times 0 = 0\),\par
又\(f\left( 2\pi \right) = f(0) = 0\),
\(f\left( \frac{\pi}{4} \right) > f\left( \frac{3\pi}{4} \right) > f(2\pi) > f\left( \frac{4\pi}{3} \right)\),\par
故\(f(t)\)的最大值为\(\frac{1}{2} + \frac{2\sqrt{2}}{3}\).\(\frac{1}{2} + \frac{2\sqrt{2}}{3}\)}
\end{question}

\section{解答题}

\begin{question}
已知\(\overrightarrow{m} = \left( \cosx, - \sqrt{3}\cosx \right),\overrightarrow{n} = \left( \sinx,\cosx \right)\),
\(f(x) = \overrightarrow{m} \cdot \overrightarrow{n}\).
\begin{enumerate}[label=(\arabic*)]
  \item 将函数\(y = f(x)\)的图象向左平移\(\frac{\pi}{3}\)个单位长度,
\item 再将图象上所有点的横坐标伸长到原来的\(2\)倍,纵坐标不变,
\item 得到函数\(y = g(x)\)的图象,求函数\(g(x)\)的解析式;
  \item 若\(f\left( \frac{\alpha}{2} \right) = - \frac{\sqrt{3}}{4},\alpha \in \left( 0,\frac{\pi}{2} \right)\),
\item 求\(\cos\alpha\).
\end{enumerate}
\topics{求图象变化前(后)的解析式;辅助角公式;三角恒等变换的化简问题;数量积的坐标表示}
\difficulty{0.4}
\answer{(1)\(g(x) = \sin\left( x + \frac{\pi}{3} \right) - \frac{\sqrt{3}}{2}\)
(2)\(\frac{\sqrt{13} - 3}{8}\)}
\explain{(1)\(f(x) = \sinx \cdot \cosx - \sqrt{3}\cos^{2}x = \frac{1}{2}\sin2x - \sqrt{3} \cdot \frac{1 + \cos2x}{2} = \sin\left( 2x - \frac{\pi}{3} \right) - \frac{\sqrt{3}}{2}\).\par
将函数\(y = f(x)\)的图象向左平移\(\frac{\pi}{3}\)个单位长度,
则\(y = \sin\left\lbrack 2\left( x + \frac{\pi}{3} \right) - \frac{\pi}{3} \right\rbrack - \frac{\sqrt{3}}{2} = \sin\left( 2x + \frac{\pi}{3} \right) - \frac{\sqrt{3}}{2}\),\par
再将图象上所有点的横坐标伸长到原来的\(2\)倍,纵坐标不变,
则有\(g(x) = \sin\left( x + \frac{\pi}{3} \right) - \frac{\sqrt{3}}{2}\).\par
(2)由题意得\(f\left( \frac{\alpha}{2} \right) = \sin\left( \alpha - \frac{\pi}{3} \right) - \frac{\sqrt{3}}{2} = - \frac{\sqrt{3}}{4}\),
所以\(\sin\left( \alpha - \frac{\pi}{3} \right) = \frac{\sqrt{3}}{4}.\because\alpha \in \left( 0,\frac{\pi}{2} \right),\therefore\alpha - \frac{\pi}{3} \in \left( - \frac{\pi}{3},\frac{\pi}{6} \right)\),\par
\(\therefore\cos\left( \alpha - \frac{\pi}{3} \right) = \sqrt{1 - \sin^{2}\left( \alpha - \frac{\pi}{3} \right)} = \frac{\sqrt{13}}{4}\cos\alpha = \cos\left\lbrack \left( \alpha - \frac{\pi}{3} \right) + \frac{\pi}{3} \right\rbrack = \cos\left( \alpha - \frac{\pi}{3} \right) \cdot \cos\frac{\pi}{3} - \sin\left( \alpha - \frac{\pi}{3} \right) \cdot \sin\frac{\pi}{3}= \frac{\sqrt{13}}{4} \cdot \frac{1}{2} - \frac{\sqrt{3}}{4} \cdot \frac{\sqrt{3}}{2} = \frac{\sqrt{13} - 3}{8}\).}
\end{question}

\begin{question}
如图,四棱锥\(P - ABCD\)的底面\(ABCD\)是菱形,
\(AB = 4\),\(\angle ABC = 60{^\circ}\),
侧棱\(PA\bot\)底面\(ABCD\)且\(PA = 4\),
\(E\)是\(PA\)的中点.
\begin{enumerate}[label=(\arabic*)]
  \item 证明:平面\(PBD\bot\)平面\(PAC\);
  \item 求\(BE\)与平面\(PAC\)所成角的正切值.
\end{enumerate}

\begin{center}
% IMAGE_TODO_START id=jiangsu-huaian-2025-2026-mock1-Q16-img1 path=/Users/muryor/code/mynote/word\\_to\\_tex/output/figures/jiangsu-huaian-2025-2026-mock1/media/image6.png width=60% inline=false question_index=16 sub_index=1
% CONTEXT_BEFORE: $$PA\bot$$底面$$ABCD$$且$$PA = 4$$,$$E$$是$$PA$$的中点.
% CONTEXT_AFTER: (1)证明:平面$$PBD\bot$$平面$$PAC$$; (2)求$$BE$$与平面$$PA
\begin{tikzpicture}[scale=1.05,>=Stealth,line cap=round,line join=round]
  % TODO: AI_AGENT_REPLACE_ME (id=jiangsu-huaian-2025-2026-mock1-Q16-img1)
\end{tikzpicture}
% IMAGE_TODO_END id=jiangsu-huaian-2025-2026-mock1-Q16-img
1
\end{center}


\begin{center}
% IMAGE_TODO_START id=jiangsu-huaian-2025-2026-mock1-Q16-img2 path=/Users/muryor/code/mynote/word\\_to\\_tex/output/figures/jiangsu-huaian-2025-2026-mock1/media/image7.png width=60% inline=false question_index=16 sub_index=1
% CONTEXT_BEFORE: 法一:设$$AC \cap BD = O$$,由(1)知$$BO\bot$$平面$$PAC$$.
% CONTEXT_AFTER: 则$$\angle BEO$$为$$BE$$与平面$$PAC$$所成角. $$\because
\begin{tikzpicture}[scale=1.05,>=Stealth,line cap=round,line join=round]
  % TODO: AI_AGENT_REPLACE_ME (id=jiangsu-huaian-2025-2026-mock1-Q16-img2)
\end{tikzpicture}
% IMAGE_TODO_END id=jiangsu-huaian-2025-2026-mock1-Q16-img
2
\end{center}


\begin{center}
% IMAGE_TODO_START id=jiangsu-huaian-2025-2026-mock1-Q16-img3 path=/Users/muryor/code/mynote/word\\_to\\_tex/output/figures/jiangsu-huaian-2025-2026-mock1/media/image8.png width=60% inline=false question_index=16 sub_index=1
% CONTEXT_AFTER: 为基底建立空间直角坐标系.则有: $$A(0,0,0),B( 2,
\begin{tikzpicture}[scale=1.05,>=Stealth,line cap=round,line join=round]
  % TODO: AI_AGENT_REPLACE_ME (id=jiangsu-huaian-2025-2026-mock1-Q16-img3)
\end{tikzpicture}
% IMAGE_TODO_END id=jiangsu-huaian-2025-2026-mock1-Q16-img
3
\end{center}

\topics{证明面面垂直;线面角的向量求法}
\difficulty{0.4}
\answer{(1)证明见解析;
(2)\(\frac{\sqrt{6}}{2}\).}
\explain{(1)连接\(AC,BD,\because\)底面\(ABCD\)是菱形,
\(\therefore BD\bot AC\).\par
\(\because PA\bot\)平面\(ABCD,BD \subset\)平面\(ABCD,\therefore BD\bot PA\).\par
又\(\because AC \cap PA = A,PA \subset\)平面\(PAC,AC \subset\)平面\(PAC\),\par
\(\therefore BD\bot\)平面\(PAC\).\par
\(\because BD \subset\)平面\(PBD,\therefore\)平面\(PBD\bot\)平面\(PAC\).\par
(2)方法一:设\(AC \cap BD = O\),
由(1)知\(BO\bot\)平面\(PAC\).\par
则\(\angle BEO\)为\(BE\)与平面\(PAC\)所成角.\par
\(\because\)底面\(ABCD\)是菱形,
\(\angle ABC = 60^{\circ},\therefore AC = 4\).\par
在直角\(\bigtriangleup PAC\)中,
\(PC = \sqrt{PA^{2} + AC^{2}} = 4\sqrt{2},OE = \frac{1}{2}PC = 2\sqrt{2}\).\par
在\(\bigtriangleup BAD\)中,由余弦定理知:\par
\(BD^{2} = AB^{2} + AD^{2} - 2AB \cdot AD \cdot \cos\angle BAD = 48,BD = 4\sqrt{3}\),\par
\(BO = \frac{1}{2}BD = 2\sqrt{3}\).\par
在\(\bigtriangleup BOE\)中,
\(\tan\angle BEO = \frac{OB}{OE} = \frac{2\sqrt{3}}{2\sqrt{2}} = \frac{\sqrt{6}}{2}\).\par
\(\therefore BE\)与平面\(PAC\)所成角的正切值为\(\frac{\sqrt{6}}{2}\).\par
方法二:取\(BC\)中点\(F\),易知\(AF\bot AD\),
以\(\left\{ \overrightarrow{AF},\overrightarrow{AD},\overrightarrow{AP} \right\}\)\par
为基底建立空间直角坐标系.则有:\par
\(A(0,0,0),B\left( 2\sqrt{3}, - 2,0 \right),C\left( 2\sqrt{3},2,0 \right),E(0,0,2),P(0,0,4)\).\par
\(\overrightarrow{BE} = \left( - 2\sqrt{3},2,2 \right),\overrightarrow{AP} = (0,0,4),\overrightarrow{AC} = \left( 2\sqrt{3},2,0 \right)\).\par
设平面\(PAC\)的一个法向量\(\overrightarrow{n} = (x,y,z)\),\par
\(\left\{ \begin{array}{r}
\overrightarrow{AP} \cdot \overrightarrow{n} = 0 \\
\overrightarrow{AC} \cdot \overrightarrow{n} = 0
\end{array} \Rightarrow \left\{ \begin{array}{r} z = 0 \\
\sqrt{3}x + y = 0
\end{array} \right.\  \right.,\par
令\(x = 1\)有\(\overrightarrow{n} = \left( 1, - \sqrt{3},0 \right)\).\par
设\(BE\)与平面\(PAC\)所成角为\(\alpha\),则有\(\sin\alpha = \frac{\overrightarrow{BE} \cdot \overrightarrow{n}}{\left| \overrightarrow{BE} \cdot \right|\left| \overrightarrow{n} \right|} = \frac{\sqrt{15}}{5}\),\par
\(\cos\alpha = \sqrt{1 - \sin^{2}\alpha} = \frac{\sqrt{10}}{5},\tan\alpha = \frac{\sin\alpha}{\cos\alpha} = \frac{\sqrt{6}}{2}\).}
\end{question}

\begin{question}
已知函数\(f(x) = x - \frac{1}{x} - a\lnx\).
\begin{enumerate}[label=(\arabic*)]
  \item 当\(a = 1\)时,
\item 求函数\(f(x)\)在\(x = 1\)处的切线方程;
  \item 若\(f(x)\)有两个极值点\(x_{1},x_{2}\).

\item ①求\(a\)的取值范围;

\item ②证明:存在\(0 < x_{0} < \frac{2}{a}\),
\item 使得\(f\left( x_{1} \right),f\left( x_{0} \right),f\left( x_{2} \right)\)成等差数列.
\end{enumerate}
\topics{求在曲线上一点处的切线方程(斜率);利用导数研究函数的零点;根据极值点求参数}
\difficulty{0.4}
\answer{(1)\(x - y - 1 = 0\)
(2)①\(a > 2\) ;②证明见解析}
\explain{(1)当\(a = 1\)时,
\(f(x) = x - \frac{1}{x} - \lnx(x > 0),f'(x) = 1 + \frac{1}{x^{2}} - \frac{1}{x}\).\par
\(f(1) = 0,f'(1) = 1\),
\(\therefore\) \(f(x)\)在\(x = 1\)处的切线方程为\(x - y - 1 = 0\).\par
(2)函数\(f(x)\)定义域为\((0, + \infty)\),
\(f'(x) = 1 + \frac{1}{x^{2}} - \frac{a}{x} = \frac{x^{2} - ax + 1}{x^{2}}\).\par
①由题意知\(f'(x) = 0\)在\((0, + \infty)\)内有两不等实根,
则有:\par
\(\frac{a}{2} > 0, \bigtriangleup > 0\),
解得\(a > 2\).\par
当\(a > 2\)时,令\(f'(x) > 0\),
\(x < \frac{a - \sqrt{a^{2} - 4}}{2}\)或\(x > \frac{a + \sqrt{a^{2} - 4}}{2}\).\par
令\(f'(x) < 0\),
\(\frac{a - \sqrt{a^{2} - 4}}{2} < x < \frac{a + \sqrt{a^{2} - 4}}{2}\)\par
\(\therefore\) \(f(x)\)在\(\left( 0,\frac{a - \sqrt{a^{2} - 4}}{2} \right),\left( \frac{a + \sqrt{a^{2} - 4}}{2}, + \infty \right)\)上单调递增,
在\(\left( \frac{a - \sqrt{a^{2} - 4}}{2},\frac{a + \sqrt{a^{2} - 4}}{2} \right)\)上单调递减,
此时\(f(x)\)有两个极值点,\par
\(a > 2\)符合题意.\par
②由①知\(x_{1} + x_{2} = a,x_{1} \cdot x_{2} = 1\).\par
\(f\left( x_{1} \right) + f\left( x_{2} \right) = x_{1} - \frac{1}{x_{1}} - a\lnx_{1} + x_{2} - \frac{1}{x_{2}} - a\lnx_{2} = x_{1} + x_{2} - \frac{x_{1} + x_{2}}{x_{1} \cdot x_{2}} - a\ln\left( x_{1} \cdot x_{2} \right) = 0\).\par
下证存在\(x_{0} \in \left( 0,\frac{2}{a} \right)\),
使\(f\left( x_{0} \right) = 0\).\par
先证\(f(0) < 0\),
取\(f\left( \mathrm{e}^{- a} \right) = \mathrm{e}^{- a} - \mathrm{e}^{a} + a^{2}\),
\(\because\) \(a > 2\),
\(\therefore\) \(f\left( \mathrm{e}^{- a} \right) < a^{2} - \mathrm{e}^{a} + 1\).\par
令\(F(a) = a^{2} - \mathrm{e}^{a} + 1\),
\(F'(a) = 2a - \mathrm{e}^{a}\),
令\(g(x) = F'(x)\),
\(g'(x) = 2 - \mathrm{e}^{a} < 0\),\par
\(\therefore\) \(F'(a)\)在\((2, + \infty)\)上单调递减.\par
\(\therefore\) \(F'(a) < F(2) = 4 - \mathrm{e}^{2} < 0\),
\(\therefore\) \(F(a)\)在\((2, + \infty)\)上单调递减.\par
\(\therefore\) \(F(a) < F(2) = 5 - \mathrm{e}^{2} < 0\),
\(\therefore\) \(f\left( \mathrm{e}^{- a} \right) < 0\).\par
再证\(f\left( \frac{2}{a} \right) > 0\).\(f\left( \frac{2}{a} \right) = \frac{2}{a} - \frac{a}{2} - a\ln\left( \frac{2}{a} \right)\),\par
令\(G(a) = f\left( \frac{2}{a} \right)\),
\(G'(a) = - \frac{2}{a^{2}} + \lna + \frac{1}{2} - \ln2\),
令\(h(x) = G'(a)\),
则\(h'(a) = 4a^{- 3} + \frac{1}{a} > 0\),
\(\therefore\) \(G'(a)\)在\((2, + \infty)\)上单调递增.\par
\(\therefore\) \(G'(a) > G'(2) = 0\),
\(\therefore\) \(G(a)\)在\((2, + \infty)\)上单调递增.\(\therefore\) \(G(a) > G(2) = 0\),
即\(f\left( \frac{2}{a} \right) > 0\).\par
由零点存在定理知存在\(0 < x_{0} < \frac{2}{a}\),
使得\(f\left( x_{1} \right),f\left( x_{0} \right),f\left( x_{2} \right)\)成等差数列.}
\end{question}

\begin{question}
在\(\bigtriangleup ABC\)中,\emph{a},\emph{b},
\emph{c}为角\emph{A},\emph{B},\emph{C}的对边,
\(\frac{\cosC}{c} = \frac{\cosB}{2a - b}\).
\begin{enumerate}[label=(\arabic*)]
  \item 求\(\cosC\);
  \item 已知\(2\overrightarrow{AD} = \overrightarrow{DB}\).

\item ①若\(\angle ADC = 60{^\circ}\),
\item 求\(\tan\angle ACD\);

\item ②求\(\frac{CD}{AB}\)的取值范围.
\end{enumerate}

\begin{center}
% IMAGE_TODO_START id=jiangsu-huaian-2025-2026-mock1-Q18-img1 path=/Users/muryor/code/mynote/word\\_to\\_tex/output/figures/jiangsu-huaian-2025-2026-mock1/media/image9.png width=60% inline=false question_index=18 sub_index=1
% CONTEXT_BEFORE: t{sin}A \neq 0$$,∴$$C = {2}$$.
% CONTEXT_AFTER: (2)①$$\because C \in \left( 0, \right),\
\begin{tikzpicture}[scale=1.05,>=Stealth,line cap=round,line join=round]
  % TODO: AI_AGENT_REPLACE_ME (id=jiangsu-huaian-2025-2026-mock1-Q18-img1)
\end{tikzpicture}
% IMAGE_TODO_END id=jiangsu-huaian-2025-2026-mock1-Q18-img
1
\end{center}

\topics{正弦定理解三角形;余弦定理解三角形;数量积的运算律}
\difficulty{0.4}
\answer{(1)\(\frac{1}{2}\)
(2)①\(\frac{\sqrt{3}}{3}\)
;②\(\left( \frac{1}{3},\frac{\sqrt{3} + 1}{3} \right\rbrack\)}
\explain{(1)由正弦定理知\(\frac{\cosC}{\sinC} = \frac{\cosB}{2\sinA - \sinB}\),\par
\(2\sinA \cdot \cosC - \cosC \cdot \sinB = \sinC \cdot \cosB\).\par
\(\therefore\) \(2\sinA \cdot \cosC = \sinC \cdot \cosB + \cosC \cdot \sinB = \sin(B + C) = \sinA\).\par
\(\because\) \(A \in \left( 0,\pi \right)\),
\(\sinA \neq 0\),
\(\therefore\) \(\cosC = \frac{1}{2}\).\par
(2)①\(\because C \in \left( 0,\pi \right),\therefore C = \frac{\pi}{3}\).设\(\angle ACD = \theta\).\par
在\(\bigtriangleup ACD\)中,
由正弦定理知\(\frac{|AD|}{\sin\theta} = \frac{|CD|}{\sin\left( \frac{2\pi}{3} - \theta \right)}\)①\par
在\(\bigtriangleup BCD\)中,
由正弦定理知\(\frac{|BD|}{\sin\left( \frac{\pi}{3} - \theta \right)} = \frac{|CD|}{\sin\theta}\)②\par
\(\because\) \(2\overrightarrow{AD} = \overrightarrow{DB}\),
\(\therefore\) \(2|AD| = |DB|\),
则有\(\frac{\sin\left( \frac{\pi}{3} - \theta \right)}{2\sin\theta} = \frac{\sin\theta}{\sin\left( \frac{2\pi}{3} - \theta \right)}\).\par
\(\Rightarrow 2\sin^{2}\theta = \sin\left( \frac{\pi}{3} - \theta \right) \cdot \sin\left( \frac{2\pi}{3} - \theta \right)\Rightarrow 2\sin^{2}\theta = \left( \frac{\sqrt{3}}{2}\cos\theta - \frac{1}{2}\sin\theta \right) \cdot \left( \frac{\sqrt{3}}{2}\cos\theta + \frac{1}{2}\sin\theta \right) = \frac{3}{4}\cos^{2}\theta - \frac{1}{4}\sin^{2}\theta\Rightarrow \tan^{2}\theta = \frac{1}{3}\)\par
又\(\because\) \(\theta \in \left( 0,\frac{\pi}{3} \right)\),
\(\therefore\) \(\tan\theta = \frac{\sqrt{3}}{3}\).\par
②\(\overrightarrow{CD} = \frac{1}{3}\overrightarrow{CB} + \frac{2}{3}\overrightarrow{CA}\),
平方得\(9{\overrightarrow{CD}}^{2} = {\overrightarrow{CB}}^{2} + 4{\overrightarrow{CA}}^{2} + 4\overrightarrow{CA} \cdot \overrightarrow{CB}\).\par
即\(9CD^{2} = a^{2} + 4b^{2} + 2ab\).\par
又\(\because AB^{2} = c^{2} = a^{2} + b^{2} - ab\).\par
\(\therefore\) \(\frac{9CD^{2}}{AB^{2}} = \frac{a^{2} + 4b^{2} + 2ab}{a^{2} + b^{2} - ab} = \frac{\left( \frac{a}{b} \right)^{2} + 2\frac{a}{b} + 4}{\left( \frac{a}{b} \right)^{2} - \frac{a}{b} + 1}\)\par
令\(t = \frac{a}{b}\),则\(t > 0\),
\(\therefore\) \(\frac{9CD^{2}}{AB^{2}} = \frac{t^{2} + 2t + 4}{t^{2} - t + 1} = 1 + 3 \cdot \frac{t + 1}{t^{2} - t + 1}\).\par
令\(x = t + 1\),则\(x > 1\),
\(\therefore\) \(\frac{t + 1}{t^{2} - t + 1} = \frac{x}{{(x - 1)}^{2} - (x - 1) + 1} = \frac{x}{x^{2} - 3x + 3} = \frac{1}{x + \frac{3}{x} - 3}\).\par
\(\because\) \(x + \frac{3}{x} - 3 \geq 2\sqrt{3} - 3 > 0\),
\(\therefore\) \(\frac{t + 1}{t^{2} - t + 1} \in \left( 0,\frac{1}{2\sqrt{3} - 3} \right\rbrack\)\par
\(\therefore\) \(\frac{9CD^{2}}{AB^{2}} = 1 + 3 \cdot \frac{t + 1}{t^{2} - t + 1} \in \left( 1,4 + 2\sqrt{3} \right\rbrack\).\par
\(\therefore\frac{CD}{AB}\)的取值范围为\(\left( \frac{1}{3},\frac{\sqrt{3} + 1}{3} \right\rbrack\)}
\end{question}

\begin{question}
已知数列\(\left\{ a_{n} \right\}\)的前\(n\)项和为\(S_{n}\),
且\(a_{n} \in N^{\text{*}}\).
\begin{enumerate}[label=(\arabic*)]
  \item 若\(\left\{ a_{n} \right\}\)为等差数列,
\item 且\(\forall m,n \in \mathbb{N}^{\text{*}}\),
\item \(S_{mn} = S_{m}S_{n}\),
\item 求数列\(\left\{ a_{n} \right\}\)的通项公式;
  \item 若对任意\(m,n \in N^{\text{*}}\),
\item 都有\(\frac{2S_{m + n}}{m + n} - \frac{2S_{m}}{m} = a_{m + n} - a_{m}\).

\item ①求证:\(\left\{ a_{n} \right\}\)是等差数列;

\item ②设\(b_{n} = 2^{a_{n}}\),
\item \(S_{2025} = 2025^{2}\),
\item \(4049 \leq \log_{2}\left( \sum_{i = 1}^{2025}b_{i} \right) < 4050\),
\item 求\(\left\{ a_{n} \right\}\)的公差\(d\)的值.
\end{enumerate}
\topics{等差数列通项公式的基本量计算;由递推关系证明数列是等差数列;求等比数列前n项和}
\difficulty{0.15}
\answer{(1)\(a_{n} = 1\)或\(a_{n} = 2n - 1\)
(2)① 证明见解析;②\(d = 2\)}
\explain{(1)因为\(\left\{ a_{n} \right\}\)是等差数列,
设公差为\(d\),
因为\(\forall m,n \in N^{\text{*}},S_{mn} = S_{m}S_{n}\),\par
则令\(m = n = 1\)得\(S_{1} = S_{1}^{2}\),
即\(a_{1} = a_{1}^{2}\),
因为\(a_{1} \in N^{\text{*}}\),所以\(a_{1} = 1\).\par
令\(m = n = 2\)得\(S_{4} = S_{2}^{2}\),
则\(4a_{1} + 6d = \left( 2a_{1} + d \right)^{2}\),\par
即\(4 + 6d = {(2 + d)}^{2}\),\par
化简得\(d^{2} - 2d = 0\),则\(d = 2\)或0.\par
当\(d = 0\)时,
\(a_{n} = 1,S_{n} = n\)满足\(S_{mn} = S_{m}S_{n}\);\par
当\(d = 2\)时,
\(a_{n} = 2n - 1,S_{n} = n^{2},S_{mn} = {(mn)}^{2} = S_{m}S_{n}\).\par
所以,\(a_{n} = 1\)或\(a_{n} = 2n - 1\).\par
(2)①令\(m = 1\)时,
\(\frac{2S_{n + 1}}{n + 1} - 2S_{1} = a_{n + 1} - a_{1}\),\par
即\(2S_{n + 1} = (n + 1)\left( a_{n + 1} + a_{1} \right)\),\par
\(\therefore\) \(2S_{n} = n\left( a_{n} + a_{1} \right)\),
\(n \geq 2\).\par
化简得:\(2a_{n + 1} = (n + 1)\left( a_{n + 1} + a_{1} \right) - n\left( a_{n} + a_{1} \right)\),\par
即\((n - 1)a_{n + 1} - na_{n} + a_{1} = 0(n \geq 2)\),\par
\(\therefore\) \((n - 2)a_{n} - (n - 1)a_{n - 1} + a_{1} = 0(n \geq 3)\).\par
化简得:\(2(n - 1)a_{n} = (n - 1)a_{n - 1} + (n - 1)a_{n + 1}\),
即\(2a_{n} = a_{n - 1} + a_{n + 1}(n \geq 3)\).\par
又\(\because\frac{2S_{3}}{3} - \frac{2S_{2}}{2} = a_{3} - a_{2}\),
\(\therefore\) \(a_{1} + a_{3} = 2a_{2}\).\par
\(\therefore\) \(\left\{ a_{n} \right\}\)为等差数列.\par
②因为\(S_{2025} = 2025^{2}\),
所以\(a_{1} + 1012d = 2025\).\par
\(\frac{b_{n + 1}}{b_{n}} = 2^{a_{n + 1} - a_{n}} = 2^{d}\),
所以\(\sum_{i = 1}^{2025}b_{i} = \frac{2^{a_{1}}\left( 1 - 2^{2025d} \right)}{1 - 2^{d}}\).\par
\(\log_{2}\left( \sum_{i = 1}^{2025}b_{i} \right) = a_{1} + \log_{2}\frac{1 - 2^{2025d}}{1 - 2^{d}}\),\par
\(= a_{1} + \log_{2}\frac{2^{2024d}\left( 1 - \frac{1}{2^{2025d}} \right)}{1 - \frac{1}{2^{d}}}= a_{1} + 2024d + \log_{2}\frac{\left( 1 - \frac{1}{2^{2025d}} \right)}{1 - \frac{1}{2^{d}}}< a_{1} + 2024d + \log_{2}\frac{1}{1 - \frac{1}{2^{d}}}\).\par
因为\(0 < \log_{2}\frac{1}{1 - \frac{1}{2^{d}}} \leq 1\),
所以\(a_{1} + 2024d = 4049\),\par
又\(a_{1} + 1012d = 2025\),所以\(d = 2\).}
\end{question}
