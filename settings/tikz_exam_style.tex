% ===================================================================
% settings/tikz_exam_style.tex
% ===================================================================
% Unified TikZ style for exam figures
% 使用方法: 在 settings/preamble.sty 中 % ===================================================================
% settings/tikz_exam_style.tex
% ===================================================================
% Unified TikZ style for exam figures
% 使用方法: 在 settings/preamble.sty 中 % ===================================================================
% settings/tikz_exam_style.tex
% ===================================================================
% Unified TikZ style for exam figures
% 使用方法: 在 settings/preamble.sty 中 % ===================================================================
% settings/tikz_exam_style.tex
% ===================================================================
% Unified TikZ style for exam figures
% 使用方法: 在 settings/preamble.sty 中 \input{settings/tikz_exam_style.tex}
% ===================================================================

% ===== 全局样式定义 =====
\tikzset{
  % 实线 (边、线段、曲线)
  examline/.style={
    line width=0.8pt,
    line cap=round,
    line join=round
  },
  % 虚线 (辅助线、隐藏线)
  examdashed/.style={
    line width=0.8pt,
    dash pattern=on 3pt off 2pt,
    line cap=round
  },
  % 坐标轴
  examaxis/.style={
    line width=0.8pt,
    -{Stealth[length=2.5mm,width=1.6mm]},
    line cap=round
  },
  % 点 (顶点、交点)
  examdot/.style={
    circle,
    fill=black,
    inner sep=0pt,
    minimum size=2pt
  },
  % 空心点
  examdot hollow/.style={
    circle,
    draw=black,
    fill=white,
    inner sep=0pt,
    minimum size=2pt,
    line width=0.6pt
  },
  % 标签文字
  examlabel/.style={
    font=\small,
    inner sep=1pt
  },
  % 向量箭头
  examarrow/.style={
    line width=0.8pt,
    -{Stealth[length=2.5mm,width=1.6mm]},
    line cap=round
  },
  % 直角标记
  examrightangle/.style={
    line width=0.6pt
  },
  % 填充区域
  examfill/.style={
    fill=black!10,
    draw=none
  },
  % 网格
  examgrid/.style={
    line width=0.3pt,
    gray!50
  }
}

% ===== 自定义命令 =====

% 直角标记 (三点定义: \examRightAngle{B}{A}{C} 表示角BAC为直角)
\newcommand{\examRightAngle}[4][0.3cm]{%
  % #1 = 大小 (可选, 默认0.3cm)
  % #2 = 顶点
  % #3 = 角的第一边上的点
  % #4 = 角的第二边上的点
  \draw[examrightangle] 
    ($(#2)!#1!(#3)$) -- 
    ($(#2)!#1!(#3)!#1!90:(#3)$) -- 
    ($(#2)!#1!(#4)$);
}

% ===================================================================
% 使用示例 (在实际文档中):
% \begin{tikzpicture}[scale=1]
%   % 坐标轴
%   \draw[examaxis] (-1,0) -- (4,0) node[below] {$x$};
%   \draw[examaxis] (0,-1) -- (0,3) node[left] {$y$};
%   
%   % 线段和点
%   \draw[examline] (0,0) -- (2,2) -- (3,0);
%   \node[examdot,label=above:$A$] at (2,2) {};
%   \node[examdot,label=right:$B$] at (3,0) {};
%   
%   % 直角标记
%   \examRightAngle{(2,2)}{(0,0)}{(3,0)}
%   
%   % 虚线
%   \draw[examdashed] (0,2) -- (2,2);
% \end{tikzpicture}
% ===================================================================

% ===================================================================

% ===== 全局样式定义 =====
\tikzset{
  % 实线 (边、线段、曲线)
  examline/.style={
    line width=0.8pt,
    line cap=round,
    line join=round
  },
  % 虚线 (辅助线、隐藏线)
  examdashed/.style={
    line width=0.8pt,
    dash pattern=on 3pt off 2pt,
    line cap=round
  },
  % 坐标轴
  examaxis/.style={
    line width=0.8pt,
    -{Stealth[length=2.5mm,width=1.6mm]},
    line cap=round
  },
  % 点 (顶点、交点)
  examdot/.style={
    circle,
    fill=black,
    inner sep=0pt,
    minimum size=2pt
  },
  % 空心点
  examdot hollow/.style={
    circle,
    draw=black,
    fill=white,
    inner sep=0pt,
    minimum size=2pt,
    line width=0.6pt
  },
  % 标签文字
  examlabel/.style={
    font=\small,
    inner sep=1pt
  },
  % 向量箭头
  examarrow/.style={
    line width=0.8pt,
    -{Stealth[length=2.5mm,width=1.6mm]},
    line cap=round
  },
  % 直角标记
  examrightangle/.style={
    line width=0.6pt
  },
  % 填充区域
  examfill/.style={
    fill=black!10,
    draw=none
  },
  % 网格
  examgrid/.style={
    line width=0.3pt,
    gray!50
  }
}

% ===== 自定义命令 =====

% 直角标记 (三点定义: \examRightAngle{B}{A}{C} 表示角BAC为直角)
\newcommand{\examRightAngle}[4][0.3cm]{%
  % #1 = 大小 (可选, 默认0.3cm)
  % #2 = 顶点
  % #3 = 角的第一边上的点
  % #4 = 角的第二边上的点
  \draw[examrightangle] 
    ($(#2)!#1!(#3)$) -- 
    ($(#2)!#1!(#3)!#1!90:(#3)$) -- 
    ($(#2)!#1!(#4)$);
}

% ===================================================================
% 使用示例 (在实际文档中):
% \begin{tikzpicture}[scale=1]
%   % 坐标轴
%   \draw[examaxis] (-1,0) -- (4,0) node[below] {$x$};
%   \draw[examaxis] (0,-1) -- (0,3) node[left] {$y$};
%   
%   % 线段和点
%   \draw[examline] (0,0) -- (2,2) -- (3,0);
%   \node[examdot,label=above:$A$] at (2,2) {};
%   \node[examdot,label=right:$B$] at (3,0) {};
%   
%   % 直角标记
%   \examRightAngle{(2,2)}{(0,0)}{(3,0)}
%   
%   % 虚线
%   \draw[examdashed] (0,2) -- (2,2);
% \end{tikzpicture}
% ===================================================================

% ===================================================================

% ===== 全局样式定义 =====
\tikzset{
  % 实线 (边、线段、曲线)
  examline/.style={
    line width=0.8pt,
    line cap=round,
    line join=round
  },
  % 虚线 (辅助线、隐藏线)
  examdashed/.style={
    line width=0.8pt,
    dash pattern=on 3pt off 2pt,
    line cap=round
  },
  % 坐标轴
  examaxis/.style={
    line width=0.8pt,
    -{Stealth[length=2.5mm,width=1.6mm]},
    line cap=round
  },
  % 点 (顶点、交点)
  examdot/.style={
    circle,
    fill=black,
    inner sep=0pt,
    minimum size=2pt
  },
  % 空心点
  examdot hollow/.style={
    circle,
    draw=black,
    fill=white,
    inner sep=0pt,
    minimum size=2pt,
    line width=0.6pt
  },
  % 标签文字
  examlabel/.style={
    font=\small,
    inner sep=1pt
  },
  % 向量箭头
  examarrow/.style={
    line width=0.8pt,
    -{Stealth[length=2.5mm,width=1.6mm]},
    line cap=round
  },
  % 直角标记
  examrightangle/.style={
    line width=0.6pt
  },
  % 填充区域
  examfill/.style={
    fill=black!10,
    draw=none
  },
  % 网格
  examgrid/.style={
    line width=0.3pt,
    gray!50
  }
}

% ===== 自定义命令 =====

% 直角标记 (三点定义: \examRightAngle{B}{A}{C} 表示角BAC为直角)
\newcommand{\examRightAngle}[4][0.3cm]{%
  % #1 = 大小 (可选, 默认0.3cm)
  % #2 = 顶点
  % #3 = 角的第一边上的点
  % #4 = 角的第二边上的点
  \draw[examrightangle] 
    ($(#2)!#1!(#3)$) -- 
    ($(#2)!#1!(#3)!#1!90:(#3)$) -- 
    ($(#2)!#1!(#4)$);
}

% ===================================================================
% 使用示例 (在实际文档中):
% \begin{tikzpicture}[scale=1]
%   % 坐标轴
%   \draw[examaxis] (-1,0) -- (4,0) node[below] {$x$};
%   \draw[examaxis] (0,-1) -- (0,3) node[left] {$y$};
%   
%   % 线段和点
%   \draw[examline] (0,0) -- (2,2) -- (3,0);
%   \node[examdot,label=above:$A$] at (2,2) {};
%   \node[examdot,label=right:$B$] at (3,0) {};
%   
%   % 直角标记
%   \examRightAngle{(2,2)}{(0,0)}{(3,0)}
%   
%   % 虚线
%   \draw[examdashed] (0,2) -- (2,2);
% \end{tikzpicture}
% ===================================================================

% ===================================================================

% ===== 全局样式定义 =====
\tikzset{
  % 实线 (边、线段、曲线)
  examline/.style={
    line width=0.8pt,
    line cap=round,
    line join=round
  },
  % 虚线 (辅助线、隐藏线)
  examdashed/.style={
    line width=0.8pt,
    dash pattern=on 3pt off 2pt,
    line cap=round
  },
  % 坐标轴
  examaxis/.style={
    line width=0.8pt,
    -{Stealth[length=2.5mm,width=1.6mm]},
    line cap=round
  },
  % 点 (顶点、交点)
  examdot/.style={
    circle,
    fill=black,
    inner sep=0pt,
    minimum size=2pt
  },
  % 空心点
  examdot hollow/.style={
    circle,
    draw=black,
    fill=white,
    inner sep=0pt,
    minimum size=2pt,
    line width=0.6pt
  },
  % 标签文字
  examlabel/.style={
    font=\small,
    inner sep=1pt
  },
  % 向量箭头
  examarrow/.style={
    line width=0.8pt,
    -{Stealth[length=2.5mm,width=1.6mm]},
    line cap=round
  },
  % 直角标记
  examrightangle/.style={
    line width=0.6pt
  },
  % 填充区域
  examfill/.style={
    fill=black!10,
    draw=none
  },
  % 网格
  examgrid/.style={
    line width=0.3pt,
    gray!50
  }
}

% ===== 自定义命令 =====

% 直角标记 (三点定义: \examRightAngle{B}{A}{C} 表示角BAC为直角)
\newcommand{\examRightAngle}[4][0.3cm]{%
  % #1 = 大小 (可选, 默认0.3cm)
  % #2 = 顶点
  % #3 = 角的第一边上的点
  % #4 = 角的第二边上的点
  \draw[examrightangle] 
    ($(#2)!#1!(#3)$) -- 
    ($(#2)!#1!(#3)!#1!90:(#3)$) -- 
    ($(#2)!#1!(#4)$);
}

% ===================================================================
% 使用示例 (在实际文档中):
% \begin{tikzpicture}[scale=1]
%   % 坐标轴
%   \draw[examaxis] (-1,0) -- (4,0) node[below] {$x$};
%   \draw[examaxis] (0,-1) -- (0,3) node[left] {$y$};
%   
%   % 线段和点
%   \draw[examline] (0,0) -- (2,2) -- (3,0);
%   \node[examdot,label=above:$A$] at (2,2) {};
%   \node[examdot,label=right:$B$] at (3,0) {};
%   
%   % 直角标记
%   \examRightAngle{(2,2)}{(0,0)}{(3,0)}
%   
%   % 虚线
%   \draw[examdashed] (0,2) -- (2,2);
% \end{tikzpicture}
% ===================================================================
