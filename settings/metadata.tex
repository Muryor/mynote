% ===================================================================
% settings/metadata.tex
% ===================================================================
% 本文件用于配置:
% 1. 当前要编译的试卷/讲义源文件路径
% 2. 文档元信息(标题、作者、日期等)
%
% 修改说明:
% - 只需修改 \examSourceFile 或 \handoutSourceFile 中的路径,
%   就能切换要编译的试卷或讲义。
% - 路径支持英文文件名(推荐)和中文文件名(macOS + XeLaTeX 测试环境)。
% - 使用相对路径(相对于项目根目录)或绝对路径均可。
% ===================================================================

% -------------------------------------------------------------------
% 试卷编译入口配置
% -------------------------------------------------------------------
% 说明:
% - \examSourceFile 指定当前要编译的试卷源文件路径
% - 修改此路径可切换不同的试卷(无需修改 main-exam.tex)
% 
% 示例(英文路径 - 推荐):
%   \newcommand{\examSourceFile}{content/exams/auto/gaokao_2025_national_2/converted_exam.tex}
%
% 示例(中文路径 - macOS 测试):
%   \newcommand{\examSourceFile}{content/exams/auto/gaokao_2025_national_2/converted_exam.tex}
%
% 当前配置:
\newcommand{\examSourceFile}{content/exams/auto/gaokao_2024_national_2_v2/converted_exam.tex}

% 可选:试卷显示名称(用于识别,非必须)
\newcommand{\examDisplayName}{安徽省江淮十校2026届高三上学期第二次联考数学试题}

% -------------------------------------------------------------------
% 讲义编译入口配置
% -------------------------------------------------------------------
% 说明:
% - \handoutSourceFile 指定当前要编译的讲义源文件路径
% - 修改此路径可切换不同的讲义(无需修改 main-handout.tex)
%
% 示例(英文路径 - 推荐):
%   \newcommand{\handoutSourceFile}{content/handouts/g3/functions/g3_functions_topic01_basic_concepts.tex}
%
% 示例(中文路径 - macOS 测试):
%   \newcommand{\handoutSourceFile}{content/handouts/高三/函数专题/高三函数专题(一).tex}
%
% 当前配置:
\newcommand{\handoutSourceFile}{content/handouts/g3/functions/g3_functions_topic01_basic_concepts.tex}

% 可选:讲义显示名称(用于识别,非必须)
\newcommand{\handoutDisplayName}{高三函数专题(一):基本概念与性质}

% -------------------------------------------------------------------
% 文档元信息(用于讲义封面等)
% -------------------------------------------------------------------
\title{高中数学讲义}
%\subtitle{圆锥曲线分册(Demo)}
\author{刘老师}
%\institute{你的学校或机构}
\date{\today}

% ===================================================================
% 注意事项:
% 1. 路径中避免使用空格,建议用下划线连接(如:高三上_期中.tex)
% 2. 如果使用绝对路径,请在 Shell 中用引号包裹
% 3. 如遇编译错误,可先用英文文件名测试排查
% 4. teacher/student 角色由 build.sh 控制,不在此配置
% ===================================================================
